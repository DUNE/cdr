
\chapter{Summary of Physics}
\label{ch:physics-summary}

The primary science goals of DUNE are drivers for the advancement of
particle physics. The questions being addressed are of wide-ranging
consequence: the origin of flavor and the generation structure of the
fermions (i.e., the existence of three families of quark and lepton
flavors), the physical mechanism that provides the CP violation needed
to generate the Baryon Asymmetry of the Universe, and the high energy
physics that would lead to the instability of matter.  Achieving these
goals requires a dedicated, ambitious and long-term program. 

Observation of $\mathcal{O}(1000)$ $\nu_\mu \rightarrow \nu_e$ events
in the DUNE LArTPCs can be achieved with moderate exposures of around
300 kt-MW-years, depending on the beamline design. When coupled with a
highly capable near detector and sophisticated analysis techniques to
control systematics to a few \%, this level of statistics will enable
discovery ($5\sigma$) of CP violation if it is near maximal, and an
unambiguous highly precise measurement of the mass hierarchy for all
possible values of \deltacp. With an optimized beam design a precision
of $10^\circ$ on $\delta_{\rm CP} = 0$ is also achievable with this
level of exposure.  Exposures of 850 to 1320 kt-MW-yr (depending on
the beam design) would be needed to reach $3\sigma$ sensitivity to CP
violation for 75\% of all values of \deltacp. No experiment can
provide coverage at 100\% of \deltacp values, since CP violation
effects vanish as $\mdeltacp\to 0$ or $\pi$. Higher exposures - with
more detector mass or higher proton beam power - will enable high
precision probes of the 3-flavor model of neutrino mixing, improving
sensitivities to new effects including the presence of sterile
neutrinos and non-standard interactions.

The DUNE far detector will significantly extend lifetime sensitivity
for specific nucleon decay modes by virtue of its high detection
efficiency relative to water Cherenkov detectors and its low
background rates.  As an example, DUNE has enhanced capability for
detecting the $p\to K^+\overline{\nu}$ channel, where lifetime
predictions from supersymmetric models extend beyond, but remain close
to, the current (preliminary) Super-Kamiokande limit of $\tau/B >
\SI{5.9e33}{year}$ (90\% CL). Supersymmetric GUT models in which
the $p\to K^+\overline{\nu}$ channel mode is dominant also favor
other modes involving kaons in the final state, thus enabling a rich 
program of searches for nucleon decay in the DUNE LArTPC detectors.

In a core-collapse supernova, over 99\% of all gravitational binding
energy of the $1.4 M_{\odot}$ collapsed core - some 10\% of its rest
mass - is emitted in neutrinos.  The neutrinos are emitted in a burst
of a few tens of seconds duration, with about half in the first
second. Energies are in the range of a few tens of MeV, and the
luminosity is divided roughly equally between the three known neutrino
flavors.  Compared to existing water Cerenkov detectors, liquid argon
has a unique sensitivity to the electron-neutrino ($\nu_e$) component
of the flux, via the absorption interaction on $^{40}$Ar. The $\nu_e$
component of the flux dominates the very early stages of the
core-collapse, including the ``neutronization'' burst. The observation
of the neutrino signal from a core-collapse supernova in the DUNE
LArTPC's will thus provide unique and unprecedented information on the
mechanics of supernovas, in addition to enabling the search for new
physics. The sensitivity of the DUNE LArTPC's to low energy $\nu_e$
will also enable unique measurements with other astrophysical
neutrinos, such as solar neutrinos.

A highly capable near neutrino detector is required to provide
precision measurements of neutrino interactions, which in the medium
to long term are essential for controlling the systematic
uncertainties in the long-baseline oscillation physics
program. Furthermore, since the near detector data will feature very
large samples of events that are amenable to precision reconstruction
and analysis, they can be exploited for sensitive studies of
electroweak physics and nucleon structure, as well as for searches for
new physics in unexplored regions (heavy sterile neutrinos,
high-$\Delta m^2$ oscillations, light Dark Matter particles, and so
on).

The DUNE experiment is a world-leading international physics
experiment, bringing together the world's neutrino community as well
as leading experts in nucleon decay and particle astrophysics to
explore key questions at the forefront of particle physics and
astrophysics. The highly capable beam and detectors will enable a
large suite of new physics measurements with potential groundbreaking
discoveries.
