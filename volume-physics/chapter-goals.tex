\chapter{LBNF/DUNE Scientific Goals}
\label{ch:physics-goals}


\fixme{We keep going back and forth on capitalizing universe; I'm trying to UNcapitalize it because
I don't like it (even though we used that in the sci opp doc; but it doesn't really matter to me, let's just agree}

\fixme{Also, standardizing on NO PUNCTUATION at end of bullet unless it's a full sentence. And use
``textit'' not ``it'' to italicize.}

LBNF/DUNE will address fundamental questions key to our understanding of the Universe. These include
\begin{itemize}
   \item {\bf What is the origin of the matter-antimatter asymmetry in the Universe?} Immediately after
                    the Big Bang, matter and antimatter were created equally, but now matter dominates.
                    By studying the properties of neutrino and antineutrino oscillations, LBNF/DUNE 
                    will pursue the current most promising avenue for understanding this asymmetry.
   \item {\bf What are the fundamental underlying symmetries of the Universe?} The patterns of mixings and masses between the particles of the Standard Model is not understood. By making precise measurements of the mixing between the neutrinos and the ordering of neutrino masses and comparing these with the quark sector, LNBF/DUNE could reveal new underlying symmetries of the Universe.
  \item{\bf  Is there a Grand Unified Theory of the Universe?} Results from a range of experiments suggest that the
                 physical forces observed today were unified into one force at the birth of the Universe.
                Grand Unified Theories (GUTs), which attempt to describe the unification of forces,
                predict that protons should decay, a process that has never been observed. DUNE will 
                search for proton decay in the range of proton lifetimes predicted by a wide range of GUT models.
   \item{\bf How do supernovae explode and what new physics will we learn from a neutrino burst?}
   Many of the heavy elements that are the key components of life were created in the super-hot cores of collapsing stars. DUNE would be able to detect the neutrino bursts from core-collapse supernova within our galaxy (should any occur). Measurements of the time, flavor and energy structure of the neutrino burst will be critical for understanding the dynamics of this important astrophysical phenomenon, as well as providing information on neutrino properties and other particle physics.
\end{itemize}

\section{Scientific Objectives of LBNF/DUNE}

The LBNF/DUNE scientific objectives are categorized into: the \textit{primary science program}, addressing the key science questions highlighted by the particle physics project prioritization panel (P5); 
a high-priority {\textit{ancillary science program} that is 
enabled by the construction of LBNF and DUNE; and \textit{additional scientific objectives}, that may require developments 
of the LArTPC technology. The goals of the primary science program define the high-level requirements for LBNF and the 
DUNE detectors. The ancillary science program provides further requirements, specifically on the design of the near 
detector, required for the full scientific exploitation of this world leading facility.

\subsection{The Primary Science Program}

The primary science program of the LBNF/DUNE experiment focuses on fundamental open questions in neutrino and astroparticle physics: 
\begin{itemize}
  \item precision measurements of the parameters that govern $\nu_{\mu} \rightarrow \nu_\text{e}$ and
           $\overline{\nu}_{\mu} \rightarrow \overline{\nu}_\text{e}$ oscillations with the goal of
  \subitem -- measuring the charge-parity (CP) violating phase $\delta_\text{CP}$ --- where a value differing from zero or $\pi$ would represent the discovery of CP-violation in the leptonic sector, providing a possible explanation for the matter-antimatter asymmetry in the universe;
  \subitem -- determining the neutrino mass ordering (the sign of $\Delta m^2_{31} \equiv m_3^2-m_1^2$), often referred to as the neutrino\textit{mass hierarchy};  
  \subitem -- precision tests of the three-flavour neutrino oscillation paradigm through studies of muon neutrino disappearance 
    and electron neutrino appearance in both $\nu_\mu$ and $\overline{\nu}_{\mu}$ beams, including the 
    measurement of the mixing angle $\theta_{23}$ and the determination of the octant in which this angle lies;
    \item search for proton decay in several important decay modes, for example $\text{p}\rightarrow\text{K}^+\overline{\nu}$, where the observation of proton decay would represent a ground-breaking discovery in physics, providing a portal to Grand Unification of the forces;
    \item detection and measurement of the $\nu_\text{e}$ flux from a core-collapse supernova within our galaxy, should any occur during the lifetime of the DUNE experiment.
\end{itemize}

\subsection{The Ancillary Science Program}

The intense neutrino beam from LBNF, the massive DUNE LArTPC far detector and the highly-capable DUNE near detector provide a rich ancillary science program, beyond the primary mission of the experiment. The ancillary science program includes:
\begin{itemize}
     \item other accelerator-based neutrino flavor transition measurements with sensitivity to Beyond Standard Model (BSM) physics, such as:    
            \subitem -- non-standard interactions (NSIs);
             \subitem -- the search for sterile neutrinos at both the near and far sites;
             \subitem -- measurements of tau neutrino appearance;
     \item measurements of neutrino oscillation phenomena using atmospheric neutrinos;
     \item a rich neutrino interaction physics program utilizing the DUNE near detector, including:
         \subitem -- a wide-range of measurements of neutrino cross sections;
         \subitem -- studies of nuclear effects, including neutrino final-state interactions;
         \subitem -- measurements of the structure of nucleons;      
         \subitem -- measurement of $\sin^2\theta_\text{W}$;  
     \item  and the search for signatures of dark matter.
\end{itemize} 
Furthermore, a number of previous breakthroughs in particle physics have been serendipitous, in the sense that they were beyond the
original scientific objectives of an experiment. The intense LBNF neutrino beam and novel capabilities for both 
the DUNE near and far detectors will probe new regions of parameter space for both the accelerator-based and astrophysical frontiers, 
providing the opportunity for discoveries that are not currently anticipated.



\subsection{Additional Scientific Objectives}
There are a number of opportunities that could be enabled by developments/improvements to the LArTPC detector technology over the course of the DUNE installation. These include:
\begin{itemize}
      \item measurements of neutrino oscillation phenomena and of solar physics using solar neutrinos;
      \item detection and measurement of the diffuse supernova neutrino flux;
      \item measurement of neutrinos from astrophysical sources at energies from GRBs, active galactic nuclei, black-hole and neutron star mergers, or other transient sources.
\end{itemize}


 

