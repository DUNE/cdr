\chapter{LBNF/Dune Scientific Goals}
\label{ch:physics-goals}

LBNF/DUNE will address fundamental questions key to our understanding of the Universe. These include:
\begin{itemize}
   \item {\bf What is the origin of the matter-antimatter asymmetry in the Universe?} Immediately after
                    the Big Bang, matter and antimatter were created equally, but now matter dominates.
                    By studying the properties of neutrino and antineutrino oscillations, LBNF/DUNE 
                    will pursue the most promising avenue for understanding this asymmetry;
   \item {\bf What are the fundamental underlying symmetries of the Universe?} The patterns of mixings and masses between the particles of the Standard Model is not understood. By making precise measurements of the mixing between the neutrinos and the ordering of neutrino masses and comparing these with the quark sector, DUNE/LNBF could reveal new underlying symmetries of the Universe;
  \item{\bf  Is there a Grand Unified Theory of the Universe?} There is experimental evidence that the
                 physical forces observed today were unified into one force at the birth of the Universe.
                Grand Unified Theories (GUTs), which attempt to describe the unification of forces,
                predict that protons should decay; a process that has never been observed. DUNE will 
                search for proton decay in the range of proton lifetimes predicted by a wide range of GUT models;
   \item{\bf How do supernovae explode?} The heavy elements that are the key components of life --
             such as carbon -- were created in the super-hot cores of collapsing stars. DUNE
             would be able to detect the neutrino burst from core-collapse supernovae within our galaxy (should one occur). 
             The time structure and energy spectrum of the neutrino burst would be
              provide information that is critical for understanding the dynamics of this important astrophysical
             phenomenon.
\end{itemize}

\section{Scientific Objectives of LBNF/DUNE}

The LBNF/DUNE scientific objectives are categorized into: the {\it primary science program}, addressing the key science questions highlighted by the particle physics project prioritization panel (P5); 
a high-priority {\it ancillary science program} that is 
enabled by the construction of LBNF/DUNE; and {\it additional scientific objectives}, that may require developments 
of the LArTPC technology. The goals of the primary science program define the high-level requirements for LBNF and the 
DUNE detectors. The ancillary science program provides further requirements, specifically on the design of the near 
detector, required for the full scientific exploitation of this world leading facility.

\subsection{The Primary Science Program}

The primary science program of the LBNF/DUNE experiment focuses on fundamental open questions in neutrino and astroparticle physics: 
\begin{itemize}
  \item precision measurements of the parameters that govern $\nu_{\mu} \rightarrow \nu_\text{e}$ and
           $\overline{\nu}_{\mu} \rightarrow \overline{\nu}_\text{e}$ oscillations with the goal of
  \subitem -- measurement of the charge-parity (CP) violating phase $\delta_\text{CP}$ -- where a non-zero value would represent the discovery of CP-violation in the leptonic sector, providing a possible explanation for the matter-antimatter asymmetry in the Universe;
  \subitem -- determination of the neutrino mass ordering (the sign of $\Delta m^2_{31} = m_3^2-m_1^2$), often referred to as the neutrino {\it mass hierarchy};  
    \item a precision test of the three-flavour neutrino oscillation paradigm through studies muon neutrino disappearance 
    and electron neutrino appearance in both $\nu_\mu$ and $\overline{\nu}_{\mu}$ beams, including the 
    measurement of the mixing angle $\theta_{23}$ and the determination of the octant in which this angle lies.
    \item the search for proton decay in several important decay modes, for example $\text{p}\rightarrow\text{K}^+\overline{\nu}$, where the observation of proton decay would represent a ground-breaking discovery in physics, providing a portal to Grand Unification of the forces;
    \item detection and measurement of the $\nu_\text{e}$ flux from a core-collapse supernova within our galaxy, should one occur during the lifetime of the DUNE experiment.
\end{itemize}

\subsection{The Ancillary Science Program}

The intense neutrino beam from LBNF, the large LArTPC DUNE far detector and the highly-capable DUNE near detector provide a rich ancillary science program, beyond the primary mission of the experiment. The ancillary science program includes:
\begin{itemize}
     \item other accelerator-based neutrino oscillation measurements with sensitivity to Beyond Standard Model (BSM) physics, such as non-standard interactions (NSIs);
     \item the search for sterile neutrinos at both the near and far sites;
     \item measurements of tau neutrino appearance;
     \item the search for signatures of dark matter; 
     \item measurements of neutrino oscillation phenomena using atmospheric neutrinos;
     \item a rich neutrino interaction physics program utilizing the DUNE near detector, including studies of neutrino-nucleus cross sections and interactions, nucleon structure, neutrino final-state interactions, and the measurement of
 $\sin^2\theta_\text{W}$.     
\end{itemize} 

\subsection{Additional Scientific Objectives}
There are a number of opportunities that could be enabled by developments/improvements to the LArTPC detector technology over the course of the DUNE installation. These include:
\begin{itemize}
      \item detection and measurement of the diffuse supernova neutrino flux;
      \item measurements of neutrino oscillation phenomena and of solar physics using solar neutrinos.
\end{itemize}


 

