
%%%%%%%%%%%%%%%%%%%%%%%%%%%%%%%%%%%%%%%%%%%%%%%%%%%%%%%%%%%%%%%%%%%%%%%%%%%%%%%%%%%%%
\chapter{Near Detector Physics}
\label{ch:physics-nd}

\section{Introduction and Motivation}
\label{sec:physics-nd-introduction}

The LBNF neutrino beam used to study neutrino oscillations in DUNE is
an extended source at the near site, therefore every single spectrum
induced by the neutrino charged (CC) and neutral (NC) current
interactions --- $\nu_\mu$-CC, $\bar \nu_\mu$-CC, $\nu_e$-CC, $\bar
\nu_e$-CC, and the NC --- is different when measured at the far
detector versus the near detector.  In order to achieve the systematic
precision for the signal and background events in the far detector,
which ideally should always be lower than the corresponding
statistical error, the near detector measurements --- including
neutrino fluxes, cross sections, topology of interactions and smearing
effects --- must be unfolded and extrapolated to the far detector
location~\cite{near-detector-REQ1},~\cite{near-detector-REQ2}.  The
charge, ID, and the momentum vector resolution of particles produced
in the neutrino interactions are %the
key to constraining the systematic uncertainties in the predictions at
far detector.
%

To this end, it is useful to recall that for the LBNF low-energy reference beam
(80-GeV protons, 1.07 MW, $1.47 \times 10^{21} $ POT/year), the 
event rates expected at the 40-kt far detector per year are 
2900 (1000) for the $\nu_\mu (\bar \nu_\mu)$
disappearance channel and 230 (45) for the $\nu_e(\bar \nu_e)$
appearance channel (for $\delta_{CP}=0$, normal hierarchy and assumed best fit values of
the mass-squared differences and mixings). For comparison, the raw
event rates per ton of near detector target mass (without detector
effects) for various neutrino interactions in the near detector at
459m from the proton beam target are summarized in
Table~\ref{tab:rates}. The rates are indicated for a ton of target
mass of Ar (Carbon) per $10^{20}$ protons-on-target.  The mass of Ar
in the near detector targets is required to have sufficient mass to
provide $\times$ 10 the statistics of the far detector. Although the
Ar-target design is preliminary, the Ar mass is expected to be
approximately 100kg.

\begin{table}[!htb]
\centering
\caption[Interaction rates, $\nu$ mode, per ton
for \SI{1e20}{\POT}, \SI{459}{\meter}, \SI{120}{\GeV}]{Estimated interaction rates on Ar (Carbon) in the neutrino (second column) and antineutrino (third column) beams per ton of detector 
  for \SI{1e20}{\POT} at \SI{459}{\meter} assuming neutrino
  cross-section predictions from GENIE~\cite{GENIE} and a \GeVadj{120}
  proton beam using the CDR reference design.  Processes are defined at the initial neutrino
  interaction vertex and thus do not include final-state effects. These estimates do not
  include detector efficiencies or acceptance~\cite{DOCDB740,DOCDB783}. 
}
\label{tab:rates}
\footnotesize{
\begin{tabular}[!htbp]{$L^r^r}%rl}  %$
\toprule
\rowtitlestyle
Production mode & $\nu_\mu$ Events  & $\overline\nu_\mu$ Events \\
\rowtitlestyle
                & on Ar (Carbon) & on Ar (Carbon)\\
\toprowrule
CC QE ($\nu_\mu n \rightarrow \mu^- p$)                                       & 52,238 (48,014) & 21,701(25,158) \\ \colhline  
NC elastic ($\nu_\mu N \rightarrow \nu_\mu N$)                                & 19,088 (19,263) & 11,399 (11,576) \\ \colhline  
CC resonant ($\nu_\mu p \rightarrow \mu^- p \pi^+$)                           & 36,137 (40,152) &      0 (0) \\ \colhline  
CC resonant ($\nu_\mu n \rightarrow \mu^- n \pi^+\,(p\pi^0)$)                 & 39,479 (35,890) &      0 (0) \\ \colhline   
CC resonant ($\bar\nu_\mu p \rightarrow \mu^+ p \pi^-\,(n\pi^0)$)             &      0 (0)      & 13,835 (15,372) \\ \colhline
CC resonant ($\bar\nu_\mu n \rightarrow \mu^+ n \pi^-$)                       &      0 (0)      & 19,776 (17,978) \\ \colhline
NC resonant ($\nu_\mu p \rightarrow \nu_\mu p \pi^0\,(n\pi^+)$)               & 11,845 (13,161) &      0 (0) \\ \colhline
NC resonant ($\nu_\mu n \rightarrow \nu_\mu n\pi^+\,(p\pi^0)$)                & 15,221 (13,837) &      0 (0) \\ \colhline
NC resonant ($\bar\nu_\mu p \rightarrow \bar\nu_\mu p\pi^-\,(n\pi^0)$)        &  0 (0)          & 6,533 (7,259)     \\ \colhline
NC resonant ($\bar\nu_\mu n \rightarrow \bar\nu_\mu n \pi^-$)                 &  0 (0)          & 7,944 (7,222)     \\ \colhline
CC DIS ($\nu_\mu N \rightarrow \mu^- X$ or 
$\overline{\nu}_\mu N \rightarrow \mu^+ X$)                                   & 155,613 (150,643) & 39,342 (40,451) \\ \colhline
NC DIS ($\nu_\mu N \rightarrow \nu_\mu X$ or 
$\overline{\nu}_\mu N \rightarrow \overline{\nu}_\mu X$)                      & 50,745 (50,441)   &  16,469 (16,498) \\ \colhline
CC coherent $\pi^+$ ($\nu_\mu A \rightarrow \mu^- A \pi^+$)                   & 1,554 (2,580)     &      0 (0) \\ \colhline
CC coherent $\pi^-$ ($\overline{\nu}_\mu A \rightarrow \mu^+ A \pi^-$)        &     0 (0)         &  1,333 (2,224) \\ \colhline
NC coherent $\pi^0$ ($\nu_\mu A \rightarrow \nu_\mu A \pi^0$ or 
$\overline{\nu}_\mu A \rightarrow \overline{\nu}_\mu A \pi^0$)                & 871 (1,404)       &  750 (1,213) \\ \colhline
%NC resonant radiative decay ($N^* \rightarrow N \gamma $)                    & 110    & 50 \\ \colhline
NC elastic electron ($\nu_\mu e^- \rightarrow \nu_\mu e^-$  
or  $\overline{\nu}_\mu e^- \rightarrow \overline{\nu_\mu} e^-$)              & 27 (29)           & 19 (21) \\ \colhline
Inverse Muon Decay ($\nu_\mu e \rightarrow \mu^- \nu_e$)                      & 15 (16)           & 0 (0) \\ \colhline
Other                                                                         & 3,306 (3,326) & 509 (511) \\ 
\toprule
\rowtitlestyle
Total CC                                                         & 288,475 (280,622) & 96,498 (101,695) \\ %81,340 \\
\rowtitlestyle
Total NC+CC                                                      & 386,271 (378,756) & 139,610 (145,483) \\%114,980 \\ 
\bottomrule
\end{tabular}
}
\end{table}



Importantly, given the scale and ambition of LBNF/DUNE, the near detector must offer a physics 
potential that is as rich as those offered by collider detectors. 
One of the main advantages of a high-resolution near detector built according to the reference design 
(detailed in Volume 4, Chapter 7 of  this CDR) is that it will offer a rich panoply of physics %composed of 
spanning an estimated 100 topics and  resulting in over 200 publications and theses during a ten-year operation. 

%%%%%%%%%%%%%%%%%%%%%%%%%%%%%%%%%%%%%%%%%%%%%%%%%%%%%%%%%%%%%%

\section{Physics Goals of the Near Detector}
\label{sec:physics-nd-goals}

The physics goals of the DUNE near detector fall under three categories: 

\begin{itemize}
\item constraining the systematic uncertainties in  oscillation studies
\item  offering a generational advance in the precision measurements of neutrino interactions, e.g., % including 
cross sections, exclusive processes, electroweak and isospin physics, structure of nucleons and nuclei % etc.;  
\item conducting searches for new physics covering unexplored regions, 
including heavy (sterile) neutrinos, large $\Delta m^2$ neutrino oscillations, light Dark Matter 
candidates, etc. 
\end{itemize}

%We stress that there is a large synergy in the above three goals. 
These three broad goals possess significant synergy. The physics requirements for the near detector are driven 
by the oscillation physics. However,  the unprecedented neutrino fluxes available at LBNF and the challenging 
constraints required by the long-baseline program, especially those related to the CP measurement, also  make the near detector imminently suitable for  short-baseline 
precision physics. And conversely, conducting precision measurements of neutrino 
interactions will actually %would 
result in a reduction of systematic uncertainties on signal and background 
predictions in the far detector~\cite{HIRESMNU, DPR, Adams:2013qkq}.  


%%%%%%%%%%%%%%%%%%%%%%%%%%%%%%%%%%%%%%%%%%%%%%%%%%%%%%%%%%%%%%
\section{The Role of the Near Detector in Oscillation Physics} 
\label{sec-nd-oscl} 

As illustrated in Chapter~\ref{ch:physics-lbnosc}, studies on the impact of different levels of systematic 
uncertainties on the oscillation analysis indicate 
that uncertainties exceeding 1\% for signal and 5\% for backgrounds may result in 
substantial degradation of the sensitivity to CP violation and mass hierarchy. 
%In the following we list the physics measurements required in the near detector in order to 
The near detector physics measurements discussed in this section are needed in order to match this level of systematic uncertainty.   

The near detector will need to determine the relative abundance and 
energy spectrum of all \textit{four} species of neutrinos in the LBNE beam.
%
This requires measurement of 
$\nu_\mu$, $\bar \nu_\mu$, $\nu_e$, and $\bar \nu_e$ via their  CC-interactions,  
which in turn demands precise measurement of $\mu^-$, $\mu^+$, $e^-$, and $e^+$ in the 
the near detector. Specifically, to measure both the small $\nu_e$ and $\bar \nu_e$ contamination in the beam with high precision, 
the detector would need to be able to distinguish $e^+$  from $e^-$. 
This last requirement is motivated by 
\begin{enumerate}
\item the need to measure and identify $\nu - e$ NC elastic scattering (and calibrate 
the corresponding backgrounds) for the absolute flux measurements
\item  the redundancy in determining the distributions 
of the parent mesons -- %and 
in particular, the $K^0$ content -- 
\fixme{together the parent meson distributions are an ingredient, or each distribution is an ingredient?}
which are an essential ingredient for predicting 
the far detector/near detector flux ratio as a function of energy
\item the measurement of the $\pi^0$ yield in CC and NC interactions 
from converted photons
\item the different composition in terms of QE, single-pion resonance, multi-pion resonance, and deep inelastic scattering (DIS) of CC and NC events originated by each of the four species in the far detector and near detector, respectively 
\end{enumerate}


Quantifying asymmetries between neutrinos and antineutrinos, such as 
energy scales and interaction topologies, which are relevant for the measurement of the CP-violating phase, is
another job of the near detector.
%
Since the reference near neutrino detector, the fine-grained tracker (FGT),  
is not identical to the far
 detector, it is not possible in long-baseline analyses to ``cancel'' the event reconstruction
 errors in a near-to-far ratio.  
The extent to which such a cancellation will limit the ultimate precision of the experiment has yet to
 be fully explored. 
Because of the low average density of the FGT (0.1 g/cm$^3$), however, DUNE will be able to 
measure the missing transverse momentum ($p_{T}$)  vector in the CC processes, in addition to accurately 
measuring the lepton and hadron energies.
This redundant missing-$p_{T}$ vector measurement provides a most important
constraint on the neutrino and antineutrino energy scales. Measurements of
exclusive topologies like quasi-elastic, resonance and
coherent meson production offer additional constraints on the neutrino energy scale. 


In the disappearance 
studies, the absolute $\nu_\mu$- and $\bar \nu_\mu$ flux should be determined to $\simeq 3\%$ precision in 
$0.5 \leq E_\nu \leq 8$~GeV so as to eliminate uncertainties in the neutrino 
and antineutrino cross sections affecting the oscillation measurements. For precision measurements of electroweak 
and QCD physics, a similar precision is required at higher energies. 


%\textbf{The  Flux at the far detector relative to that at the near detector}
 
For precision $\nu_\mu$- and $\bar \nu_\mu$-disappearance  
channels, the far detector/near detector ratio of the number of neutrinos ($\nu_\mu$  and $\bar \nu_\mu$)
at a given $E_\nu$ bin in $0.5 \leq E_\nu \leq 8$ GeV range should be known
to $\simeq 1-2\%$ precision. \fixme{Some sentence like ``The blah capability of the near detector will enable this precision.''}

%\textbf{Neutral Current (NC) versus Charged Current (CC)}

%This requirement calls for event-by-event classification of NC and CC events. 
NC processes 
constitute one of the largest backgrounds to all appearance and disappearance oscillation 
channels. It is therefore important for the near detector to make a precise measurement of the NC cross section relative to 
CC as a function of the hadronic energy, $E_{Had}$. 


A precise measurement of $\pi^0$ and photon yields by the near detector in \textit{both}  $\nu$-induced NC and CC interactions is
essential;
this is the most important background to the $\nu_e$- and $\bar \nu_e$ appearance at low energies.


%\textbf{Measurement of the $\pi^{\pm}$ content in CC and NC hadronic jets}  

The $\pi^{\pm}$ content in CC and NC hadronic jets is is the most important background to 
the $\nu_\mu$- and $\bar \nu_\mu$ disappearance coming from the hadronic   $\pi^{\pm} \rightarrow \mu^{\pm}$; 
it can also be a background to the appearance channel at lower energies. \fixme{So how does the ND help?}

%\textbf{Characterization of various exclusive (semi-exclusive) channels such as Quasi-Elastic (QE), Resonance (Res), Coherent-mesons and Deep-Inelastic-Scattering (DIS)}

Precise near detector measurements leading to characterization of 
 various exclusive (semi-exclusive) 
channels such as Quasi-Elastic (QE), resonance (Res), 
coherent-mesons and Deep-Inelastic-Scattering (DIS)
%these processes 
will yield  
 in situ constraints on the nuclear effects from both initial and final interaction (FSI). 

%\textbf{Quantification of neutrino-argon cross section by measuring interactions off Ar, Ca, C, H, etc. targets}

The near detector will quantify the neutrino-argon cross section 
by measuring interactions off Ar, Ca, C, H, etc. targets. The goal is to provide 
a consistent model, as opposed to an empirical parametrization, for the nuclear effects. 


Finally, the near detector will constrain NC and CC backgrounds to the $\tau-$ appearance in the far detector, 
which is one of the most important, and unique, capabilities of the liquid argon (LAr) detector. \fixme{Sorry, what's the unique
FD capability? This says the ND will constrain the backgrounds. Clarify.}

The above requirements suggest a high-resolution, magnetized near detector for identifying and 
measuring $e^+$, $e^-$, $\mu^-$, $\mu^+$, $\pi^0$, $\pi^{+,-}$ and protons  with high efficiency. 



%%%%%%%%%%%%%%%%%%%%%%%%%%%%%%%%%%%%%%%%%%%%%%%%%%%%%%%%%%%
\section{Precision Measurements at the Near Detector} 
\label{sec-nd-sbp} 

Over a five-year run in neutrino mode,  %focusing neutrinos,  
the intense neutrino source at LBNF will provide 
${\cal {O}}$(100) million neutrino interactions in a 7-t near detector; 
and about $0.4$ times as many in antineutrino mode. %while focusing antineutrinos. 
The high-resolution, fine-grained near detector described in \voldune would offer not only the requisite systematic 
precision for  oscillation studies, but also a generational advance in the precision measurements and unique 
searches that a neutrino beam can provide. \fixme{Anne added `beam'} This section outlines the salient physics reach of this detector; further details can be 
found in~\cite{DPR} and~\cite{Adams:2013qkq}. Discussed first are precision measurements that would support 
and impact the oscillation physics program, followed by examples of other non-oscillation-related 
physics measurements that would extend our knowledge of important aspects of particle physics. 


\subsection{Precision Measurements Related to Oscillation Physics}

% Absolute Flux Measurement:} 

Using  $\nu$-electron NC scattering, the absolute neutrino flux can be determined 
to $\leq 3\%$ precision in the range $0.5 \leq E_\nu \leq 10$ GeV. Additionally, the $\nu$-electron CC scattering leading to  inverse 
muon decay would determine the absolute flux to $\simeq 3\%$ precision in $E_\nu \geq 20$ GeV region~\cite{ABS-FLUX}. 
The DUNE near detector's ability to determine the background (primarily from $\nu_{\mu}$ quasi-elastic scattering) 
to inverse muon decay \textit{without} relying on $\bar \nu_\mu$ measurements or 
ad hoc extrapolations, such as made in CCFR~\cite{CCFR-IMD-Mishra-89},~\cite{CCFR-IMD-Mishra-90} 
and CHARM~\cite{CHARM-IMD-95}, allows such precisions, which are dominated by statistics. 
Importantly, the ability to extract the quasi-elastic 
$\bar \nu_\mu$-H interaction, via subtraction of hydrocarbon and pure carbon targets, would allow an extraction of 
the $\bar \nu_\mu$ absolute flux to a few percent precision. Furthermore, novel techniques such as the use of 
coherent-$\rho$ meson production in the near detector combined with photo-production could provide a constraint 
on the absolute flux to $\simeq 5\%$ in the intermediate $5 \leq E_\nu \leq 20$ GeV region. 
We note that this near detector will be the first %neutrino experiment 
to in-situ-constrain the absolute flux to a level approaching $\sim$2.5\% precision. 

%Relative Neutrino and Antineutrino Flux Measurement:} 
The most promising method of determining 
the shape of  $\nu_\mu$ and $\bar \nu_\mu$ flux is by measuring the low-hadronic energy CC, the Low-$\nu_0$
method~\cite{MISHRA-Nu0}. The method, when combined with the empirical parametrization of the $\pi^{\pm}$, $K^{\pm}$ 
and such hadroproduction data as would be available in the coming decade, permits a bin-to-bin precision of 1--2\% on 
the flux spanning $1 \leq E_\nu \leq 50$ GeV ~\cite{near-detector-REL-FLUX}. 
Recent model calculations by Bodek et al. ~\cite{Bodek:2012uu} confirm these estimates. 
Specifically, for the $\nu_\mu$ and $\bar \nu_\mu$ disappearance the Low-$\nu_0$ method would predict the 
far detector/near detector($E_\nu$) to 1--2\% precision. \fixme{Where does ``Relative Neutrino and Antineutrino Flux Measurement'' fit in?}

%Flavor Content of the Neutrino Source:} 
By precisely measuring the $\nu_\mu$-, $\bar \nu_\mu$-, $\nu_e$-, $\bar \nu_e$-CC, 
the near detector will decompose the $\pi^+$, $K^+$, $\pi^-$, $K^-$, $\mu^+$, $\mu^-$, and $K^0_L$ contents of the beam, thus 
allowing a precise far detector/near detector($E_\nu$) prediction to a few percent.  This ability lends a unique power to 
not only measure the cross sections of all four neutrino species, but also to allow sensitive searches for new 
physics, e.g., violation of universality and large-$\Delta$m$^2$ oscillations, as discussed %adumbrated 
below. 

\noindent
For the $\delta_{CP}$ in particular, the near detector measurements would constrain the $\nu_e$/$\nu_\mu$ to 
$<1\%$ and $\bar \nu_e$/$\bar \nu_\mu$ to $\simeq 1\%$ precision, thereby vastly reducing the associated error. 



%Determination of the $E_\nu$-Scale of Neutrinos versus Anti-Neutrinos:} 
The high resolution 
measurements of $E_\mu$ and $E_e$ and those of charged hadrons and the reconstruction of about one 
million $K^0_S$ and several million $\pi^0$ will provide a tight constraint on the (anti)neutrino energy scale. 
However, nuclear physics, including initial and final state interactions, 
 affects the $E_\nu$-scale and can affect $\nu$ differently from $\bar\nu$ interactions thus  
producing a spurious contribution to any measured CP-violating observable. 
The unique experimental handle on these seemingly intractable effects comes from a 
a precise measurement of the missing transverse momentum {\it vector},{ \bf{ $P^m_T$}} afforded by the 
high-resolution near detector. 
%Specific interaction modes are being explored to quantify the precision 
%on the $E_\nu$-scale in $\nu$ versus $\bar\nu$. 


%Event-by-Event Measurements of NC interactions:} 
The ability to determine {\bf{$P^m_T$}} affords an event-by-event 
identification of NC events. This is particularly crucial in order to decompose the background contributions to 
the $\nu_e$ or $\bar \nu_e$ appearance and the disappearance measurements. 


%Measurement of $\pi^0$, $\pi^+$, $\pi^-$, $K^+$, $K^-$, Proton, $K^0_S$ and $\Lambda$ in NC \& CC:} 
The yields and momentum vector measurements of %these 
$\pi^0$, $\pi^+$, $\pi^-$, $K^+$, $K^-$, proton, $K^0_S$ and $\Lambda$ particles in CC and NC, as a function of 
visible energy, will provide an ``event generator'' $measurement$ for the far detector and constrain the 
``hadronization'' error to $\leq 2.5\%$ associated with the far detector prediction. 

%of exclusive and semi-exclusive measurements. 


%Quasi-Elastic (QE) and Resonance Measurements:}  
In any long-baseline neutrino oscillation program, including LBNF/DUNE, 
the quasi-elastic (QE) interactions are special. First, the QE cross section is substantial, especially at the second oscillation maximum. 
Second, because of the simple topology --- a muon and a proton --- QE provides, to  first order, 
a close approximation to $E_\nu$.  
Precise momentum measurements  of this two-track topology impose direct constraints on nuclear effects 
associated with both initial and final state interactions~\cite{near-detector-QE}. 

Resonance is the second most dominant interaction mode, besides deep inelastic scattering (DIS),  
in the LBNF/DUNE oscillation range, $0.5 \leq E_\nu \leq 10$ GeV.
% Traditionally, the resonance measurements 
%have been imprecise due to the difficulty of measuring the protons and pions.
 By measuring the complete 
topology of resonance, a high-resolution near detector 
will offer an unprecedented precision on the resonance cross section, and will provide in situ constraints on the 
nuclear effects~\cite{near-detector-RES}.  


%Nucleon Structure, Parton Distribution Functions, and QCD Studies:} 
Precision measurements of  structure functions and 
differential cross sections would directly affect the oscillation measurements 
by providing accurate simulations of neutrino interactions. They would also offer an estimate of 
all background processes that are dependent upon the angular distribution of 
the outgoing particles in the far detector. 
\fixme{Next sentence is a mouthful; can't parse unambiguously; clarify} Furthermore, QCD analyses within the framework of global fits to 
extract parton distribution functions (PDF) by using the differential cross sections 
measured in near detector data provide a crucial input,  because they  constrain systematic errors in 
precision electroweak measurements.

% not only in neutrino physics but also in hadron-collider measurements.\\
Under the rubric of nucleon structure, the topics \fixme{for QCD analysis?} include: 
\begin{enumerate}
\item Measurement of form factors and structure functions
\item QCD analysis,  tests of perturbative QCD and quantitating the non-perturbative 
QCD effects
\item d/u Parton distribution functions at large $x$, which is the limiting error in the 
$\nu_\tau$-CC measurements/searches at the far detector
\item Sum rules and the strong coupling constant; and v) Quark-hadron duality
\end{enumerate}

%Neutrino-Argon Interactions and Nuclear Effects:} 
An integral part of the near detector physics program is a set of detailed measurements of (anti)neutrino 
interactions in argon and in a variety of nuclear targets including calcium, carbon, hydrogen (via subtraction), 
and steel~\cite{near-detector-NUCL}. 
The goals are twofold, (1) obtain a model-independent direct measurement of nuclear effects in Ar 
using the FGT's ability to isolate $\nu (\bar \nu)$ interactions off free hydrogen 
via subtraction of hydrocarbon and carbon targets; and
(2) measure the neutrino-nuclear interactions as to allow an accurate modeling 
of initial and final state effects. The studies would include 
the nuclear modification of form factors and structure functions, effects in coherent and incoherent 
regimes, nuclear dependence of exclusive and semi-exclusive processes, and nuclear 
effects including short-range correlations, pion-exchange currents, pion absorption, shadowing, 
initial-state interactions and final-state interactions. 

%%%%%%%%%%%%%%%%%%%%%%%%%%%%%%%%%%%%%%%%%%%%%%%%%
\subsection{Other Precision Measurements}

The near detector may be able to make precision measurements of electroweak physics. \fixme{check} 
%Neutrinos and antineutrinos are the most effective probes for investigating electroweak physics.  
Interest in a precise determination of the weak mixing angle ($\sin^2 \theta_W$) at DUNE 
energies via neutrino scattering is twofold: (1) it provides a direct measurement of neutrino couplings to 
the $Z$ boson, and (2) it probes a different scale of momentum transfer than LEP did by virtue
of not being at the $Z$ boson mass peak. 
% 
The weak mixing angle can be extracted experimentally from several independent NC physics processes:
(1) deep inelastic scattering off quarks inside nucleons: $\nu N \to \nu X$; (2) elastic scattering off electrons: $\nu e^- \to \nu e^-$; 
(3) elastic scattering off protons: $\nu p \to \nu p$; iv) coherent $\rho^0$ meson production. 
Note that these processes involve
substantially different scales of momentum transfer, providing a tool
to test the running of $\sin^2 \theta_W$ wihin a single experiment. 

The most sensitive channel for $\sin^2 \theta_W$ 
in the DUNE near detector is expected to be the $\nu N$ DIS through a precision measurement 
of the NC/CC cross-section ratio~\cite{near-detector-EW}. This measurement will be dominated by systematic uncertainties, which can be 
accurately constrained by dedicated in situ measurements using the large CC samples and employing corresponding 
improvements in theory that will have evolved over the course of the experiment. Using the existing knowledge of 
structure functions and cross-sections we expect a relative precision of about $0.35\%$ on $\sin^2 \theta_W$, with 
the default low-energy LBNF beam. An increase of the fluxes with a beam upgrade and/or a one year run with a high 
energy tuning of the neutrino spectrum would allow a substantial reduction of uncertainties down to about $0.2\%$. 
This level of precision is comparable to colliders (LEP) and offers a shot at discovery.
 
The various independent channels measured in the DUNE near detector can be combined though global electroweak fits, 
further optimizing the sensitivity to electroweak parameters. The level of precision achievable as well as the richness of 
the physics measurements put the DUNE near detector electroweak program on par with the gold standard electroweak measurements at LEP.   



% Isospin Physics and the Adler Sum Rule:} 
One of the most compelling physics topics 
accessible to the DUNE-near detector in the LBNF-beam is the isospin physics 
using neutrino and antineutrino interactions. Given the statistics and a commensurate 
resolution of near detector, for the first time we have a chance to test the Adler sum-rule to a 
few percent level, and perhaps claim a discovery. 
Precision test of sum-rules is a rich ground for finding something new, refuting the prevalent wisdom. 
An added motivation is the possibility of isospin asymmetry in nucleons . 

To accomplish this, neutrino and anti-neutrino scattering off hydrogen is needed. 
Whereas the Adler sum rule is the prize, the $\bar \nu_\mu$-H and $\nu_\mu$-H scattering
will provide (a)  the 
absolute flux normalization via low-Q$^2$ $\bar \nu_\mu$-QE interactions,  
and (b) will be crucial to achieve a model-independent measurement of nuclear effects in the 
neutrino-nuclear interactions. 

%Measurement of Nucleon's Strangeness Content:} 
The question of whether strange
  quarks contribute substantially to the vector and axial-vector
  currents of the nucleon remains unresolved. A large observed value of the
  strange-quark contribution to the nucleon spin (axial current),
  $\Delta s$, would enhance our understanding of the proton structure.

The strange \emph{axial vector} form factors are poorly 
determined. The most direct measurement of $\Delta s$, which does not rely on the difficult
measurements of the $g_1$ structure function at very small values of the Bjorken variable $x$, 
can be obtained from (anti)neutrino NC elastic scattering off protons.  %

The low-density magnetized tracker in DUNE near detector can provide a good proton reconstruction efficiency as well as
high resolution on both the proton angle and energy, down to $Q^2\sim0.07$~GeV$^2$. 
This capability will reduce the uncertainties in the extrapolation of the form factors to the limit
$Q^2 \to 0$. About $2.0 (1.2) \times 10^6$ $\nu p
(\overline{\nu} p)$ events are expected after the selection cuts in
the low-density tracker, yielding a statistical precision on the order
of 0.1\%.


%%%%%%%%%%%%%%%%%%%%%%%%%%%%%%%%%%%%%%%%%%%%%%%%%
\section{New Physics Searches} 
\label{sec-nd-np} 

A search for heavy neutrinos is intriguing. 
The most economic way to handle the problems of neutrino masses, dark matter and baryon asymmetry of the universe in a unified way may be to add to the SM three Majorana singlet fermions with masses roughly on the order of the masses of known quarks and leptons. 
%The appealing feature of this theory (called the $\nu$-MSM for ``Neutrino Minimal SM'') is the fact that there every 
%left-handed fermion has a right-handed counterpart, leading to an symmetric way of treating quarks and leptons.
 The lightest of the three new leptons is expected to have a mass from 1~keV to 50~keV and play the role of the Dark Matter particle (for details and additional references, see~\cite{DPR} and ~\cite{Adams:2013qkq}).

The most effective  mechanism of sterile neutrino production is through weak two body and three body decays of heavy mesons and baryons. In the search for heavy neutrinos, the strength of the proposed high-resolution near detector, compared to earlier experiments lies in reconstructing the exclusive decay modes, including electronic, hadronic and muonic channels. Furthermore, the detector provides a means to constrain and measure the backgrounds using control samples. Preliminary investigations of these issues are ongoing and  suggest that the FGT  will have an order of magnitude higher sensitivity in exclusive channels than previous experiments did. 

%Search for Large $\Delta$m$^2$ Oscillation:} 

The near detector could potentially search for large $\Delta$m$^2$ oscillations. As has become evident over the past decade or more, there may be evidence from several distinct experiments that points towards the existence of sterile neutrinos with mass in the range 1~eV$^2$ (for details and additional references, see~\cite{DPR} and~\cite{Adams:2013qkq}).  A short-baseline neutrino program has been initiated at Fermilab and elsewhere to clear the questions raised by these varying pieces of evidence.  

Since the DUNE near detector is located at a baseline of several hundred meters and uses the LE beam, it has values of  $L/E \sim 1$, which render it sensitive to these oscillations --- if they exist. Due to the differences between neutrinos and antineutrinos,  four possibilities have to be considered in the analysis: 
$\nu_\mu$ disappearance,  $\bar \nu_\mu$ disappearance, $\nu_e$ appearance and $\bar \nu_e$  appearance. 
It must be noted that  the search for the high
 $\Delta$m$^2$ oscillations must be performed simultaneously with the in situ determination of the fluxes. 
To this end, it is necessary to obtain an independent prediction of the $\nu_e$  and $\bar \nu_e$  fluxes starting from the measured $\nu_\mu$ and  $\bar \nu_\mu$ CC distributions, since the  $\nu_e$  and $\bar \nu_e$ CC distributions could be distorted by the appearance signal. An iterative procedure has been developed to handle this, details of which can be found in~\cite{DPR} and~\cite{Adams:2013qkq}.\\

%Light (sub-GeV) Dark Matter (DM) Searches in the Neutrino Beam at DUNE}
Recently, a great deal of interest has been generated in searching for DM at low-energy, fixed-target experiments.  High-flux neutrino beam experiments, as DUNE is planned to be, have been shown to provide coverage of the  DM and DM-mediator parameter space which can be covered by neither direct detection nor collider experiments. Upon striking the target, the proton beam can produce  dark photons either directly through $pp(pn)\rightarrow \bf {V}$  or indirectly through the production of a $\pi^{0}$ or a $\eta$ meson which then promptly decays into a SM photon and a dark photon. For the case of $m_{V}\geq 2m_{DM}$, the dark photons will quickly decay into a pair of DM particles.  These relativistic DM particles from the beam will travel along with the neutrinos to the DUNE near detector.  The DM particles can then be detected through neutral-current-like interactions either with electrons or nucleons in the detector. 

Since the signature of DM events looks just like those of the neutrinos, the neutrino beam provides the major source of background for the DM signal. Several ways have been proposed to suppress neutrino backgrounds by using the unique characteristics of the DM beam. Since DM, due to much higher maas,  will travel much more slowly than the neutrinos,  the timing  in the near detector becomes a discriminator.  In addition, since the electrons struck by DM will be much more forward in direction, the angles of these electrons may be used to reduce backgrounds, taking advantage of the fine angular resolution of the DUNE near detector.  Finally, a special run can be devised to turn off the focusing horn to significantly reduce the charged particle flux that will produce neutrinos. Further studies are required to determine appropriate  hardware-parameter choices that could benefit these searches, including granularity, absorbers, timing resolution, DAQ-speed, etc. Studies are also required to determine if DUNE will effectively cover the important region in parameter space between the MiniBooNE exclusion and the direct detection region of  the most popular candidates.
 
%Distortion of Neutrino Flux at High Energies ($E_\nu > 10$ GeV):} 
%LBNF/DUNE 
DUNE will be the first long-baseline experiment possessing a large 
statistics of high-energy $\nu_\mu$,  $\bar \nu_\mu$, and $\nu_e$ + $\bar \nu_e$ CC and NC events 
measured with high precision in the liquid argon far detector. An obvious venue for discovery is to 
search for distortion at energies greater than 10~GeV, not envisioned by the 
PMNS mixing. 
%The role of near detector will be to provide a precise prediction of 
%the neutrino spectra at higher energies, including NC.  

%Measurement of $\nu_\tau$-CC at far detector:} 
%LBNF/DUNE 
DUNE will also be the first long-baseline experiment with an ability to 
reconstruct $\nu_\tau$-appearance with high statistics. The paucity of the number of measured 
$\nu_\tau$-CC motivates searching for new physics. 
The role of near detector --- where no $\tau$ is expected --- will be to accurately ``calibrate'' background topologies 
in the NC and CC interactions, most notably at large $x_{bj}$. 


\section{Summary}
\label{sec:physics-nd-summary}


%{\it The DUNE-near detector, as embodied in the FGT reference design, will offer a rich physics portfoilio which will not only  support and buttress the oscillation program at the far detector, but also extend our knowledge of cross sections, structure functions, interaction topologies, weak interaction parameters, sum rules and possibly lead to the discovery of new phenomena. It will  do so by providing  tracking of charged particles at levels of precision unattained by previous experiments.  In the above, we have tried to summarize these capabilities and provide a glimpse of its impact on the overall DUNE physics program.}
The DUNE near detector, as embodied in the FGT reference design, will offer a rich physics portfolio that will not 
only support and buttress the oscillation program at the far detector, but also extend our knowledge of fundamental interactions 
and of the structure of nucleons and nuclei, possibly leading to the discovery of new phenomena. It will do 
so by providing tracking of charged particles at levels of a precision unattained by previous experiments.  
In the above, we have tried to summarize these capabilities and provide a glimpse of its 
impact on the overall DUNE physics program. 



%\end{document}


