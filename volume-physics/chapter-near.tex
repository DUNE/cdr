
%%%%%%%%%%%%%%%%%%%%%%%%%%%%%%%%%%%%%%%%%%%%%%%%%%%%%%%%%%%%%%%%%%%%%%%%%%%%%%%%%%
\chapter{Near Detector Physics}
\label{ch:physics-nd}

Because neutrinos oscillate and because the neutrino beam is an extended  source, every single 
spectrum,  induced by the neutrino charged (CC) and neutral (NC) current interactions -- $\nu_\mu$-CC, 
$\bar \nu_\mu$-CC, $\nu_e$-CC, $\bar \nu_e$-CC, and the NC ---, measured at the far detector (FD)  is different from that at the near 
detector (ND). 
In order to achieve the systematic precision for the signal and background events  in the FD,  
which ideally should always be lower than the corresponding statistical error, the ND 
measurements,  including neutrino fluxes,  cross-sections, topology of 
interactions and smearing effects, must be unfolded and extrapolated to the far detector location 
~\cite{ND-REQ1}, ~\cite{ND-REQ2}.  The charge, ID,  and the momentum vector resolution of 
particles, produced in the neutrino interactions,  are the key to constraining  the systematic 
uncertainties in the predictions at FD. 

%The particle ID and the energy, angular and spatial resolutions of the ND 
%are key factors in reducing the systematic 
%uncertainties on the predicted FD spectra, allowing   a high precision in the measurements of the
%elements of the neutrino mixing matrix and, importantly, affording  
%redundancy to establish a discovery should something entirely unexpected be observed in the FD.  

Importantly, given the scale and ambition of LBNF/DUNE, the ND must offer a physics 
potential that is as rich as those offered by collider detectors. 
One of the main advantages of a high resolution ND built according to the reference design 
(detailed in Volume 4, section 7 of  this document) is that it will offer a rich panoply of physics composed 
of an estimated 100 topics,  resulting in over 200 publications and  theses during a ten-year operation. 

%%%%%%%%%%%%%%%%%%%%%%%%%%%%%%%%%%%%%%%%%%%%%%%%%%%%%%%%%%%%%%

\section{Physics Goals of the Near Detector}
\label{sec:physics-nd-goals}

The physics goals of the DUNE-ND fall under three broad categories: 

\noindent
{\boldmath $ {\cal A:}$} Constraining the systematic uncertainties in  oscillation studies;  

\noindent
{\boldmath $ {\cal B:}$} Offering a generational advance in the precision measurements of neutrino interactions, including 
cross-sections, exclusive processes, electroweak and isospin physics, structure of nucleons and nuclei etc.;  

\noindent
{\boldmath $ {\cal C:}$} Conducting searches for new physics covering unexplored regions, 
including heavy (sterile) neutrinos, large $\Delta m^2$ neutrino oscillations, light Dark Matter 
candidates etc. \\
\noindent
We stress that there is a large synergy in the above three goals. The physics requirements for the ND are driven 
by the oscillation physics. However,  the unprecedented neutrino fluxes available at LBNF and the challenging 
constraints required by the long-baseline program, especially those related to the CP measurement,   make the ND imminently suitable for  short-baseline 
precision physics. Conversely, conducting precision measurements of neutrino 
interactions would result in a reduction of systematic uncertainties on signal and background 
predictions in the FD ~\cite{HIRESMNU, DPR, Adams:2013qkq}. 



\subsection{The Role of ND in Oscillation Physics}  
\label{sec-nd-oscl} 

%For neutrino oscillation studies ($\nu$OSCL) in LBNF/DUNE ---  
%be it $\nu_e$ or $\bar \nu_e$ appearance leading to CP-violating $\delta$ and mass hierarchy (MH), or 
%$\nu_\mu$ or $\bar \nu_\mu$ disappearance, or distortion in the NC spectrum,  or the non-standard oscillation --  
%the associated systematic error determined from the extrapolation of the ND measurements 
%must be less than the corresponding statistical error in the FD. 
As illustrated in Section~\ref{ch:physics-lbnosc}, studies on the impact of different levels of systematic 
uncertainties on the oscillation analysis indicate 
that uncertainties exceeding 1\% for signal and 5\% for backgrounds may result in 
substantial degradation of the sensitivity to CP violation and mass hierarchy. 
In the following we list the physics measurements required in the ND in order to 
match this level of systematic uncertainty:   

\vspace{0.25cm} 
\noindent
{\bf (1):}  Determination of the relative abundance and 
energy spectrum of all  {\bf four} species of neutrinos in the LBNE beam: Measurement of 
$\nu_\mu$, $\bar \nu_\mu$, $\nu_e$, and $\bar \nu_e$ via their  CC-interactions.  
The requirement demands precise measurement of $\mu^-$, $\mu^+$, $e^-$, and $e^+$ in the 
the ND. Specifically, to measure both the small $\nu_e$ and $\bar \nu_e$ contamination in the beam with high precision, 
the detector would need to be able to distinguish $e^+$  from $e^-$. 

% this would require a low density detector with a commensurately long physical radiation length. 


\vspace{0.25cm} 
\noindent
{\bf (2):} Quantify asymmetries between neutrinos and anti-neutrinos, such as 
energy-scales and interaction topologies, which are relevant for the measurement of the CP-violating phase. 


\vspace{0.25cm} 
\noindent
{\bf (3):} Determination of the absolute $\nu_\mu$- and $\bar \nu_\mu$-flux so as to  eliminate uncertainties in the neutrino and anti-neutrino cross-sections 
affecting the oscillations measured  in LBNF/DUNE In the  disappearance 
studies, the absolute flux should be determined to $\simeq 3\%$ precision in 
$0.5 \leq E_\nu \leq 8$~GeV. For precision measurements of electroweak 
and QCD physics a similar precision is required at higher energies. 

\vspace{0.25cm} 
\noindent
{\bf (4):} The  Flux at FD Relative to ND: For precision $\nu_\mu$- and $\bar \nu_\mu$-disappearance  
channels, the ratio of number of neutrinos ($\nu_\mu$  and $\bar \nu_\mu$) at FD to ND (FD/ND) 
at given $E_\nu$ bin in $0.5 \leq E_\nu \leq 8$ GeV range should be known 
to $\simeq 1\%$ precision. 

\vspace{0.25cm} 
\noindent
{\bf (5):} Neutral Current (NC) versus Charged Current (CC): 
The requirement calls for event by event classification of NC  
CC events leading to a precise measurement of 
the NC cross-section relative to 
CC as a function of the hadronic energy, $E_{Had}$, since NC processes 
constitute one of the largest background to all appearance and disappearance oscillation 
channels.

\vspace{0.25cm} 
\noindent
{\bf (6):} Precise measurement 
of $\pi^0$ and photon yields  in {\bf both}  $\nu$-induced 
NC and CC interactions: This is the most important background to the $\nu_e$- and $\bar \nu_e$-appearance; 


\vspace{0.25cm} 
\noindent
{\bf (7)} Measurement of the $\pi^{\pm}$ content in 
CC and NC hadronic jets:  This is the most important background to 
the $\nu_\mu$- and $\bar \nu_\mu$-disappearance coming from  the hadronic   $\pi^{\pm} \rightarrow \mu^{\pm}$; 

\vspace{0.25cm} 
\noindent
{\bf (8)} Characterization of various exclusive (semi-exclusive) 
channels such as Quasi-Elastic (QE), Resonance (Res), 
Coherent-mesons and Deep-Inelastic-Scattering (DIS): 
Precise measurements of these processes yields  
 in situ constraints on the nuclear effects from both initial and final interaction (FSI); 

\vspace{0.25cm} 
\noindent 
{\bf (9)} Quantification of neutrino-argon cross-section 
by measuring interactions off Ar, Ca, C, H, etc. targets: The goal is to provide 
a consistent model, as opposed to an empirical parametrization, for the nuclear effects. 



\vspace{0.25cm} 
\noindent 
{\bf (10):} Constrain NC and CC backgrounds to the {\boldmath $\tau-$}Appearance in FD, 
which is one of the most important, and unique, capabilities of the Liquid-Argon (LAr) detector. 
%This is a sufficiently important physics topic to warrant that FD should 'calibrate' topologies 
%in the NC and CC interactions at ND, where no $\tau$ is expected, to allow FD to conduct 
%the measurement which will be dominated by the FD-statistics.

\noindent
The above requirements suggest a high resolution, magnetized ND for identifying and 
measuring $e^+$, $e^-$, $\mu^-$, $\mu^+$, $\pi^0$, $\pi^{+,-}$ and protons  with high efficiency. 



%%%%%%%%%%%%%%%%%%%%%%%%%%%%%%%%%%%%%%%%%%%%%%%%%%%%%%%%%%%
\subsection{Precision Measurements at the ND} 
\label{sec-nd-sbp} 

\noindent
Over a 5 year run focusing neutrinos,  the intense neutrino source at LBNF would provide 
${\cal {O}}$(100) million neutrino interactions in a 7 tonne ND; 
and about half as many while focusing anti-neutrinos. 
%An ND possessing the possessing 
%resolution and redundancy  to match such a statistical precision, the physics possibilities are enormous. 
A high resolution, fine grained ND such as described in Volume-4 would offer not only the requisite systematic 
precision for  oscillation studies but a generational advance in the precision measurements and unique 
searches  that neutrinos provide. Here we outline the salient physics reach of such an ND; further details can be 
found in ~\cite{DPR} and ~\cite{Adams:2013qkq}. 

\vspace{0.25cm} 
\noindent 
{\bf Absolute Flux Measurement:} Using  $\nu$-electron NC scattering, the absolute flux can be determined 
to $\leq 3\%$ precision in the range $0.5 \leq E_\nu \leq 10$ GeV. Additionally, the $\nu$-electron CC scattering leading to  inverse 
muon decay would determine the absolute flux to $\simeq 3\%$ precision in $E_\nu \geq 20$ GeV region 
~\cite{ABS-FLUX}. 
The DUNE-ND's ability to determine the background $without$ relying on $\bar \nu_\mu$ measurements or 
ad hoc extrapolations, such as made in CCFR ~\cite{CCFR-IMD-Mishra-89}, ~\cite{CCFR-IMD-Mishra-90} 
and CHARM ~\cite{CHARM-IMD-95}, allows such precisions which are dominated by statistics. 
Importantly, the ability to extract the quasi-elastic 
$\bar \nu_\mu$-H interaction, via subtraction of hydrocarbon and pure carbon targets, would allow an extraction of 
the $\bar \nu_\mu$ absolute flux to a few percent precision. Furthermore, novel techniques such as the use of 
coherent-$\rho$ meson production in ND combined with the photo-production, could provide a constraint 
to the absolute flux to $\simeq 5\%$ in the intermediate $5 \leq E_\nu \leq 20$ GeV region. 
We note that the DUNE-ND will be the first neutrino experiment to in situ constrain the absolute flux 
to a level approaching $~2.5\%$ precision. 

\vspace{0.25cm} 
 
\noindent 
{\bf Relative Neutrino and Antineutrino Flux Measurement:} The most promising method of determining 
the shape of  $\nu_\mu$ and $\bar \nu_\mu$ flux is by measuring the low-hadronic energy CC, the Low-$\nu_0$ method ~\cite{MISHRA-Nu0}. The method, when combined with the empirical parametrization of the $\pi^{\pm}$, $K^{\pm}$ 
and such hadroproduction data as would be available in the coming decade, permits a bin-to-bin precision of 1--2\% on 
the flux spanning $1 \leq E_\nu \leq 50$ GeV ~\cite{ND-REL-FLUX}. 
Recent model calculations by Bodek et al. ~\cite{Bodek:2012uu} confirm these estimates. 
Specifically, for the $\nu_\mu$ and $\bar \nu_\mu$ disappearance the Low-$\nu_0$ method would predict the 
FD/ND($E_\nu$) to 1--2\% precision. 

\vspace{0.25cm} 
\noindent 
{\bf Flavor Content of the Neutrino Source:} By precisely measuring the $\nu_\mu$-, $\bar \nu_\mu$-, $\nu_e$-, $\bar \nu_e$-CC, 
the ND would decompose the $\pi^+$, $K^+$, $\pi^-$, $K^-$, $\mu^+$, $\mu^-$, and $K^0_L$ contents of the beam, thus 
allowing a precise FD/ND($E_\nu$) prediction to a few percent.  This ability lends a unique power to 
not only measure the cross-sections of all four neutrino species but also allow sensitive searches for new 
physics, such a violation of universality and large-$\Delta$m$^2$ oscillations,  as adumbrated below. 

For the $\delta_{CP}$ in particular, the ND measurements would constrain the $\nu_e$/$\nu_\mu$ to 
$<1\%$ and $\bar \nu_e$/$\bar \nu_\mu$ to $\simeq 1\%$ precision, thereby vastly reducing the associated error. 



\vspace{0.25cm} 
\noindent 
{\bf Determination of the $E_\nu$-Scale of Neutrinos versus Anti-Neutrinos:} The high resolution 
measurements of $E_\mu$ and $E_e$ and those of charged hadrons and the reconstruction of about one 
million $K^0_S$ and several million $\pi^0$ provide a tight constraint on the (anti)neutrino energy scale. 
However,  nuclear physics, including initial and final state interactions, 
 affects the $E_\nu$-scale. Also, it can affect $\nu$ differently from $\bar\nu$ interactions thus  
producing a spurious contribution to any measured CP-violating observable. 
The unique experimental handle on these seemingly intractable effects comes from a 
a precise measurement of the missing transverse momentum {\it vector},{ \bf{ $P^m_T$}} afforded by the 
high resolution ND. 
%Specific interaction modes are being explored to quantify the precision 
%on the $E_\nu$-scale in $\nu$ versus $\bar\nu$. 


\vspace{0.25cm} 
\noindent 
{\bf Event-by-Event Measurements of NC interactions:} The ability to determine  { \bf{ $P^m_T$}} affords an event-by-event 
identification of NC events. This is particularly crucial to decompose the background contributions to 
the $\nu_e$ or $\bar \nu_e$ appearance and the disappearance measurements. 


\vspace{0.25cm} 
\noindent 
{\bf Measurement of $\pi^0$, $\pi^+$, $\pi^-$, $K^+$, $K^-$, Proton, $K^0_S$ and $\Lambda$ in NC \& CC:} 
The yields and momentum vector measurements of these particles in CC and NC, as a function of 
visible energy will provide an "event generator" $measurement$ for the FD and constrain the 
`hadronization' error to $\leq 2.5\%$ associated with the FD-prediction. 

%of exclusive and semi-exclusive measurements. 



\vspace{0.25cm} 
\noindent 
{\bf Quasi-Elastic (QE) and Resonance Measurements:}  
In any long-baseline neutrino oscillation program, including LBNF/DUNE, 
the quasi-elastic (QE) interactions are special. First, the QE cross section is substantial, especially at the second oscillation maximum. 
Second, because of the simple topology -- a muon and a proton -- QE provides, to the first order, 
a close approximation to $E_\nu$.  
Precise momentum measurements  of this two track topology impose direct constraints on nuclear effects 
associated with both initial and final state interactions~\cite{ND-QE}. 

Resonance is the second most dominant interaction mode, besides deep inelastic scattering (DIS),  
in the LBNF/DUNE oscillation range, $0.5 \leq E_\nu \leq 10$ GeV.
% Traditionally, the resonance measurements 
%have been imprecise due to the difficulty of measuring the protons and pions.
 By measuring the complete 
topology of resonance, a high resolution ND 
would offer an unprecedented precision on the resonance cross section, and will provide in situ constraints on the 
nuclear effects ~\cite{ND-RES}.  




\vspace{0.25cm} 
\noindent 
{\bf Nucleon Structure, Parton Distribution Functions, and QCD Studies:} 
Precision measurements of  structure functions and 
differential cross sections would directly affect the oscillation measurements 
by providing accurate simulations of neutrino interactions and offer an estimate of 
all background processes that are dependent upon the angular distribution of 
the outgoing particles in the FD. \\
Furthermore, QCD analyses within the framework of global fits to 
extract parton distribution functions (PDF) by using the differential cross sections 
measured in ND data provide a crucial input,  because they  constrain systematic errors in 
precision electroweak measurements.\\
% not only in neutrino physics but also in hadron-collider measurements.\\
Under the rubric of nucleon-structure, the topics include: i) Measurement of form factors and structure functions; ii) QCD analysis,  tests of perturbative QCD,  and quantitating the non-perturbative 
QCD effects; iii) d/u Parton distribution functions at large x which is the limiting error in the 
$\nu_\tau$-CC measurements/searches at FD ; iv) Sum rules and the strong coupling constant; and v) Quark-hadron duality. 
%\noindent
%$\bullet$ Measurement of form factors and structure functions; 
%
%
%\noindent
%$\bullet$ QCD analysis,  tests of perturbative QCD,  and quantitating the non-perturbative 
%QCD effects; 
%
%\noindent
%$\bullet$ d/u Parton distribution functions at large x which is the limiting error in the 
%$\nu_\tau$-CC measurements/searches at FD ; %
%\noindent
%$\bullet$ Sum rules and the strong coupling constant 
%\noindent
%$\bullet$ Quark-hadron duality 

\vspace{0.25cm} 
\noindent 
{\bf Neutrino-Argon Interactions and Nuclear Effects:} 
An integral part of the ND physics program is a  detailed measurements of (anti)neutrino 
interactions in Argon and in a variety of nuclear targets including Calcium, Carbon, Hydrogen (via subtraction), 
and Steel~\cite{ND-NUCL}. 
The goals are twofold: a) obtain a model-independent direct measurement of nuclear effects in Ar (from 
the Hydrogen target); b) measure the neutrino-nuclear interactions as to allow an accurate modeling 
of initial and final state effects. The studies would include 
the nuclear modification of form-factors and structure functions, effects in coherent and incoherent 
regimes, nuclear dependence of exclusive and semi-exclusive processes, and nuclear 
effects including short-range correlations, pion-exchange currents, pion absorption, shadowing, 
initial-state interactions, and final-state interactions. 

\vspace{0.25cm} 
\noindent 
{\bf Precision measurements of  Electroweak Physics:} 
%Neutrinos and antineutrinos are the most effective probes for investigating electroweak physics.  
Interest in a precise determination of the weak mixing angle ($\sin^2 \theta_W$) at DUNE 
energies via neutrino scattering is twofold: (1) it provides a direct measurement of neutrino couplings to 
the $Z$ boson and (2) it probes a different scale of momentum transfer than LEP did by virtue
of not being at the $Z$ boson mass peak. 
% 
The weak mixing angle can be extracted experimentally from several independent NC physics processes:
i) deep inelastic scattering off quarks inside nucleons: $\nu N \to \nu X$; ii) elastic scattering off electrons: $\nu e^- \to \nu e^-$; 
iii) elastic scattering off protons: $\nu p \to \nu p$; iv) coherent $\rho^0$ meson production. 
Note that these processes involve
substantially different scales of momentum transfer, providing a tool
to test the running of $\sin^2 \theta_W$ wihin a single experiment. 

The most sensitive channel for $\sin^2 \theta_W$ 
in the DUNE ND is expected to be the $\nu N$ DIS through a precision measurement 
of the NC/CC cross-section ratio~\cite{ND-EW}. This measurement will be dominated by systematic uncertainties, which can be 
accurately constrained by dedicated in situ measurements using the large CC samples and employing corresponding 
improvements in theory that will have evolved over the course of the experiment. Using the existing knowledge of 
structure functions and cross-sections we expect a relative precision of about $0.35\%$ on $\sin^2 \theta_W$, with 
the default low-energy LBNF beam. An increase of the fluxes with a beam upgrade and/or a one year run with a high 
energy tuning of the neutrino spectrum would allow a substantial reduction of uncertainties down to about $0.2\%$. 
This level of precision is comparable to colliders (LEP) and offers a shot at discovery. 

The various independent channels measured in the DUNE ND can be combined though global electroweak fits, 
further optimizing the sensitivity to electroweak parameters. The level of precision achievable as well as the richness of 
the physics measurements put the DUNE ND electroweak program on par with the gold standard electroweak measurements at LEP.   



\vspace{0.25cm} 
\noindent 
{\bf Isospin Physics and the Adler Sum Rule:} One of the most compelling physics topics 
accessible to the DUNE-ND in the LBNF-beam is the isospin physics 
using neutrino and antineutrino interactions. Given the statistics and a commensurate 
resolution of ND, for the first time we have a chance to test the Adler sum-rule to a 
few percent level, and perhaps claim a discovery. 
Precision test of sum-rules is a rich ground for finding something new, refuting the prevalent wisdom. 
An added motivation is the possibility of isospin asymmetry in nucleons . 

To accomplish this, neutrino and anti-neutrino scattering off hydrogen is needed. 
Whereas the Adler sum rule is the prize, the $\bar \nu_\mu$-H and $\nu_\mu$-H scattering
will provide (a)  the 
absolute flux normalization via low-Q$^2$ $\bar \nu_\mu$-QE interactions,  
and (b) will be crucial to achieve a model-independent measurement of nuclear effects in the 
neutrino-nuclear interactions. 
%In the proposed ND, neutrino-H scattering will be obtained 
%via statistical subtraction between the radiator targets (C$_3$H$_6$)$_n$ and a pure carbon (graphite) 
%target. The resulting physics is expected to be limited by 
%the statistical precision of the samples.  

\vspace{0.25cm} 
\noindent 
{\bf Measurement of Nucleon's Strangeness Content:} The question is whether the strange
  quarks contribute substantially to the vector and axial-vector
  currents of the nucleon still remains unresolved. A large observed value of the
  strange-quark contribution to the nucleon spin (axial current),
  $\Delta s$, would enhance our understanding of the proton structure.
  %The spin structure of the nucleon also affects the couplings of axions and
%supersymmetric particles to dark matter. 

The strange \emph{axial vector} form factors are poorly 
determined. The most direct measurement of $\Delta s$, which does not rely on the difficult
measurements of the $g_1$ structure function at very small values of the Bjorken variable $x$, 
can be obtained from (anti)neutrino NC elastic scattering off protons.  %
%Systematic uncertainties can be reduced by measuring the ratios of NC to CC (quasi-elastic) 
%for both neutrinos and antineutrinos as a function of $Q^2$. 

The low-density magnetized tracker in DUNE ND can provide a good proton reconstruction efficiency as well as
high resolution on both the proton angle and energy, down to $Q^2\sim0.07$~GeV$^2$. 
This capability will reduce the uncertainties in the extrapolation of the form factors to the limit
$Q^2 \to 0$. About $2.0 (1.2) \times 10^6$ $\nu p
(\overline{\nu} p)$ events are expected after the selection cuts in
the low-density tracker, yielding a statistical precision on the order
of 0.1\%.



\subsection{New Physics Searches} 
\label{sec-nd-np} 

\vspace{0.25cm} 
\noindent 
{\bf Search for Heavy Neutrinos:} 
The most economic way to handle the problems of neutrino masses, dark matter and baryon asymmetry of the Universe in a unified way may be to add to the SM three Majorana singlet fermions with masses roughly on the order of the masses of known quarks and leptons. 
%The appealing feature of this theory (called the $\nu$-MSM for ``Neutrino Minimal SM'') is the fact that there every 
%left-handed fermion has a right-handed counterpart, leading to an symmetric way of treating quarks and leptons.
 The lightest of the three new leptons is expected to have a mass from 1 keV to 50 keV and play the role of the dark matter particle (for details and additional references, see ~\cite{DPR} and ~\cite{Adams:2013qkq}).

The most effective  mechanism of sterile neutrino production is through weak two body and three body decays of heavy mesons and baryons. In the search for heavy neutrinos, the strength of the proposed high-resolution ND, compared to earlier experiments lies in reconstructing the exclusive decay modes, including electronic, hadronic and muonic channels. Furthermore, the detector provides a means to constrain and measure the backgrounds using control samples. Preliminary investigations of these issues are ongoing and  suggest that the FGT  will have an order of magnitude higher sensitivity in exclusive channels than previous experiments did. 

\vspace{0.25cm} 
\noindent 
{\bf Search for Large $\Delta$m$^2$ Oscillation:} 

As has become evident over the past decade or more, there may be evidence from several distinct experiments pointing towards the existence of sterile neutrinos with mass in the range $1$ eV$^2$ (for details and additional references, see ~\cite{DPR} and ~\cite{Adams:2013qkq}).  A short baseline neutrino program has been initiated at Fermilab and elsewhere to clear the questions raised by these varying pieces of evidence.  

Since the ND at DUNE  is located at a baseline of several hundred meters  and uses the LE beam, it has values of   $L/E \sim 1$ which render it sensitive to these oscillations if they exist. Due to the differences between neutrinos and antineutrinos,  four possibilities have to be considered in the analysis: $\nu_\mu$ disappearance,  $\bar \nu_\mu$ disappearance, $\nu_e$ appearance and $\bar \nu_e$  appearance. It must be noted that  the search for the high
 $\Delta$m$^2$ oscillations has to be performed simultaneously with the in situ determination of the fluxes. 
To this end, we need to obtain an independent prediction of the $\nu_e$  and $\bar \nu_e$  fluxes starting from the measured $\nu_\mu$ and  $\bar \nu_\mu$ CC distributions since the  $\nu_e$  and $\bar \nu_e$ CC distributions could be distorted by the appearance signal. We have evolved an iterative procedure to handle this, details of which can be found in ~\cite{DPR} and ~\cite{Adams:2013qkq}.\\

\vspace{0.25cm}
\noindent
{\bf Light (sub-GeV) Dark Matter (DM) Searches in the Neutrino Beam at DUNE}
Recently, a great deal of interest has been generated in searching for DM at low-energy, fixed-target experiments.  High flux neutrino beam experiments, such as DUNE, have been shown to provide coverage ofthe  DM and DM mediator parameter space which cannot be covered by either direct detection or collider experiments. Upon striking the target, the proton beam can produce  dark photons either directly through $pp(pn)\rightarrow \bf {V}$  or indirectly through the production of a $\pi^{0}$ or a $\eta$ meson which then promptly decays into a SM photon and a dark photon. For the case where $m_{V}\geq 2m_{DM}$, the dark photons will quickly decay into a pair of DM particles.  These relativistic DM particles from the beam will travel along with the neutrinos to the DUNE near detector.  The DM particles can then be detected through neutral-current like interactions either with electrons or nucleons in the detector. Since the signature of DM events looks just like those of the neutrinos, the neutrino beam provides the major source of background for the DM signal. Several ways have been proposed to suppress neutrino backgrounds by using the unique characteristics of the DM beam. Since DM, due to much higher maas,  will travel much slower than the neutrinos,  the timing  in the near detector becomes a discriminator.  In addition, since the electrons struck by DM will be much more forward in direction, the angles of these electrons may be used to reduce backgrounds, taking advantage of the fine angular resolution of the DUNE-ND.  Finally, a special run can be devised to turn off the focusing horn to significantly reduce the charged particle flux that will produce neutrinos. Further studies are required to determine appropriate  hardware-parameter choices that could benefit these searches, including granularity, absorbers, timing resolution, DAQ-speed etc. Studies are also required  to determine if DUNE will effectively cover the important region in parameter space between the MiniBooNE exclusion and the direct detection region of  the most popular candidates.
 
\vspace{0.25cm} 
\noindent 
{\bf Distortion of Neutrino Flux at High Energies ($E_\nu > 10$ GeV):} 
LBNF/DUNE will be the first long baseline experiment possessing a large 
statistics of high energy $\nu_\mu$,  $\bar \nu_\mu$, and $\nu_e$ + $\bar \nu_e$ CC and NC events 
measured with high precision in the liquid argon FD. An obvious venue for discovery is to 
search for distortion at energies greater than 10 GeV, not envisioned by the 
PMNS mixing. 
%The role of ND will be to provide a precise prediction of 
%the neutrino spectra at higher energies, including NC.  


\vspace{0.25cm} 
\noindent 
{\bf Measurement of $\nu_\tau$-CC at FD:} 
LBNF/DUNE will be the first long baseline experiment with an ability to 
reconstruct $\nu_\tau$-appearance with high statistics. The paucity of the number of measured 
$\nu_\tau$-CC motivates searching for new physics. 
The role of ND -- where no $\tau$ is expected -- will be to accurately 'calibrate' background topologies 
in the NC and CC interactions, most notably at large-x$_{bj}$. 


