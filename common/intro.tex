% Intro shared by all subsections

\section{An International Physics Program}

The global neutrino physics community is developing a multi-decade
physics program to measure unknown parameters of the Standard Model of
particle physics and search for new phenomena.  
The program will be carried out by an international,
leading-edge, dual-site experiment
for neutrino science and proton decay studies, which is known as the Deep Underground
Neutrino Experiment (DUNE), hosted at Fermilab in Batavia, IL. The
facility required for this experiment, the Long-Baseline Neutrino
Facility (LBNF), will be an internationally designed, coordinated and
funded program, comprising the world's highest-intensity neutrino beam
at Fermilab and the infrastructure necessary to support the massive
DUNE cryogenic far detectors installed deep underground at the Sanford
Underground Research Facility (SURF), \num{800} miles (\SI{1300}{\km}) downstream,
in Lead, SD. LBNF will provide the facilities to house the DUNE near
detectors on the Fermilab site. %LBNF and DUNE will be tightly coordinated as DUNE collaborators design the detectors and infrastructure that will carry out this experimental program. (moved down)

%  5/14 Anne added the following from the global strategy section, with edits
The strategy for executing the experimental program presented in this Conceptual 
Design Report (CDR) has been developed to meet the requirements 
set out in the P5 report~\cite{p5report} and takes into account the recommendations of the European 
ESPP strategy~\cite{ESPP-2012}, adopting a model where U.S. and international funding agencies 
share costs on the DUNE detectors, and CERN and other participants provide in-kind contributions 
to the supporting infrastructure. LBNF and DUNE will be tightly
coordinated as DUNE collaborators design the detectors and
infrastructure that will carry out this experimental program.
  
The LBNF scope includes
\begin{itemize}
\item an intense neutrino beam aimed at the far site
\item conventional facilities at both the near and far sites
\item and cryogenics infrastructure to support the DUNE
  liquid argon time-projection chamber (LArTPC) detectors at SURF
\end{itemize}

The DUNE scope includes
\begin{itemize}
\item a high-performance neutrino detector and beamline monitoring system
located a few hundred meters downstream of the neutrino source
\item and a massive LArTPC neutrino detector located deep underground at
  the far site
\end{itemize}

With the facilities provided by LBNF and the detectors
provided by DUNE, the DUNE Collaboration proposes to mount a focused
attack on the puzzle of neutrinos with broad sensitivity to neutrino
oscillation parameters in a single experiment.  The focus of the
program will be the determination of the neutrino mass hierarchy and the explicit demonstration of leptonic CP violation,
if it exists, by precisely measuring the asymmetric oscillations of
muon-type neutrinos and antineutrinos into respectively electron-type neutrinos and
antineutrinos.  Siting the far detector deep underground will provide
exciting additional research opportunities in nucleon decay, neutrino
astrophysics and studies of neutrino bursts from supernovae occurring
in our galaxy.

\section{A Roadmap of the Conceptual Design Report}

The LBNF/DUNE CDR describes the proposed physics program and 
technical designs at the conceptual design stage.  At this stage, the design is
still undergoing development and the CDR therefore presents a \textit{reference design} for each element as well as any 
\textit{alternative designs} that are under consideration.

The CDR is composed of four volumes and is supplemented
by several annexes that provide details on the physics program and technical designs. The volumes are as follows:

\begin{itemize}
\item \volintro provides an executive summary of and strategy for the experimental program and of the CDR as a whole.
\item \volphys outlines the scientifc objectives and describes the physics studies that the DUNE Collaboration will undertake to address them.
\item \vollbnf describes the LBNF Project, which includes design and construction of the beamline at Fermilab, the conventional facilities at both Fermilab and SURF, and the cryostat and cryogenics infrastructure required for the DUNE far detector.
\item \voldune describes the DUNE Project, which includes the design, construction and commissioning of the near and far detectors. 
\end{itemize}
\fixme{check vollbnf title}
Annexes to these volumes are listed at \fixme{provide URL}:

