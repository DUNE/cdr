% Intro shared by all subsections

\section{Introduction to LBNF and DUNE}
\label{sec:intro-lbnfdune-eachvol}

The global neutrino physics community is developing a multi-decade physics program to measure unknown parameters of the Standard Model of particle physics and search for new phenomena.   It is based on a leading-edge, dual-site experiment
%The global neutrino physics community is coming together to develop a leading-edge, dual-site 
for neutrino science and proton decay studies, the Deep Underground Neutrino Experiment 
(DUNE), hosted at Fermilab in Batavia, IL. The facility required for this experiment, the Long-Baseline 
Neutrino Facility (LBNF), will be an internationally designed, coordinated and funded program, comprising 
the world's highest-intensity neutrino beam at Fermilab and the infrastructure necessary to support 
DUNE's massive, cryogenic far detectors installed deep underground at the Sanford Underground 
Research Facility (SURF), 800 miles (1,300 km) downstream, in Lead, SD. LBNF will also provide the 
facilities to house the experiment's near detectors on the Fermilab site. LBNF and DUNE will be tightly coordinated as DUNE collaborators design the detectors that will carry out its experimental program. 
  
The LBNF scope includes 
\begin{itemize}
\item an intense neutrino beam aimed at a far site
\item construction of conventional facilities at both the near and far sites
\item cryogenics infrastructure at the far site to support the DUNE liquid argon time-projection
chamber (LArTPC) detector
\end{itemize}

The DUNE scope includes
\begin{itemize}
\item %a fine-grained neutrino detector 
a high-performance neutrino detector and beamline monitoring system located a few hundred meters downstream of the neutrino source
\item a massive LArTPC neutrino detector located deep underground at the far site
\end{itemize}

With the facilities provided by the LBNF Project and the detectors provided by DUNE, the DUNE Science Collaboration proposes to mount a broad attack on the science of neutrinos with sensitivity 
to all known parameters in a single experiment.  The focus of the program will be the explicit 
demonstration of leptonic CP violation, if it exists, by precisely measuring the asymmetric oscillations of 
muon-type neutrinos and antineutrinos into electron-type neutrinos and antineutrinos.
Siting the far detector deep underground will provide
opportunities for research in additional areas of physics, such as nucleon decay and neutrino
astrophysics, in particular, studies of neutrino bursts from supernovae occurring in our galaxy.

