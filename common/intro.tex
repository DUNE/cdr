% Intro shared by all subsections

%\input{volume-1-intro/chapter-execsum} -- copied from there but heading levels changed

\section{Introduction to the LBNE Project}
\label{sec:intro-lbne-each-vol}

The Long-Baseline Neutrino Experiment (LBNE) Project team has prepared this Conceptual
Design Report (CDR) which describes a world-class facility to enable a compelling research program in neutrino physics. The ultimate goal in
the operation of the facility and experimental program is to measure fundamental physical
parameters, explore physics beyond the Standard Model and better elucidate the nature of matter
and antimatter. 

Although the Standard Model of particle physics presents a remarkably accurate
description of the elementary particles and their interactions, it is known that the current
model is incomplete and that a more fundamental underlying theory must exist. Results from the
last decade, revealing that the three known types of neutrinos have nonzero mass, mix with one
another and oscillate between generations, point to physics beyond the Standard Model.
Measuring the mass and other properties of neutrinos is fundamental to understanding the deeper,
underlying theory and will profoundly shape our understanding of the evolution of the universe.


\subsection{About this Conceptual Design Report}
The LBNE Conceptual Design Report is intended to describe, at a conceptual level, the scope and design of the experimental and conventional facilities that the LBNE Project plans to build to address a well-defined set of neutrino-physics measurement objectives.  At this Conceptual Design stage the LBNE Project presents a {\em Reference Design} for all of the planned components and facilities, and alternative designs that are still under consideration for particular elements. 
%the science goals listed in Chapter~\ref{v1ch:sci-objectives}.  
The scope includes 
\begin{itemize}
\item an intense neutrino beam aimed at a far site
\item detectors located at the near site just downstream of the neutrino source
\item a massive neutrino detector located at the far site
\item construction of conventional facilities at both the near and far sites
\end{itemize}
The selected near and far sites are Fermi National Accelerator Laboratory (Fermilab), in Batavia, IL and  Sanford Underground Laboratory at Homestake (Sanford Laboratory), respectively. The latter is the site of the formerly proposed Deep Underground Science and Engineering
Laboratory (DUSEL) in Lead, South Dakota.

This CDR is organized into six stand-alone volumes, one to describe the overall LBNE Project and one for each of its component subprojects: 
\begin{itemize}
\item Volume 1: The LBNE Project
\item Volume 2: The Beamline at the Near Site
\item Volume 3: Detectors at the Near Site
\item Volume 4: The Liquid Argon Detector at the Far Site
\item Volume 5: Conventional Facilities at the Near Site
\item Volume 6: Conventional Facilities at the Far Site
\end{itemize}

Volume 1 is intended to provide readers of varying backgrounds an introduction to LBNE and to the following volumes of this CDR.  It contains high-level information and refers the reader to topic-specific volumes and supporting documents, also listed in Section~\ref{intro-supp-doc}. 
Each of the other volumes contains a common, brief introduction to the overall LBNE Project, an introduction to the individual subproject and a detailed description of its conceptual design. 

\subsection{LBNE and the U.S. Neutrino-Physics Program}

In its 2008 report, the Particle Physics Project Prioritization Panel (P5) recommended a world-class
neutrino-physics program as a core component of the U.S. particle physics program \cite{p5report}. Included
in the report is the long-term vision of a large detector at the Sanford Laboratory and a high-intensity neutrino source at  Fermilab.

On January 8, 2010, the Department of Energy (DOE) approved the Mission Need for a new long-baseline
neutrino experiment that would enable this world-class program and firmly establish the
U.S. as the leader in neutrino science. The LBNE Project is designed to meet this Mission Need.

With the facilities provided by the LBNE Project, the LBNE Science Collaboration proposes to mount a broad attack on the science of neutrinos with sensitivity to all known parameters in a single experiment.  The focus of the program will be the explicit demonstration of leptonic CP violation, if it exists, by precisely measuring the asymmetric oscillations of muon-type neutrinos and antineutrinos into 
electron-type neutrinos and antineutrinos.

The experiment will result in the most precise measurements of the three-flavor neutrino-oscillation parameters over a very long baseline and a wide range of neutrino energies, in particular, the CP-violating phase in the three-flavor framework.  The unique features of the experiment -- the long baseline, the broad-band beam, and the high resolution of the detector -- will enable the search for new physics that manifests itself as deviations from the expected three-flavor neutrino-oscillation model.

The configuration of the
LBNE facility, in which a large neutrino detector is located deep underground, could also provide
opportunities for research in other areas of physics, such as nucleon decay and neutrino
astrophysics, including studies of neutrino bursts from supernovae occuring in our galaxy. The
scientific goals and capabilities of LBNE are outlined in Volume 1 of this CDR and described fully in 
the LBNE Case Study Report
(Liquid Argon TPC Far Detector)~\cite{caseStudy} and the 2010 Interim Report of
the Long-Baseline Neutrino Experiment Collaboration Physics Working Groups~\cite{PWGIReport}.



\subsection{LBNE Project Organization}
The LBNE Project Office at Fermilab is headed by the Project Manager and assisted by the Project
Engineer, Project Systems Engineer and Project Scientist. Project Office support staff include a Project Controls Manager
and supporting staff, a Financial Manager, an Environment, Safety and Health (ES\&H) Manager, a Computing Coordinator, Quality Assurance and
Risk Managers, a documentation team and administrative support. 
The Beamline, Liquid Argon Far Detector and Conventional Facilities subprojects are managed by the Project Office at Fermilab, while the Near Detector Complex subproject is managed by a Project Office at Los Alamos National Laboratory (LANL).

More information on Project Organization can be found in Volume~1 of this CDR. A full description of LBNE 
Project management is contained in the LBNE Project Management Plan~\cite{PMP-2453}.

\subsection{Principal Parameters of the LBNE Project}

The principal parameters of the major Project elements are given in Table~\ref{table:param-summ-fd}. 

\begin{table}[htpb]
\caption{LBNE Principal Parameters}
\label{table:param-summ-fd}
\centering
 \begin{tabular}[htbp]{|l|| p{6cm} |}
\hline
Project Element Parameter & Value  \\
\hline\hline
Near- to Far-Site Baseline &  1,300~km\\
\hline
Primary Proton Beam Power &  708~kW, upgradable to 2.3~MW\\
\hline
Protons on Target per Year &   $6.5 \times 10^{20}$  \\
\hline
Primary Beam Energy &  60 -- 120 GeV (tunable) \\
\hline
Neutrino Beam Type &  Horn-focused with decay volume\\
\hline
Neutrino Beam Energy Range &  0.5 -- 5~GeV \\ 
\hline
Neutrino Beam Decay Pipe Diameter $\times$ Length &  4~m $\times$ 200~m \\
\hline
Near Site Neutrino Detector Type & Liquid Argon Time Projection Chamber (LArTPC) Tracker \\
\hline
Near Site Neutrino Detector Active Mass &  18~ton \\
\hline
Far Detector Type &  LArTPC \\
\hline
Far Detector Active (Fiducial) Mass &  54 (40)~kton\\
\hline
Far Detector Depth &  1480~m \\
\hline
\end{tabular} 
\end{table}

\subsection{Supporting Documents}
\label{intro-supp-doc}
%\chapter{Supporting Documents}
% AH 12/30/11: Since I'm pulling this in as a chapter in vol 1 and as a subsection in section 1.1 of vols 2-6, I'm taking the chapter heading out and putting it directly into the file that inputs all the content of volume 1
A host of information related to the CDR is available in a set of supporting documents. Detailed information on risk analysis and mitigation, value engineering, ES\&H, costing, project management and other topics not directly in the design scope can be found in these documents, listed in 
Table~\ref{table:cd-1-doc-list}. Each document is numbered and stored in LBNE's document database, accessible via a username/password combination provided by the Project. Project documents stored in this database are made available to internal and external review committees through Web sites developed to support individual reviews.

%\fixme{Either need descriptions, or get rid of Description column; need to finalize table. AH}

\begin{center}
\begin{longtable}{|p{10cm}|p{4cm}|} %{|l|l|l|}
\caption{LBNE CD-1 Documents}
\label{table:cd-1-doc-list} \\
   \multicolumn{1}{p{10cm}}{\textbf{Title}} & % 2/23/12 AH tried upping col width from 5 and 2.
   \multicolumn{1}{p{4cm}}{\textbf{LBNE Doc Number(s)}} \\
\hline
\endfirsthead
Acquisition Plan & 5329 \\
\hline
Alternatives Analysis  & 4382    \\
\hline
Case Study Report; Liquid Argon TPC Detector &  3600   \\
\hline
Configuration Management Plan & 5452  \\
\hline
DOE Acquisition Strategy for LBNE & 5442 \\
\hline
Integrated Environment, Safety and Health Management Plan & 4514 \\
\hline
LAr-FD Preliminary ODH Analysis &  2478   \\
\hline
Global Science Objectives, Science Requirements and Traceback Reports &  4772   \\
\hline
Preliminary Hazard Analysis Report  &  4513  \\
\hline
Preliminary Project Execution Plan &   5443 \\
\hline
Preliminary Security Vulnerability Assessment Report & 4826 \\
\hline
Project Management Plan  & 2453    \\
\hline
Project Organization Chart & 5449    \\
\hline
Quality Assurance Plan & 2449    \\
\hline
Report on the Depth Requirements for a Massive Detector at Homestake & 0034  \\
\hline
Requirements, Beamline &  4835   \\
\hline
Requirements (Parameter Tables), Far Detector &  3747 (3383)\\
\hline
Requirements, Far Site Conventional Facilities  &   4408  \\
\hline
Requirements, Near Detectors & 5579 \\
\hline
Requirements, Near Site Conventional Facilities & 5437  \\
\hline
Risk Management Plan & 5749    \\
\hline
Value Engineering Report & 3082   \\
\hline
Work Breakdown Structure (WBS) & 4219    \\
\hline
%Cost Book List, Beamline &  http://lbne.fnal.gov/reviews/beam-costbooklist.shtml  \\
%\hline
%Cost Book List, Near Detector & http://lbne.fnal.gov/reviews/nd-costbooklist.shtml  \\
%\hline
%Cost and Schedule documents, LAr-FD &  5131  \\
%\hline
%\end{tabular} 
\end{longtable}
\end{center} %Seems funny that I need the "common/", but it works. AH

%%%%%%%%%%%%%%%%%%%%
\nomenclature{CDR}{Conceptual Design Report}
\nomenclature{CP}{charge parity}
\nomenclature{DOE}{Department of Energy}
\nomenclature{DUSEL}{Deep Underground Science and Engineering Laboratory}
\nomenclature{P5}{Particle Physics Project Prioritization Panel}
