% This holds definitions of macros to enforce consistency in names.

% This file is the sole location for such definitions.  Check here to
% learn what there is and add new ones only here.  

% also see units.tex for units.  Units can be used here.

%%% Common terms

% Check here first, don't reinvent existing ones, add any novel ones.
% Use \xspace.

%%%%% Anne adding macros for referencing CDR volumes and annexes Apr 20, 2015 %%%%%
\def\expshort{DUNE\xspace}
\def\explong{The Deep Underground Neutrino Experiment\xspace}

%\def\thedocsubtitle{LBNF/DUNE Conceptual Design Report (DRAFT)}
\def\thedocsubtitle{Long-Baseline Neutrino Facility (LBNF) and 
Deep Underground Neutrino Experiment (DUNE)}% 
\def\cdrtitle{Conceptual Design Report}

% For the document titles (not italicized)
\def\volintrotitle{Volume 1: The LBNF and DUNE Projects\xspace}
\def\volphystitle{Volume 2: The Physics Program for DUNE at LBNF\xspace}
\def\vollbnftitle{Volume 3: The Long-Baseline Neutrino Facility for DUNE\xspace}
\def\voldunetitle{Volume 4: The DUNE Detectors at LBNF\xspace}
\def\anxratestitle{Annex 4B: Expected Data Rates for the DUNE Detectors\xspace}
\def\anxrecotitle{Annex 4C: Simulation and Reconstruction\xspace}
\def\anxndreftitle{Annex 4G: Near Detector Reference Design\xspace}
\def\anxcryotitle{Annex 3D: Detailed Report on the LBNF Cryostat and Cryogenics System\xspace}

% For use within volumes (italicized)
\def\volintro{Volume 1: \textit{The LBNF and DUNE Projects\xspace}}
\def\volphys{Volume 2: \textit{The Physics Program for DUNE at LBNF\xspace}}
\def\vollbnf{Volume 3: \textit{The Long-Baseline Neutrino Facility for DUNE\xspace}}
\def\voldune{Volume 4: \textit{The DUNE Detectors at LBNF\xspace}}

\def\anxlbnesci{Annex 2A: \textit{LBNE: Exploring Fundamental Symmetries of the Universe\xspace}}

\def\anxlbnefd{Annex 4A: \textit{The LBNE Design for a Deep Underground Single-Phase Liquid Argon TPC\xspace}}
\def\anxrates{Annex 4B: \textit{Expected Data Rates for the DUNE Detectors\xspace}}
\def\anxreco{Annex 4C: \textit{Simulation and Reconstruction\xspace}}
\def\anxlbnoa{Annex 4D: \textit{LAGUNA/LBNO Part 1\xspace}}
\def\anxlbnob{Annex 4E: \textit{LAGUNA/LBNO Part 2\xspace}}
\def\anxaltinstall{Annex 4F: \textit{Alternative Far Detector Design Installation and Commissioning Details\xspace}}
\def\anxndref{Annex 4G: \textit{Near Detector Reference Design\xspace}}
\def\anxdualrpt{Annex 4H: \textit{Progress report on LBNO-DEMO/WA105 (2015)\xspace}}
\def\anxdualtdr{Annex 4I: \textit{WA105 TDR\xspace}}
\def\anxcernproto{Annex 4J: \textit{CERN Single-phase Prototype Detector Proposal\xspace}}
\def\anxcryo{Annex 3D: \textit{Detailed Report on the LBNF Cryostat and Cryogenics System\xspace}}

\def\cernsingleproto{\textit{Full-Scale Detector Engineering Test and Test Beam Calibration of a Single-Phase LArTPC\xspace}}
\def\cerndualproto{\textit{Long Baseline Neutrino Observatory Demonstration (WA105)\xspace}}

% Things about oscillation
%
% example: \dm{12}
\newcommand{\dm}[1]{$\Delta m^2_{#1}$\xspace}
% example \sinstt{12}
\newcommand{\sinstt}[1]{$\sin^22\theta_{#1}$\xspace}
% example \sinst{12}
\newcommand{\sinst}[1]{$\sin^2\theta_{#1}$\xspace}
% example \deltacp
\newcommand{\deltacp}{$\delta_{\rm CP}$\xspace}
% example \nuxtonux{\mu}{e}
\newcommand{\nuxtonux}[2]{$\nu_{#1} \to \nu_{#2}$\xspace}
\newcommand{\numutonumu}{\nuxtonux{\mu}{\mu}}
\newcommand{\numutonue}{\nuxtonux{\mu}{e}}
\newcommand{\numu}{$\nu_\mu$\xspace}
\newcommand{\nue}{$\nu_e$\xspace}
\newcommand{\nutau}{$\nu_\tau$\xspace}
\newcommand{\anumu}{$\bar\nu_\mu$\xspace}
\newcommand{\anue}{$\bar\nu_e$\xspace}
\newcommand{\anutau}{$\bar\nu_\tau$\xspace}

\newcommand{\mdeltacp}{\delta_{\rm CP}}

% Names
\newcommand{\cherenkov}{Cherenkov\xspace}
\newcommand{\kamland}{KamLAND\xspace}
\newcommand{\kamiokande}{Kamiokande\xspace}
\newcommand{\superk}{Super--Kamiokande\xspace}
\newcommand{\miniboone}{MiniBooNE\xspace}
\newcommand{\minerva}{MINER$\nu$A\xspace}
\newcommand{\nova}{NO$\nu$A\xspace}
\newcommand{\SURF}{Sanford Underground Research Facility\xspace}

\def\Ar39{$^{39}$Ar}
\def\driftvelocity{\SI{1.6}{\milli\meter/\micro\second}\xspace}