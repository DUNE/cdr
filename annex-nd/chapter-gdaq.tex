\chapter{Data Acquisition and Computing for the Near Detector System} % (NDS) DAQ and Computing}
\label{ch:nd-gdaq}

\section{Introduction}
\label{sec:nd-gdaq-intro}

The Near Detector System Data Acquisition system (NDS-DAQ) collects raw data from each detector 
in the Near Detector System (NDS), issues global \fixme{global in what sense here?} triggers, adds precision timing 
data from a global positioning system (GPS), and builds events. Each NDS detector has its 
own data acquisition system that connects to the NDS-DAQ.
The NDS-DAQ is made up of two parts: NDS Master DAQ (NDS-MDAQ) and Beamline Measurements 
DAQ (BLM-DAQ). A third component, the Near Neutrino Detector DAQ (NND-DAQ), is 
connected to the NDS-MDAQ and is part of the Near Neutrino Detector system.
Figure~\ref{fig:DAQ_Block} shows a block diagram for the DAQ.

\begin{cdrfigure}[Near Detector System DAQ block diagram]{DAQ_Block}{Near Detector System DAQ block diagram: The NDS-DAQ consists 
of two parts: NDS Master DAQ (green blocks) and Beamline Measurement DAQ (yellow summary 
block). The NearSite Detectors DAQ (orange block) is described in a separate chapter. The 
NDS-DAQ connects to other sections of the LBNE project, shown here in other colors (blue, 
light red, tan).}
\includegraphics[width=6in,angle=0]{DAQ_Block}
\end{cdrfigure}

The NDS computing system encompasses two major activities: online computing (the required
slow-control systems) and offline computing for further development of the measurement strategy and for simulation work on technical systems.

\subsection{NDS Master DAQ and Computing}
\label{sec:nd-master-daq}

The NDS-DAQ design consists of equipment designed to provide a high level user interface 
for local run control and data taking and secure remote control and monitoring.   It will 
serve as the primary interface to the slow control system, online data and DAQ performance 
monitoring, raw data collection, building of events, and data storage between all Near 
Neutrino Detector DAQs and Beamline Measurements DAQs.  This includes hardware triggering 
for two-way trigger processing between the NDS-DAQ and all detectors, and GPS for precision 
time stamping and global clock synchronization.  It is currently based on a channel count 
estimate of approximately 430,000 channels of neutrino detectors, plus $<1000$ channels of 
beamline detectors.  Custom electronics components for the NDS-DAQ are based on existing 
custom designs from other experiments, e.g. T2K or ATLAS, and have all commercial 
components:  Trigger modules, clock and timing synchronization, GPS, environmental 
monitoring.

The computing system encompasses two major activities: online computing with required 
slow-control systems, and offline computing for further development of the measurement 
strategy and for simulation work on technical systems. The computing components are based 
on currently available commercial computing and gigabit networking technology, which is 
likely to improve over the next years without driving costs up for the final design.  

\subsection{Beamline Measurements DAQ (BLM-DAQ)}

Similar to the NDS Master DAQ, the BLM-DAQ will mainly consist of a scalable back-end 
computer array, inter-connected to the individual beamline measurement detector DAQs via 
Gigabit Ethernet and specialized electronics modules for trigger processing and clock 
synchronization. It interfaces to the NDS Master DAQ (NDS-MDAQ) for run control and global 
data collection. It will also have its own local run-control setup, consisting of a number 
of desktop workstations to allow independent local runs that include beamline measurement 
detectors only (useful during detector commissioning), calibration runs, stand-alone cosmic 
runs or other runs where the beam is stopped or not needed.


\section{NDS Computing}
\label{sec:nd-gdaq-global-computing}

The computing system encompasses two major activities: %The first is 
online computing (% which consists of 
the required slow-control systems) and %.  The second is 
offline 
computing which is costed off-Project, but is important to consider.
Offline computing is needed to complete 
the work outlined in the Measurement Strategy described in Chapter~\ref{ch:meas-strat} and the simulation work %important 
for the technical systems.

The required slow-control (online) computing systems will be defined when the Project moves 
from the conceptual-design to the preliminary-design phase.

For offline computing, resources are currently being provided by Fermilab, 
LANL and various universities.  Project-wide resources are currently 
being developed at Fermilab and Brookhaven.
