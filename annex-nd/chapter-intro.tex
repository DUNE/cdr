\chapter{Introduction}  
\label{ch:nd-annex-intro}

DUNE collaborators have prepared this detailed design document which describes the reference design for 
the DUNE Near Detector Systems (NDS) and serves as an annex to the more abbreviated discussion in the DUNE 
CDR.  The NDS design pulls heavily from both the LBNE near detector design of March 2012, before the LBNE reconfiguration effort eliminated this detector from the project scope, and from the design proposed by Indian collaborators in their DPR of late 2013.

The role of the DUNE Near Detector Systems (NDS) is to 
minimize the systematic uncertainties of 
the long-baseline oscillation program and to thus maximize the 
oscillation-physics potential of the Far Detector. It is important that the
 Near Detectors improve the sensitivity of the 
DUNE long-baseline neutrino-oscillation measurements. 
%\fixme{I've never liked this sentence, it sounds negative, like you'd expect the detectors to ruin the experiment. I changed the next sentence because I didn't like it either..}
The enhanced sensitivity they provide will aid in both the analysis of electron-neutrino 
appearance, the primary oscillation channel, and muon-neutrino disappearance. 

The NDS is made up of the following components:

\begin{itemize}
\item Fine-Grained Tracker (FGT) near neutrino detector
\item Beamline Measurement System
\item Near Detector System Data Acquisition system
\item External Measurements
\end{itemize}

The DUNE Fine-Grained Tracker (FGT) near detector consists of a straw-tube
tracking detector (STT) and electromagnetic calorimeter (ECAL) inside of a 0.4-T
dipole magnet. In addition, Muon Identifiers (MuIDs) are located in the
steel of the magnet, as well as upstream and downstream of the STT. The FGT
is designed to make precision measurements of the neutrino fluxes, 
cross sections, signal rates and background rates. 

The Beamline Measurement System (BLM) will be located in the region of the Absorber Complex at 
the downstream end of the decay region to measure the muon fluxes from hadron decay. The 
absorber itself is part of the LBNF Beamline. 
The BLM is intended to determine the neutrino fluxes and spectra
and to monitor the beam profile on a spill-by-spill basis, and will operate for the life of the
experiment. 

The Near Detector System Data Acquisition system (NDS-DAQ) collects raw data from each NDS detector's
individual DAQ, issues 
%\fixme{global in what sense here?} 
triggers, adds precision timing 
data from a global positioning system (GPS), and builds events. 
The NDS-DAQ is made up of three parts: NDS Master DAQ (NDS-MDAQ), the Beamline Measurements 
DAQ (BLM-DAQ) and the Near Neutrino Detector DAQ (NND-DAQ).

In addition, external pion producion measurements will improve the simulation of
neutrino fluxes.
%\fixme{Add something about external measurements if needed.}

%%%%%%%%%%%%%%%%%%%%%%%%%%% Data for the Acro/Abbrev/Units list  %%%%%%%%%%%%%%%%%

\nomenclature{ADC}{analog-to-digital converter}
\nomenclature{ASIC}{application-specific integrated circuit}
\nomenclature{BEB}{back-end board}
\nomenclature{BLM}{beamline-measurement system}
\nomenclature{B-GDAQ}{beamline global data acquisition}
\nomenclature{CC}{charged current (interaction)}
\nomenclature{CCQE}{charged current quasi-elastic (interaction)}
\nomenclature{CERN}{European Organization for Nuclear Research}
\nomenclature{CNGS}{Neutrino Beam to Gran Sasso (at CERN)}
\nomenclature{DAQ}{data acquisition}
\nomenclature{DUNE}{Deep Underground Neutrino Experiment}
\nomenclature{ECAL}{electromagnetic calorimeter}
\nomenclature{ESH}{Environment, Safety and Health}
\nomenclature{eV}{electron-Volt, unit of energy (also keV, MeV, GeV, etc.)}
\nomenclature{FEB}{front-end board}
\nomenclature{FGT}{Fine-Grained Tracker}
\nomenclature{FPGA}{field-programmable gate array}
\nomenclature{FRA}{Fermi Research Alliance}
\nomenclature{g-2}{the New Muon g-2 Experiment at Fermilab}
\nomenclature{GDAQ}{global data acquisition}
\nomenclature{GPS}{Global Position System}
\nomenclature{HEP}{high energy physics}
\nomenclature{ICARUS}{Imaging Cosmic And Rare Underground Signals (experiment at LNGS)}
\nomenclature{K2K}{``From KEK to \superk,'' long-baseline neutrino-oscillation experiment}
\nomenclature{LANL}{Los Alamos National Laboratory}
\nomenclature{LAr}{liquid argon}
\nomenclature{LArTPC}{Liquid Argon Time Projection Chamber}
\nomenclature{LArTPCT}{Liquid Argon Time Projection Chamber Tracker system}
\nomenclature{LAr-FD}{(LBNE) Liquid Argon Far Detector}
\nomenclature{LBNF}{Long-Baseline Neutrino Facility}
\nomenclature{LHC}{Large Hadron Collider (at CERN)}
\nomenclature{LNGS}{Gran Sasso National Laboratory}
\nomenclature{M-GDAQ}{master global data acquisition}
\nomenclature{m}{meter (also nm, micron, mm, cm, km) }
\nomenclature{MicroBooNE}{A 100-ton LArTPC located along Fermilab's Booster neutrino beamline}
\nomenclature{MINERvA}{A neutrino-scattering experiment that uses the NuMI beamline at Fermilab}
\nomenclature{MiniBooNE}{Booster Neutrino Experiment (at Fermilab)}
\nomenclature{MINOS}{Main Injector Neutrino Oscillation Search experiment at Fermilab}
\nomenclature{MIPP}{Main Injector Particle Production Experiment (at Fermilab)}
\nomenclature{MPPC}{multi-pixel photon counter}
\nomenclature{MuID}{muon-identification detector}
\nomenclature{NC}{neutral current (interaction)}
\nomenclature{ND}{(Near Site) neutrino detector }
\nomenclature{NDS}{Near Detector Systems; refers to the L2 Project under LBNE }
\nomenclature{NOMAD}{Neutrino Oscillation Magnetic Detector, experiment at CERN}
\nomenclature{NA61}{NA61/SHINE, experiment at CERN that studies hadron production in hadron-nucleus and nucleus-nucleus collisions}
\nomenclature{N-GDAQ}{neutrino global data acquisition}
\nomenclature{NOvA}{NuMI Off-Axis Neutrino Appearance experiment at Fermilab}
\nomenclature{NuMI}{Neutrino beam to MINOS}
\nomenclature{PMT}{photomultiplier tube}
\nomenclature{QE}{quasi-elastic (interaction)}
\nomenclature{STT}{straw-tube tracker}
\nomenclature{Super-K}{Super-Kamiokande, a neutrino detector in Japan}
\nomenclature{T2K}{Tokai-to-Kamioka, a long-baseline neutrino oscillation experiment in Japan}
\nomenclature{T2K ND280}{Near Detector of the Tokai-to-Kamioka (T2K) experiment, located 280m from the beamline target}
\nomenclature{TDC}{time-to-digital converter}
\nomenclature{TFB}{T2K front-end board}
\nomenclature{TPC}{time projection chamber}
\nomenclature{TR}{transition radiation}
\nomenclature{TRIP-t}{Trigger and Pipeleine with timing (full custom ASIC designed at Fermilab)}
\nomenclature{WBS}{Work Breakdown Structure}
\nomenclature{T}{Tesla; unit of magnetic field strength}
\nomenclature{UA1}{Experiment at CERN that ran from 1986 to 1993}
\nomenclature{UV}{ultra-violet}
\nomenclature{V}{Volt}
\nomenclature{W}{watt (also mW, kW, MW) }
\nomenclature{WLS}{wavelength shifter}

