
%%%%%%%%%%%%%%%%%%%%%%%%%%%%%%%%%%%%%%%%%%%%%%%%%%%%%%%%%%%%%%%%%%%%%%%%%%%%%%%%%%
\chapter{Technical Designs}
\label{v1ch:tech-designs}

%
%Here is a sample figure:
%
%\begin{cdrfigure}[short]{label}{long}
%%\includegraphics[width=\linewidth]{file}
%\end{cdrfigure}
%
%Here is a sample reference to a figure (Figure~\ref{fig:label}). Notice: ``fig:'' is not present in the label as written in the figure code itself.
%
%Here is a sample table:
%
%\begin{cdrtable}[short]{cc}{label}{long} %The third argument (reads {cc}) can use c, l, r or p{some length}
%% but please do not include lines like “|c|l|l|”. It CAN look like {cll} or {llp{3cm}}, for instance.
%Header Column1 & Header Column 2 \\ \toprowrule
%Row 1 & First \\ \colhline
%Row 2 & Second \\ \colhline
%Row 3 & Third \\
%\end{cdrtable}
%
%Here is a sample reference to a table Table~\ref{tab:label}). Notice: ``tab:'' is not present in the label as written in the table code itself.
%
%


\section{LBNF Project}

LBNF will provide facilities at Fermilab and at SURF to enable the scientific program of DUNE. These facilities are geographically separated into the Far Site Facilities, those to be constructed at SURF, and the Near Site Facilities, those to be constructed at Fermilab. These are summarized as such in sections below. 

LBNF is managed through a Project Office where management and functions common to both Far Site and Near Site Facilities occur. LBNF coordinates requirements and interfaces with the DUNE Project through the Experiment-Facility Interface Group as well as working teams comprised of members of both projects. 

\subsection{Near Site Facilities}

The scope of LBNF at Fermilab is to provide the Beamline plus the Conventional Facilities for this Beamline as well as for the DUNE Near Detector. The science requirements as determined by the DUNE Science Collaboration drive the performance of the Beamline and Near Detector, which then provide requirements for the components, space, and functions necessary to construct, install, and operate the Beamline and Near Detector. ES\&H and facility operations (programmatic) requirements also provide input to the design.

The Beamline is designed to provide a neutrino beam of sufficient intensity and appropriate energy range to meet the goals of DUNE for long-baseline neutrino-oscillation physics. The design is a conventional, horn-focused neutrino beamline. The components of the beamline will be designed to extract a proton beam from the Fermilab Main Injector (MI) and transport it to a target area where the collisions generate a beam of charged particles. 

The facility is designed for initial operation at proton-beam power of 1.2 MW, with the capability to support an upgrade to 2.4 MW. The plan is for twenty years of operation, while the lifetime of the Beamline Facility, including the shielding, is for thirty years. The conservative assumption is that for the first five years will be at 1.2 MW operations and the remaining fifteen years at 2.4 MW.  
The experience gained from the various neutrino projects has been employed extensively in the reference design. In particular, the NuMI beamline serves as the prototype design. Most of the subsystem designs and their integration follow, to a large degree, from previous projects. 

The proton beam is extracted at a new point at MI-10. After extraction, this primary beam establishes a horizontally straight compass heading west-northwest toward the Far Detector, but will be bent upward to an apex before being bent downward at the appropriate angle. The primary beam will be above grade to minimize expensive underground construction and significantly enhances ground-water radiological protection. The design requires, however, construction of an earthen embankment, or hill, whose dimensions are commensurate with the bending strength of the dipole magnets required for the beamline. 

The target marks the transition from the intense, narrowly-directed proton beam to the more diffuse, secondary beam of particles that in turn decay to produce the neutrino beam. After collection and focusing, the pions and kaons that did not initially decay need a long, unobstructed volume in which to decay. This decay volume in the reference design is a pipe of circular cross section with its diameter and length optimized such that decays of the pions and kaons result in neutrinos in the energy range useful for the experiment. The decay volume is followed immediately by the absorber, which removes the remaining beam hadrons. 

Radiological protection is integrated into the LBNF beamline reference design in two important ways. First, shielding is optimized to reduce exposure of personnel to radiation dose and to minimize radioisotope production in ground water within the surrounding rock. Secondly, the handling and control of tritiated ground water produced in or near the beamline drives many aspects of the design. 

Beamline Conventional Facilities include an enclosure connecting to the existing Main Injector at MI-10, concrete underground enclosures for the primary beam, targetry, horns, absorber, and related technical support systems. Service buildings will be constructed to provide support utilities the primary proton beam at LBNF 5 and to support the absorber at LBNF 30. The enclosure for the Target Hall will be both below and above grade and is identified as LBNF 20.  Utilities will be extended from nearby existing services, including power, domestic and industrial water, sewer, and communications. 

Near Detector Conventional facilities include a small muon alcove area in the Beamline Absorber Hall and a separate underground Near Detector Hall that houses the Near Detector. A service building called LBNF 40 with two shafts to the underground supports the Near Detector. The underground hall is sized for the reference Near Neutrino Detector.

\subsection{Far Site Facilities}

The scope of LBNF at SURF includes both Conventional Facilities and Cryogenic Infrastructure to support the DUNE Far Detector. The requirements derive from DUNE Science Collaboration science requirements, which drive the space and functions necessary to construct and operate the Far Detector.  ES\&H and facility operations (programmatic) requirements also provide input to the design. The Far Detector is modularized into four 10-kt fiducial mass detectors [determine if reference should be for total mass instead of fiducial]. The caverns and the services to the caverns will be similar to one another as much as possible to enable repetitive designs for efficiency in design and construction as well as operation. 

The scope of the Conventional Facilities includes design and construction for faciliites on the surface and underground. The underground conventional facilities includes new excavated spaces at the 4850L for the detector, utility spaces for experimental equipment, utility spaces for facility equipment, drifts for access, as well as construction-required spaces. Underground infrastructure provided by Conventional Facilities for the experiment includes power to experimental equipment, cooling systems, and cyberinfrastructure. Underground infrastructure necessary for the facility includes domestic (potable) water, industrial water for process and fire suppression, fire detection and alarm, normal and standby power systems, a sump pump drainage system for native and leak water around the detector, water drainage to the facility-wide pump discharge system, and cyberinfrastructure for communications and security.
In addition to providing new spaces and infrastructure underground, Conventional Facilities will enlarge and provide infrastructure in some existing spaces for use, such as the access drifts from the Ross Shaft to the new caverns. New piping will be provided in the shaft for cryogens (gas argon transfer line and the compressor suction and discharge lines) and domestic water as well as power conduits for normal and standby power and cyberinfrastructure. 

The existing Sanford Laboratory has many surface buildings and utilities, some of which will be utilized for LBNF. The scope of the above ground conventional facilities includes only that work necessary for LBNF, and not for the general rehabilitation of buildings on the site, which remains the responsibility of the Sanford Laboratory. Electrical substations and distribution will be upgraded to increase power and provide standby capability for life safety. Additional surface scope includes a small control room in an existing building and a new building to support cryogen transfer from the surface to the underground near the existing Ross Shaft.

To reduce risk during the construction and installation period, several SURF infrastructure operations/maintenance activities are included as early activities in LBNF. These include completion of the Ross Shaft rehabilitation, rebuilding of hoist motors, and replacement of the Oro Hondo fan; if not addressed, this aging infrastructure could limit or stop access to the underground if equipment failed. 
The scope of the LBNF Cryogenics Infrastructure includes the design, fabrication, and installation of four cryostats to contain the liquid argon (LAr) and the detector components and a comprehensive cryogenic system that meets the performance requirements for purging, cooling down and filling the cryostats, achieving and maintaining the LAr temperature, and purifying the LAr outside the cryostats. 

Each cryostat will be comprised of a free-standing steel-framed structure, to be constructed in one of the four caverns, with a membrane cryostat vessel installed inside. The interior dimension of the cryostat is 15.1 m width, 14.0 m height and 62.0 m length. It contains a total mass of 17.1 kt of LAr. Each membrane tank includes a stainless-steel liner to contain the liquid cryogen. The pressure loading of the liquid cryogen is transmitted through rigid foam insulation to the surrounding structural steel frame which provides external support for the membrane. The membrane system provides full containment of the LAr. The hydrostatic load of the LAr in the cryostat is carried by the steel frame on sides and bottom. Everything else within the cryostat (TPC planes, electronics, sensors, cryogenic- and gas-plumbing connections) is supported by the steel top plate [check this is true with steel frame]. All piping and electrical penetrations into the interior of the cryostat are made through this top plate to minimize the potential for leaks.

Cryogenic system components are located on the surface or within the cavern. The cryogen receiving station will be located on the surface near the Ross shaft to allow for receipt of LAr deliveries for the initial filling period, as well as a buffer volume to accept liquid argon during the extended fill period. A large vaporizer at the surface will vaporize the liquid argon from the storage dewar prior to the argon gas being transferred by uninsulated piping down the Ross shaft. 

A liquid nitrogen dewar also located at the surface is used to accept nitrogen deliveries for the initial charging and startup of the nitrogen refrigerator as well as for pressure control of the liquid argon storage dewar. A large vaporizer for the nitrogen circuit will vaporize nitrogen to nitrogen gas for feeding the compressors of the nitrogen refrigerator. Four compressors, the only refrigerator components on the surface, and supporting systems are located in a compressor building near the Ross shaft and cryogen receiving area. The compressors discharge high pressure nitrogen gas into pipes that run down the Ross shaft. The compressors will be located on the surface because the electrical power requirement and cooling requirement is much less than for similar equipment at the 4850L.  

The detector cavern at 4850L contains the rest of the nitrogen refrigerator, liquid nitrogen vessels, argon condensers, external liquid argon recirculation pumps, and filtration equipment. Filling each cryostat with liquid argon in a reasonable period of time is a driving factor for the refrigerator and condenser sizing.  Each cryostat will have its own argon recondensers, argon-purifying equipment and overpressure protection system also located in the central utility cavern. Recirculation pumps will be placed outside of each membrane cryostat to circulate liquid from the bottom of the tank through the purifier.



\section{The DUNE Detectors}

The DUNE detectors to be installed at SURF (the far location) and FNAL (the near location) will enable the scientific program of DUNE.  The detector 
requirements derive from these DUNE science goals.

\subsection{The Far Detector}
The  Far Detector (FD) will be located deep underground at the 4850L and have
a  fiducial mass of 40-kt to perform sensitive studies of long-baseline oscillations with a 1300~km baseline as well as a rich astroparticle physics programme and nucleon decay searches. The FD  will be composed of four identical modules, each instrumented as Liquid Argon Time Projection Chambers (LArTPC)
The concept of the LAr TPC provides
excellent tracking and calorimetry performance, hence ideal for massive neutrino detectors such as DUNE which require a high signal efficiency and effective background discrimination,  an excellent capability to identify and measure precisely neutrino events over a wide range of energies, and an excellent reconstruction of the kinematical properties
with a high resolution. The full imaging of events will allow to study neutrino interactions and
other rare events in an unprecedented way. The huge mass will allow collecting sufficient statistics for precision
studies, as was already introduced in Chapter~\ref{v1ch:science}.

The LAr TPC, pioneered in the context of the ICARUS project, is a mature technology. It is the outcome
of several decades of R\&D executed worldwide.  Nonetheless, the size of a single 10-kt DUNE module represents an extrapolation by approximately one order of magnitude compared to the largest operated detector, the ICARUS~T600. To address this challenge, DUNE is developing two far detector options, the reference design and the alternate design, and is engaged in a 
comprehensive prototyping effort. At this stage, the development of two FD options is a strength and an added-value 
made possible by the merging of the worldwide neutrino community into DUNE.

Interactions in liquid argon produce ionization charge and scintillation light.
The charge is drifted with a constant electric field away from the cathode
plane and towards the segmented anode plane. 
The prompt scintillation light
is detected by photo-detectors and provide the absolute time of the event.
The reference design adopts a single-phase readout, where the readout anode is composed wire planes in the liquid argon volume. 
In the alternate design, the  dual-phase approach is considered, where the 
ionisation charges are extracted, amplified and detected in gaseous argon above the liquid surface. 
The dual phase design would allow for a finer readout pitch (3-mm), 
a lower detection-energy threshold, and better pattern reconstruction of the events.
The reference and alternate design adopt complementary design 
to collect the scintillation light.

The 10\,kt reference design TPC active volume is 12\,m high, 14.5\,m wide and 
58\,m long, instrumented with APAs, 
which are 6\,m high and 2.3\,m in width, and CPAs, 3\,m high by 2.5\,wide. Groups of
two APAs and four CPAs are stacked vertically to instrument 
the 12\,m height of the active volume. The 12.5\,m width of the detector is 
spanned three stacks of APAs and two stacks of CPAs in an APA:CPA:APA:CPA:APA
arrangement, with four 3.6\,m drift volumes, while the 58\,m length of the active volume
is spanned 25 such stack arrangements placed edge-to-edge. Hence the 10\,kt 
far detector consists of 150 APAs and 200 CPAs. The CPAs are held a -180\,kV, such that 
ionization electrons drift a maximum distance of 3.6\,m in the electric field of 500\,V\,cm$^{-1}$.
The highly modular nature of detector design enables manufacturing to be distributed across a number of sites.

A  comprehensive prototyping strategy for both designs is actively pursued.
The reference design, closer to the original ICARUS design, is currently being validated in the 35-t prototype LAr detector at Fermilab.  The alternate design, representing a novel approach, has been proven on several
small scale prototypes, and presently
a 20-t dual-phase prototype is being constructed at CERN (WA105 3x1x1) and should be operated in 2016. 
The ultimate validation of the engineered solutions for both designs of the FD is foreseen at 
the CERN neutrino platform around the year 2018, where full scale engineering prototypes will be 
assembled and commissioned. After this milestone, a test beam data 
campaign will be executed in the following years to collect large sample of charge particle interactions
to study the response of the detector with high precision.
A comprehensive list of synergies between the reference and alternative design has been identified. Common solutions for DAQ, electronics, HV feed-throughs, and so on, will pursued and implemented, independent of the details of the TPC design. The ongoing efforts and those at the CERN Neutrino Platform will
provide the ideal environment to exploit such synergies and implement common solutions.
There is recognition that the LArTPC technology will continue to evolve with (1) the large-scale prototypes at the CERN Neutrino Platform and the experience from the Fermilab SBN program, and (2) the experience gained during the construction and commissioning of the first 10-kt module. 
The staged approach with the deployment of consecutive modules will
give access to an early science programme while allowing for improvements and developments to be implemented during the life of the experiment.
The strategy to implement
the Far Detector is presented in Chapter~\ref{v1ch:strategy}.

\subsection{The Near Detectors}
The spectrum and flavor composition of the neutrino beam will be measured with high precision 
in order to reach the ultimate sensitivity for the long-baseline neutrino oscillation studies.
The separation between fluxes of neutrinos and antineutrinos requires a magnetised neutrino detector to 
charge discriminate electrons and muons produced in the neutrino charged current interactions.
This is the primary role of the DUNE near detector system, however, being exposed to an intense flux of neutrinos
will also provides the opportunity to collect an unprecedentedly high statistics of neutrino 
interactions  for an extended science program. 
The near detector will therefore provide an opportunity for a wealth of fundamental neutrino interaction 
measurements, which are an important part of the secondary scientific goals of the DUNE collaboration. 
The reference design for the neutrino near detector (NND) design is the NOMAD-inspired fine-grained tracker (FGT). The subsystems of NND comprise a central 
straw-tube tracker and an electromagnetic calorimeter embedded in a 0.4-T dipole field. The steel of the
magnet yoke will be instrumented with muon identifiers. The strategy to implement
the Near Detector is presented in Chapter~\ref{v1ch:strategy}.

The near detector system will be complemented with a Beamline Measurement System (BLM) located in the region of the beam absorber at the downstream end of the decay region. BLM aims at measuring the muon fluxes from hadron decay.
The BLM is intended to monitor the beam profile on a spill-by-spill basis, and will operate for the life of the experiment.


