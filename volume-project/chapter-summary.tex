

\chapter{Summary}
\label{ch:project-summary}

LBNF/DUNE will be a world-leading facility for pursuing a cutting-edge programme of neutrino physics and astroparticle physics. The 
combination of the intense wide-band neutrino beam, the massive LArTPC far detector and the highly capable near detector will provide 
the opportunity to discover CP violation in the neutrino sector as well as determining the neutrino mass ordering and providing a 
precision test of the three-flavor oscillation paradigm. The massive deep-underground far detector will provide unprecedented sensitivity 
for proton decay for theoretically favoured decay modes as well as providing the potential to observe electron neutrinos from a core-
collapse supernova should one occur in our galaxy during the operation of the experiment.

In addition to summarising the compelling scientific case for LBNF/DUNE, this document has presented an overview of the technical 
designs of the facility and experiment and the strategy for their implementation. 
This strategy delivers the science goals described in the 2014 report of the Particle Physics Project Prioritisation Panel (P5) on 
competitive timescale. Furthermore, a detailed management plan for the 
organization of LBNF as a U.S. hosted facility and the DUNE experiment 
as a broad international collaboration has been developed, thus satisfying the goal of internationalizing the project as highlighted in the 
P5 report.

