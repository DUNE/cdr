\chapter{Organization and Management}
\label{v1ch:org-mgmt}

%%%%%%%%%%%%%%%%%%%%%%%%%%%%%%%%%%%%%%%%%%%%%%%%%%%%%%%%%%%%%%
\section{Overview}

To accommodate a variety of international funding model constraints, LBNF and DUNE are organized as separate projects. As mentioned in the Introduction, the LBNF Project is responsible for design and construction of the conventional facilities, beamlines, and cryogenic infrastructure needed to support the experiment.  The DUNE Project is responsible for the construction and commissioning of the detectors used to pursue the scientific program.  LBNF is organized as a DOE/Fermilab project incorporating international partners.   DUNE is an international project organized by the DUNE Collaboration with appropriate oversight from stakeholders including the DOE.

%%%%%%%%%%%%%%%%%%%%%%%%%%%%%%%%%%%%%%%%%%%%%%%%%%%%%%%%%%%%%%
\section{LBNF}

%%%%%%%%%%%%%%%%%%%%%%%%%%%%%%%
\subsection{Project Structure and Responsibilities}

The LBNF Project is charged by Fermilab and DOE to design and construct the conventional and technical facilities needed to support the DUNE Collaboration.  LBNF works in close coordination with the DUNE project to ensure that the scientific requirements of the program are satisfied through the mechanisms described in Section~\ref{sec:lbnf-dune-interface}. LBNF also works closely with SURF management to coordinate the design and construction of the underground facilities required for the DUNE far detector. 

LBNF consists of two major L2 subprojects coordinated through a central Project Office located at Fermilab: Far Site Facilities and Near Site Facilities. Each L2 Project incorporates several large L3 subprojects as detailed in the WBS structure presented in Figure~\ref{fig:lbnf-wbs}.

The Project team consists of members from Fermilab, CERN, SDSTA, and BNL. \fixme{Are we adding a list of acronyms? If not, make sure these are defined in text.} The team, including members of the Project Office as well as the L2 and L3 managers for the individual subprojects, is assembled by the Project Director. 
%%The Project team to WBS Level~3 of the WBS is shown in Figure~\ref{fig:lbnf-org}. 
%%\fixme{Remove or chop this figure?}  
Line management for environment, safety and health, and quality assurance flows through the Project Director. 

%%\begin{cdrfigure}[LBNF Organization to WBS L3]{lbnf-org}{LBNF Organization to WBS L3; note that the WBS numbers reflect the pre-CD1 WBS}
%%  \includegraphics[width=0.8\textwidth]{lbnf-org-to-level3}
%%\end{cdrfigure}

Through their delegated authority and in consultation with major stakeholders, the L2 Project Managers determine which of their lower-tier managers will be Control Account Managers (CAMs) for the Project WBS. L2 and L3 Project Managers are directly responsible for generating and maintaining the cost estimate, schedule, and resource requirements for their subprojects and for meeting the goals of their subprojects within the accepted baseline cost and schedule. 

\begin{cdrfigure}[LBNF Work Breakdown Structure to WBS Level 3]{lbnf-wbs}{LBNF Work Breakdown Structure to WBS Level 3}
  \includegraphics[width=0.8\textwidth]{lbnf-wbs-to-level3}
\end{cdrfigure}

The design and construction of LBNF is supported by other laboratories and consultants/contractors that provide scientific, engineering, and technical expertise. A full description of LBNF Project Management is contained within the LBNF Project Management Plan \fixme{[ref]}.

%%%%%%%%%%%%%%%%%%%%%%%%%%%%%%%
%%\subsection{Fermilab}

%%\fixme{Some intro text about how Fermilab is the Near Site...}

%%%%%%%%%%%%%%%%%%%%%%%%%%%%%%%
%\subsection{South Dakota Science and Technology Authority and SURF}
\subsection{SDSTA and SURF}

LBNF plans to construct facilities at SURF to house the DUNE far detector. SURF is owned by the state of South Dakota and managed by the South Dakota Science and Technology Authority (SDSTA). \fixme{define SURF, SDSTA earlier}

Current SURF activities include operations necessary for allowing safe access to the 4850L of the mine, which houses the existing and under-development science experiments. The DOE is presently funding SDSTA ongoing operations through Lawrence Berkeley National Laboratory (LBNL) and its SURF Operations Office through FY16; this is expected to change to funding through Fermilab starting in FY17. 

The LBNF Far Site Facilities Manager is also an employee of SDSTA and is contracted to Fermilab to provide management and coordination of the Far Site Conventional Facilities (CF) and Cryogenics Infrastructure subprojects. LBNF contracts directly with SDSTA for the design of the required CF at SURF; whereas the actual construction of the CF will be directly contracted from Fermilab. Coordination between SDSTA and the LBNF Project is necessary to ensure efficient operations at SURF. This will be facilitated via an agreement being developed between SDSTA and Fermilab regarding the LBNF Project \fixme{[new reference]} that defines responsibilities and methods for working jointly on LBNF Project design and construction. A separate agreement will be written for LBNF Operations. 

%%%%%%%%%%%%%%%%%%%%%%%%%%%%%%%
\subsection{CERN}

The European Organization for Nuclear Research (CERN) will participate in the LBNF Project by providing cryogenic facilities and equipment to support the far detectors as well as some technical components required for the neutrino beamline. As a key partner in the Cryogenics Infrastructure subproject, CERN will provide engineering and technical support for the design and production of specific components and coordinate with others in LBNF on the installation of the identified deliverables. CERN engineers and scientists will participate in the LBNF project as assigned managers for the CERN contributions.
%as outlined in the sections below. 
Details of the agreements with CERN will be contained in \fixme{[name the agreements here]}.  

%%%%%%%%%%%%%%%%%%%%%%%%%%%%%%%
\subsection{Coordination within LBNF}

The LBNF Project organization is headed by the LBNF Project Director who is also the Fermilab Deputy Director for LBNF and reports directly to the Fermilab Director. The Project Director also is the head of the Fermilab Divisions containing the resources needed to execute the Far Site Facilities and Near Site Facilities subprojects.  Any personnel working more than half-time on these sub-projects would typically be expected to become a member of one of these divisions, while other contributors will likely be matrixed in part-time roles from other Fermilab Divisions.  The heads of the other Fermilab Divisions work with the L1 and L2 project managers to supply the needed resources on an annual basis.  The management structure described above is currently being transitioned into and will not be fully in place until the Fall of 2015.  

The LBNF WBS defines the scope of the work. All changes to the WBS must be approved by the LBNF Project Manager prior to implementation. At the time of CD-1-Refresh, the LBNF WBS is in transition. Both the current and the post CD-1-R WBS is shown in Figure~\ref{fig:lbnf-wbs} to demonstrate how the scope will map from one WBS to the other. 

SDSTA assigns engineers and others as required to work on specific tasks required for the LBNF project at the SURF site. This is listed in the resource-loaded schedule as contracted work from Fermilab for Far Site CF activities. 

CERN and Fermilab are developing a common cryogenics team to design and produce the Cryogenics Infrastructure subproject deliverables for the far site. CERN provides engineers and other staff as needed to complete their agreed-upon deliverables.  

\fixme{Something about ``LBNF has formed several management groups with responsibilities as described below.''}

\textbf{Project Management Board:} LBNF uses a Project Management Board to provide formal advice to the Project Director on matters of importance to the LBNF Project as a whole. Such matters include (but are not limited to) those that:
\begin{itemize}
\item have significant technical, cost, or schedule impact on the Project
\item have impacts on more than one L2 subproject
\item affect the management systems for the Project
\item have impacts on or result from changes to other Projects on which LBNF is dependent
\item result from external reviews or reviews called by the Project Director
\end{itemize}
The Management Board serves as the
\begin{itemize}
\item LBNF Change Control Board, as described in the Configuration Management Plan \fixme{[ref]}
\item Risk Management Board, as described in the Fermilab Risk Management Plan  \fixme{[ref]}
\end{itemize}

\textbf{Beamline Technical Board:} The role of the LBNF Beamline Technical Board (TB) is to provide recommendations and advice to the Beamline Project Manager on important technical decisions that affect the design and construction of the Beamline. The members of the Technical Board must have knowledge of the Project objectives and priorities in order to perform this function. The Beamline Project Manager chairs the Beamline TB. The Beamline Project Engineer is the Scientific Secretary of the Board and co-chairs the Beamline TB as needed. 

\textbf{FSCF Neutrino Cavity Advisory Board:} The FSCF Project has engaged three international experts in hard rock underground construction to advise it periodically through the design and construction process regarding excavation at SURF. The board meets at the request of the FSCF-PM, generally on site to discuss specific technical issues. The board produces a report with its findings and conclusions for project information and action. 

%%%%%%%%%%%%%%%%%%%%%%%%%%%%%%%%%%%%%%%%%%%%%%%%%%%%%
\section{DUNE}

%%%%%%%%%%%%%%%%%%%%%%%%%%%%%%
\subsection{DUNE Collaboration Structure}

The DUNE Collaboration brings together the members of the international science community 
interested in participating in the DUNE experiment.  The Collaboration defines the scientific goals of the experiment and subsequently the requirements on the experimental facilities needed to achieve these goals.  The Collaboration also provides the scientific effort required for the design and construction of the DUNE detectors, operation of the experiment, and analysis of the collected data. There are four main entities within the DUNE organizational structure:  
\begin{itemize}
\item DUNE Collaboration, including the General Assembly of the collaboration and the DUNE Institutional Board     
\item DUNE Management, consisting of the two co-spokespersons, the Technical Coordinator, and the Resource Coordinator, who along with the chair of the Institutional Board and five additional members of the collaboration form the DUNE Executive Committee, as well as the collaboration Technical and Finance Boards 
\item DUNE Project, containing the Project Office, headed by the Project Manger, and the lead managers of the high-level sub-projects 
\item DUNE Science, incorporating the coordinators of the collaboration detector working groups as well as the Physics and Software/Computing Coordinators 
\end{itemize}
The connections between the different members of these entities is illustrated in Figure~\ref{fig:dune-org}.

\begin{cdrfigure}[DUNE Project and Collaboration Organization]{dune-org}{DUNE Project and Collaboration Organization}
  \includegraphics[width=0.95\textwidth]{dune-collaboration-org}
\end{cdrfigure}

\fixme{(for Anne) command to keep figure from floating ...}
\begin{cdrfigure}[DUNE Work Breakdown Structure]{dune-wbs}{DUNE Work Breakdown Structure}
  \includegraphics[width=0.8\textwidth]{dune-wbs-to-level3}
\end{cdrfigure}

%%%%%%%%%%%%%%%%%%%%%%%%%%%%%%
\subsection{Responsibilities of the DUNE Leadership}

The main responsibilities of the different roles are summarized below:
\begin{itemize}
  \item \textbf{DUNE General Assembly} is composed of the full membership of the collaboration.  It is consulted on all major strategic decisions through open plenary sessions at collaboration meetings and is provided regular updates on issues affecting the collaboration at weekly collaboration meetings.  The collaboration general assembly elects the co-spokespeople through a process defined by the Institutional Board. 
  \item \textbf{DUNE Institutional Board (IB)} is the representative body of the collaboration institutes. It is composed of one representative from each of the member institutions and holds responsibility for the governance of the collaboration.  The IB has final authority over collaboration membership issues and defines the requirements for inclusion of individuals within the DUNE authorship list. The IB is also responsible    
for establishing and monitoring the election process used to select the co-spokespersons.  
  \item \textbf{DUNE co-spokespersons} are elected by the collaboration to serve as its leaders.  They direct collaboration activities on a day-to-day basis and represent the collaboration in interactions with the host laboratory, funding agencies, and the broader scientific community.
  \item \textbf{DUNE Executive Committee (EC)} is the primary decision-making body of the collaboration and is chaired by the longest serving co-spokesperson.  The membership of the EC consists of the co-spokespeople, the Technical Coordinator, the Resource Coordinator, the chair of the IB, and five additional members of the collaboration (three elected IB representatives and two additional members selected by the co-spokespeople).  The EC operates as a decision-making body through consensus.  For issues involving change control within the DUNE project, the Technical Coordinator is the ultimate arbitrator in cases where a consensus on the EC cannot be reached.  This choice is made to preserve a single line of authority within the project organization.  For all other issues on which the EC cannot reach a consensus, final decision-making authority is assigned to the co-spokespeople.   
  \item \textbf{Technical Coordinator (TC)} is jointly appointed by the co-spokespeople and the Fermilab director and has reporting responsibilities to both.  The TC serves as the project director and is responsible for implementing the scientific and technical strategy of the collaboration in the context of the international DUNE project.  The TC also serves as project director for the DOE-funded portion of the DUNE project, which is interwoven within the international project.  In addition to managing the Project Office, the TC chairs the collaboration Technical Board which coordinates activities associated with the design, construction, installation, and commissioning of the different detector elements.   
  \item \textbf{Technical Board (TB)} is chaired by the TC and has a membership that includes the coordinators of the detector working groups and managers of the high-level sub-projects within the DUNE project.  It may also include additional members of the collaboration, nominated by the TC and approved by the EC, who are expected to bring useful knowledge and expertise to its discussions on technical issues.  The TB is the primary collaboration body where issues related to the design, construction, installation, and commissioning are discussed.  This body serves as a project change control board for change requests with schedule and cost impacts that lie below pre-determined thresholds for EC approval.  Change requests that have impacts on interfaces with the LBNF project,   potentially impact DUNE science requirements, or necessitate changes to formal Memoranda of Understanding (MOU) with one or more funding agencies contributing to the project are discussed within the TB but automatically passed to the EC.  The TB is also the primary forum for discussions on technological design choices faced by the collaboration.  Based on these discussions, the TB is expected to make a recommendation to the TC, who is then charged with making a final recommendation to the EC.           
  \item \textbf{Resource Coordinator (RC)} is jointly appointed by the co-spokespeople and the Fermilab director and has reporting responsibilities to both.  Working jointly with the TC, the RC is responsible for dividing project responsibilities among the many institutions within the collaboration and preparing the formal MOU that define the contributions and responsibilities of each institution.  The RC is also responsible for management of the common financial resources of the collaboration (common fund).  The RC takes proposed project change requests approved by the EC, which involve the modification of MOU with one or more of the participating funding agencies, first to the collaboration finance board for discussion and then in cases where consensus is obtained to the Resource Research Board for final approval.  
  \item \textbf{Finance Board (FB)} is chaired by the RC and has a membership that includes a single representative from each group of collaborating institutions whose financial support for participating in the DUNE experiment originates from a single, independent funding source.  These collaboration representatives are either nominated through their respective group of institutions and approved by the associated funding agency, or directly appointed by the funding agency.  The FB discusses issues related to  collaboration resources such as contributions to project common funds and division of project responsibilities among the collaborating institutions.  The FB is also used for vetting proposed project change requests prior to their submission to the Resource Research Board for approval.   
  \item \textbf{DUNE Science Coordinators} include the coordinators of detector working groups as well as the coordinators of the DUNE physics and computing/software efforts.  Science coordinators are nominated by the co-spokespeople (jointly with the TC in the case of detector working group coordinators) and approved by the EC.  These coordinators are expected to establish additional collaboration sub-structures within their assigned areas to cover the full scope of collaboration activities within their areas of responsibility.  Detector working group coordinators report to the EC through the TB, while coordinators of   the physics and software/computing efforts report directly to the EC.   
  \item \textbf{DUNE Project Office (PO)} provides the project management for the design, construction, installation, and commissioning of the DUNE near and far detectors. DUNE will be run as an international project matching DOE requirements. This implies maintaining a full cost and schedule for the entire project, from which the DOE-funded portion can be extracted and monitored in a manner that satisfies DOE reporting requirements. The DUNE Project Office will have direct control over DOE project funds and any common fund collected from the U.S. and international stakeholders. International contributions to the DUNE project will be in the form of deliverables as defined in formal MOU. These contributions will be tracked through detailed sub-project milestones. The entire Project (including international contributions) will be subject to the DOE critical decision process incorporating a CD-2 approval of its baseline cost and schedule and a CD-3 approval for moving forward with construction.  The current high-level WBS structure of the Project is illustrated in Figure~\ref{fig:dune-wbs}.
  \item \textbf{DUNE Technical Working Groups} provide the required interface between the DUNE project and the members of the collaboration contributing to these efforts.  These working groups are jointly chaired by the DUNE collaboration detector coordinators and the DUNE sub-project leaders.  All matters related to the design, construction, installation, and commissioning of the individual detector elements are discussed within these working groups.  The sub-project leader within each working group is tasked with implementing the plans developed within their group.  Coordination of the detector working groups and discussions of technical issues that impact multiple groups are handled through the TB. 
\end{itemize}


%%%%%%%%%%%%%%%%%%%%%%%%%%%%%%%%%%%%%%%%%%%%%%%%%%%%%
\section{LBNF/DUNE Advisory and Coordinating Structures}
\label{sec:lbnf-dune-interface}

The LBNF and DUNE projects are overseen by a number of advisory and
coordinating bodies as shown in Figure~\ref{fig:lbnfdune-org}.
The role of the different bodies are described in the following sections.  

\subsection{International Advisory Committee (IAC) }

The International Advisory Committee (IAC) that is composed of
regional representatives, such as CERN, and representatives of
funding agencies making major contributions to LBNF infrastructure or to DUNE
acts as the highest-level international advisory body to the U.S.
Department of Energy and the FNAL Directorate. The IAC facilitates
high level global coordination across the entire enterprise (LBNF and DUNE),
and provides primary oversight of the two projects through
the Resource Review Board (RRB, see below). 
The IAC is chaired by the DOE Office of Science Associate Director
for High Energy Physics and includes the FNAL Director in its membership.  
The committee meets and provides pertinent advices to the LBNF and DUNE
projects through Fermilab Director as needed.  

Specific responsibilities of the IAC include, but are not limited to,
the following:


\begin{quote}
$\bullet$ During the formative stages of LBNF and DUNE,
the IAC helps coordination of the sharing of responsibilities among
the agencies for the construction of LBNF and DUNE.
Individual agency responsibilities for LBNF will be established in
bilateral international agreements with DOE. Agency contributions to
DUNE will be formalized through separate agreements.\\
$\bullet$ The IAC assists in resolving issues, especially the
issues that cannot be resolved at the RRB level, such as those that
require substantial redistributions of responsibilities amongst the
funding agencies.\\
$\bullet$ The IAC assists as needed in the coordination,
synthesis, and evaluation of input from project reports charged by
individual funding agencies, LBNF and DUNE project management,
and/or the IAC itself, leading to recommendations for action by
the managing bodies.
\end{quote}

The initial membership of the IAC is as follows:
James Siegrist (DOE HEP, Chair),
Sergio Bertolucci (CERN),
Purniah Boddapati (DAE),
Carlos Henrique de Brito Cruz (FAPESP),
Fernando Ferroni (INFN),
Fabiola Gianotti (CERN),
Rolf Heuer (CERN),
Stavros Katsanevas (ApPEC),
Frank Linde (ApPEC),
Nigel Lockyer (FNAL),
John Womersley (STFC) and
Agnieszka Zalewska (IFJ)

\subsection{Resource Research Board (RRB)}

The Resource Research Board (RRB) is composed of representatives of all
funding agencies that sponsors the LBNF and DUNE project, and the Fermilab
management. The RRB serves as the operational arm of the IAC
in order to provide more focused monitoring and detailed oversight
of each of the projects. The Fermilab Director in coordination
with the DUNE RC defines its membership. A representative of the
Fermilab director chairs the board and
organizes regular meetings to ensure that the needed flow of resources
for the smooth progress of the projects and for the successful completion
of the projects. The management of the
DUNE collaboration and the LBNF project participates in the RRB meetings
and make regular reports to the RRB on technical, managerial,
financial and administrative matters, as well as status and
progress of the DUNE Science Collaboration.

There are two subgroups of the RRB: RRB-LBNF and RRB-DUNE. Each of
these groups monitors progress and deal with the issues specific to
LBNF and DUNE, respectively, while the whole RRB deals with matters
that concerns the entire enterprise (LBNF and DUNE). A typical RRB meeting
will start with a plenary opening session which is followed by two
successive RRB-LBNF and RRB-DUNE sessions.

The RRB can employ standing DUNE and LBNF Scrutiny Groups as needed
to assist it in its responsibilities. The scrutiny groups operate
under the RRB, and provide detailed information on financial and
personnel resources, costing, and other elements under the purview of the RRB.

Roles of the RRB includes:

\begin{quote}
$\bullet$ Assisting the IAC, through DOE and the FNAL Directorate,
with coordinating and developing any required international
agreements between partners.\\
$\bullet$ Monitoring and overseeing the Common Projects and the
use of the Common Funds.\\
$\bullet$ Monitoring and overseeing general financial and personnel support.\\
$\bullet$ Assisting the IAC through the DOE and FNAL Directorate
with resolving issues that may require reallocation of responsibilities
among the project’s funding agencies.\\
$\bullet$ Reaching consensus on a maintenance and operation procedure,
and monitoring its function.\\
$\bullet$ Endorsing the annual construction and maintenance and operation
budgets of LBNF and DUNE.\\
\end{quote}

\subsection{Fermilab, the Host Laboratory}

As the host laboratory, Fermilab has a direct responsibility for the design,
construction, commissioning, and operation of the facilities and
infrastructure that support the program.  In this capacity, Fermilab reports
directly to the DOE through the Fermilab Site Office (FSO).
Fermilab also has an important oversight role for the DUNE project
itself as well as an important coordination role in ensuring that
interface issues between the two projects are completely understood.

Fermilab's oversight of the DUNE collaboration and detector
construction project is carried out, through:
\begin{quote}
$\bullet$ Regular meetings with the Collaboration leadership\\
$\bullet$ Approving the selection of Collaboration Spokespeople\\
$\bullet$ Providing the Technical and Resource Coordinators\\
$\bullet$ Convening and chairing the Resource Review Board\\
$\bullet$ Regular scientific reviews by the PAC and LBNC\\
$\bullet$ Director’s Reviews of specific management, technical,
cost and schedule aspects of the detector construction project\\
$\bullet$ Other reviews as needed
\end{quote}

\subsection{DUNE Collaboration}	

The collaboration, in consultation with the Fermilab Director,
is responsible for forming the international project team responsible
for designing and constructing the detectors.  The Technical Coordinator
(TC) and Resource Coordinator (RC) serve as the lead managers
of this international project team and are selected jointly by
the spokespeople and the Fermilab director.  Because the international
project incorporates contributions from a number of different
funding agencies, the international DUNE project is responsible for
satisfying individual tracking and reporting requirements associated
with each of the different contributions.

\subsection{Long Baseline Neutrino Committee (LBNC)}

The Long Baseline Neutrino Committee (LBNC) that is composed
of internationally prominent scientists with relevant expertise
provides external scientific peer review for the two projects regularly.
The LBNC reviews the scientific, technical and managerial
decisions and preparations of the neutrino program.
It acts effectively as an adjunct to the Fermilab Physics Advisory Committee
(PAC), meeting on a more frequent basis than the PAC.
The LBNC may employ DUNE and LBNF Scrutiny Groups for more
detailed reports and evaluations. The LBNC members are appointed by the
Fermilab director, and the current membership of the LBNC is:
David MacFarlane (SLAC, Chair),
Ursula Bassler (IN2P3),
Francesca Di Lodovico (Queen Mary),
Patrick Huber (Virginia Tech),
Mike Lindgren (FNAL),
Naba Mondal (TIFR),
Tsuyoshi Nakaya (Kyoto),
Dave Nygren (UT Arlington),
Stephen Pordes (FNAL),
Kem Robinson (LBNL),
Nigel Smith (SNOLAB) and
Dave Wark (Oxford and STFC).

\subsection{Experiment-Facility Interface Group (EFIG)}

Close and continuous coordination between DUNE and LBNF is
required to ensure the success of the combined enterprise.
An Experiment-Facility Interface Group (EFIG) was established
in January 2015 to oversee and ensure the required coordination
both during the design and construction and the operational
phases of the program. This group covers areas including:
\begin{quote}
$\bullet$ Interface between the near and far detectors and the
corresponding conventional facilities.\\
$\bullet$ Interface between the detector systems provided by
DUNE and the technical infrastructure provided by LBNF.\\
$\bullet$ Design and operation of the LBNF neutrino beamline.
\end{quote}

The EFIG is chaired by a designate of the Fermilab Director.
Its membership includes the LBNF Project Director and Project Manager,
the DUNE Co-Spokespersons, Technical Coordinator and Resource Coordinator.
In consultation with the DUNE and LBNF management, the EFIG Chair will
extend the membership as deemed necessary to carry out the coordination
function. In addition, the DOE Federal Project Director for LBNF,
the Fermilab Chief Project Officer, and a designated representative
of the South Dakota Science and Technology Authority (SDSTA) will
serve ex officio. The EFIG Chair designates a Secretary of the EFIG,
who keeps minutes of the meetings and performs other tasks as
requested by the Chair.

It is the responsibility of the EFIG Chair to report EFIG proceedings
to the Fermilab Director and other stakeholders. It is the responsibility
of the DUNE spokespeople to report EFIG proceedings to the rest of
the collaboration. The EFIG meets weekly or as needed.

The current membership of the EFIG is:
Joe Lykken (representing Fermilab Director, Chair),
André Rubbia (DUNE co-spokesperson),
Mark Thomson (DUNE co-spokesperson),
Elaine McCluskey (LBNF Project Manager),
Eric James (DUNE Technical Coordinator),
Chang Kee Jung (DUNE Resource Coordinator),
Marzio Nessi (CERN),
David Lissauer (BNL),
Jim Stewart (BNL),
Jeff Dolph (BNL, Secretary),
Mike Lindgren (FNAL Chief Project Officer, ex officio),
Pepin Carolan (DOE, ex officio), and 
Mike Headley (SDSTA, ex officio).
rmilab Director, Chair),
André Rubbia (DUNE co-spokesperson),
Mark Thomson (DUNE co-spokesperson),
Elaine McCluskey (LBNF Project Manager),
Eric James (DUNE Technical Coordinator),
Chang Kee Jung (DUNE Resource Coordinator),
Marzio Nessi (CERN),
David Lissauer (BNL),
Jim Stewart (BNL),
Jeff Dolph (BNL, Secretary),
Mike Lindgren (FNAL Chief Project Officer, ex officio),
Pepin Carolan (DOE, ex officio), and 
Mike Headley (SDSTA, ex officio).

	
