
%%%%%%%%%%%%%%%%%%%%%%%%%%%%%%%%%%%%%%%%%%%%%%%%%%%%%%%%%%%%%%%%%%%%%%%%%%%%%%%%%%
\chapter{Organization and Management}
\label{v1ch:org-mgmt}


\section{LBNF Organization and Management}

\section{DUNE Organization and Management}

%\subsection{DUNE Science Collaboration and Project Office}
The international DUNE experiment is managed through two tightly-coupled structures: the DUNE Science Collaboration and the DUNE Project Office.  The DUNE Science Collaboration is responsible for the scientific and experimental strategy, while the DUNE Project Office % team 
is responsible for the implementation of this strategy within the context of the available international resources.

%An overview of the organization of the two entities is given in the following:
%\begin{itemize}
%\item {\bf DUNE Science Collaboration}: 
\subsection{DUNE Science Collaboration}

The DUNE Science Collaboration is the organized structure that brings together the members of the international science community interested in participating in the DUNE experiment.  The Collaboration defines the scientific goals of the experiment and subsequently the requirements on the experimental facilities needed to achieve these goals.  The Collaboration also provides the scientific effort required for the design of the detectors and facilities, operation of the experiment, and analysis of the collected data.  

An Institutional Board (IB), composed of one representative from each of the member institutions, has responsibility for Collaboration governance.  The IB has final authority over Collaboration membership issues and defines requirements for inclusion of individuals within the DUNE authorship list. The IB is also responsible for establishing and monitoring the process through which the co-spokespeople are selected to serve as leaders of the collaboration.   

The DUNE Executive Committee (EC) advises the co-spokespeople in dealing with strategic issues affecting the experiment such as technological design choices and the distribution of project responsibilities among the participating institutions.  Membership of the EC includes the co-spokespeople, DUNE Project Office leaders, IB chair, and five additional Collaboration members (three elected IB representatives and two additional %selections 
members selected by the co-spokespeople).

%\item {\bf DUNE Project Office}: 
\subsection{DUNE Project Office}

The DUNE Project is organized as an international entity that manages all contributions to the design, construction, installation, and commissioning of the DUNE near and far detectors.  The U.S. DOE-funded portion of the Project is contained within the international entity, and the management of this portion is a direct responsibility of the international DUNE project team.  The entity will be run as an international project matching DOE requirements. This will imply maintaining a full cost and schedule for the entire project, from which the DOE-funded portion can be extracted and monitored in a manner that satisfies DOE reporting requirements. 

The Project will have direct control over DOE project funds and any common fund (e.g., construction and operation) collected from the U.S. and international stakeholders.  Unless requested otherwise, Project funding obtained through other agencies will be distributed and monitored through management groups set up by those agencies.  However, the entire Project (including international contributions) will be subject to the DOE critical decision process incorporating a CD-2 approval of its baseline cost and schedule and a CD-3 approval for moving forward with construction.  

The international DUNE Project will be responsible for monitoring and reporting on the status of all contributions to the Project, independent of their funding source, at least to the level of subproject milestones tied to the critical decision process.  Non-DOE partners will report progress to the Project through detailed milestones related to their deliverables as defined in formal Memoranda of Understanding (MOU).     
%\end{itemize}


\fixme{add Work Breakdown Structure ?}


\fixme{add Project Schedule}



