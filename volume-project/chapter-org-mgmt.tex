
%%%%%%%%%%%%%%%%%%%%%%%%%%%%%%%%%%%%%%%%%%%%%%%%%%%%%%%%%%%%%%%%%%%%%%%%%%%%%%%%%%
\chapter{Organization and Management}
\label{v1ch:org-mgmt}

%
%Here is a sample figure:
%
%\begin{cdrfigure}[short]{label}{long}
%%\includegraphics[width=\linewidth]{file}
%\end{cdrfigure}
%
%Here is a sample reference to a figure (Figure~\ref{fig:label}). Notice: ``fig:'' is not present in the label as written in the figure code itself.
%
%Here is a sample table:
%
%\begin{cdrtable}[short]{cc}{label}{long} %The third argument (reads {cc}) can use c, l, r or p{some length}
%% but please do not include lines like “|c|l|l|”. It CAN look like {cll} or {llp{3cm}}, for instance.
%Header Column1 & Header Column 2 \\ \toprowrule
%Row 1 & First \\ \colhline
%Row 2 & Second \\ \colhline
%Row 3 & Third \\
%\end{cdrtable}
%
%Here is a sample reference to a table Table~\ref{tab:label}). Notice: ``tab:'' is not present in the label as written in the table code itself.
%
%

\section{LBNF Organisation and Management}

\section{DUNE Organisation and Management}

\subsection{DUNE Science Collaboration and Project Office}
The international DUNE experiment is managed through two tightly-coupled structures: the DUNE Science collaboration and the DUNE Project office.  The DUNE Science collaboration is responsible for the scientific and experimental strategy, while the DUNE Project team is responsible for the implementation of this strategy within the context of the available international resources.

An overview of the organisation of the two entities is given in the following:
\begin{itemize}
\item {\bf DUNE Science Collaboration}: 
The DUNE collaboration is the organized structure that brings together the members of the international science community interested in participating in the DUNE experiment.  The collaboration defines the scientific goals of the experiment and subsequently the requirements on the experimental facilities needed to achieve these goals.  The collaboration also provides the scientific effort required for the design of the detectors and facilities, operation of the experiment, and analysis of the collected data.  An Institutional Board (IB), composed of one representative from each of the member institutions, has responsibility for collaboration governance.  The IB has final authority over collaboration membership issues and defines requirements for inclusion of individuals within the DUNE authorship list. The IB is also responsible for establishing and monitoring the process through which the co-spokespeople          are selected to serve as leaders of the collaboration.   The DUNE Executive Committee (EC) advises the co-spokespeople in dealing with strategic issues affecting the experiment such as technological design choices and the distribution of project responsibilities among the participating institutions.  Membership of the EC includes the co-spokespeople, DUNE project office leaders, IB chair, and five additional collaboration members (three elected IB representatives and two additional selections of the co-spokespeople).

\item {\bf DUNE Project Office}: 
The DUNE project is organized as an international entity that manages all contributions to the design, construction, installation, and commissioning of the DUNE near and far detectors.  The U.S. DOE-funded portion of the project is contained within the international entity, and the management of this portion is a direct responsibility of the international DUNE project team.  The entity will be run as an international project matching DOE requirements. This will imply maintaining a full cost and schedule for the entire project, from which the DOE-funded portion can be extracted and monitored in a manner that satisfies DOE reporting requirements.  The project will have direct control over DOE project funds and any common fund (e.g. construction and operation) collected from the US and international stakeholders.  Unless requested otherwise, project funding obtained through other agencies will be distributed and monitored through management groups set up by those agencies.  However, the entire project (including international contributions) will be subject to the DOE critical decision process incorporating a CD-2 approval of its baseline cost and schedule and a CD-3 approval for moving forward with construction.  The international DUNE project will be responsible for monitoring and reporting on the status of all contributions to the project, independent of their funding source, at least to the level of sub-project milestones tied to the critical decision process.  Non-DOE partners will report progress to the project through detailed milestones related to their deliverables as defined in formal Memoranda of Understanding (MOU).     
\end{itemize}

\section{DUNE Science Collaboration}

\subsection{Institutional Board (IB)}

\subsubsection{Role and membership of the Institutional Board}
The Institutional Board is responsible for establishing governance rules of the DUNE collaboration and regulating the governance related issues. The responsibilities of the IB include, but are not limited to: admission of new collaborating institutions and members; oversight of Co-Spokesperson election process; election of at-large members of the Executive Committee; approval of common funds assessments; establishment and governance of detector operations shifts; and establishment of procedure for publication of scientific results and authorship rules. IB can request presentations on and discussion of specific issues relating to the collaboration or the experiment. The results of such discussions may be recorded and passed to the Co-Spokespersons as formal advice.

All collaborating institution except the host lab will be given one seat in the IB regardless of their varying sizes. The Fermilab will be given two seats to reflect the lab�s large DUNE membership as well as its role as the host lab of the DUNE experiment. Each institute designates its IB representative. The Executive Committee members are non-voting members of the IB.

\subsubsection{Chair of the Institutional Board}
The IB Chair is elected by the IB in every odd year starting 2015 for a two-year term. The term is renewable. The IB Chair election process should commence in March in the election year. The IB Chair election will be carried out by an IB-appointed ad hoc three-member election committee. The election committee members may not stand for the election. The committee will receive up to three nominations from each IB representative. The committee will contact all nominees with three or more nominations to ascertain their willingness to stand for the election.
The election will be held electronically by using the Preferential Voting System (PVS) in which an IB submits a fully ranked list of the candidates.

\subsection{Co-Spokespersons}

\subsubsection{Number and role of Co-Spokespersons}
There will be two equal Co-Spokespersons of the collaboration. The Co-Spokespersons are the scientific leaders and representatives of the collaboration. The Co-Spokespersons will equally share leadership, authority and responsibilities. The longest serving of the Co-Spokespersons serves as the Executive Committee Chair, and the shortest serving of the Co-Spokespersons serves as the Collaboration General Assembly Chair. The Co-Spokespersons are responsible for preparing agendas for collaboration meetings. The Co-Spokespersons will work very closely with
the Executive Committee on all major collaboration matters, seek advices from the Institutional Board, and bring major issues to the Collaboration General Assembly for discussions. The Co- Spokespersons will also work closely with the host lab Director, the LBNF/DUNE International Joint Oversight Group (IJOG) and the LBNF management team. The Co-Spokespersons will bear authority and responsibilities for the day-to-day matters of the experiment.

\subsubsection{Election of Co-Spokespersons}
The Co-Spokespersons elected prior to the effective date of this document will serve two and three year terms, respectively, from March 2015 in order to create a staggered election of a new Co-Spokesperson each year. The Co-Spokesperson elected hereafter will serve a two-year term, and the term is renewable. However, it is renewable only once. The Co-Spokesperson election process should commence in January every year. The IB Chair will initiate the election process by establishing a Co-Spokesperson Search Committee (CSSC). The composition of the search committee will be proposed by the IB chair and will be approved by the IB. The CSSC members may not stand for the election. The committee will choose its own Chair. CSSC will then receive nominations from all eligible voters (up to three nominations per nominator). The committee will contact all nominees with three or more nominations to produce an initial list of candidates who are willing to stand for the election. The list may contain additional candidates whom the committee finds well suited for the Co-Spokesperson position. The committee will then interview the candidates and produce a short list (nominally comprises three candidates) which will be presented to the collaboration for a general election. In this process, the committee will seek input from the host lab Director.

The election will be held electronically by using the Preferential Voting System (PVS) in which a voter submits a fully ranked list of the candidates. All collaborators (Ph. D.�s, engineers, technicians and graduate students) who have been the members of the DUNE collaboration for at least one year prior to the election opening date will be eligible to vote. Further requirements for voter eligibility may be imposed by the IB.

\subsection{Executive Committee (EC)}

\subsubsection{Role of Executive Committee}
The Executive Committee assists the Co-spokespeople in setting scientific and technical objectives and priorities of the DUNE collaboration taking account of the financial aspects as well as other resources. The EC  advises the Co-Spokespersons on all major collaboration matters.
The EC establishes procedures for making technical choices, and will oversee progresses and developments in various projects. 

\subsubsection{Membership}
The EC consists of three at-large elected members, two appointed members by the Co- Spokespersons and the following five ex-officio members: the Co-Spokespersons, the Technical Coordinator, the Resource Coordinator and the IB Chair. If the Deputy Spokesperson is appointed by the Co-Spokespersons with an IB confirmation, s/he will be additional ex-officio member of the EC. The longest serving of the Co-Spokespersons serves as the EC Chair.

\subsection{Collaboration General Assembly (CGA)}
The DUNE Collaboration General Assembly is a forum of all DUNE collaboration members. All major decisions by the EC and IB as well as collaboration-wide issues should be reported and discussed at the CGA. This provides a forum where individual collaborators can express their opinions on particular issues. It also ensures that the major decisions of collaboration are made with input from general collaborators as well as the collaborators are properly informed on major issues. In general, the CGA will reach consensus on any major issues and will not conduct a
formal vote. Typically the DUNE collaboration meetings will end with a CGA meeting. The shortest serving of the Co-Spokespersons serves as the CGA Chair.

\subsection{Dissenting Opinions and Procedure to Appeal}
Any collaborators who object to the decisions made by the various DUNE leaderships can submit their objections to the co-spokespersons and the EC. If the co-spokespersons and the EC cannot resolve the situation, then ultimately the issue shall be brought in front of the IB, which shall make the final decision by consensus preferably or by vote if necessary.


\section{The DUNE Project Office}

\subsection{Oversight}

Oversight of the international DUNE Project is the joint responsibility of Fermilab in its role as host laboratory and the DUNE Science collaboration.  Members of the project management team are appointed jointly by the Fermilab director and the DUNE co-spokespeople.   Direct line management responsibility for the DOE-funded portion of the project is through the Fermilab director and the Fermilab DOE site office.  

\subsection{Role of Fermilab}

	As the host laboratory, Fermilab has both an oversight role over the international DUNE Project and direct line-management responsibility for the DOE-funded portion of the project.  To satisfy this oversight role, DUNE project management has reporting responsibilities to the Fermilab director and other members of the laboratory management team.  In particular, the Chief Project Officer (CPO) chairs the Performance Oversight Group (POG), which reviews and monitors all major projects hosted by the laboratory.  In addition, the head of the Fermilab Neutrino Division, host division for the DUNE project, chairs a Project Management Group (PMG), which tracks the project and works to ensure that the project has access to needed Fermilab resources.
       
\subsection{Role of the DUNE Science Collaboration}

	The DUNE Science collaboration has a complementary oversight role over the international DUNE Project.  Oversight is provided through DUNE project office reporting responsibilities to the IB, EC, and co-spokespeople of the collaboration.  Important project issues such as technology choices and the division of project responsibilities among members of the collaboration are discussed in the DUNE EC with the objective of forming consensus on these issues.       

In addition to its oversight role, the DUNE collaboration provides technical and organizational support to the Technical and Resource Coordinators to assist them in their responsibilities
(see Sections \ref{v1sec:dunetboard} and \ref{v1sec:dunefboard}).


\subsection{Project Management Team}
The managers of the DUNE Project are the Technical Coordinator (TC) and Resource Coordinator (RC) who are jointly appointed by the Fermilab director and by the co-spokespeople of the DUNE Science Collaboration.  The TC serves as director of the DUNE Project and is responsible for appointing additional project team members as necessary to assist the TC and RC in satisfying their project management responsibilities.  

\subsection{Technical Coordinator}

The Technical Coordinator (TC) is the Director of the DUNE Project and is ultimately responsible for all technical decisions on the DUNE near and far detectors. 
The TC monitors and coordinates design, construction, installation, and commissioning activities associated with each of the detector subsystems and is responsible for the overall technical performance, planning, and scheduling of the experiment.  
The TC prepares and chairs the meetings of the Technical Board of the experiment collaboration, where the technical planning of all subsystems is discussed and approved.
The TC organizes internal reviews in the process of drafting the Technical Design Report (TDR) for each subsystem of the experiment. The TC is responsible for the submission of the TDRs.
The host laboratory Chief Project Officer organizes the evaluation of the TDRs by external scientific/technical expert committees. The TC is the representative of the collaboration in the communication with the Chief Project officer regarding technical issues and the feedback of the external referees.

The TC also serves as the Director of the DOE-funded portion of the project.  

\subsection{Project Manager}
The Project Manager (PM) is appointed by the Technical Coordinator with the approval of the Fermilab director and DUNE co-spokespeople to help the project satisfy its numerous reporting and monitoring requirements.  In this role, the DUNE PM also serves as the project manager for the U.S. DOE-funded portion of the project.
 
\subsection{Resource Coordinator}
The Resources Coordinator (RC) is responsible for coordinating the financial planning and other resources issues of the collaboration. The RC is responsible in particular for the management of the common resources of the DUNE Collaboration (common fund).
The RC prepares and chairs the meetings of the Finance Board (internal) of the DUNE collaboration. The Finance Board is responsible for dealing with matters related to the costs and resources of the Collaboration, evaluation of the contributions, relations with the funding agencies and all administrative matters. The RC is responsible for the preparation of the Memoranda of Understanding of the Collaboration e.g. Interim MoU (preparatory phase) and Construction MoU (construction phase).

Moreover the RC prepares and organizes together with the host laboratory Deputy Director the Resource Review Board (RRB) meetings for the experiment.
The RRB is the body where the experiment resources are approved upon proposals from the collaboration, and monitored. It is composed of representatives of the national funding agencies, the host laboratory management and the collaboration management. It is chaired by the host laboratory Deputy Director. The RRB discusses the different national contributions and the Memoranda of Understanding. In-kind contributions to the Common Projects have to be approved in the RRB.


\subsection{Deputy Technical Coordinator}

	A Deputy Technical Coordinator may be added to the project management team from within the DUNE collaboration to assist the TC in carrying out their responsibilities.  The candidate is jointly nominated by the co-spokespeople    and TC and approved by the DUNE Executive Committee.

\subsection{Additional Project Team Members}

	Other project team members such as project engineers, finance and controls personnel, safety and quality assurance officers, and administrative support staff are appointed to the project team as necessary by the TC.            

\subsection{DUNE Technical Board}
\label{v1sec:dunetboard}

	The DUNE Technical Board (TB), chaired by the TC, consists of eight to ten members of the collaboration.   Members of the TB are jointly nominated by the TC and DUNE co-spokespersons and approved by the DUNE EC.  The role of the TB is to provide guidance to the TC, who has ultimate responsibility for the overall technical performance, planning, and scheduling of the experiment.  The TB is also charged with monitoring and coordinating the design, construction, installation, and commissioning activities associated with each of the detector subsystems.   
 
\subsection{DUNE Finance Board}
\label{v1sec:dunefboard}

The DUNE Finance Board (FB), chaired by the RC, is made up of collaboration members representing each of the participating funding agencies.  Each of the FB members represents a group of institutions that obtain their financial support for participating in the DUNE experiment from a single funding source.  Thus, a single collaborating country may have multiple representatives on the DUNE FB in cases where there are multiple independent funding agencies supporting their participation.  The FB deals with issues related to finances such as project common funds and overall resources of the collaboration.  FB members are designated by their respective groups of institutions and approved by their associated funding agency or appointed directly by that funding agency.          


