
%%%%%%%%%%%%%%%%%%%%%%%%%%%%%%%%%%%%%%%%%%%%%%%%%%%%%%%%%%%%%%%%%%%%%%%%%%%%%%%%%%
\chapter{Organization and Management}
\label{v1ch:org-mgmt}


\section{LBNF Organization and Management}

\section{DUNE Organization and Management}

%\subsection{DUNE Science Collaboration and Project Office}
%The international DUNE experiment is managed through two tightly-coupled structures: the DUNE Science Collaboration and the DUNE Project Office.  The DUNE Science Collaboration is responsible for the scientific and experimental strategy, while the DUNE Project Office is responsible for the implementation of this strategy within the context of the available international resources.
%
%%An overview of the organization of the two entities is given in the following:
%%\begin{itemize}
%%\item {\bf DUNE Science Collaboration}: 
%\subsection{DUNE Science Collaboration}

The DUNE Collaboration brings together the members of the international science community 
interested in participating in the DUNE experiment.  The Collaboration defines the scientific goals of the experiment and subsequently 
the requirements on the experimental facilities needed to achieve these goals.  The Collaboration also provides the scientific and 
technical effort required for the design and construction of the DUNE detectors, operation of the experiment, and analysis of the 
collected data. There are four main elements in the DUNE organizational structure:  
{\it the DUNE collaboration}, comprised of the General Assembly of the collaboration and the DUNE Institute Board;     
{\it DUNE management}, the two co-spokespersons, the Technical Coordinator (TC), the Resource Coordinator (RC), who along
  with the IB chair and five other members of the collaboration form the DUNE Executive Committee (EC); 
{\it the DUNE Project Office (PO)}; and
{\it the DUNE Science Team}, led by the Physics and Software/Computing coordinators. 

The main responsibilities of the different roles are summarised below:
\begin{itemize}
  \item {\bf The DUNE General Assembly} is composed of all members of the collaboration, it is consulted on major strategic decisions 
    through open plenary sessions at collaboration meetings and is informed through regular collaboration phone calls;
  \item {\bf The DUNE Institutional Board} represents the institutes of the collaboration. It is composed of one representative from each 
    of the member institutions and has responsibility for Collaboration governance.  The IB has final authority over Collaboration 
    membership issues and defines requirements for inclusion of individuals within the DUNE authorship list. The IB is also responsible    
    for establishing and monitoring the process through which the co-spokespeople are selected to serve as leaders of the collaboration.   
  \item {\bf The DUNE co-spokespersons} are accountable to the collaboration. They 
    are responsible for the day-to-day running of the collaboration and for representing the collaboration to Fermilab, funding 
    agencies and the broader scientific community.
  \item{\bf  The DUNE Executive Committee (EC)} is chaired by the longest serving co-spokesperson and is the primary 
    decision-making body of the collaboration. Membership of the EC includes the co-spokespeople, DUNE Project Office leaders, IB 
    chair, and five additional Collaboration members (three elected IB representatives and two additional members selected by the co-
    spokespeople). The EC will work by consensus. In the cases where consensus cannot be reached, 
    the authority lies with the spokespeople. If the co-spokespeople disagree, the TC will arbitrate.
  \item {\bf The Technical Coordinator (TC)} reports to the spokespersons and the Fermilab director. 
     The TC acts as the project director 
    and is responsible for the implementation of the scientific and technical strategy of the collaboration through the DUNE project office.
    The TC is also responsible for the management of the DOE contributions to the DUNE project.  
     The Technical Coordinator prepares and chairs the meetings of the Technical Board of the experiment collaboration.
     \item{\bf The Technical Board (TB)} discusses and approves the technical planning for all subsystems of the DUNE detector;
       \item {\bf The Resource Coordinator (RC)} reports to the spokespersons and the Fermilab director. The RC
    The Resources Coordinator is responsible for coordinating the financial planning and other
resources issues of the collaboration. The Resource Coordinator is responsible in particular for
the management of the common resources of the Collaboration (common fund).
The Resources Coordinator prepares and chairs the meetings of the Finance Board (internal) of
the experiment collaboration. The Resources Coordinator is responsible for the
preparation of the Memoranda of Understanding of the Collaboration.
    \item{\bf The Finance Board (FB)} is responsible for dealing with matters related to
the costs and resources of the Collaboration, evaluation of the contributions, relations with the
funding agencies and all administrative matters.  
    \item{\bf The DUNE Science Team} is led by the physics coordinator and the software/computing coordinator and is responsible for the management of the DUNE scientific WGs;
    \item{\bf The DUNE Project Office (PO)} provides the project management for the design, construction, installation, and commissioning of the DUNE near and far detectors. DUNE will be run as an international project matching DOE requirements. This will imply maintaining a full cost and schedule for the entire project, from which the DOE-funded portion can be extracted and monitored in a manner that satisfies DOE reporting requirements. The DUNE Project Office will have direct control over DOE project funds and any common fund collected from the U.S. and international stakeholders. International contributions to DUNE will be in the form of deliverables as defined in formal Memoranda of Understanding (MOU). These contributions will be tracked through detailed sub-project milestone. The entire Project (including international contributions) will be subject to the DOE critical decision process incorporating a CD-2 approval of its baseline cost and schedule and a CD-3 approval for moving forward with construction.  
    \item{\bf The DUNE Technical WGs} The organization of the technical working groups of the DUNE collaboration is the responsibility of the L2 managers in the DUNE project.
\end{itemize}


\fixme{add Work Breakdown Structure ?}


\fixme{add Project Schedule}



