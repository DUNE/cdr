
%%%%%%%%%%%%%%%%%%%%%%%%%%%%%%%%%%%%%%%%%%%%%%%%%%%%%%%%%%%%%%%%%%%%%%%%%%%%%%%%%%
\chapter{Strategy}
\label{v1ch:strategy}

\fixme{Some dates in this section are being revised as the resource-loaded schedule is matched to DOE funding guidance. Revised dates are expected soon.}

Recommendation 12 of the Report of the Particle Physics Project Prioritization Panel (P5) 
states that for a Long-Baseline Neutrino Oscillation Experiment ``The 
minimum requirements to proceed are the identified capability to reach an exposure 
of \num{120}\ktMWyr{} by the 2035 timeframe, the far detector situated underground 
with cavern space for expansion to at least 40~kt LAr fiducial volume, and 1.2~MW 
beam power upgradable to multi-megawatt power. The experiment should have the demonstrated 
capability to search for supernova bursts and for proton decay, providing a significant 
improvement in discovery sensitivity over current searches for the proton lifetime.'' 
Based on the resource-loaded schedules for the reference designs of the facility (\vollbnf)
and the detectors (\voldune), the strategy presented here meets these criteria. 
The P5 recommendations are also in line with the CERN European Strategy for Particle 
Physics (ESPP) of 2013, which classified the long-baseline neutrino program as 
one of the four scientific objectives with required international infrastructure.

\section{Global DUNE-LBNF Strategy}

The project strategy presented in this CDR has been developed to meet the requirements 
set out in the P5 report and %taking 
takes into account the recommendations of the European 
ESPP strategy, adopting a model where the U.S. DOE and international funding agencies 
share costs on the DUNE detectors, and CERN provides in-kind contributions 
to the supporting infrastructure.


The Long-Baseline Neutrino Facility (LBNF) provides:
\begin{itemize}
\item excavation of four underground caverns, each capable of hosting a cryostat 
with a 10-kt fiducial mass LArTPC, is planned to be completed 
by 20yy %2022 
under a single contract

\item surface, shaft, and underground infrastructure to support 
the outfitting of the caverns with four free-standing steel-supported cryostats 
and the required cryogenic systems. The first two cryostats will be available for filling by
20yy, %2024
allowing for a rapid deployment of the first two 10-kt far detector modules. 
The intention is to install third and fourth cryostats as rapidly as funding will 
allow.

\item  the conventional facilities for the near detector systems at Fermilab 

\item  the conventional and technical facilities for a 1.2-MW neutrino beam utilizing the PIP-II upgrade of the Fermilab accelerator 
complex, operational at the latest by 20yy %2026
and upgradable to 2.4\,MW with the proposed 
PIP-III upgrade
\end{itemize}

The Deep Underground Neutrino Experiment (DUNE) will provide:
\begin{itemize}

\item four LArTPCs, each with a fiducial mass of at least 10\,kt. The division of 
the far detector into four equal-mass detectors provides the project flexibility 
in the installation and funding (DOE vs. non-DOE) in the case of new resources being 
identified; this division also mitigates risks and allows for an early and graded science return.

\item the near detector systems, consisting of a highly-capable neutrino detector 
and the muon monitoring system necessary to reach the precision requirements needed to fully 
exploit the statistical power of the very massive far detector coupled to the powerful MW-class 
neutrino beam

\end{itemize}

Based on the reference design described below and in Volumes 2, 3 and 4 of the LBNF/DUNE 
CDR, the resource-loaded schedule will see the first two 10\,kt far detector modules 
operational by 20yy, %2025
with first beam shortly afterward. At that time the cavern 
space for all four 10-kt far detector modules will be available, allowing for 
an accelerated installation schedule, if sufficient resources for
the experiment can be established on an accelerated timescale.  

\vspace{6pt}
The project strategy described above meets these goals, reaching an exposure of 
\num{120}\ktMWyr{} by 2032, and potentially earlier if additional resources are identified. 
The P5 recommendation of sensitivity to CP violation of 3$\sigma$ for 75\% of $\delta_\text{CP}$
values can be reached with an exposure of \num{850}\ktMWyr{} with an optimized beam.

\section{A Strategy for Implementing the DUNE Far Detector}

The LBNF project will provide four separate cryostats to be located on the 4850L at the 
Sanford Underground Research Facility (SURF).  Instrumentation of the first cryostat 
will commence in 20yy. %2021/2022 
As part of the deployment and risk mitigation strategy, 
the cryostat for the second detector must be available when the first cryostat 
is filled. The aim is to install third and fourth cryostats as rapidly thereafter as funding 
allows.

The DUNE collaboration aims to deploy four 10-kt (fiducial) mass FD modules based 
on the Liquid Argon Time Projection Chamber (LArTPC) technology. The viability 
of the basic LArTPC technology has been proven by the ICARUS experiment. Neutrino 
interactions in liquid argon produce ionization and scintillation signals. While 
the basic detection method is the same, DUNE contemplates two options for the readout 
of the ionization signals: single-phase readout, where the ionization is detected 
using readout (wire) planes in the liquid argon volume; and the dual-phase approach, where 
the ionization signals are amplified and detected in gaseous argon above the liquid 
surface. The dual-phase approach, if demonstrated, would allow for a 3-mm readout 
pitch, a lower detection energy threshold, and potentially better reconstruction of 
the events. The DUNE single-phase readout design is being validated 
in the 35-t prototype detector at Fermilab. A 20-t dual-phase readout prototype is being 
constructed at CERN and will operate in 2016. An active development program for 
both technologies is being pursued in the context of the Fermilab Short-Baseline Neutrino (SBN)
program and 
the CERN Neutrino Platform. 
%Currently, the development of the dual-phase technology 
%is being funded from outside the DOE project. 
A flexible 
approach to the DUNE far detector designs offers the potential to bring additional 
interest and resources into the experimental collaboration. 

\section{Guiding Principles for the DUNE Far Detector}

\begin{itemize}
\item The lowest-risk design for the first 10-kt module satisfying the requirements 
will be adopted, allowing for its installation at SURF to commence in 20yy. %2021/2022. 
Installation  of the second 10-kt module should commence before 20yy. % 2023.

\item  Recognition that the LArTPC technology will continue to evolve with: (1) the 
large-scale prototypes at the CERN Neutrino Platform and the experience from the 
Fermilab SBN program, and (2) the experience gained during the construction and 
commissioning of the first 10-kt module. It is assumed that all four modules 
will be similar but not necessarily identical.

\item  In order to start installation on the timescale of 20yy, % 2021/2022,
the first  10-kt module will be based on the APA/CPA design, which is currently the lowest
risk option. There will be a clear and transparent decision process (organized by the DUNE 
technical board) for the design 
of the second and subsequent far detector modules, allowing for evolution of the 
LArTPC technology to be implemented. The decision will be 
based on physics performance, technical and schedule risks, costs and funding 
opportunities.

\item The DUNE Collaboration will instrument the second cryostat as soon as possible.

\item A comprehensive list of synergies between the reference and alternative designs 
has been identified and summarized in \voldune. Common solutions for DAQ, 
electronics, HV feed-throughs, etc., will be pursued and implemented, independent 
of the details of the TPC design.

\end{itemize}

\subsection{Strategy for the First 10-kt Far Detector TPC}



The viability of wire-plane LArTPC readout has already been demonstrated by the ICARUS T600 
experiment, where data were successfully accumulated over a period of three years. 
An extrapolation of the observed performance and the implementation of improvements 
in the design (such as immersed cold electronics) will allow the single-phase 
approach to meet the LBNF/DUNE far detector requirements. In order to start the FD installation
by 20yy, % 2022, 
the first 10-kt module will be based on the single-phase design using anode and cathode
plane assemblies (APAs and CPAs), described in Chapter 4 of \voldune. 
Based on previous experience and the 
future development path in the Fermilab SBN program and at the CERN Neutrino Platform, 
this choice represents the lowest-risk option for installation of the first 10-kt FD module by 
20yy. % 2021/2022. 
For these reasons, the APA/CPA single-phase wire plane LArTPC readout 
concept %, described in Volume 4 of the DUNE CDR, 
is the \textit{reference design} 
for the far detector. 
The design is already relatively advanced for the conceptual  
stage. From this point on, modifications to the reference design will require approval
by the DUNE Technical Board. A preliminary design review will take place as early 
as possible, utilizing the experience from the DUNE 35-t prototype; the design 
review will define the baseline design that will form the basis of the TDR (CD-2). 
At that point, the design will be put under a formal change-control 
process. 

A single-phase engineering prototype,
comprising six full-sized drift cells of the TDR engineering baseline,
will be validated at the CERN neutrino platform in 2018 (pending approval by CERN). 
This %single-phase engineering prototype at CERN 
prototype is a central part of the risk-mitigation 
strategy for the first 10-kt module and is part of the DOE-funded portion of the DUNE project. 
Based on the  performance of this prototype at the CERN Neutrino Platform, a final design review will 
take place towards the end of 2018 and construction of the readout planes will 
commence in 2019, to be ready for first installation in 20yy. % 2021/2022. 
The design reviews 
will be organized by the DUNE Technical Coordinator. 

In parallel with preparation for construction of the first 10-kt far detector module, 
\fixme{Do we want to use FD and ND or spell them out each time?}
the DUNE collaboration recognizes the potential of the dual-phase technology and 
strongly endorses the already approved development program at the CERN Neutrino 
Platform (the WA105 experiment), which includes the operation of the 20-t prototype 
in 2016 and the 6$\times$6$\times$6\,m\textsuperscript{3} demonstrator in 2018. Participation 
in the WA105 experiment is open to all DUNE collaborators. A concept for the dual-phase 
implementation of a far detector module is presented as an \textit{alternative 
design} in \voldune. This alternative design, if demonstrated, 
could form the basis of the second or subsequent 10-kt modules, in 
particular to achieve improved detector performances in a cost-effective way. 

\subsection{DUNE at the CERN Neutrino Platform}

WA105 has signed an MoU with the CERN Neutrino Platform to provide a large \textasciitilde{} 
8$\times$8$\times$8\,m\textsuperscript{3} cryostat by October 2016 in the new EHN1 extension, 
and it is foreseen that a second large cryostat to house the single-phase LArTPC will 
be provided on a similar timescale. Both will be exposed to charged-particle test 
beam spanning a range of particle types and energies.   

The DUNE collaboration will instrument one of these cryostats with an arrangement 
of six APAs and six CPAs, in a APA:CPA:APA configuration providing an engineering 
test of the full-size drift volume. These will be produced at two or more sites with the cost 
shared between the DOE project and international partners. The CERN prototype thus 
provides the opportunity for the production sites to validate the manufacturing 
procedure ahead of large-scale production for the far detector. Three major operational 
milestones are defined for this single-phase prototype: (1) engineering validation 
--- successful cool-down;( 2) operational validation --- successful TPC readout with 
cosmic-ray muons; and (3) physics validation with test-beam data. Reaching milestone 
2, scheduled for early 2018, will allow the retirement of a number of technical 
risks for the construction of the first 10-kt module. The proposal for the DUNE 
single-phase prototype will be presented to the CERN SPSC in June 2015. 

In parallel, the WA105 experiment approved by the CERN Research Board in 2014 and supported 
by the CERN Neutrino Platform has a funded plan to construct and operate a large-scale 
demonstrator utilizing the dual-phase readout in the test beam by October 2017. 
Successful operation and demonstration of long-term stability of the WA105 demonstrator 
will establish this technological solution as an option for the second or subsequent 
far detector modules. The DUNE dual-phase design is based on independent 3$\times$3\,m$^2$
charge readout planes (CRP) placed at the gas-liquid interface. Each module provides 
two perpendicular ``collection'' views with 3-mm readout pitch. A 10-kt module 
would be composed of 80 CRPs hanging from the top of the cryostat, decoupled from 
the field cage and cathode. The WA105 demonstrator will contain four 3$\times$3m$^2$ 
CRPs of the DUNE type giving the opportunity to validate the manufacturing procedure 
ahead of large-scale production. WA105 is presently constructing a 3$\times$1\,m$^2$ 
CRP to be operated in 2016. The same operational milestones (engineering, operational, 
physics) are defined as for the single-phase prototype.

The DUNE program at the CERN Neutrino Platform will be coordinated by a single 
L2 manager. Common technical solutions will be adopted wherever possible for the 
DUNE single-phase engineering prototype and the dual-phase (WA105) demonstrator. 
The charged-particle test-beam data will provide essential calibration samples 
for both technologies and will enable a direct comparison of the relative physics 
capabilities of the single-phase and dual-phase TPC readout. 

\subsection{Strategy for the Second and Subsequent 10-kt Far Detector Modules}

For the purposes of cost and schedule, the reference design for the first module 
is taken as the reference design for the subsequent three modules. However, 
the experience with the first 10-kt module and the development activities at 
the CERN Neutrino Platform are likely to lead to the evolution of the TPC technology, both 
in terms of refinements to single-phase design and the validation of the operation 
of the dual-phase design. The DUNE technical board will instigate a formal review 
of the design for the second module in 20yy; %2020; 
the technology choice 
will be based on risk, cost (including the potential benefits of additional 
non-DOE funding) and physics performance (as established in the CERN charged-particle 
test beam). After the decision, the design of the second module will come under formal 
change control. This process will be repeated for the third and fourth modules 
in 20yy.%2022.

This strategy allows flexibility with respect to international contributions, 
enabling the DUNE collaboration to
adopt evolving approaches for subsequent modules. This approach provides the possibility of attracting interest 
and resources from a broader community, and space for flexibility to respond to 
the funding constraints from different sources. 

\section{A Strategy for Implementing the DUNE Near Detector(s)}

The LBNF project will provide the facilities for the DUNE near detector systems 
(muon monitors and near neutrino detector). The primary scientific motivation for 
the DUNE near detector system is to determine the beam spectrum for the long-baseline 
neutrino oscillation studies. The near detector, which is exposed to an intense 
flux of neutrinos, also enables a wealth of fundamental neutrino 
interaction measurements, which are an important part of the  scientific 
goals of the DUNE collaboration. Within the former LBNE collaboration the neutrino 
near detector (NND) design was the NOMAD-inspired fine-grained tracker (FGT), which 
was established through a strong collaboration of U.S. and Indian institutes.

\subsection{Guiding Principles for the DUNE Near Detector}

\begin{itemize}
%%\item Recognition of the central importance of the reference design for NND;  
\item  The primary design consideration of the DUNE neutrino near detector is the 
ability to adequately constrain the systematic errors in the DUNE LBL oscillation 
analysis; this requires the capability to precisely measure exclusive neutrino
interactions.

\item An additional design consideration for the DUNE NND is the self-contained non-oscillation 
neutrino physics program.

\item It is recognized that a detailed cost-benefit study of potential near detector options 
has yet to take place and such a study is of high priority to the DUNE project. \fixme{DUNE Project or Collaboration?}
\end{itemize}

\subsection{The DUNE Near Detector Reference Design }

The NOMAD-inspired fine-grained tracker (FGT) concept is the \textit{reference 
design} for CD-1 review. The cost and resource-loaded schedule for CD-1 review 
will be based on this design, as will the near site conventional facilities. The 
Fine-Grained Tracker consists of:  central straw-tube tracker (STT) of volume 
3.5\,m$\times$3.5\,m$\times$6.4\,m; a lead-scintillator sandwich sampling electromagnetic calorimeter 
(ECAL); a large-bore warm dipole magnet, with inner dimensions of 
4.5\,m$\times$4.5\,m$\times$8.0\,m, surrounding the STT and ECAL and providing a magnetic field of 0.4\,T; 
and RPC-based muon detectors (MuIDs) located in the steel of the magnet, as well 
as upstream and downstream of the STT. The reference 
design is presented in Chapter 
7 of \voldune. 

For ten years of operation in the LBNF 1.2-MW beam (5 years neutrinos + 5 years 
antineutrinos), the near detector will record a sample of more than 100 million 
neutrino interactions and 50 million antineutrino interactions. These vast samples 
of neutrino interactions shall provide the necessary strong constraints on the 
systematic uncertainties for the LBL oscillation physics --- the justification is 
given in Section 6.1.1 of Volume \volphys. The large samples of neutrino 
interactions will also provide significant physics opportunities, including 
numerous topics for PhD theses.  

\subsection{DUNE Strategy for the Near Detector}

The contribution of Indian institutions to the design and construction of the DUNE 
FGT neutrino near detector is a vital part of the strategy for the construction 
of the experiment. The reference design will provide a rich self-contained physics 
program. From the perspective of an ultimate LBL oscillation program, there may 
be benefits of augmenting the FGT with, for example, a relatively small LArTPC 
in front of the FGT that would allow for a direct comparison with the far detector. 
A second line of study would be to augment the straw-tube tracker  with 
a High-Pressure Gaseous Argon TPC. At this stage, the benefits of such options 
have not been studied; alternative designs for the NND are not presented in 
the CDR and will be the subject of detailed studies in the coming months. 

\subsection{DUNE Near Detector Task Force}

A full end-to-end study of the impact of the FGT NND design on the LBL oscillation 
systematics has yet to be performed. Many of the elements of such a study are in 
development, for example the Monte Carlo simulation of the FGT and the adaptation 
of the T2K framework for implementing ND measurements as constraints in the propagation 
of systematic uncertainties to the far detector. 

After the CD-1-R review, the DUNE collaboration will initiate a detailed study 
of the optimization of the NND system. To this end a new task force will be set 
up with the charge of:

\begin{itemize}
\item delivering the simulation of the NND reference design and possible alternatives

\item undertaking an end-to-end study to provide a quantitative understanding of 
the power of the NND designs to constrain the systematic uncertainties on the LBL 
oscillation measurements

\item quantifying the benefits of augmenting the reference design with a LArTPC 
or a high-pressure gaseous argon TPC
\end{itemize}

High priority will be placed on this work and the intention is to engage a broad 
cross section of the collaboration in this process. The task force will be charged 
to deliver a report by July 2016. Based on the report of this task force and input 
from the DUNE Technical Board, the DUNE Executive Board will refine the DUNE strategy 
for the near detector.

\section{A Strategy for Developing the LBNF Beamline}
 
The neutrino beamline described in this CDR is a direct outgrowth of the design~\cite{lbnecdr} developed for the 
CD 1 in 2012.  That design was driven by the need to minimize cost, while delivering the performance required to meet the scientific objectives of the long-baseline neutrino program.  It includes many features that followed directly from the 
successful NuMI beamline design as updated for the NOvA experiment.  It utilizes a target and horn system based on NuMI designs, with the spacing of the target and two horns set to maximize flux at the first, and to the extent possible, second 
oscillation maxima, subject to the limitations of  the NuMI designs for these systems.  The target chase volume --- length and width --- are set to the minimum necessary to accommodate this focusing system, and the temporary morgue space to store 
used targets and horns is sized based on the size of the NuMI components.  Following the NuMI design, the decay pipe is helium-filled, while the target chase is air-filled.  
 
The LBNF beamline is designed to utilize the Main Injector proton beam with energy between 60 and 120~GeV and beam power from 1.0 to 1.2~MW respectively, as will be delivered after the PIP-II upgrades~\cite{pip2-2013}.  The ability to vary the 
proton beam energy is important for optimizing the flux spectrum and to understand systematic effects in the beam production, and to provide flexibility to allow the facility to address future questions in neutrino physics which may require a 
different flux spectrum.  To allow for higher beam power that may be enabled by future upgrades to the Fermilab accelerator complex beyond PIP-II, those elements of the beamline and supporting conventional facilities that cannot be changed once 
the facility is built and has been irradiated are designed to accommodate beam power in the range of 2.0 to 2.4~MW for proton beam energy of 60 to 120~GeV.  These elements include the primary beam, target hall, decay pipe and absorber, as 
well as all of the shielding for them.  Components that can be replaced, such as targets and horns, are designed for the \MWadj{1.2} initial operation.  In general, additional R\&D, which is not part of the LBNF Project, is required to develop those components to be able to operate at the higher beam power.
 
Since the 2012 CD-1 review, the beamline design has evolved in a number of areas, as better understanding of the design requirements and constraints has been developed.  Some of these design changes have come to full maturity and are 
described in this CDR.  Others require further development and evaluation to determine if and how they might be incorporated into the LBNF neutrino beamline design.  They offer the possibility of higher performance, flexibility in 
implementation of future ideas, or greater reliability. The beamline facility is designed to have an operational lifetime of 20 years, and it is important that it be designed to allow future upgrades and modifications that will allow it to 
exploit new 
technologies and/or adapt the neutrino spectrum to address new questions in neutrino physics over this long period. The key alternatives and options under consideration and the strategy for evaluating and potentially implementing them are summarized here.  They are described in more detail elsewhere in this CDR or its Annexes.  
               
 
Further optimization of the target-horn system has the potential to substantially increase the neutrino flux at the first and especially second oscillation maxima and to reduce wrong-sign neutrino background, thereby increasing the sensitivity to CP 
violation and mass hierarchy determination, as discussed in \volphys.  The optimization work there is ongoing and may yield further improvements beyond those currently achieved. Engineering studies of the proposed horn designs and methods 
of integrating the target into the first horn must be performed to turn these concepts into real buildable structures that satisfy other requirements such as reliability and longevity.  These studies will be carried out between CD-1 and CD-2 to 
determine the baseline design for the LBNF target-horn system.  Since targets and horns must be replaceable, it is also possible to continue development of the target-horn system in the future and replace the initial system with a more advanced 
one or one optimized for different physics.  Such future development, beyond that necessary to establish the baseline design at CD-2, would be done outside of the LBNF Project.
 
The more advanced focusing system, called the ``optimized beam configuration'' in \volphys, utilizes horns that are longer and larger in diameter and that are spaced farther apart than the reference design, which would require a target chase approximately 9~m longer and 0.6~m wider 
than the reference design.  It cannot be ruled out that further optimization, or or future designs to explore new questions may require additional space beyond this.  Also, the larger horns will require larger 
space for temporary storage of used, irradiated components, requiring, in turn, an increase in the size of the morgue or a revision of the remote handling approach.  Between CD-1 and CD-2, studies will be done to determine not only the geometric 
requirements from the final baseline target-horn system, but also to estimate the dimensions needed to accommodate potential future designs.
 
The material, geometry and the structure of the target assembly itself can have significant impact both on the effective pion production and the energy spectrum of pions, which in turn affect the neutrino spectrum, and on the reliability and longevity of 
the target, which affects the integrated beam exposure.  Potential design developments range from incremental (e.g., changing from the reference design rectangular cross section, water-cooled graphite target to a cylindrical 
helium-cooled target), to more substantial (e.g., changing target material from graphite to beryllium), to radical (e.g., implementing a hybrid target with lighter material upstream and heavier material downstream and perhaps constructed of a set of spheres captured in a 
cylindrical skin).  New designs beyond the current reference design are also needed in order to accommodate the higher beam power (up to 2.4~MW) that will be provided by the PIP-II upgrade.  Target development will largely be carried out in the 
context of worldwide collaborations on high-power targetry such as the Radiation Damage In Accelerator Target 
Environments (RaDIATE) collaboration, and not within the LBNF Project. The LBNF design must be such that it can fully exploit future developments in target design.
 
The length and diameter of the decay pipe also affect the neutrino flux spectrum.  A longer decay pipe increases the total neutrino flux with a larger increase at higher energies; a larger diameter allows the capture and decay of lower-energy pions, 
increasing the neutrino flux at lower energy as described in \volphys. The dimensions also affect the electron-neutrino and wrong-sign backgrounds.  Unlike targets and horns, the decay pipe cannot be modified after the facility is built, making the 
choice of geometry particularly important.  The reference design values of \SI{204}{\meter} length and \SI{4}{\meter} diameter appear well matched to the physics of DUNE but studies to determine the optimal dimensions continue.  The cost of increasing the decay 
pipe length or diameter  is relatively large, including 
the impact on the absorber of increasing the diameter.
Therefore, studies of the decay pipe must include 
evaluation of the relative advantages of
investment in the decay pipe versus investment in 
other systems, e.g., a larger target hall complex, more advance target-horn systems, or more far detector mass.  Ongoing studies will continue to be carried out jointly by LBNF and DUNE between CD-1 and CD-2 to determine the baseline decay pipe geometry.
