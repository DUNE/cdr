

\chapter{Introduction to LBNF and DUNE}
\label{ch:project-overview}

%% Intro shared by all subsections

\section{An International Physics Program}

The global neutrino physics community is developing a multi-decade
physics program to measure unknown parameters of the Standard Model of
particle physics and search for new phenomena.  The program will be carried out as an international,
leading-edge, dual-site experiment for neutrino science and proton decay studies, which 
is known as the Deep Underground Neutrino Experiment (DUNE).
The detectors for this experiment will be designed, built, commissioned and operated by the international DUNE Collaboration. The facility required to support this experiment, the Long-Baseline Neutrino Facility (LBNF), is hosted by Fermilab and its design and construction is organized as a DOE/Fermilab project incorporating international partners. Together LBNF and DUNE will comprise the world's highest-intensity neutrino beam at Fermilab, in Batavia, IL, a high-precision near detector on the Fermilab site, a massive liquid argon time-projection chamber (LArTPC) far detector installed deep underground at the Sanford Underground Research Facility (SURF) \SI{1300}{\km} away in Lead, SD, and all of the conventional and technical facilities necessary to support the beamline and detector systems. 


The strategy for executing the experimental program presented in this Conceptual 
Design Report (CDR) has been developed to meet the requirements 
set out in the P5 report~\cite{p5report} and takes into account the recommendations of the European Strategy for Particle Physics~\cite{ESPP-2012}. It adopts a model where U.S. and international funding agencies 
share costs on the DUNE detectors, and CERN and other participants provide in-kind contributions 
to the supporting infrastructure of LBNF. LBNF and DUNE will be tightly coordinated as DUNE collaborators 
design the detectors and infrastructure that will carry out the scientific program.
  
The scope of LBNF is
\begin{itemize}
\item an intense neutrino beam aimed at the far site
\item conventional facilities at both the near and far sites
\item cryogenics infrastructure to support the DUNE
  liquid argon time-projection chamber (LArTPC) detectors at SURF
\end{itemize}

The DUNE detectors include
\begin{itemize}
\item a high-performance neutrino detector and beamline monitoring system
located a few hundred meters downstream of the neutrino source
\item a massive LArTPC neutrino detector located deep underground at the far site
\end{itemize}

With the facilities provided by LBNF and the detectors provided by
DUNE, the DUNE Collaboration proposes to mount a focused attack on the
puzzle of neutrinos with broad sensitivity to neutrino oscillation
parameters in a single experiment.  The focus of the scientific
program is the determination of the neutrino mass hierarchy and the
explicit demonstration of leptonic CP violation, if it exists, by
precisely measuring differences between the oscillations of muon-type
neutrinos and antineutrinos into electron-type neutrinos
and antineutrinos, respectively. Siting the far detector deep underground will
provide exciting additional research opportunities in nucleon decay,
studies utilizing atmospheric neutrinos, and neutrino astrophysics,
including measurements of neutrinos from a core-collapse supernova
should such an event occur in our galaxy during the experiment's
lifetime.

%%%%%%%%%%%%%%%%%%%%%%%%%%%%%%%%%%%%%%%%%%%%%%%%%%%%%%%%%%%%%%%
\section{The LBNF/DUNE Conceptual Design Report Volumes}

%%%%%%%%%%%%%%%%%%%%%%%%%%%%%%%%%%%
\subsection{A Roadmap of the CDR}

The LBNF/DUNE CDR describes the proposed physics program and 
technical designs at the conceptual design stage.  At this stage, the design is
still undergoing development and the CDR therefore presents a \textit{reference design} 
for each element as well as \textit{alternative designs} that are under consideration.

The CDR is composed of four volumes and is supplemented by several annexes that 
provide details on the physics program and technical designs. The volumes are as follows

\begin{itemize}
\item \volintro{} provides an executive summary of and strategy for the experimental 
program and of the CDR as a whole.
\item \volphys{} outlines the scientific objectives and describes the physics studies that 
the DUNE Collaboration will undertake to address them.
\item \vollbnf{} describes the LBNF Project, which includes design and construction of the 
beamline at Fermilab, the conventional facilities at both Fermilab and SURF, and the cryostat
 and cryogenics infrastructure required for the DUNE far detector.
\item \voldune{} describes the DUNE Project, which includes the design, construction and 
commissioning of the near and far detectors. 
\end{itemize}

More detailed information for each of these volumes is provided in a set of annexes listed on the \href{https://web.fnal.gov/project/LBNF/ReviewsAndAssessments/LBNF-DUNE%20CD-1-Refresh%20Directors%20Review/SitePages/Home.aspx}{review website}. 

%%%%%%%%%%%%%%%%%%%%%%%%%%%%%%%%%%%
%\subsection{About this Volume}  <----- follows in overview chapter file of indiv volume



\fixme{Removed the common intro; put new information here; 5/14}

\section{A Roadmap of the Conceptual Design Report}

The LBNF/DUNE CDR describes the proposed physics program and 
technical designs at the conceptual design stage.  At this stage, the design is
still undergoing development and the CDR therefore presents a \textit{reference design} for each element as well as any 
\textit{alternative designs} that are under consideration.

The CDR is composed of four volumes and is supplemented
by several annexes that provide details on the physics program and technical designs. The volumes are as follows:

\begin{itemize}
\item \volintro provides an executive summary of and strategy for the experimental program and of the CDR as a whole.
\item \volphys outlines the scientifc objectives and describes the physics studies that the DUNE Collaboration will undertake to address them.
\item \vollbnf describes the LBNF Project, which includes design and construction of the beamline at Fermilab, the conventional facilities at both Fermilab and SURF, and the cryostat and cryogenics infrastructure required for the DUNE far detector.
\item \voldune describes the DUNE Project, which includes the design, construction and commissioning of the near and far detectors. 
\end{itemize}
\fixme{check vollbnf title}
Annexes to these volumes are listed at \fixme{provide URL}:



\section{About this Volume}

This introductory volume of the LBNF/DUNE Conceptual Design Report provides an overview of DUNE's science program (Chapter~\ref{v1ch:science}) and the technical designs of the facilities and the detectors 
(Chapter~\ref{v1ch:tech-designs}). It also describes the LBNF and DUNE organization and management structures 
(Chapter~\ref{v1ch:org-mgmt}) and the strategy (Chapter~\ref{v1ch:strategy})  that is being developed to construct, install and commission the conventional and experimental facilities in accordance with the requirements set out by the P5 report of 2014, which, in turn, is in line with the CERN
European Strategy for Particle Physics (ESPP) of 2013. \fixme{cite these documents}.

\section{LBNF/DUNE Schedule}
\fixme{This is probably not in the right place. We will have to rearrange after Andre does his updates}


The schedule for the design and construction work for LBNF and DUNE has two parallel critical paths: one for the Far Site scope at SURF and one the Near Site scope at Fermilab. The initial work is driven by Conventional Facilities design and construction at each site. With limited DOE funding available early in the LBNF Project for this work, the Far Site CF is advanced first. Far Site CF final design starts in fall 2015. Early site preparation is timed to be completed so that excavation is ready to start when the Ross Shaft rehabilitation work completes in late 2017. As each detector pit is excavated and sufficient utilities installed, cryostat and cryogenic system proceeds, followed by detector installation, filling, and commissioning. The Far Detector \#1 is completed in fall 2024. The Far Detector \#4 is completed 3.5 years later in early 2027. 

The Near Site work is delayed with respect to the Far Site due to available DOE funding. The Near Site CF and Beamline essentially slows to almost no effort until design restarts in late 2017. Optimization decisions about the Beamline that affect the CF design will need to be made by late 2018 to be ready for the CF design process. The Embankment is constructed and then allowed to settle for at least 12 months before the majority of the Beamline CF work proceeds. The Beneficial Occupancies of the Beamline CF are staggered to allow Beamline installation to begin as soon as possible. The Beamline installation is completed slightly more than one year after Far Detector \#1. 

The Near Detector CF construction overlaps with the Beamline CF construction but lags due to available funding. The Near Detector assembly begins on the surface before Beneficial Occupancy, after which the detector is installed, completing about the same time as Far Detector \#4. 

The DOE project management process requires approvals at Critical Decision milestones which allows the project to move to the next step. In fall 2015 the Far Site Facilities will seek CD-3a approval for construction of some of the conventional facilities and cryogenic systems at SURF. In spring 2018 LBNF Near Site Facilities will seek CD-3b construction approval for Advanced Site Preparation to build the Embankment. In 2020 LBNF and DUNE will seek to baseline the LBNF/DUNE scope, cost, and schedule as well as construction approval for the balance of the LBNF scope as well as the DUNE scope. The project concludes with CD-4 approval to start operations \fixme{need to determine if one or two CD-4s}. 
