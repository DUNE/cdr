

\chapter{An Experimental Program in Neutrinos, Nucleon Decay and Astroparticle Physics Enabled by the Fermilab Long-Baseline Neutrino Facility}
\label{ch:project-overview}

%<<<<<<< Updated upstream
%%% Intro shared by all subsections

\section{An International Physics Program}

The global neutrino physics community is developing a multi-decade
physics program to measure unknown parameters of the Standard Model of
particle physics and search for new phenomena.  The program will be carried out as an international,
leading-edge, dual-site experiment for neutrino science and proton decay studies, which 
is known as the Deep Underground Neutrino Experiment (DUNE).
The detectors for this experiment will be designed, built, commissioned and operated by the international DUNE Collaboration. The facility required to support this experiment, the Long-Baseline Neutrino Facility (LBNF), is hosted by Fermilab and its design and construction is organized as a DOE/Fermilab project incorporating international partners. Together LBNF and DUNE will comprise the world's highest-intensity neutrino beam at Fermilab, in Batavia, IL, a high-precision near detector on the Fermilab site, a massive liquid argon time-projection chamber (LArTPC) far detector installed deep underground at the Sanford Underground Research Facility (SURF) \SI{1300}{\km} away in Lead, SD, and all of the conventional and technical facilities necessary to support the beamline and detector systems. 


The strategy for executing the experimental program presented in this Conceptual 
Design Report (CDR) has been developed to meet the requirements 
set out in the P5 report~\cite{p5report} and takes into account the recommendations of the European Strategy for Particle Physics~\cite{ESPP-2012}. It adopts a model where U.S. and international funding agencies 
share costs on the DUNE detectors, and CERN and other participants provide in-kind contributions 
to the supporting infrastructure of LBNF. LBNF and DUNE will be tightly coordinated as DUNE collaborators 
design the detectors and infrastructure that will carry out the scientific program.
  
The scope of LBNF is
\begin{itemize}
\item an intense neutrino beam aimed at the far site
\item conventional facilities at both the near and far sites
\item cryogenics infrastructure to support the DUNE
  liquid argon time-projection chamber (LArTPC) detectors at SURF
\end{itemize}

The DUNE detectors include
\begin{itemize}
\item a high-performance neutrino detector and beamline monitoring system
located a few hundred meters downstream of the neutrino source
\item a massive LArTPC neutrino detector located deep underground at the far site
\end{itemize}

With the facilities provided by LBNF and the detectors provided by
DUNE, the DUNE Collaboration proposes to mount a focused attack on the
puzzle of neutrinos with broad sensitivity to neutrino oscillation
parameters in a single experiment.  The focus of the scientific
program is the determination of the neutrino mass hierarchy and the
explicit demonstration of leptonic CP violation, if it exists, by
precisely measuring differences between the oscillations of muon-type
neutrinos and antineutrinos into electron-type neutrinos
and antineutrinos, respectively. Siting the far detector deep underground will
provide exciting additional research opportunities in nucleon decay,
studies utilizing atmospheric neutrinos, and neutrino astrophysics,
including measurements of neutrinos from a core-collapse supernova
should such an event occur in our galaxy during the experiment's
lifetime.

%%%%%%%%%%%%%%%%%%%%%%%%%%%%%%%%%%%%%%%%%%%%%%%%%%%%%%%%%%%%%%%
\section{The LBNF/DUNE Conceptual Design Report Volumes}

%%%%%%%%%%%%%%%%%%%%%%%%%%%%%%%%%%%
\subsection{A Roadmap of the CDR}

The LBNF/DUNE CDR describes the proposed physics program and 
technical designs at the conceptual design stage.  At this stage, the design is
still undergoing development and the CDR therefore presents a \textit{reference design} 
for each element as well as \textit{alternative designs} that are under consideration.

The CDR is composed of four volumes and is supplemented by several annexes that 
provide details on the physics program and technical designs. The volumes are as follows

\begin{itemize}
\item \volintro{} provides an executive summary of and strategy for the experimental 
program and of the CDR as a whole.
\item \volphys{} outlines the scientific objectives and describes the physics studies that 
the DUNE Collaboration will undertake to address them.
\item \vollbnf{} describes the LBNF Project, which includes design and construction of the 
beamline at Fermilab, the conventional facilities at both Fermilab and SURF, and the cryostat
 and cryogenics infrastructure required for the DUNE far detector.
\item \voldune{} describes the DUNE Project, which includes the design, construction and 
commissioning of the near and far detectors. 
\end{itemize}

More detailed information for each of these volumes is provided in a set of annexes listed on the \href{https://web.fnal.gov/project/LBNF/ReviewsAndAssessments/LBNF-DUNE%20CD-1-Refresh%20Directors%20Review/SitePages/Home.aspx}{review website}. 

%%%%%%%%%%%%%%%%%%%%%%%%%%%%%%%%%%%
%\subsection{About this Volume}  <----- follows in overview chapter file of indiv volume


%
%\fixme{Removed the common intro; put new information here; 5/14}
%

\section{A worldwide convergence}

During the last decade, several independent worldwide efforts have attempted to develop paths towards a next generation long-baseline experiment, including in the US with LBNE, in Europe with LBNO and in Japan with HyperK. Such  studies have generally recognised that achieving the necessary conditions to execute this challenging science program in a comprehensive way, requires the previously independent worldwide experimental options to converge on a single facility. They 
also call for the creation of a unique international collaboration reuniting all the expertise and the capabilities to develop a world-class experiment.
In this context, DUNE represents the convergence of the worldwide community around the opportunity provided by the large investment planned by the DOE to support by 2023 a significant expansion of the underground infrastructure at the Sanford Underground Research Facility in South Dakota, 1,300 km from Fermilab and to create a megawatt neutrino beam facility at Fermilab by 2026.  The PIP-II accelerator upgrade at Fermilab will provide 1.2 MW of power to drive the new neutrino beam line at Fermilab, with a plan that could further upgrade the 
accelerator complex to enable it to provide up to 2.4 MW of beam power by 2030.  

This document therefore represents 
a Conceptual Design Report by a global neutrino community to pursue 
LBNF/DUNE (Deep Underground Neutrino Experiment at the FNAL Long Base Neutrino Facility),
a groundbreaking science experiment for long-baseline neutrino oscillation studies as well as neutrino astrophysics and nucleon decay searches, with a very large modular liquid argon TPC (LAr-TPC) detector located deep underground and a high-resolution near detector, coupled to a  multi-MW power wide-band neutrino beam.

The study of the properties of the neutrino has provided many surprises, including the first evidence in particle physics beyond the Standard Model of elementary particles and interactions.   The phenomenon of flavor oscillations, whereby neutrinos change flavor as they propagate through space and time, is now well established. Important conclusions that follow from these discoveries include that neutrinos have mass and that the mass eigenstates are mixtures of the flavor eigenstates.

Speculations on the origin of neutrino masses and mixings are wide-ranging. 
Completing the puzzle will require more precise and detailed experimental information with neutrinos and antineutrinos and with sensitivity to matter effects. With the exception of a few anomalies, the current data can be described in terms of the three-neutrino paradigm, in which the quantum-mechanical mixing of three mass eigenstates produces the three known neutrino-flavor states.  The mixing is described by the Pontecorvo-Maki-Nakagawa-Sakata (PMNS) matrix, a parameterization that includes a CP-violating phase. 
The primary science goals of DUNE are to carry out a comprehensive investigation of neutrino oscillations to test CP violation in the lepton sector, determine the ordering of the neutrino masses, and to test the three-neutrino paradigm.
DUNE will be able to measure the neutrino transitions with the necessary precision to determine the CP-violating phase and determine the neutrino mass hierarchy by measuring independently the  propagation of neutrinos and antineutrinos through matter.

At the same time, the DUNE far detector, consisting of four liquid argon TPCs located deep underground with masses an order of magnitude larger than ever realised, will have unique capabilities for addressing diverse physics topics. 
DUNE will exploit the large, high-resolution, underground far detector for non-accelerator physics topics including atmospheric neutrino measurements, searches for nucleon decay, and measurement of astrophysical neutrinos especially those from a core-collapse supernova.
Interactions produced by such supernova and atmospheric neutrinos will be detected and measured in ways that will bring new insight on these natural phenomena. 
An attractive conjecture is that neutrino masses are related to a new ultra-high-energy scale that may be associated with the unification of matter and forces. Such theories are able to describe the absence of antimatter in the Universe in terms of the properties of ultra-heavy particles as well as offering a description of cosmological inflation in terms of the phase transitions associated with the breaking of symmetries at the ultra-high-energy scale. DUNE will be able to probe the very high-energy scales with its capability to detect and study rare events such as nucleon decays in an unbiased way that was not possible before. 

At the same time, the construction of LBNF and the DUNE will enable a high-priority ancillary science program, such as 
the very precise measurements of neutrino interactions and cross sections, the studies of nuclear effects in such interactions, measurements of the structure of nucleons, as wel as precise tests of the electroweak theory. The precise knowledge of such processes will be mandatory to achieve the ultimate sensitivity of the long-baseline neutrino oscillation studies.

Finally, opportunities could be enabled by further developments of far detector technology during course of DUNE construction such as the detection of
solar neutrinos or the measurement of diffuse supernova neutrino flux.

%With the availability of space for expansion and improved access at the Sanford laboratory, this international collaboration will develop the necessary framework to design, build and operate a world-class deep-underground neutrino and nucleon decay observatory. Fermilab will act as the host laboratory. 
%This plan is aligned with the European Strategy Report and the US HEPAP Particle Physics Project Prioritization Panel (P5) report.

\section{Global LBNF/DUNE Strategy}

The project strategy presented in this CDR has been developed to meet the requirements 
set out in the P5 report and %taking 
takes into account the recommendations of the European 
ESPP strategy.

With the availability of space for expansion and improved access at the Sanford laboratory, 
the DUNE international collaboration proposes to construct a deep-underground neutrino observatory based on four independent 10-kt liquid-argon (LAr) time-projection chamber (TPC) at the Sanford Underground Research Facility.  
The goal is the deployment of two 10-kt fiducial mass detectors on a relatively short timeframe, followed by future expansion to the full detector size as soon as possible. 

Several designs for LAr TPCs are under development by different groups worldwide, involving both single- and dual-phase readout technology.
The DUNE international Collaboration has the necessary expertise, technical knowledge, and critical mass to design and implement this exciting discovery experiment. 

The Long-Baseline Neutrino Facility (LBNF) provides:
\begin{itemize}
\item excavation of four underground caverns, each capable of hosting a cryostat 
with a 10-kt fiducial mass LArTPC, is planned to be completed 
by 20yy %2022 
under a single contract

\item surface, shaft, and underground infrastructure to support 
the outfitting of the caverns with four free-standing steel-supported cryostats 
and the required cryogenic systems. The first two cryostats will be available for filling by
20yy, %2024
allowing for a rapid deployment of the first two 10-kt far detector modules. 
The intention is to install third and fourth cryostats as rapidly as funding will 
allow.

\item  the conventional facilities for the near detector systems at Fermilab 

\item  the conventional and technical facilities for a 1.2-MW neutrino beam utilizing the PIP-II upgrade of the Fermilab accelerator 
complex, operational at the latest by 20yy %2026
and upgradable to 2.4\,MW with the proposed 
PIP-III upgrade
\end{itemize}

The Deep Underground Neutrino Experiment (DUNE) will provide:
\begin{itemize}

\item four LArTPCs, each with a fiducial mass of at least 10\,kt. The division of 
the far detector into four equal-mass detectors provides the project flexibility 
in the installation and funding (DOE vs. non-DOE) in the case of new resources being 
identified; this division also mitigates risks and allows for an early and graded science return.

\item the near detector systems, consisting of a highly-capable neutrino detector 
and the muon monitoring system necessary to reach the precision requirements needed to fully 
exploit the statistical power of the very massive far detector coupled to the powerful MW-class 
neutrino beam

\end{itemize}

Based on the reference design described below and in Volumes 2, 3 and 4 of the LBNF/DUNE 
CDR, the resource-loaded schedule will see the first two 10\,kt far detector modules 
operational by 20yy, %2025
with first beam shortly afterward. At that time the cavern 
space for all four 10-kt far detector modules will be available, allowing for 
an accelerated installation schedule, if sufficient resources for
the experiment can be established on an accelerated timescale.  

\vspace{6pt}
The project strategy described above meets these goals, reaching an exposure of 
\num{120}\ktMWyr{} by 2032, and potentially earlier if additional resources are identified. 
The P5 recommendation of sensitivity to CP violation of 3$\sigma$ for 75\% of $\delta_\text{CP}$
values can be reached with an exposure of \num{850}\ktMWyr{} with an optimized beam.

\section{The international organisation and responsibilities}
The successful model used by CERN for managing the construction and exploitation of the LHC and its experiments was used as a starting point for the joint management of LBNF and the experimental program.  Fermilab, as the host laboratory, has the responsibility for the facilities, and for oversight of the experiment and its operations.  Mechanisms to ensure input from and coordination among all of the funding agencies supporting collaboration, modelled on the CERN Resource Review Board, has been adopted. The same (or similar) structure is employed to coordinate among funding agencies supporting the LBNF construction and operation.  

The LBNF/DUNE project will be organised as two distinct entities. The LBNF part is funded primarily
by the the U.S. DOE acting on behalf of the hosting country. The DUNE part is organised
as an international collaboration, adopting a model where the DOE and international funding agencies  share costs on the DUNE detectors. CERN provides in-kind contributions to the supporting cryogenic infrastructure needed for the far detector.

The DUNE Collaboration is responsible for:
\begin{itemize}
\item The definition of the scientific goals and corresponding scientific and technical requirements on the detector systems and neutrino beam line.
\item The design, construction, commissioning and operation of the detector systems.
\item The scientific research program conducted with the detectors and LBNF neutrino beam.
\end{itemize}

Fermilab will provide the high-intensity proton source that will drive the long-baseline neutrino beam, utilizing the existing Main Injector with upgraded injectors (PIP-II).  PIP-II is also being planned with significant international collaboration.  Fermilab, working with and with the support of international partners, is responsible for the Long-Baseline Neutrino Facility, including:
\begin{itemize}
\item Design, construction and operation of the LBNF beamline, including the primary proton beam beamline and the neutrino beamline including target, focusing structure (horns), decay pipe, absorber, and corresponding beam instrumentation. 
\item Design, construction and operation of the conventional facilities and technical infrastructure on the Fermilab site required for the Near Detector complex.
\item Design, construction and operation of the conventional facilities and technical infrastructure within the Sanford site, including cryostat and cryogenic systems, required for the Far Detector.
\item Fermilab, as the host lab, will pay for the operating costs of the facility.
\end{itemize}


\section{A vigorous schedule}
\fixme{please suggest a better word than vigorous??..Andr\'e}

The schedule for the design and construction work for LBNF and DUNE has two parallel critical paths: one for the Far Site scope at SURF and one the Near Site scope at Fermilab. The initial work is driven by Conventional Facilities design and construction at each site. 

Within the anticipated DOE funding profile, in particular during the initial phase of the Project, the Far Site CF is advanced first. Far Site CF final design starts in fall 2015. Early site preparation is timed to be completed so that excavation is ready to start when the Ross Shaft rehabilitation work completes in late 2017. As each detector pit is excavated and sufficient utilities installed, cryostat and cryogenic system proceeds, followed by detector installation, filling, and commissioning. The Far Detector \#1 is completed in fall 2024. The Far Detector \#4 is completed 3.5 years later in early 2027. 

The Near Site work is delayed with respect to the Far Site due to the anticipated funding profile. The Near Site CF and Beamline essentially slows to almost no effort until design restarts in late 2017. Optimization decisions about the Beamline that affect the CF design will need to be made by late 2018 to be ready for the CF design process. The Embankment is constructed and then allowed to settle for at least 12 months before the majority of the Beamline CF work proceeds. The Beneficial Occupancies of the Beamline CF are staggered to allow Beamline installation to begin as soon as possible. With this timescale, the science with the Far Detectors start without the beam, focusing on non-accelerator based science, since
the Beamline installation is completed slightly more than one year after Far Detector \#1. 

The Near Detector CF construction overlaps with the Beamline CF construction but lags due to available funding. The Near Detector assembly begins on the surface before Beneficial Occupancy, after which the detector is installed, completing about the same time as Far Detector \#4. 

The DOE project management process requires approvals at Critical Decision milestones which allows the project to move to the next step. In fall 2015 the Far Site Facilities will seek CD-3a approval for construction of some of the conventional facilities and cryogenic systems at SURF. In spring 2018 LBNF Near Site Facilities will seek CD-3b construction approval for Advanced Site Preparation to build the Embankment. In 2020 LBNF and DUNE will seek to baseline the LBNF/DUNE scope, cost, and schedule as well as construction approval for the balance of the LBNF scope as well as the DUNE scope. The project concludes with CD-4 approval to start operations.
\fixme{need to determine if one or two CD-4s}. 


\section{A Roadmap of the Conceptual Design Report}

The LBNF/DUNE CDR describes the proposed physics program and 
technical designs at the conceptual design stage.  At this stage, the design is
still undergoing development and the CDR therefore presents a \textit{reference design} for each element as well as any 
\textit{alternative designs} that are under consideration.

The CDR is composed of four volumes and is supplemented
by several annexes that provide details on the physics program and technical designs. The volumes are as follows:

\begin{itemize}
\item \volintro provides an executive summary of and strategy for the experimental program and of the CDR as a whole.
\item \volphys outlines the scientifc objectives and describes the physics studies that the DUNE Collaboration will undertake to address them.
\item \vollbnf describes the LBNF Project, which includes design and construction of the beamline at Fermilab, the conventional facilities at both Fermilab and SURF, and the cryostat and cryogenics infrastructure required for the DUNE far detector.
\item \voldune describes the DUNE Project, which includes the design, construction and commissioning of the near and far detectors. 
\end{itemize}
\fixme{check vollbnf title}
Annexes to these volumes are listed at \fixme{provide URL}:



\section{About this Volume}

This introductory volume of the LBNF/DUNE Conceptual Design Report provides an overview of DUNE's science program (Chapter~\ref{v1ch:science}) and the technical designs of the facilities and the detectors 
(Chapter~\ref{v1ch:tech-designs}). It also describes the LBNF and DUNE organization and management structures 
(Chapter~\ref{v1ch:org-mgmt}) and the strategy (Chapter~\ref{v1ch:strategy})  that is being developed to construct, install and commission the conventional and experimental facilities in accordance with the requirements set out by the P5 report of 2014, which, in turn, is in line with the CERN
European Strategy for Particle Physics (ESPP) of 2013. \fixme{cite these documents}.

