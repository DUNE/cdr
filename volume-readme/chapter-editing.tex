\chapter{Reviewing and Editing}


\section{Markup}

While reviewing, it is possible to mark up the document with simple
\LaTeX{} macros as provided by the ``todonotes'' class.
This class has many features but a few are more important.

Note: if you do not see the examples in this section your copy may
have been built with the ``\texttt{final}'' option turned on.


\subsection{Inline Fixme}

You may prefer to place an inline note to mark up the text.
This can be accomplished with a \fixme{Think of something critical
  about this sentence} \verb|\fixme{...}| command.

%%%%%%%%%%%%%%%%%%%%%%%%%%%%%%%%%%%%%%%%%%%%%%%%%%%%%%%%%%%%%%%%%%%%
\subsection{Margin notes}

You can add notes to the \todo{run spell checker}margine easily which
are associated with some text using:

\begin{verbatim}
\todo{run spell checker.}
\end{verbatim}

%%%%%%%%%%%%%%%%%%%%%%%%%%%%%%%%%%%%%%%%%%%%%%%%%%%%%%%%%%%%%%%%%%%%
\subsection{Highlighting}

If you wish to make add a comment on some section of text you may
\hlfix{highlight it}{This shows how.}with a comment.


\begin{verbatim}
\hlfix{highlight it}{This shows how.}
\end{verbatim}


