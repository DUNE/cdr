\chapter{Technical}

This chapter describes some of the more technical parts of the CDR.

%%%%%%%%%%%%%%%%%%%%%%%%%%%%%%%%%%%%%%%%%%%%%%%%%%%%%%%%%%%%%%%%%%%%
\section{The \LaTeX{} CDR class}

All of the \LaTeX{} configuration for the document that pertains to the general CDR style and not to the actual content is in the \texttt{cdr.cls} file.  The class takes some options that control high-level style:

\begin{description}
\item[\texttt{draft}] produce markup to assist in editing (line numbers, draft water mark, label tags)
\item[\texttt{print}] print quality (remove editing markup)
\end{description}

%%%%%%%%%%%%%%%%%%%%%%%%%%%%%%%%%%%%%%%%%%%%%%%%%%%%%%%%%%%%%%%%%%%%
\section{Topic-specific \LaTeX{} files}

There are three files which are specific to the topic of the CDR but generic to each volume that should be provided:

\begin{description}
\item[\texttt{common/preamble.tex}] placed before the document begins and should provide all macros to define common terms or units.
\item[\texttt{common/init.tex}] placed immediately after the document starts and should provide things like title page, author list, toc/lof/lot, etc.
\item[\texttt{common/final.tex}] placed immediately before the document ends and should provide the bibliography setup or any other common trailing matter.
\end{description}

%%%%%%%%%%%%%%%%%%%%%%%%%%%%%%%%%%%%%%%%%%%%%%%%%%%%%%%%%%%%%%%%%%%%
\section{Volume main file}

To start a new volume, copy the \texttt{volume-readme.tex} to a new
name and edit as directed by the comments.  

\begin{enumerate}
\item Set the graphics path
\item Redefine the volumes sub-title
\item Input each chapter file.
\end{enumerate}

