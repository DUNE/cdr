\chapter{Implementation Strategy}
\label{ch:detectors-strategy}

\section{Overview}

Recommendation 12 of the Report of the Particle Physics Prioritization
Panel (P5) states that for a Long-Baseline Neutrino Oscillation
Experiment to proceed ``The minimum requirements to proceed are the
identified capability to reach an exposure of
120~kt$\times$MW$\times$yr by the 2035 timeframe, the far detector
situated underground with cavern space for expansion to at least 40~kt
LAr fiducial volume, and 1.2~MW beam power upgradable to
multi-megawatt power. The experiment should have the demonstrated
capability to search for supernova bursts and for proton decay,
providing a significant improvement in discovery sensitivity over
current searches for the proton lifetime''.  The strategy presented
here meets these criteria.  The P5 recommendation is in line with CERN
European Strategy for Particle Physics (ESPP) of 2013, which
classified the long baseline neutrino program as one of the four
scientific objectives with required international infrastructure.

\section{Strategy for implementing the DUNE far detectors}

The LBNF project will provide four cryostats at 4850L of the Sanford
Underground Research Facility (SURF) in which the DUNE collaboration
will deploy four 10~kt (fiducial) mass far detector (FD) Liquid Argon Time
Projection Chambers (LArTPC).  DUNE contemplates two options for the
read out of the ionization signals: single-phase readout, where the
ionization is detected using wire planes in the liquid argon volume;
and dual-phase readout, where the ionization signals are amplified and
detected in gaseous argon above the liquid surface.  An active
development program for both technologies is being pursued in the
context of the Fermilab SBN program and the CERN neutrino platform.

The viability of the LArTPC technology has been proven by the ICARUS
experiment with single-phase wire plane LArTPC readout, where data was
successfully accumulated over a period of three years.  An
extrapolation of the observed performance and implementation of
improvements in the design (e.g. cold electronics) will allow the
single-phase approach to meet DUNE requirements. For these reasons,
the APA/CPA single-phase wire plane LArTPC (see
Chapter~\ref{ch:detectors-fd-ref}) is adopted as the far detector
\textit{reference design} and the first 10~kt LArTPC will be based on
this design.

The reference design is already relatively advanced for a
conceptual design report. Modifications of the reference design will
be approved by the DUNE technical board. A preliminary design
review will take place as early as possible, utilizing the experience
from the DUNE 35-t prototype; the design review will define the
baseline design that will form the basis of the TDR (CD-2).  Once
defined, changes to the baseline will fall under a formal
change-control process. An engineering prototype consisting of
six full-sized drift cells will be validated at the CERN neutrino
platform\footnote{The proposal for the DUNE single-phase prototype
  will be presented to the CERN SPSC in June 2015.}.  This engineering
prototype at CERN is a central part of the risk mitigation strategy
for the first 10~kt FD module. Following experience at the CERN
neutrino platform, the DUNE technical coordinator will organize a
final design review. The CERN prototype provides the opportunity for
production sites to validate manufacturing procedures ahead of
large-scale production for the far detector. Three major operational
milestones are defined for this single-phase prototype: 1) engineering
validation --- successful cool-down; 2) operational validation ---
successful TPC readout with cosmic-ray muons; and 3) physics
validation with test beam data. Reaching milestone 2, will allow the
retirement of a number of technical risks for the construction of the
first 10~kt FD module.

In parallel with preparation for construction of the first 10~kt far
detector module, the DUNE collaboration recognizes the potential of
the dual-phase technology and strongly endorses the development
program at the CERN neutrino platform (WA105 experiment), which
includes the operation of the 20-ton prototype and the
6$\times$6$\times$6~m$^3$ demonstrator which has already been
approved. Many DUNE Collaborators are participants in the WA105
experiment. A concept for the dual-phase implementation of a FD module
is presented in detail as an \textit{alternative design} in
Chapter~\ref{ch:detectors-fd-alt}. This alternative design, if
demonstrated, could form the basis of the second or subsequent 10~kt
far detector modules, to achieve improved detector performances in a
cost-effective way.

The DUNE program at the CERN neutrino platform will be coordinated by
a single L2 manager. Common technical solutions will be adopted
wherever possible.  The charged-particle test-beam data will provide
essential calibration samples for both technologies and will enable a
direct comparison of the relative physics benefits.

For the purposes of cost and schedule, the reference design for the
first FD module is adopted as the reference design for the subsequent
three FD modules. However, the experience with the first 10~kt FD
module and the development activities at the CERN platform are likely
to lead to the evolution of the TPC technology, both in terms of
refinements to single-phase design and the validation of the operation
of the dual-phase design.  The technology choice for the second and
subsequent LArTPCs will be based on risk, cost (including the
potential benefits of additional non-DOE funding) and physics
performance (as established in the CERN charged-particle test beam).

This strategy allows flexibility with respect to international
contributions and provides the possibility of attracting interest and
resources from a broader community with space for flexibility to
respond to the funding constraints from different sources.

\section{Strategy for implementing the DUNE near detector(s)}

The LBNF project will provide the civil facilities for the DUNE near
detector systems (muon monitors and near neutrino detector). The
primary scientific motivation for the DUNE near detector system is to
constrain the beam spectrum for the long-baseline neutrino oscillation
studies. It also provides large data samples for precision studies of
neutrino-argon interactions. The near detector, which is exposed to an
intense flux of neutrinos, also provides an opportunity for a wealth
of fundamental neutrino interaction measurements, which are an
important part of the secondary scientific goals of the DUNE
collaboration. Within the former LBNE collaboration the neutrino near
detector (NND) design was the NOMAD-inspired fine-grained tracker
(FGT), which was built through a strong collaboration of U.S. and
Indian institutes. DUNE adopts the FGT concept as the
\textit{reference design}:
\begin{itemize}
%\item Recognition of the central importance of the reference design for NND;
\item The primary design consideration of the DUNE neutrino near
  detector is the ability to adequately constrain the systematic
  errors in the DUNE long baseline oscillation analysis;
\item The secondary design consideration for the DUNE NND is the
  self-contained non-oscillation neutrino physics program;
\item It is recognized that a detailed cost-benefit study of potential
  ND options has yet to take place and such a study is of high
  priority to the DUNE project.
\end{itemize}
The cost and resource-loaded schedule are based on this design (see
Chapter~\ref{ch:detectors-nd-ref}).

The contribution of Indian institutions to the design and construction
of the DUNE FGT neutrino near detector is a vital part of the strategy
for the construction of the experiment. The reference design will
provide a rich, self-contained physics program. From the perspective of
an ultimate long baseline oscillation program, there may be benefits
of augmenting the FGT with a relatively small LArTPC that would allow
for a direct comparison with the far detector or a high-pressure
gaseous Argon TPC. At this stage, the benefits of such options have
not been studied and alternative designs for the NND are not presented
in the CDR and will be the subject of detailed studies in the coming
months.

A full end-to-end study of the impact of the FGT NND design on the
oscillation systematics has yet to be performed. Many of the elements
of such a study are in development, for example the Monte Carlo
simulation of the FGT and the adaption of the T2K framework for
implementing ND measurements as constraints in the propagation of
systematic uncertainties to the FD.  After the CD-1-R review, the DUNE
collaboration will initiate a detailed study of the optimization of
the NND system. To this end a task force will be set up with the
charge of:
\begin{itemize}
\item Delivering the simulation of the reference design of the NND and
  possible alternatives;
\item Undertaking an end-to-end study to provide a quantitative
  understanding of the power of the NND designs to constrain the
  systematic uncertainties on the long baseline oscillation measurements;
\item Quantifying the benefits of augmenting the reference design with
  a LArTPC or high-pressure gaseous argon TPC.
\end{itemize}
High priority will be placed on this work and the intention is to
engage a broad cross section of the collaboration in this process. The
task force will be charged to deliver a report by July 2016. Based on
the report of this task force and input from the DUNE Technical Board,
the DUNE executive board will refine the DUNE strategy for the Near
Detector.
