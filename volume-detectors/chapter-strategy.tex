\chapter{Implementation Strategy}
\label{ch:detectors-strategy}

\section{Overview}

Recommendation 12 of the Report of the Particle Physics Prioritization Panel (P5) 
states that for a Long-Baseline Neutrino Oscillation Experiment to proceed ``The 
minimum requirements to proceed are the identified capability to reach an exposure 
of 120 kt x MW x yr by the 2035 timeframe, the far detector situated underground 
with cavern space for expansion to a least 40 kt LAr fiducial volume, and 1.2 MW 
beam power upgradable to multi-megawatt power. The experiment should have the demonstrated 
capability to search for supernova bursts and for proton decay, providing a significant 
improvement in discovery sensitivity over current searches for the proton lifetime''. 
The strategy presented here meets these criteria. 
The P5 recommendations are also in line with CERN European Strategy for Particle 
Physics (ESPP) of 2013, which classified the long baseline neutrino program as 
one of the four scientific objectives with required international infrastructure.

\section{A Strategy for Implementing the DUNE Far detectors}

The LBNF project will provide four separate cryostats on the 4850L at the 
Sanford Underground Research Facility (SURF).  
The DUNE collaboration aims to deploy four 10 kt (fiducial) mass FD modules based 
on the Liquid Argon Time Projection Chamber (LAr-TPC) technology. The viability 
of the LAr-TPC technology has been proven by the ICARUS experiment. Neutrino 
interactions in liquid argon produce ionization and scintillation signals. While 
the basic detection method is the same, DUNE contemplates two options for the read 
out of the ionization signals: single-phase readout, where the ionization is detected 
using wire planes in the liquid argon volume; and the dual-phase approach, where 
the ionization signals are amplified and detected in gaseous argon above the liquid 
surface. The dual-phase approach, if demonstrated, would allow for a 3mm readout 
pitch, a lower detection energy threshold, and better pattern reconstruction of 
the events. The DUNE single-phase read-out design is being currently being validated 
in the 35 t detector at Fermilab. A 20 t dual-phase read out prototype is being 
constructed at CERN and will operate in 2016. An active development program for 
both technologies is being pursued in the context of the Fermilab SBN program and 
the CERN neutrino platform. At this moment, the development of the dual-phase technology 
is being fully funded from outside the DOE project, although interest from DOE 
groups has been expressed in the context of the CERN Neutrino Platform. A flexible 
approach to the DUNE Far Detectors designs offers the potential to bring additional 
interest and resources into the experimental collaboration. The vision for the 
FD is consolidated in line with the requirements set by the CD-milestones.

\begin{itemize}
\item The lowest-risk design for the first 10 kt FD module satisfying the requirements 
will be adopted, allowing for installation at SURF to commence in 2021/2022. Installation 
of the second 10 kt module should commence before 2023.

\item  Recognition that the LAr TPC technology will continue to evolve with: i) the 
large-scale prototypes at the CERN Neutrino Platform and the experience from the 
Fermilab SBN program, and ii) the experience gained during the construction and 
commissioning of the first 10 kt FD module. It is assumed that all four FD detectors 
will be similar but not necessarily totally identical.

\item  In order to start the FD installation on the timescale of 2021/2022, the first 
10 kt module will be based on the APA/CPA design.
There will be a clear and transparent decision process for the design 
of the second and subsequent far detector modules, allowing for evolution of the 
LAr TPC technology to be implemented in the FD detectors. The decision will be 
based on physics performance, technical and schedule risks, costs, and funding 
opportunities.

\item The DUNE collaboration will instrument the second cryostat as soon as possible.

\item A comprehensive list of synergies between the reference and alternative design 
has been identified and summarized in the CDR Vol. 4. Common solutions for DAQ, 
electronics, HV feed-throughs, etc., will pursued and implemented, independent 
of the details of the TPC design.

\end{itemize}

Wire plane LAr-TPC readout has already been demonstrated in by the ICARUS T600 
experiment, where data was successfully accumulated over a period of three years. 
An extrapolation of the observed performance and the implementation of improvements 
in the design (such as e.g. immersed cold electronics) will allow the single-phase 
approach to meet detector requirements. In order to start the FD installation on 
by 2022, the first 10 kt module will be based on the single-phase APA/CPA design, 
subject to risks identified in the register. Based on previous experience and the 
future development path in the Fermilab SBN program and at the CERN neutrino platform, 
this choice represents the lowest risk option for the first 10 kt FD Module by 
2021/2022. For these reasons, the APA/CPA single-phase wire plane LAr-TPC readout 
concept, described in Volume 4 of the DUNE CDR, is the \textit{reference design} 
for the far detector. The 10 kt TPC active volume is 12 m high, 14.5 m wide and 
58 m long. The active volume is instrumented with anode plane assemblies (APAs), 
which are 6 m high and 2.3 m in width. Two APAs are stacked vertically to instrument 
the 12 m height of the active volume. Three stacks of APAs span the width of the 
detector. With 25 APA stacks, placed edge-to-edge, spanning 58 m active length 
of the detector. Cathode plane assemblies (CPAs) at -180 kV are located in between 
the APAs. The CPAs are formed from 3 m high by 2.5 wide cathode planes stacked 
in fours to reach the 12 m height of the detector.  The width of the detector is 
thus spanned by four 3.6 m long drift regions (APA:CPA:APA:CPA:APA). The 10 kt 
far detector consists of 150 APAs and 200 CPAs. Ionization electrons drift a maximum 
distance of 3.6 m in the electric field of 500 V/cm. The highly modular nature 
of modular design enables manufacturing to be distributed across a number of sites.

The \textit{reference design} is already relatively advanced for a conceptual design 
report. At this stage modifications of the reference design will need to be approved 
by the DUNE technical board. A preliminary design review will take place as early 
as possible, utilizing the experience from the DUNE 35-ton prototype; the design 
review will define the baseline design that will form the basis of the TDR (CD-2). 
Once defined, the changes to the baseline will fall under a formal change-control 
process. The validation of six full-sized drift cells of the TDR engineering design 
will be validated at the CERN neutrino platform in 2018 (pending approval by CERN). 
This single-phase engineering prototype at CERN is a central part of the risk mitigation 
strategy for the first 10 kt FD module and is part of the DOE-funded DUNE project. 
Based on the experience at the CERN neutrino platform, a final design review will 
take place towards the end of 2018 and construction of the readout planes will 
commence in 2019, ready for first installation in 2021/2022. The design reviews 
will be organized by the DUNE technical coordinator.

In parallel with preparation for construction of the first 10 kt far detector module, 
the DUNE collaboration recognizes the potential of the dual-phase technology and 
strongly endorses the already approved development program at the CERN neutrino 
platform (WA105 experiment), which includes the operation of the 20-ton prototype 
in 2016 and the 6 x 6 x 6 m\textsuperscript{3} demonstrator in 2018. Participation 
to the WA105 experiment is open to all DUNE Collaborators. A concept for the dual-phase 
implementation of a FD module is presented in detail as an \textit{alternative 
design} in Chapter~\ref{ch:detectors-fd-alt}. This alternative design, if demonstrated, 
could form the basis of the second or subsequent 10 kt far detector modules, in 
particular to achieve improved detector performances in a cost-effective way. 

WA105 has signed an MoU with the CERN Neutrino Platform to provide a large \textasciitilde{} 
8 x 8 x 8 m\textsuperscript{3} cryostat by October 2016 in the new EHN1 extension, 
and it is foreseen a second large cryostat to house the single phase LAr-TPC will 
be provided on a similar timescale. Both will be exposed to charged-particle test 
beam in spanning a range of particle types and energies.   

The DUNE collaboration will instrument one of these cryostats with an arrangement 
of six APAs and six CPAs, in a APA:CPA:APA configuration providing an engineering 
test of the full-size drift volume. The first 10 kt far detector module will contain 
150 APAs and 200 CPAs. These will be produced in two or more sites with the cost 
shared between the DOE project and international partners. The CERN prototype thus 
provides the opportunity for the production sites to validate the manufacturing 
procedure ahead of large-scale production for the far detector. Three major operational 
milestones are defined for this single-phase prototype: 1) engineering validation 
--- successful cool-down; 2) operational validation --- successful TPC readout with 
cosmic-ray muons; and 3) physics validation with test beam data. Reaching milestone 
2, scheduled for early 2018, will allow the retirement of a number of technical 
risks for the construction of the first 10 kt FD module. The proposal for the DUNE 
single-phase prototype will be presented to the CERN SPSC in June 2015.

The WA105 experiment approved by the CERN Research Board in 2014 and supported 
by the CERN Neutrino Platform has a funded plan to construct and operate a large-scale 
demonstrator utilizing the dual-phase readout in the test beam by October 2017. 
Successful operation and demonstration of long-term stability of the WA105 demonstrator 
will establish this technological solution as an option for the second or subsequent 
far detector modules. The DUNE double phase design is based on independent 3x3m$^2$
charge readout planes (CRP) placed at the gas-liquid interface. Each module provides 
two perpendicular ``collection'' views with 3mm readout pitch. A 10 kt FD module 
would be composed of 80 CRPs hanging from the top of the cryostat, decoupled from 
the field cage and cathode. The WA105 demonstrator will contain four 3x3m$^2$ 
CRPs of the DUNE type giving the opportunity to validate the manufacturing procedure 
ahead of large-scale productions. WA105 is presently constructing a 3x1m$^2$ 
CRP to be operated in 2016. The same operational milestones (engineering, operational, 
physics) are defined as for the single phase.

The DUNE program at the CERN neutrino platform will be coordinated by a single 
L2 manager. Common technical solutions will be adopted wherever possible for the 
DUNE single-phase engineering prototype and the dual-phase (WA105) demonstrator. 
The charged-particle test-beam data will provide essential calibration samples 
for both technologies and will enable a direct comparison of the relative physics 
benefits of the single-phase and dual-phase TPC readout. 


For the purposes of cost and schedule, the reference design for the first FD module 
is adopted as the reference design for the subsequent three FD modules. However, 
the experience with the first 10 kt FD module and the development activities at 
the CERN platform are likely to lead to the evolution of the TPC technology, both 
in terms of refinements to single-phase design and the validation of the operation 
of the dual-phase design. The DUNE technical board will instigate a formal review 
of the design for the second FD TPC module in 2020. The technology choice for the 
second TPC will be based on risk, cost (including the potential benefits of additional 
non-DOE funding) and physics performance (as established in the CERN charged-particle 
test beam). After the decision, the design of the second TPC will come under formal 
change control. This process will be repeated for the third and fourth FD module 
in 2022.

This strategy allows flexibility with respect to international contributions with 
a path where the DUNE collaboration decides that the second and third TPCs respectively 
adopt evolving approaches. This option provides the possibility of attracting interest 
and resources from a broader community, and space for flexibility to respond to 
the funding constraints from different sources. 

\section{A Strategy for Implementing the DUNE Near Detector(s)}

The LBNF project will provide the civil facilities for the DUNE near detector systems 
(muon monitors and near neutrino detector). The primary scientific motivation for 
the DUNE near detector system is to constrain the beam spectrum for the long-baseline 
neutrino oscillation studies. It also provides high-statistics for precision studies 
of neutrino-argon interactions. The near detector, which is exposed to an intense 
flux of neutrinos, also provides an opportunity for a wealth of fundamental neutrino 
interaction measurements, which are an important part of the secondary scientific 
goals of the DUNE collaboration. Within the former LBNE collaboration the neutrino 
near detector (NND) design was the NOMAD-inspired fine-grained tracker (FGT), which 
was built through a strong collaboration of U.S. and Indian institutes.

\begin{itemize}
\item Recognition of the central importance of the reference design for NND;

\item  The primary design consideration of the DUNE neutrino near detector is the 
ability to adequately constrain the systematic errors in the DUNE LBL oscillation 
analysis; 
\item The secondary design consideration for the DUNE NND is the self-contained non-oscillation 
neutrino physics program;

\item It is recognized that a detailed cost-benefit study of potential ND options 
has yet to take place and such a study is of high priority to the DUNE project.
\end{itemize}

The NOMAD-inspired fine-grained tracker (FGT) concept is the \textit{reference 
design} for CD-1 review. The cost and resource-loaded schedule for CD-1 review 
will be based on this design, as will the near site conventional facilities. The 
Fine-Grained Tracker consists of:  central straw-tube tracker (STT) of volume 3.5 
m x 3.5 m x 6.4 m; a lead-scintillator sandwich sampling electromagnetic calorimeter 
(ECAL); a large-bore warm dipole magnetic, with inner dimensions of 4.5 m x 4.5 
m x 8.0 m, surrounding the STT and ECAL and providing a magnetic field of 0.4 T; 
and RPC-based muon detectors (MuIDs) located in the steel of the magnet, as well 
as upstream and downstream of the STT. The reference design is presented in Chapter 
7 of Vol. 4 of the DUNE CDR. 

For ten years of operation in the LBNF 1.2 MW beam (5 years neutrinos + 5 years 
antineutrinos), the near detector will record a sample of more than 100 million 
neutrino interactions and 50 million antineutrino interactions. These vast samples 
of neutrino interactions shall provide the necessary strong constraints on the 
systematic uncertainties for the LBL oscillation physics - the justification is 
given in Section 6.1.1 of Volume 2 of the DUNE CDR. The large samples of neutrino 
interactions will also provide a wealth of physics opportunities beyond the main 
scientific goal of long-baseline oscillations for DUNE collaborators, including 
numerous topics for PhD theses.  

The contribution of Indian institutions to the design and construction of the DUNE 
FGT neutrino near detector is a vital part of the strategy for the construction 
of the experiment. The reference design will provide a rich self-contained physics 
program. From the perspective of an ultimate LBL oscillation program, there may 
be benefits of augmenting the FGT with, for example, a relatively small LAr-TPC 
in front of the FGT that would allow for a direct comparison with the far detector. 
A second line of study would be to augment the straw-tube tracker of the NND with 
a High-Pressure Gaseous Argon TPC. At this stage, the benefits of such options 
have not been studied and alternative designs for the NND are not presented in 
the CDR and will be the subject of details studies in the coming months. 

A full end-to-end study of the impact of the FGT NND design on the LBL oscillation 
systematics has yet to be performed. Many of the elements of such a study are in 
development, for example the Monte Carlo simulation of the FGT and the adaption 
of the T2K framework for implementing ND measurements as constraints in the propagation 
of systematic uncertainties to the FD. 

After the CD-1-R review, the DUNE collaboration will initiate a detailed study 
of the optimization of the NND system. To this end a new task force will be set 
up with the charge of:

\begin{itemize}
\item Delivering the simulation of the reference design of the NND and possible alternatives;

\item Undertaking an end-to-end study to provide a quantitative understanding of 
the power of the NND designs to constrain the systematic uncertainties on the LBL 
oscillation measurements;

\item Quantifying the benefits of augmenting the reference design with a LAr-TPC 
or a high-pressure gaseous argon TPC.
\end{itemize}

High priority will be placed on this work and the intention is to engage a broad 
cross section of the collaboration in this process. The task force will be charged 
to deliver a report by July 2016. Based on the report of this task force and input 
from the DUNE Technical Board, the DUNE executive board will refine the DUNE strategy 
for the Near Detector.
