%%%%%%%%%%%%%%%%%%%%%%%%%%%%%%%%
\section{Installation and Commissioning}
\label{sec:detectors-fd-alt-install}

\subsection{Preparatory Work}

Before the installation of the detector can begin, a number of
preparations must be conducted and boundary conditions must be met in
order to enable a safe and efficient process that minimizes the
manpower required for the underground detector construction. A careful
study of the installation sequence and tools and an optimization of
the size and characteristics of the detector elements has been
performed in order to facilitate the installation procedures. Some of the
most important preparation and installation steps are listed in this
section. More detailed explanation and figures are available in
\anxlbnob.

\subsection{Detector Installation Sequence}

This installation procedure has been extensively studied in the
framework of the LAGUNA-LBNO design study. A construction sequence for
the dual-phase DUNE LArTPC detector module installation was defined based on
the use of a slightly modified scaffolding arrangement with respect to
the LBNO design, discussed in \anxlbnob\ (Design: courtesy Technodyne
Ltd., Eastleigh, UK). The proposed construction sequence assumes
completion of the membrane and insulation system installation together
with completion of all tank internal pipework and cable trays.  At
this stage, sections of the scaffolding will be removed and replaced
by the Alimak-Hek or similar climbing access platforms to provide
increased functionality to the installation.  On completion of the
scaffolding revisions and the climbing access platform installation,
the entire tank and scaffolding systems will be cleaned in preparation
for the detector module installation. The proposed far detector module construction
sequence consists of:
\begin{enumerate}
\item{Complete installation of insulation \& membrane, install cable trays from top to bottom for photomultipliers (PMTs) electrical cables.}
\item{Adjust scaffolding platforms, add Alimak-Hek platform and floor protection.}
\item{Air purge top level installation.}
\item{Install hanging columns for detector.}
\item{Install lowest field shaping coil to stabilize columns, and then install first top 15 levels of field shaping coils.}
\item{Thoroughly clean top level assembly.}
\item{Install Charge Readout (CRP) from top scaffolding platform (detailed sequences described in \anxlbnob.}
\item{Thoroughly clean top level assembly.}
\item{Screen off top level to protect Anode (protective screen 1).}
\item{Air purge top level (allow bleed air into middle \& lower levels).}
\item{Continue installing Field shaping coils.}
\item{Complete installation of Field shaping coils.}
\item{Thoroughly clean field shaping coils, remove protective screen (screen 1) to top level \& progressively remove all scaffolding \& Alimak-Hek platforms.}
\item{Screen off field shaping coils (including CRO): protective screen 2.}
\item{Air purge top \& middle levels.}
\item{Construct Cathode from Modules. Cathode to be raised 300mm off tank bottom during construction.}
\item{Thoroughly clean cathode \& space used for fabrication.}
\item{Remove protective screen 2.}
\item{Fit cathode to field cage using suitable jacks.}
\item{Screen off entire detector: protective screen 3.}
\item{Remove floor protection.}
\item{Add cable trays, junction boxes \& cables for PMTs.}
\item{Install PMTs to tank Bottom (pre-assembled L-flanges). Check out \& test PMTs.}
\item{Clean air purge bottom level.}
\item{Install temporary enclosure around TCO inside \& outside with air lock within the enclosure.}
\item{ Remove protective screen (screen 3) using air lock system to prevent contamination of detector}.
\item{Close temporary construction openings.}
\item{Thoroughly clean TCO areas.}
\item{Remove temporary enclosures.}
\item{Remove all tools, equipment, etc., through tank roof.}
\item{Exit via room manways.}
\item{Close all tank roof openings.}
\end{enumerate}

\subsection{Detector construction program and installation schedule}

Both the 3$\times$3$\times$1~m$^3$ prototype detector and the
6$\times$6$\times$6~m$^3$ WA105 demonstrator are planned to be built in
advance of the larger 4$\times$10-kt modules of the experiment at
SURF. It is expected that valuable information will be gathered
from the construction of both the prototype and demonstrator detectors
that will ultimately benefit both the planning and construction
forecasting for the larger far detector. 

For comparison the 20-kt LAr detector for Pyhasalmi was designed for a
drift surface (roughly the CRP and cathode area) of 824~m$^2$ in an
octagonal shape with a drift length of 20~m. The dual-phase DUNE
LArTPC experiment at SURF has an equivalent area of 12$\times$60 =
720~m$^2$ (for one module of 12~kt) with a drift length of 12--15~m
(fiducial mass 1~kt / 1~m drift, total 12--15~kt). The construction
program calculated by Rockplan Ltd, Alan Auld Ltd and Rhyal Eng. Ltd
can be seen as a conservative approach for the SURF site, as most
of the time corresponds more closely to
the instrumented surface required  rather than to 
the drift length, but neither specific SURF site-related
effects nor effects of US legal procedure are  taken into account
in this construction program.

The DUNE detector module installation program has been divided into three distinct (and separate)
stages:
\begin{enumerate}
\item{Design,}
\item{Manufacture/Fabrication, and} 
\item{Assembly.}
\end{enumerate}  

The detector design must be done together with the tank deck design,
as the complete detector is suspended from the deck.  Fabrication and
manufacturing can be started while the tank construction is still
on-going.  The total time for manufacture/fabrication and construction
is calculated to be around six years for a DUNE far detector module, of which
\begin{itemize}
\item{14 months	is for	for manufacture/fabrication off-site, }
\item{20 months	is for	for construction/installation + testing, and }
\item{32 months	is for	total works (with partial overlap). }
\end{itemize}
