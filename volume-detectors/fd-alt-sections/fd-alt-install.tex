%%%%%%%%%%%%%%%%%%%%%%%%%%%%%%%%
\section{Installation and Commissioning}
\label{sec:detectors-fd-alt-install}

\subsection{Prepatory works}

Before the installation or the detector works can be initiated, many preparations must be conducted and boundary conditions must be met in order to allow one for both a safe as well a cost and time efficient realization of the detector installation.  Some of the most important preparation items are briefly explained in the Annex belonging to this Chapter.

\subsection{Detector Installation Sequences}
Note: Design: courtesy Technodyne Ltd., Eastleigh, United Kingdom

A Construction Sequence for the double phase TPC LAr Detector installation was defined based on the use of a slightly modified scaffolding arrangement as discussed in the Annex.  The proposed Construction Sequence assumes completion of the membrane and insulation system installation together with completion of all tank internal pipework and cable trays.  At this stage, sections of the scaffolding will be removed and replaced by the Alimak Hek or similar climbing access platforms to provide increased functionality to the installation.  On completion of the scaffolding revisions and the climbing access platform installation, the entire tank and scaffolding systems will be cleaned in preparation for the Detector installation.  The proposed LAr Detector Construction Sequence consists of:

\begin{enumerate}
\item{Complete installation of insulation \& membrane, install cable trays from top to bottom from PMT electrical cables}
\item{Adjust scaffolding platforms, add Alimak-Hek platform and floor protection}
\item{Air purge top level installation}
\item{Install hanging columns for detector}
\item{Install lowest field shaping coil to stabilise columns + Install first top 15 levels of field shaping coils}
\item{Thoroughly clean top level assembly}
\item{Install Charge Readout (CRP) from top scaffolding platform (detailed sequences described below figure 7)}
\item{Thoroughly clean top level assembly}
\item{Screen off top level to protect Anode (protective screen 1)}
\item{Air purge top level (allow bleed air into middle \& lower levels)}
\item{Continue installing Field shaping coils}
\item{Complete installation of Field shaping coils}
\item{Thoroughly clean field shaping coils, remove protective screen (screen 1) to top level \& progressively remove all scaffolding \& Alimak-Hek platforms}
\item{Screen off field shaping coils (incl. CRO): protective screen 2}
\item{Air purge top \& middle levels}
\item{Construct Cathode from Modules. Cathode to be raised 300mm off tank bottom during construction}
\item{Thoroughly clean cathode \& space used for fabrication}
\item{Remove protective screen 2}
\item{Fit cathode to field cage using suitable jacks}
\item{Screen off entire detector: protective screen 3}
\item{Remove floor protection}
\item{Add cable trays, junction boxes \& cables for PMTs}
\item{Install PMTs to tank Bottom (pre-assembled L-flanges). Check out \& test PMTs}
\item{Clean air purge bottom level}
\item{Install temporary enclosure around TCO inside \& outside with air lock within the enclosure}
\item{ Remove protective screen (screen 3) using air lock system to prevent contamination of detector}
\item{Close temporary construction openings}
\item{Thoroughly clean TCO areas}
\item{Remove temporary enclosures}
\item{Remove all tools, equipment etc. through tank roof}
\item{Exit via room manways}
\item{Close all tank roof openings}
\end{enumerate}

More detailed explanation and figures are presented in the Annex belonging to this chapter.

\subsection{Construction programme of detector installation}
Both the $3\times3\times1$ $m^3$ prototype detector [Location CERN, B.182] and the $6\times 6\times 6$ $m^3$  (= 216 $m^3$ = 0.3kT) demonstrator [Location CERN, B.887 in the North Area] are planned to be built in advance of the larger $4\times 10kT$ experiment at Homestake.  It is envisaged that valuable information will be gathered from the construction of both the Prototype and Demonstrator experiments that will ultimately benefit both the planning and construction forecasting for the larger experiments later on in the project.

For comparison the 20kT Lar detector, designed for Pyhasalmi had a drift surface (roughly the CRP and Cathode area) of 824 $m^2$ in octagonal shape with a drift length of 20m. The double phase TPC LAr experiment at Homestake has an equivalent area of 12 by 60 = 720 $m^2$ (for one module of 10+ kT) with a drift length of 13 to 15m (fiducial mass 1 kT / 1m drift, total 13 to 15kT). The construction programme calculated by Rockplan Ltd, Alan Auld Ltd and Rhyal Eng. Ltd can be seen as a conservative approach for the Homestake site, as most of the time is linked with the instrumented surface required and not so much with the drift length, but specific Homestake site related effects and effects of US legal procedure are not taken into account in this construction programme.

The Detector programme has been divided into 3 distinct [and separate] stages, 1) Design, 2) Manufacture/Fabrication and 3) Construction.  The detector design must be done together with the tank deck design, as the complete detector is suspended from the deck.  Fabrication and manufacturing can be started while the tank construction is still on-going.  The total time for manufacture / fabrication and construction is calculated to be around 6 years for the LAr detector, of which:
\begin{itemize}
\item{14 months		for manufacture/fabrication off-site}
\item{20 months		for construction/installation + testing}
\item{32 months		total works (partially overlap)}
\end{itemize}


