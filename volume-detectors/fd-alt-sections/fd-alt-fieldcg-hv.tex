%%%%%%%%%%%%%%%%%%%%%%%%%%%%%%%% 
\section{The Field Cage and High Voltage System} 
\label{sec:detectors-fd-alt-hv}

This section describes the design of the high voltage system, field cage and cathode for the TPC.  It is inspired from the  
LBNO 20-kt and 50-kt detector design, described in \anxlbnoa and \anxlbnob,  which can be simplified and down-sized for the DUNE detector 
(12~kt or 15~kt),  due to the shorter drift path and the rectangular aspect ratio of the detector. The much shorter transversal dimension of DUNE with respect to the LBNO design permits a lighter cathode structure (less sag requires less compensation) and a simpler hanging system for the drift cage and the cathode.

In the LBNO design the field cage is composed of equally spaced octagonal rings that create a uniform drift field which have an intensity adjustable in a range going from 500 to 1000~V/cm. This leads, when operating at the maximal field intensity of 1~kV/cm over a drift distance of 20~m, to a cathode voltage of up to 2~MV.

Two different approaches have been developed for the drift-field HV system. The first approach uses an external HV power supply and uses HV feedthroughs to penetrate into the detector volume. The  WA105 demonstrator will use this approach  for its drift
of 6~m and a cathode voltage up to 600~kV.  The second approach is based on a HV generator directly inside the LAr volume, which is the  the Greinacher HV multiplier. It is an innovative technique,  with some advantages relative to the first approach. This technique particularly suits giant-scale detectors that require a very high voltage of $\sim$1--2~MV. 

The DUNE detector (12-m drift) requires a voltage of 600~kV  in order to operate at a field intensity of 0.5~kV/cm. This voltage is a factor of 3.3 higher than that for the reference design (Chapter~\ref{ch:fd-ref}) and it will be tested during the WA105 detector operation at 1~kV/cm over a 6-m drift. 

The field cage designed for the LBNO 20-kt detector (20-m drift path) is composed of 99 octagonal field-shaping coils manufactured from 316L stainless-steel tubing and long radius elbows to EN 10217-7 shop. The straight pipes and the elbows are assembled together to form the coils by using a combination of welded and clamped joints. 

The coils are supported by 32 off-hanging columns of G-10CR glass fibre/epoxy-laminated sheet insulating material, built in the form of chains and suspended from the tank deck structure. 

Each  coil is designed as a series of fully welded infill tubes intended to fit between pairs of hanging support columns to form one section of the field cage.  Short sections of the field-shaping coils are integrated as pins into links to assemble each chain.  Longer sections and  corner sections of the field-shaping coil are then fixed between the hanging columns to complete each coil (Figure~\ref{fig:LBNO_FC}). The combined assembly of 99 sets of field shaping coils within the 32 off-hanging columns provides a complete field cage.

\begin{cdrfigure}[Assembly concept of the field cage]{LBNO_FC}
{\small Left: assembly of the elements of a hanging chain. Right: 
construction of a field cage section from the hanging chains and the field shaping coil elements.}
\includegraphics[width=.5\linewidth]{LBNO_chains} \hfill
\includegraphics[width=.4\linewidth]{LBNO_FC2}
\end{cdrfigure}

The infill tube specifications assume an outer diameter of 139.7~mm, which is common to the cathode structure.  This allows a thinner wall, 1.6~mm,  to be used  in non-structural parts of the  coils.  Although a non-standard size, the total length of this tube can be manufactured as a special mill run.  This will make it possible to save 21~tons of material relative to the standard tubes (wall thickness of 2.0~mm). The wall thickness of the link pins  is 2.6~mm; this will provide sufficient stiffness to resist to  the bending torques across the link pins.  All specialist preparation and welding of the link pins will be carried out in shop facilities under controlled fabrication.  This includes the rough machining of the end fittings, preparation of the tube ends and the jig-welding of the complete assemblies. Further machining, after welding, will be carried out to ensure correct alignment and tolerance levels in conjunction with the hanging columns.  Vent holes will be incorporated into the tubes as required to facilitate construction and to allow purging with GAr/LAr on commissioning.

Manufacturing, transportation and underground construction considerations were a fundamental part of the field-shaping coil design process, in
collaboration with the LAGUNA-LBNO industrial partners. The requirement of construction in a clean-room environment within the completed membrane tank  presented considerable challenges in terms of logistics and the development of the overall concept for fabrication.  It was concluded that a modular construction approach would be required in order to  (1) maximize off-site shop fabrication and minimize on-site assembly,  and (2) ensure the cleanliness of construction and to minimize the installation time. The breakdown of each field-shaping coil is made into sets of three main modules.  Although separate modules, the field shaping coils and the cathode structure share identical features and dimensions.  Thus, the maintenance of common interfaces is an important advantage  of the overall field cage design.
 
The DUNE  12-kt detector field cage would include 60 rectangular rings used to cover the 12~m of drift volume.

The LBNO cathode design for the 50-kt detector follows an extensive review of options and analysis. The design incorporates features to minimize the static deflection of the cathode and to maximize the electrostatic performance. (To avoid regions with high electric fields, the electric field is limited to 50~kV/cm.)  Similar to the field-shaping coil, and for the same reasons, the cathode is designed as a modular structure fulfilling ensuring a minimal on-site assembly time. The cathode is designed as a fully welded tubular rectangular grid structure in 2~m$\times$2~m bays, 1~m deep in the vertical direction supported only from the  periphery (Figure~\ref{fig:LBNO_cathode}).  

\begin{cdrfigure}[Cathode design]{LBNO_cathode}
{\small Left: cathode plane design for the LBNO detector. Right: breakdown of 
the cathode structure in construction modules.}
\includegraphics[width=.4\linewidth]{LBNO_cathode} \hfil
\includegraphics[width=.5\linewidth]{LBNO_cathode_elements}
\end{cdrfigure}

The top and bottom grid structures are manufactured from 139.7-mm OD tubes with wall thickness  2.6~mm  to EN 10217-7  in 316 stainless steel.  The bracing structure is manufactured from 60.3-mm OD tubes with wall thickness 2.6~mm, also to EN 10217-7 in 316 stainless steel.  A grid structure comprising   10-mm OD tubes with wall thickness 1~mm, arranged in a single plane at 100-mm centers, will be fitted to the top of the cathode only. The maximum module size for this structure is 6~m, comprising three full 2~m$\times$2~m$\times$1~m deep bays of the cathode structure.  All specialist nodes preparation and welding of the modules will be carried out in controlled facilities at the fabrication shop.  Vent holes are incorporated into the structures to
facilitate construction and to allow purging with GAr/LAr on commissioning. High levels of quality control will be possible with the modular construction design, and following inspection, each module will be cleaned to ISO 8 cleanliness standard and double-wrapped prior to dispatch and transportation to the site for installation and final assembly. 

The cathode outer top tubular structure is identical to the bottom field-shaping coil; they use tubes of the same outer diameter
(139.7~mm).  The spans are the same (48~m for the 50-kt LBNO detector) and the vertical distance separating these components is the same as for
the remaining field-shaping coils (200-mm centers). The cathode will be attached at the bottom of each hanging column by a split link in
G-10CR. The cathode attachment points will also incorporate locally thickened sections of tube (as in the hanging chains) included as part
of the peripheral structure nodes.
 
The complete assembly procedure, logistics and tooling for the field cage and cathode is described in \anxlbnob.
It is expected that the general design, adapted to the rectangular geometry, and the basic elements for the cathode construction would be similar for the 12-kt DUNE detector.  However the basic elements for the cathode construction  could be down-sized since sagging effects are only on 12~m span and not 40~m, as in the case of LBNO. This less stringent requirement translates in a lighter and cheaper structure.

