%%%%%%%%%%%%%%%%%%%%%%%%%%%%%%%% 
\section{The Field Cage and High Voltage System} 
\label{sec:detectors-fd-alt-hv}

A detailed description of the design of the HV system, field cage and
cathode for the LBNO 20~kt and 50~kt detectors is provided in the
deliverable document 3.1 (LAr experiment) of the LAGUNA-LBNO design
study\cite{LAGUNA-LBNO-deliv}. This design can be applied to the DUNE
12~kt or 15~kt detectors with some simplifications and down-sizing
due to the shorter drift path and the rectangular aspect ratio of the
detector. The much shorter transversal dimension with respect to LBNO
allows to construct a lighter cathode structure (less sagging effect
to be compensated with respect to the octagonal detector) and to
simplify the hanging system of the drift cage and the cathode.

The field cage in the LAGUNA-LBNO dual-phase detector design is
composed of equally spaced octagonal rings which create a uniform
drift field with strength of 500--1000~V/cm. A voltage up to 2~MV at
the cathode is then needed in order to provide a the drift field at
the maximal intensity of 1~kV/cm over a drift distance of 20~m.

Two different approaches have been considered and realized for the
drift field HV system. The first type uses an external HV power supply
and feeds it into the detector volume using HV feedthroughs. This
approach is going to be applied in the WA105 demonstrator for a drift
of 6~m and a cathode voltage up to 600~kV.  The second approach is
based on an internal HV generator directly inside the LAr volume same
as the Greinacher HV multiplier. It is an innovative technique to
generate a HV for creating a drift electric field in a LArTPC, with
some advantages compared to the first option. This technique
particularly suits giant-scale detectors, where a very HV of
$\sim$1--2 MV is needed. For the DUNE detector (12~m drift) a voltage
of 600~kV will be needed in order to operate at a field intensity of
0.5~kV/cm. This voltage is a factor 3.3 higher than the one foreseen
in the reference design and it is in the scope of the WA105 detector
operation at 1~kV/cm over 6~m drift.

The field cage designed for the LBNO 20~kt detector (20~m drift
path) is composed of 99 octagonal field shaping coils manufactured
from 316L stainless steel tubing and long radius elbows to EN 10217-7
shop using a combination of welded and clamped joints. Each field
shaping coil is designed as a series of fully welded infill tubes to
fit between hanging columns to form one section of the field cage. The
coils are then supported by 32 off hanging columns suspended from the
tank deck structure by a special hanging support system. The design
foresees hanging columns built in the form of chains. Short sections
of the field shaping coils are integrated as pins into links of G-10CR
glass fibre/epoxy laminated sheet insulating material to assemble each
one of these chains. Longer sections and corner sections of the field
shaping coil are then fixed in between the hanging columns to complete
each coil (Figure~\ref{fig:LBNO_FC}). The combined assembly of 99 sets
of field shaping coils within the 32 off hanging columns thus provides
a complete field cage.
\begin{cdrfigure}[Assembly concept of the field cage]{LBNO_FC}
{\small Left: assembly of the elements of a hanging chain. Right: 
construction of a field cage section from the hanging chains and the field shaping coil elements.}
\includegraphics[width=.5\linewidth]{LBNO_chains} \hfill
\includegraphics[width=.4\linewidth]{LBNO_FC2}
\end{cdrfigure}

The tubes specifications foresee an outer diameter of 139.7~mm which
is common to the cathode structure but allow a thinner wall section to
be used where required in a non-structural parts pf the field shaping
coils.  Although a non-standard size, the total length of 1.6~mm wall
thickness tube required for the field shaping coils can be
manufactured as a special mill run.  Using this approach it will be
possible to save 21~t of material as compared with the standard
2.0~mm wall thickness of the tubes. The link pins have 2.6~mm wall
thickness. This will provide sufficient section modulus to react the
bending moments across the link pins. All specialist preparation and
welding of the link pins will be carried out under controlled
fabrication shop facilities.  This will comprise the rough machining
of the end fittings, preparation of the tube ends and the jig welding
of the complete assemblies. Further machining, after welding will be
carried out to ensure correct alignment and tolerance levels in
conjunction with the hanging columns.  Vent holes will be incorporated
into the tubes as required to facilitate construction and to allow
purging with GAr/LAr on commissioning.

A fundamental part of the field shaping coil design process, in
collaboration with the LAGUNA-LBNO industrial partners, involved the
consideration of how such tubes could be manufactured to a high level
of accuracy, transported to site, moved underground and then
constructed in a 'Clean Room' environment within the completed
membrane tank.  This requirement presented considerable challenges in
terms of logistics and the development of the overall concept for
fabrication.  It was concluded that a modular construction approach
would be required in order to maximize off-site shop fabrication and
minimize on-site assembly.  This approach was also considered
essential in order to ensure the cleanliness of construction and to
minimize the installation time. The breakdown of each field shaping
coil is made into sets of 3 main modules.  Although separate modules,
the field shaping coils and the cathode structure share identical
features and dimensions.  Thus, the maintenance of common interfaces
forms an essential part of the overall detector design. The field cage
design for the DUNE far detector is based on the same concept
developed for LBNO, including 60 rectangular rings used to cover the
12~m of drift volume of the 12~kt detector.

The cathode design for LBNO was developed following an extensive
review of options and analysis to optimize the design for the specific
requirements of the 50~kt LAr Detector.  The design incorporates
features to minimize the static deflection of the cathode and to
maximize the electrostatic performance (avoid regions with high
electric fields, maximum allowed electric field limited to 50~kV/cm). 
Also the cathode is designed in a modular structure for
maximum off-site shop fabrication and minimum on-site assembly in
order the ensure cleanliness of construction, by using the cryostat ad
clean room for the assembly, and to minimize the installation
time. The cathode is designed as a fully welded tubular rectangular
(square) grid structure in 2~m$\times$2~m bays, 1~m deep supported only from
the outer periphery (Figure~\ref{fig:LBNO_cathode}).  
\begin{cdrfigure}[Cathode design]{LBNO_cathode}
{\small Left: cathode plane design for the LBNO detector. Right: breakdown of 
the cathode structure in construction modules.}
\includegraphics[width=.4\linewidth]{LBNO_cathode} \hfil
\includegraphics[width=.5\linewidth]{LBNO_cathode_elements}
\end{cdrfigure}
The top and bottom grid structures are manufactured from 139.7~mm OD x
2.6~mm wall tubes to EN 10217-7 in 316 stainless steel.  The bracing
structure is manufactured from 60.3~mm OD$\times$2.6~mm wall tubes also to EN
10217-7 in 316 stainless steel.  A grid structure comprising 10~mm OD x
1~mm wall tubes arranged in a parallel single plane at 100~mm centers
will be fitted to the top of the cathode only. The maximum module size
is the 6~m module comprising 3 full 2~m$\times$2~m$\times$1~m deep bays
of the cathode structure.  All specialist node preparation and welding
of the modules will be carried out under controlled fabrication shop
facilities.  Vent holes are incorporated into the structures to
facilitate construction and to allow purging with GAr/LAr on
commissioning. High levels of quality control will be possible with
the modular construction design and following inspection, each module
will be cleaned to ISO 8 cleanliness standard and double wrapped prior
to dispatch and transportation to site for installation and final
assembly of the cathode. The complete assembly procedure logistics and
tooling for the field cage and cathode is described in details the
LAGUNA-LBNO deliverable document.

The cathode outer top tubular structure is identical to the bottom
field shaping coil by using tubes with the same outer diameter
(139.7~mm).  The spans are the same (48~m for the 50~kt LBNO detector)
and the vertical distance separating these components is the same as
the remaining field shaping coils (200~mm centers). The cathode will be
attached at the bottom of each hanging column by a split link in
G-10CR. The cathode attachment points will also incorporate locally
thickened sections of tube (as in the hanging chains) included as part
of the peripheral structure nodes


The cathode structure for the 12~kt DUNE far detector follows the
same design concepts as the one studied for LBNO with a down-sizing of
the elements since is is not needed to limit sagging effects over 48~m
span but on only 12~m. This translates in a lighter and cheaper
structure.
