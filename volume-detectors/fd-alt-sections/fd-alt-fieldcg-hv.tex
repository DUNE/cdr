%%%%%%%%%%%%%%%%%%%%%%%%%%%%%%%% 
\section{The Field Cage and High Voltage System} 
\label{sec:detectors-fd-alt-hv}

A detailed description of the design of the HV system, field cage and cathode for the LBNO 20kton and 50 kton detectors is provided in the deliverable document of the LAGUNA-LBNO design study \cite{LAGUNA-LBNO-deliv}. This design can be applied to the DUNE 12kton or 15kton detector with some simplifications due to the shorter drift path and the rectangular aspect ratio of the detector, with a much shorter transversal dimension which allows to construct a lighter cathode structure (less sagging effect to be compensated with respect to the octagonal detector) and to simplify the hanging system of the drift cage and the cathode.

The field cage in the LAGUNA-LBNO double-phase detector design  is composed of equally spaced octagonal rings which create a uniform drift field with strength between 500 and 1000 V/cm. To provide high voltage to the drift field over a drift distance of 20 m, a voltage up to 2 MV at the cathode is needed. 

Two different approaches have been considered and realized for the drift field HV system. The first type uses an external HV power supply and feeds it into the detector volume using HV feed-throughs. This approach is going to be applied in the WA105 demonstator for a drift of 6m and a cathode voltage up to 600 kV.  The second type has an internal HV generator directly inside the LAr volume same as the Greinacher HV multiplier. It is an innovative technique to generate a HV for creating a drift electric field in a LAr-TPC, with some advantages compared to the first option. This technique particularly suits giant-scale detectors, where a very HV of ~1 MV to ~2 MV is needed.

The LBNO Field Cage comprises 99 off Field Shaping Coils supported by 32 off Hanging Columns from the Membrane Tank Deck Structure by a Special Hanging Support System.  The design is based on the Hanging Chain Type Columns where short sections of Field Shaping Coil are integrated as pin into links of G-10CR composite insulating sheet material to produce a chain.  Longer sections and corner sections of Field Shaping Coil are fixed between the Hanging Columns to form the Field Cage.  The design incorporates features developed in conjunction with the Scientific Partners to maximise the electrostatic performance of the Field Cage. The proposed Field Cage is designed for maximum off-site shop fabrication and minimum on-site assembly in order the ensure cleanliness of construction and to minimise the installation timeline.  These features have been developed in conjunction with the Industrial Partners.

The Field Shaping Coils comprise a series of 99 off octagonal rings manufactured from 316L stainless steel tubing and long radius elbows to EN 10217-7 using a combination of welded and clamped joints.  Each Field Shaping Coil is designed as a series of fully welded infill tubes to fit between the Hanging Columns to form one section of the Field Cage.

The tube specification selected will provide an outer diameter which is common to the Cathode structure but will allow a thinner wall section to be used where required in a non-structural application for the Field Shaping Coils.  Although a non-standard size, the total length of 1.6mm wall thickness tube required for the Field Shaping Coils will be manufactured as a special mill run.  Using this approach it will be possible to save 21tonne of material as compared with the standard 2.0mm wall thickness.

A fundamental part of the Field Shaping Coil design process involved the consideration of how such tubes could be manufactured to a high level of accuracy, transported to site, transported to the cavern site and then constructed in a 'Clean Room' environment within the completed Membrane Tank.  This requirement presented considerable challenges in terms of logistics and the development of the overall concept for fabrication.  It was concluded that a Modular Construction approach would be required in order to maximise off-site shop fabrication and minimise on-site assembly.  This approach was also considered essential in order to ensure the cleanliness of construction and to minimise the installation timeline.

The proposed breakdown of each Field Shaping Coil is made into sets of 3 main modules.   Although separate Modules, the Field Shaping Coils and the Cathode structure share identical features and dimensions.  Thus, the maintenance of common interfaces forms an essential part of the overall Detector design.

The Hanging Columns comprise 32 off  Chain Type columns manufactured from G-10CR glass fibre/epoxy laminated sheet material and 316L stainless steel tubing to EN 10217-7 shop assembled into appropriate lengths.  Each Hanging Column is designed to provide a fixing point for the attachment of the Field Shaping Coils as a series of infill tubes.  The combined assembly of 99 sets of Field Shaping Coils within the 32 off Hanging Columns thus provides a complete Field Cage.
The stainless steel tube specification selected will provide a 139.7mm outer diameter which is common to the Cathode structure and the Field Shaping Coils and by using 2.6mm wall thickness, this will provide sufficient section modulus to react the bending moments across the link pins. .  All specialist preparation and welding of the link pins will be carried out under controlled fabrication shop facilities.  This will comprise the rough machining of the end fittings, preparation of the tube ends and the jig welding of the complete assemblies. Further machining, after welding will be carried out to ensure correct alignment and tolerance levels in conjunction with the Hanging Columns.  Vent holes will be incorporated into the tubes as required to facilitate construction and to allow purging with GAr/LAr on commissioning.

The proposed Cathode design was developed following an extensive review of options and analysis to optimise the design for the specific requirements of the 50ktonne LAr Detector.  The design incorporates features developed in conjunction with the Scientific Partners to minimise the static deflection of the cathode and to maximise the electrostatic performance. The proposed Cathode is designed for maximum off-site shop fabrication and minimum on-site assembly in order the ensure cleanliness of construction and to minimise the installation timeline.

The Cathode is designed as a fully welded tubular rectangular (square) grid structure in 2m x 2m bays, 1m deep supported only from the outer periphery.  The top and bottom grid structures will be manufactured from 139.7mm OD x 2.6mm wall tubes to EN 10217-7 in 316 stainless steel.  The bracing structure will be manufactured from 60.3mm OD x 2.6mm wall tubes also to EN 10217-7 in 316 stainless steel.  A grid structure comprising 10mm OD x 1mm wall tubes arranged in a parallel single plane at 100mm centres will be fitted to the top of the cathode only.

