%%%%%%%%%%%%%%%%%%%%%%%%%%%%%%%%
\section{Overview}
\label{sec:detectors-fd-ref-ov}

The alternate detector design benefits of 13 years of R\&D activity and two programs funded by the European Union for a total budget of 17 Meur which resulted in the LAGUNA-LBNO design study concluded in August 2014. The LAGUNA-LBNO design study allowed to define, in collaboration with several industrial partners, an optimized configuration for a long baseline experiment  experiment with associated technological developments, innovative solutions and full costing. In particular, since the very beginning, these studies were focused on the underground implementation of a very large liquid argon detector (GLACIER) aiming at optimizing the construction and installation costs and making affordable the detector implementation and operation. This optimization concerned not only the dual-phase readout concept but also many technical aspects related to the underground design and implementation such as the tank, the field cage, the cathode  and the definition of the logistics and the detector assembly sequence. All these technical developments are described in the deliverable documents submitted to the EU at the conclusion of the design study \cite{LAGUNA-LBNO-deliv}.

Following the GLACIER concept (see Figure~\ref{fig:LBNO_50}), the dual-phase LAr TPC detector has a fully homogeneous liquid argon volume, where electrons are drifted vertically towards the liquid-vapor interface. They are then extracted from the liquid into the gas phase, amplified, and collected on a segmented anode \cite{Badertscher:2013wm,Badertscher:2012dq,Badertscher:2010zg}. The electron amplification in the gas phase is one of the novel features of the dual-phase TPC  allowing to obtain a robust and tunable signal to noise ratio. Amplification of the ionization charges in extremely pure argon gas is achieved by using micro-pattern avalanche gas detectors call LEM (Large Electron Multiplier). The detector configuration is similar to a single-phase LAr TPC but the electron extraction to the gas phase and multiplication before collection, combined with a modular anode structure, allows to achieve very long drift paths and  large detector dimensions while minimizing the number of readout channels.

\begin{cdrfigure}[The 50kton LBNO detector. ]{LBNO_50}{The 50 kton LBNO detector.}
\includegraphics[width=\linewidth]{LBNO50kton2}
\end{cdrfigure}

An extraction efficiency of  100\% of the electrons from the liquid to the gas phase is achieved with an electric field of the order of 2 kV/cm created by submersed grid at the liquid-gas interface. The ionization charge is then amplified by avalanches occurring in the gas in the micro-pattern structure of the LEM and collected in a 2-dimensional readout plane on the top of the volume. Readout units, combining both the amplification and collection functions have an area of $0.5\times 0.5$~m$^2$ finely segmented with a 3.125~mm pitch strips. The typical amplification achieved in this configuration is around a few tens. It allows to compensate for the charge losses due to the presence of electronegative impurities along very log drift paths. The S/N ratio can exceed 100 for a particle at the ionization minimum (m.i.p.) after a drift path of 12m. Although the dual-phase TPC as longer drift length, it should be noted that it requires a liquid argon purity around 0.1~ppb = 100~ppt of oxygen equivalent, similarly as for the single-phase TPC,
starting from commercially available ppm-level bulk argon and non-evacuated vessel \cite{WA105_TDR}.

The drift field ( $E{\simeq} 0.5-1.0$~kV/cm) inside the fully active LAr volume  is  produced by applying high voltage  to the bottom cathode plane and it is kept uniform by a stack of field shaping electrodes (field cage) polarized at linearly decreasing voltage from the cathode voltage to ground. These electrodes are  made by several equally-spaced stainless-steel tubes along a rectangular path with round corners. The field cage is held in place by insulating mechanical structures hung from the top deck of the vessel.  The cathode structure is suspended from the filed cage. The cathode plane is gridded and it is transparent to allow the detection of the scintillation light by uniform arrays of photomultipliers mounted at the bottom of the tank , $1m$  below the cathode surface. 

The drift field is closed at the top by the anode used for the charge readout which is segmented in modular units: Charge Readout Planes (CRPs). Each CRP is composed of several  $0.5\times 0.5$~m$^2$ readout units and it  is independently suspended with stainless-steel ropes linked to the top deck. This suspension system allows to adjust the CRP distance and parallelism with respect to the LAr surface. The electrical signals from the collected charges are then passed outside the tank via a set of dedicated signal feedthrough chimneys traversing the top layer of insulation. These chimneys house at their bottom the cryogenic  Front End (FE) electronics.  The front-end electronics is based on analog preamplifiers implemented in CMOS ASIC circuits for high integration and large scale affordable production. The FE  is cooled at a temperature around 110 K and it is isolated with respect to the LAr vessel by a cold feedthrough. This feedthrough is then connected to the CRP via flat cables of 0.5 m length. The chimneys design allows to access and replace the FE from outside without contaminating the LAr volume. The digital electronics and DAQ system are completely outside the cryostat in micro-TCA racks mounted on each signal feedthrough chimney. Other feedthrough chimneys are foreseen for the cathode HV connection, the CRPs suspension and level adjustment, th high voltage and signal readout of the PMTs and the monitoring instrumentation (level meters, temperature probes, strain gauges, etc.)

Situated in the vapor phase on top of the LAr volume the CRP  provides an adjustable charge gain and two independent readout views, each with al pitch of 3.125 mm.  Combined with the time information provided by the LAr scintillation readout by the PMT arrays (T0), the CRPs provide three-dimensional (3D) tracks imaging with dE/dx information. The S/N ratio is increased by at least one order of magnitude by the possibility of amplifying the charges with avalanches in the gas phase.  This high S/N significantly improves the event reconstruction quality. It also lowers the threshold for small energy depositions and provides a better resolution per volumetric pixel (voxel) compared to a conventional single-phase LAr-TPC. 

The dual-phase TPC concept is well suited for large detector sizes since the charge attenuation on long drift paths is compensated by the charge amplification in the CRPs.  This configuration also simplifies the construction by optimally exploiting long vertical dimensions of the cryostat, reducing the number of read-out channels, increasing the detector size, and lowering costs.  This provides inside the TPC field cage a large homogeneous fiducial volume for neutrino interactions, consisting only of liquid argon with no embedded passive materials. The charge produced in this volume is collected on the top CRPs surface in a projective way, with practically no dead regions. In each CRP the charge is readout in two collection views with no use of induction views, which is also an important asset in the case of complicated topologies. 

The dual-phase readout scheme, has been successfully demonstrated on several prototypes  by an R\&D activity which spans over more than 10 years.The design of very large (20 to 50 kton) underground detectors based on this concept has been developed in great detail in the context of the LAGUNA and LAGUNA-LBNO design studies.  The CERN WA105 experiment is intended to show a full scale implementation of this technique, as well as of other technologies developed for the construction of large underground TPC detectors.  All the key subsystems have been designed taking into account their scalability to a large detector.  In 2018, this demonstrator will be tested and calibrated with a charged particles beam (pion, muons, electrons).


\subsection{DUNE Dual-Phase LAr Detector Configuration}

The proposed detector module, based on the dual-phase design, optimally exploits the inner space foreseen for the DUNE cryostats ($14 (w) \times 14.1 (h) \times 62  (l)~m^3$) with an active area of  ($12 \times 60 ~m^2$) and a drift length of 12 m corresponding to 12.096 kton of LAr (10.643 kton fiducial). The proposed design is based on the one for 20 kton LAr detector of the LAGUNA-LBNO design study with a CRP unit size re-adapted to the dimensions on the active area. The height of the DUNE cryostat could be increased in order to achieve 15m  drift. In that case a single detector would reach an active mass of 15.12 kton (13.444 kton fiducial).  This 15.1 kton  detector configuration, apart the longer drift distance and field cage, would have the same characteristics of the 12.1 kton baseline configuration, given that  the covered active area is exactly the same. With these transverse dimensions every additional meter of drift length provides 1 kton increase in the active mass with a moderate additional cost.

The inner detector for the 12.1 kTon active mass is built as a single active volume 60m long, 12m wide and 12m high. The active volume (see Figure~\ref{fig:DP_det1} and Figure~\ref{fig:DP_det2}) is surrounded by a vertical field cage made by a set of 60 rectangular tubular rings (tube diameter 140mm, vertical spacing 200mm).

\begin{cdrfigure}[Douple-phase detector 3D view]{DP_det1}{The DUNE dual-phase detectorwith cathode,PMTs, field cage and the anode plane with chimneys.}
\includegraphics[width=\linewidth]{DUNE12_FC}
\end{cdrfigure}

The cathode plane (on the bottom) is made by a reinforced frame, to guarantee its planarity, filled by a tubular grid (not visible in Figure~\ref{fig:DP_det2}), to allow the optical transparency for the scintillation light towards an array of 180 PMTs (1 per $4m^2$) located at the bottom of the vessel.

\begin{cdrfigure}[Douple-phase detector 3D view (partially open)]{DP_det2}{The DUNE dual-phase detector (partially open) with cathode,PMTs, field cage and the anode plane with chimneys.}
\includegraphics[width=\linewidth]{DUNE12_open}
\end{cdrfigure}

The ionization electrons in the liquid phase, drift towards the CRPs (Charge Readout Planes)  in a uniform electric field. The extraction of the electrons from the liquid to vapor phase is performed thanks to a submersed horizontal extraction grid, integrated in the  CRP structure.  Each CRP is composed by LEM/Anode Sandwiches (LAS) ($0.5m\times 0.5m$) embedded in  a mechanically reinforced frame of FR-4 and Stainless Steel.  Each sandwich includes a  LEM (Large Electron Multiplier) for the charge amplification via avalanches occurring in a micro-pattern structure in pure gas argon and a anode Printed Circuit Board (PCB) for the collection of the amplified charges on two independent orthogonal views with interleaved X and Y strips.  Thicknesses and possible biasing voltages for the different layers are indicated, as example, in Figure~\ref{fig:CRP_struct}. 

\begin{cdrfigure}[Charge Readout Plane (CRP) structure]{CRP_struct}{Thicknesses and HV values for e- extraction from LAr to GAr, their multiplication by LEMs and their collection on the X-Y CRP plane. The HV values are indicated for a drift field of 0.5 kV/cm in LAr.}
\includegraphics[width=\linewidth]{CRP_gaps.png}
\end{cdrfigure}

The anode plane, at the top of the active volume, is made by an array of 45 independent CRP modules, $3\times3$ $m^2$ each, as shown in Figure~\ref{fig:CRP_unit1} and Figure~\ref{fig:CRP_unit2} Each CRP unit includes 36 ($0.5m\times 0.5m$) LAS. 

Signals in each CRP unit are collected via 3 signal feedthrough chimneys. Each chimney collects 640 readout channels and hosts at its bottom the front-end cards with the cryogenic electronics (ASIC amplifiers), just before a cold feedthrough isolating them from the ultra-pure LAr volume. The front-end cards work at a temperature of 110 K and can be removed from the top of the chimney without contaminating the LAr volume. The LEM/Anode sandiwiches in the same CRP unit are interconnected with short flat cables so that each readout channel corresponds to a total strips length of 3m.
  
Each CRP unit is independently suspended  by 3 stainless steel ropes. The  vertical level of each CRP unit can then be automatically adjusted with respect to the LAr level via 3 suspension feedthroughs, electrically operated from outside. A Slow Control feedthrough + chimney, one per CRP unit, is used for the signals readout for level meters and the temperature probes and to apply the HV bias on the two sides of the LEMs  and on the extraction grid.

The extraction grid integrated in the CRP  is made by an array X and Y oriented stainless steel wires, 0.1mm in diameter with 3.125mm pitch.  Wires, ~3 long in X and ~3m long in Y directions, have sags minimized to ~0.1mm by supporting them by X and Y oriented suspending combs blades ( Figure~\ref{fig:Wires_comb}) inserted between anode planes of $1m \times 1m$ size. The blade array, penetrating in the liquid, splits in sectors the LAr surface helping in maintaining it still.

\begin{cdrfigure}[DUNE CRP units $3m\times 3m$).]{CRP_unit1}{ Two DUNE CRP $3m\times 3m$ units side by side. On the left one of the 79 equal CRP units, on the right the 1st CRP unit with a chamfered LEM/Anode Sandwich for the insertion of the HV FT.}
\includegraphics[width=.7\linewidth]{DUNE12_units}
\end{cdrfigure}

\begin{cdrfigure}[Signal collection in the X and Y views by the 3 SFT chimneys.]{CRP_unit2}{Signal collection in the X and Y views of the  $3\times3$ $m^2$ DUNE CRP unit by the 3 SFT chimneys.}
\includegraphics[width=.5\linewidth]{DUNE12_CRP}
\end{cdrfigure}

\begin{cdrfigure}[Wires hanging comb blade.]{Wires_comb}{Wires hanging comb blade.}
\includegraphics[width=.6\linewidth]{Wires_comb.png}
\end{cdrfigure}

The numbers of components and the parameters for the 12kT (15kT) dual-phase  LArTPC are summarized in Tables~\ref{tab:DP_params} and~\ref{tab:DP_numbers}.

 
\begin{cdrtable}[Sizes and Dimensions for the 12kT (15kT) dual-phase LAr TPC]{lll}{DP_params}{Sizes and Dimensions for the 12kT (15kT) dual-phase  LAr TPC}  Item & Value(s) &  \\ \toprowrule
Active volume width and length & W = 12m &  L = 60m \\ \colhline
Active volume height &  H = 12~m (H = 15~m)  &  \\ \colhline
Active volume/LAr mass & 8640 (10800)~m$^3$ &  12096 (15120) metric ton \\ \colhline
Field ring vertical spacing & 200~mm  \\ \colhline
Field ring tube diameter & 140~mm \\ \colhline
Anode plane size & W = 12m & L = 60~m \\ \colhline
CRP unit size & W =3m & L = 3~m  \\ \colhline
HV for vertical drift & 600 - 900~kV \\ \colhline
Resistor value & 100~M$\Omega$ \\ 
\end{cdrtable}


\begin{cdrtable}[Quantities of Items for the 12kT (15kT) dual-phase LArTPC]{ll}{DP_numbers}{Quantities of Items for the 12kT (15kT) dual-phase  LAr TPC}  Item & Number    \\ \toprowrule
Field rings & 60  (75)  \\ \colhline
CRP units & 4 $\times$ 20 = 80 \\ \colhline
LEM/Anode sadwiches per CRP unit & 36 \\ \colhline
LEM/Anode sandwiches (total) & 2880 \\ \colhline
SFT chimneys / CRP unit & 3 \\ \colhline
SFT chimneys (total) & 240 \\ \colhline
Read-out channels / SFT chimney & 640  \\ \colhline
Read-out channels (total) & 153600 \\ \colhline
Suspension FT / CRP unit & 3  \\ \colhline
Suspension FTs (total) & 240  \\ \colhline
Slow Control FT / sub-anode & 1  \\ \colhline
Slow Control FTs (total) & 80 \\ \colhline
HV feedthrough & 1  \\ \colhline
Voltage degrader resistive chains & 4 \\ \colhline
Resistors (total) & 240 (300)  \\ \colhline
PMTs (total) & 180 (1 / 4~m$^2$) \\ 
\end{cdrtable}