\chapter{Summary of the DUNE Detectors}
\label{ch:physics-summary}

The DUNE experiment is a world-leading international physics
experiment, bringing together the world's neutrino community as well
as leading experts in nucleon decay and particle astrophysics to
explore key questions at the forefront of particle physics and
astrophysics. The highly capable detectors will enable a
large suite of new physics measurements with potential groundbreaking
discoveries.

The Far Detector (FD) will be located deep underground at the 4850L of
SURF and have a fiducial mass of 40-kt to perform sensitive studies of
long-baseline oscillations with a 1300~km baseline as well as a rich
astroparticle physics programme and nucleon decay searches. The FD
will be composed of four identical modules, each instrumented with
Liquid Argon Time Projection Chambers (LArTPC). The LAr
TPC provides excellent tracking and calorimetry performance, hence
ideal for massive neutrino detectors such as DUNE which require a high
signal efficiency and effective background discrimination, an
excellent capability to identify and measure precisely neutrino events
over a wide range of energies, and an excellent reconstruction of the
kinematical properties with a high resolution. The full imaging of
events will allow to study neutrino interactions and other rare events
in an unprecedented way.

The spectrum and flavor composition of the neutrino beam will be
measured with high precision in the Near Detector (ND) in order to
reach the ultimate sensitivity for the long-baseline neutrino
oscillation studies.  The separation between fluxes of neutrinos and
antineutrinos requires a magnetised neutrino detector to charge
discriminate electrons and muons produced in the neutrino charged
current interactions.  This is the primary role of the ND, however,
being exposed to an intense flux of neutrinos will also provides the
opportunity to collect unprecedentedly high statistics of neutrino
interactions for an extended science program.  The near detector will
therefore provide an opportunity for a wealth of fundamental neutrino
interaction measurements, which are an important part of the secondary
scientific goals of the DUNE collaboration.  The reference design for
the neutrino near detector (NND) design is the NOMAD-inspired
fine-grained tracker (FGT). The subsystems of NND comprise a central
straw-tube tracker and an electromagnetic calorimeter embedded in a
0.4-T dipole field. The steel of the magnet yoke will be instrumented
with muon identifiers.
