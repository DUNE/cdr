\chapter{Summary of DUNE Detectors}
\label{ch:physics-summary}

The DUNE experiment is a world-leading international physics
experiment, bringing together the world's neutrino community as well
as leading experts in nucleon decay and particle astrophysics to
explore key questions at the forefront of particle physics and
astrophysics. The highly capable detectors will enable a
large suite of new physics measurements with potential groundbreaking
discoveries.

The Far Detector (FD) will be located deep underground at the 4850L of
SURF and have a fiducial mass of 40-kt to perform sensitive studies of
long-baseline oscillations with a 1300~km baseline as well as a rich
astroparticle physics program and nucleon decay searches. The FD
will be composed of four identical cryostats, each instrumented with
Liquid Argon Time Projection Chambers (LArTPC). The LArTPC provides
excellent tracking and calorimetry performance ideal for the massive
DUNE detectors which require high signal efficiency, effective
background discrimination, excellent capability to precisely measure
neutrino events and high resolution reconstruction of kinematical
properties. The full imaging of events will allow study of neutrino
interactions and other rare events to unprecedented levels.

The spectrum and flavor composition of the neutrino beam will be
measured with high precision in the Near Detector (ND) in order to
reach the ultimate sensitivity for the long-baseline neutrino
oscillation studies.  The separation between fluxes of neutrinos and
antineutrinos requires a magnetized neutrino detector to charge
discriminate electrons and muons produced in the neutrino charged
current interactions.  This is the primary role of the ND, however,
being exposed to an intense flux of neutrinos will also provide the
opportunity to collect unprecedentedly high neutrino
interaction statistics for an extended science program.  The near detector will
provide an opportunity for a wealth of fundamental neutrino
interaction measurements, which are an important part of the secondary
scientific goals of the DUNE collaboration.  The reference design for
the neutrino near detector (NND) design is the NOMAD-inspired
fine-grained tracker. The NND subsystems consist of a central
straw-tube tracker and an electromagnetic calorimeter embedded in a
0.4-T dipole field. The steel of the magnet yoke will be instrumented
with muon detectors.
