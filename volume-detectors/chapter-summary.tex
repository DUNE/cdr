\chapter{Summary of DUNE Detectors}
\label{ch:detectors-summary}

The DUNE experiment is a world-leading, international physics
experiment, bringing together a global neutrino community as well as
leading experts in nucleon decay and particle astrophysics to explore
key questions at the forefront of particle physics and
astrophysics. The massive, high-reolution near and far detectors will
enable an extensive suite of new physics measurements that are
expected to result in groundbreaking discoveries.

The far detector will be located deep underground at the 4850L of
SURF.  Its 40-kt fiducial mass of LAr will enable sensitive studies of
long-baseline oscillations with a 1,300~km baseline, as well as a rich
program in astroparticle physics and nucleon decay searches.  The far
detector configuration consists of four LArTPCs.  They provide
excellent tracking and calorimetry performance, high signal efficiency
and effective background discrimination, all of which converge to
provide an overall excellent capability to precisely measure neutrino
events and reconstruct kinematical properties with high
resolution. The full imaging of events will enable study of neutrino
interactions and other rare events to unprecedented levels.


The magnetized, high-resolution near detector will measure the
spectrum and flavor composition of the neutrino beam extremely
precisely, due to its ability to separate neutrino and antineutrino
fluxes, as well as its ability to discriminate neutrino flavor through
charge discrimination of electrons and muons produced in the neutrino
charged current-interactions enabling DUNE to reach unprecedented
sensitivity in its long-baseline neutrino oscillation studies.  This
is the primary role of the near detector, however, its exposure to an
intense flux of neutrinos will provide an opportunity to collect
unprecedentedly high neutrino interaction statistics, making possible
a wealth of fundamental neutrino interaction measurements, an
important component of the DUNE Collaboration's ancillary scientific
goals.
