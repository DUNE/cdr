\chapter{Software and Computing}
\label{ch:detectors-sc}

\section{Overview}

\fixme{Might want to add an overview of chapter. Anne 5/7}

\section{Computing Infrastructure}
\label{sec:detectors-sc-infrastructure}

%\subsection{Overview}
There are many factors that influence various characteristics of the
data (e.g. rates, volume etc) to be collected and processed in DUNE.
A special Annex document for data rates was created in order to
provide a systematic reference to the parameters and assumptions used
in the estimating these characteristics (``\anxrates'').  The Annex
contains information on both the Reference Design and Alternative
Design versions of the Far Detector.

\subsection{Raw Data Rates}
\label{sec:detectors-sc-infrastructure-data-rates}


\subsubsection{Types of data (using Supernova Burst example)}
DUNE is a multipurpose apparatus and will pursue a variety of physics goals during its operation.
This will be reflected in different
characteristics of data streams processed and collected in real time
as well as off-line, and different strategies and algorithms for handling
these streams.

As one example, consider the difference between
neutrino oscillation physics with beam neutrinos on one hand and search for
Supernova neutrino bursts (SNB) on the other.  Signals produced by
``beam events'' will be characterized by total energy in the GeV
range, so that various aspects of handling the signal and the data (e.g. thresholds for zero suppression etc)
can be optimized for  minimum-ionizing particles.

By comparison, the energy scale of signals
produced by SNB neutrino interactions is in the range of tens of MeV.
For that reason, it is expected that lower thresholds will need to be applied while processing these data in real time,
which will result in considerable contribution from radiological backgrounds.
For this readon, the data rate that needs
to be handled in the process of SNB search can be expected to be quite significant. 

Another differentiating SNB feature is that \textit{multiple neutrinos are
  expected to arrive and interact in the detector volume} during the
possible rare Supernova burst event within seconds from each other, as
opposed to a single vertex produced by a beam neutrino (or any other
localized interactions and/or decays). This opens an opportunity to apply the DAQ architecture
presented in \ref{sec:detectors-fd-ref-daq} to make these data ``self-triggering'', i.e.
to use the buffer memory in the LArTPC detector readout to detect a corresponding
signature in the data stream and trigger recording of the potential SNB event.

Characteristic
time scale for such SNB data capture will be $\sim$\SI{10}{\second}.
Given the large amount of
data arriving within this time period (see the Annex) and practical limits on the bandwidth of
the connection between the RCE data processors and frontend computers in DAQ, local
storage attached to data processors will be necessary to record the
data at full-stream (no zero-suppression).
The buffer
in the data processor will not have sufficient capacity due to design and cost considerations (it can effectively
buffer about \SI{0.4}{\second} of streaming data which is likely enough for the trigger decision but not for the complete
supernova event capture).
Preliminary estimates indicate that a storage device such as a SSD (one or two per board) will have
speed sufficient for this purpose.
With trigger properly tuned, the number of times data are written to
the SSD can be kept sufficiently low so as to ensure their
longevity.
Once captured in this manner, the data can then be
transmitted to the rest of the farm within the available bandwidth.

The core elements of the DAQ system now exist as prototypes.
The system as a whole with capabilities such as descibed above is
in the conceptual design stage and information will be added to DUNE planning documents as it is developed.




%This section focuses on data streams present in oscillation physics studies with beam
%neutrinos, since these data will constitute the bulk of what's committed
%to mass storage, transmitted over networks, processed offline and in general have most significant
%infrastructure and cost implications.
%Issues and parameters related to other classes of data
%are covered in ``\anxrates'', and also in the DAQ and other sections.

\subsubsection{Assumptions}
\label{sec:detectors-sc-infrastructure-assumptions}
According to the present baseline design, the Far Detector will
consist of four identical modules of \tpcmodulemass each.  For
purposes of estimating data characteristics in this document the issue
of possible variations in the design of these modules shall not be
addressed. A few basic assumptions will need to be used:
\begin{itemize}
\item Estimates presented below correspond to the ``full detector'',
  i.e. is effectively normalized to \dunedetectormass.
\item Accelerator spill cycle is \beamspillcycle with beam expected
  for \beamrunfraction of each calendar year.
\item Zero-Suppression thresholds will be set at levels that preserve
  signals from minimum-ionizing particles while effectively removing
  data due to electronics noise.
\item The DAQ will be able to trigger based on spill times and will be
  able to reject isolated $^{39}Ar$ decays on at least a per-APA
  basis. (for DAQ details see~\ref{sec:detectors-fd-ref-daq})
\end{itemize}

\subsubsection{Far Detector LAr TPC}
The information presented below is based on the parameters listed in
``\anxrates''.
\begin{itemize}
\item TPC channel count: \dunenumberchannels (i.e. four times
  \daqchannelspermodule which is the channel count for each \tpcmodulemass module)
\item Maximum drift Time: \tpcdrifttime
\item Number of drift time windows in one DAQ readout cycle: \daqdriftsperreadout
\item ADC clock frequency: approx. \daqsamplerate
\item ADC resolution (bits): 12
\end{itemize}

In addition to these basic parameters, there are other factors
affecting data rates and volumes, such as implementation of Zero
Suppression (ZS) in the DAQ RCE processors,
contribution from Radiological and Cosmological Backgrounds, and DAQ
trigger configuration (cf. the case of low-energy events).

Non-ZS maximum event size (corresponding to a snapshot of the complete TPC) can be calculated as a product of the following numbers:
\begin{itemize}
\item Channel count
\item Number of ADC samples per total drift (collection) time
\item Drift time windows in one DAQ cycle
\item ADC resolution
\end{itemize}

This results in a total of \dunefsreadoutsize worth of TPC data in one readout.

Zero suppression greatly reduces the event size.  An overly
conservative estimate (leaning to the higher end of the range of
values) based on a LArSoft Monte Carlo simulation of GeV-scale events
suggests a zero-suppressed and uncompressed event size of
$\sim$\beameventsize.  After compression this event size is expected
to be $\sim$\beameventsizecompressed.  This particular simulation
employed a less than optimal schema for packing data and it is
expected that with optimization these sizes can be further reduced.

Some of the driving zero-suppressed (ZS) and full-stream (FS) annual
data volumes are summarized in Table~\ref{tab:sc-zs-summary}. It is
important to note that the numbers in the row characterizing $^{39}$Ar
are given for information only and do not represent our estimates of
actual data to be committed to mass storage.
Once DAQ-level rejection of isolated $^{39}Ar$ decay events is invoked
a residual amount of data is accepted when the decay is accidentally
coincident with beam-$\nu$ activity.
The data required to record this background is reduced to 3\% of the
``with-beam-$\nu$'' estimate of table~\ref{tab:sc-zs-summary} and is
thus negligible being an order of magnitude smaller than the data
associated with the beam neutrino interactions themselves.

%%%% I think this simply restates our simple assumption that isolated
%%%% 39Ar can be killed by the DAQ.
%%%%  The "high threshold" approach makes people nervous, so I don't
%%%%  think we need to invoke it.
% There is a range of possible approaches to data volume reduction in
% real time without losing much of information due to low-amplitude
% signals, which may be important for more precise particle
% identification or in search for unusual signatures that may exist due
% to new phenomena. For example, implementation of the ``trigger
% stream'' in the DAQ system (see~\ref{sec:detectors-fd-ref-daq})
% allows to trigger on event candidates using
% ``high threshold'', while capturing data in the potential event
% without applying any threshold at all. In that case, one can estimate
% the total size of the TPC data to be collected annually by multiplying
% the predicted number of beam neutrino interactions per year (estimated
% here to be \beamrate) by the full stream event size quoted above
% (\dunefsreadoutsize).  This results in an estimate of
% \beamdatayearfs of this type of data per year.


% do not edit, this is generated by dune-params

\begin{cdrtable}[Annual data volume estimations for zero-suppressed (ZS) data from various sources.]{rrrrr}{tab:sc-zs-summary}{Annual data volume estimations for zero-suppressed (ZS) data from various sources. An additional full-stream (FS) data estimation is given for supernova burst (SNB).}
Source & Event Rate & Event Size & Data Rate & Annual Data Volume \\ \toprowrule
$^{39}Ar$ (ZS) & \SI[round-mode=places,round-precision=1]{11.4}{\mega\hertz} & \SI[round-mode=places,round-precision=0]{36.0}{\byte} &
\SI[round-mode=places,round-precision=1]{0.4104}{\giga\byte\per\second} &  \SI[round-mode=places,round-precision=0]{12.95096242}{\peta\byte}\\
\colhline
cosmic-$\mu$ (ZS) & \SI[round-mode=places,round-precision=3]{0.258947264}{\hertz} &
\SI[round-mode=places,round-precision=1]{2.5}{\mega\byte} & \SI[round-mode=places,round-precision=1]{647.36816}{\kilo\byte\per\second} &
\SI[round-mode=places,round-precision=0]{20.4289491035}{\tera\byte} \\
\colhline
beam-$\nu$ (ZS) & \SI[round-mode=places,round-precision=0]{8770.19567714}{\per\year} & \SI[round-mode=places,round-precision=1]{2.5}{\mega\byte} &
\SI[round-mode=places,round-precision=2]{0.694791666667}{\kilo\byte\per\second} & \SI[round-mode=places,round-precision=0]{21.9254891928}{\giga\byte} \\
beam-$\nu$ (FS) & \SI[round-mode=places,round-precision=0]{8770.19567714}{\per\year} & \SI[round-mode=places,round-precision=1]{24.8832}{\giga\byte} &
\SI[round-mode=places,round-precision=0]{6.915456}{\mega\byte\per\second} & \SI[round-mode=places,round-precision=0]{218.230533073}{\tera\byte} \\
\colhline
SNB cand. (ZS) & \SI[round-mode=places,round-precision=0]{12.0}{\per\year} & \SI[round-mode=places,round-precision=1]{4.104}{\giga\byte} &
\SI[round-mode=places,round-precision=0]{1560.60828103}{\byte\per\second} & \SI[round-mode=places,round-precision=0]{49.248}{\giga\byte} \\
SNB cand. (FS) & \SI[round-mode=places,round-precision=0]{12.0}{\per\year} & \SI[round-mode=places,round-precision=1]{46.08}{\tera\byte} &
\SI[round-mode=places,round-precision=1]{17.5226192958}{\mega\byte\per\second} & \SI[round-mode=places,round-precision=0]{552.96}{\tera\byte} \\
\end{cdrtable}

\subsubsection{Far Detector Photon Detector (PD)}
There are variations in the basic parameters of the Photon Detector
currently in the R\&D stage, so the numbers presented below need to be
considered as ballpark values to be made more precise at a later time:

\begin{itemize}
\item Readout channel count: \num{24000} (i.e. four times \num{6000} which is the channel count for each 10kt module)
\item Trigger rate is uncertain at this point due to ongoing investigation; one approach that exists assumes 1 trigger per spill cycle
\item ADC resolution (bits): 12
\item ADC digitization frequency: \SI{150}{\MHz}
\end{itemize}
It is assumed that a few dozen samples will be recorded in each
channel, and there will be zero suppression of channels with signals
below a chosen threshold, resulting in an order of magnitude reduction
of the data volume.  This results in \SI{360}{\kilo\byte} per spill cycle, and should
be considered negligible from the point of view of requirement to data
handling, compared to other data sources.

\subsubsection{Near Detector Data Rates}
Once again, the Near Detector is subject to an intense R\&D effort and
its parameters are being optimized and not fixed at this point. The
most relevant parameters of the Fine-Grained Tracker (FGT):
\begin{itemize}
\item   Straw Tube Tracker (STT) readout channel count: \ndsstchannels
\item STT Drift Time: 120ns
\item STT ADC clock frequency and resolution (bits): \SI{3}{\ns} intervals, 10 bit
\item ECAL channel count: \ndecalchannels
\item Muon Detector Resistive Plane Chambers (RPC) channel count: \ndmuidchannels
\item Average expected event rate per spill: \textasciitilde 1.5
\end{itemize}
\ Since there are large uncertainties in estimates of the detector
occupancy levels per event, broad assumptions must be made to roughly
estimate its data rate. Current estimate (as quoted in the Near
Detector section of the ``Expected Data Rates'' Annex is
\textasciitilde \nddatarate, which translates into \textasciitilde
\SI{30}{\tera\byte\per\year}.

% Assuming that 10\% occupancy in the STT and 40 samples
%per trigger, one arrives to \textasciitilde 1MB of data per event. Under same assumption, ECAL will contribute \textasciitilde 0.25MB
%and the Muon Detector \textasciitilde  0.75MB.
%Based on these parameters, the upper limit of the ND data rate can be estimated as 1.5MB/s. This translates into \textasciitilde 45TB/year. 

\subsection{Processed Data}
\label{sec:detectors-sc-infrastructure-processed-data}
For the purposes of this document, processed data is defined as most
data which is not considered ``raw'', i.e. it's data derived from raw
(including possibly multiple stages of calibration and reconstruction)
as well as data produced as a result of Monte Carlo studies.

There are uncertainties in anticipated quantities of all of these
types of data. Table \ref{tab:sc-zs-summary} contains a range of
numbers reflecting limiting cases such as ZS vs FS.  Depending on the
exact optimum readout strategy, an annual raw data volume of
\SI{1}{\tera\byte} to \SI{1}{\peta\byte} may be collected.  Assuming
that the data will undergo a few processing stages, one can expect the
need to handle as much as \textasciitilde \SI{2}{\peta\byte} of data
annually for reconstruction and a lesser volume for final analysis
purposes.

%\fixme{Amir is right, as we are mostly guessing the MC data volume - but I don't have more guidance at this point. We are already stating it's more or less a guess --mxp--}
For Monte Carlo, at the time of writing typical annual volume of data
produced has been of the order of a few tens of terabytes.  Initial
expectations are that the MC sample size for beam events will need to
be 10--100$\times$ that of the data.  With Collaboration growing
and more detailed studies (e.g. of systematics) are undertaken, our
expectation is that this estimate will increase.

\subsection{Computing Model}
\label{sec:detectors-sc-infrastructure-computing-model}

\subsubsection{Distributed Computing}

Given the fact the Collaboration is large and widely dispersed
geographically, a fully distributed approach to computing is planned,
based on experience gained during the operation of the LHC
experiments. This includes not only ``traditional'' Grid technologies
in the form they were deployed during the first decade of this
century, but also more recent expansion into Cloud Computing and Big
Data methodology. This will allow the DUNE Collaboration to better
leverage resources and expertise from many of its member institutions
and improve the overall long-term scalability of its computing
platform.

DUNE will operate a distributed network of federated resources, for
both CPU power and storage capability. This will allow for streamlined
incorporation of computing facilities as they become available at
member institutions, and thus is particularly amenable to accommodate
staged construction and commissioning of the detector subsystems. A
modern Workload Management System will be deployed on top of Grid and
Cloud resources to provide computing power to DUNE researchers.

\subsubsection{Raw Data Transmission and Storage Strategy}
FNAL will be the principal data storage center for the experiment. It
will serve as a hub where the data from both the Facility (e.g. beam
and target) and the various detector systems (such as the Far and Near
Detectors) are collected, catalogued and committed to mass
storage. This will obviously require transmission of data over
considerable distances (certainly for the Far Detector). In addition,
the DAQ systems of the Far Detector are being designed to be located
in the vicinity of the Far Detector (in the cavern), which results in
an additional step of transmitting the data from 4850L to the surface.

Raw data to be collected from the detectors in DUNE are considered
``precious'' due to high cost of operating the both the facility at
FNAL and the detectors that are part of DUNE. This leads to three
basic design elements in the data transmission and storage chain:
\begin{itemize}
\item Buffering:
\begin{itemize}
\item Adequate buffers will be provided for the DAQ systems to
  mitigate possible downtime of the network connection between 4850L
  and the surface.
\item Buffers will be provided at the surface facility to mitigate
  downtime of the network connection between the Far Site and FNAL.
\end{itemize}
\item Robust transmission: data transfer needs to be instrumented with
  redundant checks (such as checksum calculation), monitoring, error
  correction and retry logic.
\item Redundant replicas: it is a common practice in industry and
  research (cf. the LHC experiments) to have a total of three copies
  of ``precious'' data, which are geographically distributed.  Such
  geographical distribution of the replicas may include countries
  other than the United States, where the data will be collected.
  This provides protection against catastrophic events (such as
  natural disasters) at any given data center participating in this
  scheme, and facilitates rebuilding (``healing'') lost data should
  such event does happen.
\end{itemize}
% \fixme{Amir - yes I do mean industry like in industry, what's wrong with that? --mxp--}


\subsubsection{Data Management}
\label{sec:detectors-sc-infrastructure-computing-model-data-mgt}

Data will be placed into mass storage at FNAL. Along the lines
described above, additional copies (replicas) will be distributed to
other computing centers possessing sufficient resources.  A single
additional copy does not necessarily need to reside in its entirety on
a single data center; the replicas can be ``striped'' across a few
data centers if that becomes optimal at the time of implementation of
the Computing Model. For example, consideration is given to both
Brookhaven National Laboratory and NERSC as candidates for the
placement of extra replicas.

Recent progress in network and storage technologies made possible
\textit{federation of storage} across multiple data centers located at
member institutions. In this approach, data can be effectively shared
and utilized via the network (cf. ``data in the Cloud''). One example
of an advanced system of this type of XRootD.

For data distribution, a combination of managed data movement between
sites (such as ``dataset subscription'', primarily for managed
production), and a network of XRootD servers to cache processed data
and for analysis will be used.  A file catalog and a Meta-Data system
will be required for efficient data management at scale, and an effort
will be made to leverage experience of member institutions in this
area, making an effort to reuse existing systems or design ideas where
possible.

\subsection{Computing Implications of the Alternative Design}
\label{sec:detectors-sc-alternate}
Parameters of the alternative design of the Far Detector (based on the
dual-phase technology) are listed in Chapter 2 of the
``\anxrates''. Here's a brief summary of some of its characteristics:

\begin{itemize}
	\item Readout channel count: 614,400 (i.e. four times 153,600 which is the channel count
	for each 10kt module)
	\item Drift Time: 7.5ms
\end{itemize}
\
For the Photon Detector in the dual-phase design:
\begin{itemize}
	\item Readout channel count: 720 (i.e. four times 180 which is the channel count
	for each 10kt module)
\end{itemize}
\ According to some estimates listed in the Annex, the ``Full Stream''
readout will produce 16.09GB of data for each candidate event. This is
about 65\% of the data volume in one readout cycle, in the reference
design.  Although signal processing strategies may be implemented
differently in the alternative design, it can be argued that the total
data rate will be of the same order of magnitude or less than in the
reference design.
% \fixme{Can still improve here but I doubt we'll get much more detail in time --mxp--}

\section{Near Detector Physics Software}
\label{sec:detectors-sc-nd-physics-software}

A longer description of the current status of the near detector simulation and reconstruction are given
in \anxreco, and only an abbreviated summary is given here.

Two approaches are being simultaneously pursued for the simulation of
the DUNE Near Detector.  The first is a fast Monte Carlo based on
parameterized detector responses. The GENIE~\cite{GENIE} generator is
used to model the interactions of neutrinos with nuclei in the
detector, and a parameterization of the achieved NOMAD reconstruction
performance is used to model the detector response.  The second
approach is a full GEANT4-based simulation, which is under
development.  The fast Monte Carlo tool is based on work done for the
Far Detector~\cite{Adams:2013qkq} (Appendix A.3) and is capable of
rapidly evaluating the sensitivity of the detector design for a broad
variety of analyses targeting specific final states.  The full
GEANT4-based simulation and subsequent reconstruction chains will be
used to inform the parameterized responses of the fast Monte Carlo, as
well as being indispensable tools for simulating and extracting
results from the Near detector.
Figure~\ref{fig:ndeventdisplaychaptersc} shows the trajectory of a
negatively-charged muon with an initial momentum of 1~GeV propagating
in the straw tube tracker, as simulated using GEANT4.

\begin{cdrfigure}[The trajectory of a 1 GeV $\mu^-$ simulated in the near detector.]{ndeventdisplaychaptersc}
{The trajectory of a 1 GeV $\mu^-$ produced by the GEANT4 simulation of the near detector's straw-tube tracker.}
\includegraphics[width=0.7\textwidth]{hisoftndmuon.png}
\end{cdrfigure}

\section{Far Detector Physics Software}
\label{sec:detectors-sc-physics-software}

Longer descriptions of the single-phase and dual-phase far detector
simulations and reconstruction are given in \anxreco, and only an
abbreviated summary is given here.

%\subsection{Simulation}
%\label{sec:detectors-sc-physics-software-simulation}
% Beam simulation in the Beam Requirements chapter
%\subsubsection{Beam Simulation}
%\label{sec:detectors-sc-physics-software-simulation-beam}
% ND simulation and reconstruction in the ND chapter
%

\subsection{Far Detector Simulation}
\label{sec:detectors-sc-physics-software-simulation-fd}

Detailed GEANT4-based~\cite{GEANT4:NIM,GEANT4} Monte Carlo simulations
have been developed for the single-phase and dual-phase far detector
designs, incorporating both the TPC modules and the photon detection
systems. These simulations provide a basis for detailed studies of
detector design and performance, and also enable the development of
automated event-reconstruction algorithms.

The single-phase detector simulation is implemented in
LArSoft~\cite{Church:2013hea}, which provides a common simulation
framework for LArTPC experiments.  LArSoft is based on the {\it art}
framework~\cite{Green:2012gv}, and is supported by the Fermilab
Scientific Computing Division.  The comparison of data from
ArgoNeuT~\cite{Anderson:2012vc,Anderson:2012mra} with LArSoft
simulations gives confidence in the reliability of the detector
simulation.  Future data from
LArIAT~\cite{Adamson:2013/02/28tla,Cavanna:2014iqa},
MicroBooNE~\cite{Chen:2007ae,Jones:2011ci,microboonecdr}, and the
35-ton prototype (Sec.~\ref{sec:proto-35t}) will allow further tuning
of the LArSoft simulation as experience is gained.  The dual-phase
detector simulation and hit-level reconstruction are based on the
Qscan~\cite{lussi:thesis} package, which has been developed over the
past decade, and is currently being used for technical design and
physics studies for the \cerndualproto{} program.

Events are generated using either the GENIE~\cite{GENIE} simulation of 
neutrino-nucleus interactions, the CRY~\cite{Cosmic-CRY,Cosmic-CRY-protons,CRY-url} cosmic-ray generator, 
a radiological decay simulator written specifically for LArSoft, using the decay spectra
in Ref.~\cite{docdb-8797}, a particle gun, or one of several
text-file-based particle input sources. GEANT4 is then used to simulate the trajectories
of particles and to model their energy deposition.  
Custom algorithms have been developed to propagate the drifting charge
and scintillation photons through the detector, and to simulate the
response characteristics of the TPC wires, readout electronics, and photon detectors.
Figure~\ref{fig:larsofteventdisplays} shows some examples of simulated 
accelerator neutrino interactions in the MicroBooNE detector.

\begin{cdrfigure}[Simulated neutrino interactions in MicroBooNE]{larsofteventdisplays}
{Examples of accelerator neutrino interactions, simulated by LArSoft in the 
MicroBooNE detector. The panels show 2D projections of different event types.
The top panel shows a $\nu_{\mu}$ charged-current interaction with a stopped muon followed
by a decay Michel electron; the middle panel shows a $\nu_{e}$ charged-current 
quasi-elastic interaction with a single electron and proton in the final state;
the bottom panel shows a neutral-current interaction with a $\pi^{0}$ in the final state
that decayed into two photons with separate conversion vertices.}
\includegraphics[width=\textwidth]{numuCC_annotated.pdf}
\includegraphics[width=\textwidth]{nueQE_annotated.pdf}
\includegraphics[width=\textwidth]{nc_pi0sep_annotated.pdf}
\end{cdrfigure}

\subsection{Far Detector Reconstruction}
\label{sec:detectors-sc-physics-software-reconstruction-fd}

The reconstruction of particle interactions in LArTPC
detectors is an active area of research that has advanced significantly in recent years.
In particular, the analysis of the data from the ICARUS~\cite{Amerio:2004ze,icarus-url,ICARUS-pizero,Antonello:2012hu} 
and ArgoNEUT experiments~\cite{Adamson:2013/02/28tla,argoneut-url,Acciarri:2013met}
required the development of a variety of new reconstruction techniques,
forming the basis for precision neutrino physics measurements.
Accurate reconstruction is needed not only of neutrino scattering events from the beam, but also atmospheric neutrino events,
supernova burst neutrino interactions, and nucleon decay events, each with its own requirements.
With the advance of both single-phase and dual-phase technologies,
and expansion of the experimental program to include MicroBooNE~\cite{Chen:2007ae,microboone-url},
the 35-ton prototype and the CERN test experiments,
the reconstruction tools have grown in both volume and sophistication,
supported by powerful software frameworks such as LArSoft and Qscan.

Fully automated chains of event-reconstruction algorithms
are being developed for the DUNE far detector, both for the single-phase and dual-phase designs.
The first stage of reconstruction involves the processing of the
ADC wire signals, and the identification of pulses, or ``hits,'' in the 
two-dimensional space of wire number and charge arrival time. 
These hits provide the input for a series of pattern-recognition algorithms,
which form 2D and 3D clusters, representing individual particle tracks and showers.
A set of high-level algorithms is then used to reconstruct the 
3D vertex and trajectory of each particle, identify the type of particle, and determine the four-momentum.
While each stage of the reconstruction chain has been implemented, the algorithms -- in
particular those addressing the higher level aspects of reconstruction such as particle identification -- are
rather preliminary and are in active development.


\begin{cdrfigure}[Dual-phase LArTPC reconstructed events for data and MC]{lbnoeventdisplay}
{
%Badertscher:2012dq
Dual-phase LArTPC reconstructed events for data and MC.
Top: Cosmic ray event displays for an hadronic shower candidate.
Bottom: Reconstructed hits for a MC simulation of a 5 GeV $\nu_{\mu}$ interaction.
The secondary particles produced in the two interactions are 
distinguished with different colors based on the MC truth information (blue=muon, green=electron, red=proton, cyan=pion).
From Ref.~\cite{Badertscher:2012dq}.
}
\includegraphics[width=0.99\textwidth]{LArTPC-3L-LEM-cosmics.png}
\includegraphics[width=0.99\textwidth]{LBNO-MC-nu_mu_event.png}
\end{cdrfigure}

\subsubsection{TPC Signal Processing, Hit Finding, and Disambiguation}

The signal-processing steps in the single-phase and dual-phase detectors are similar but
are accomplished with separate software.  Both proceed first by decompressing the raw data
and filtering the noise using a frequency-based filter.  The single-phase signal-processing
algorithm also deconvolves the detector and electronics responses at this step.
Both the single-phase and dual-phase hit-finding algorithms then subtract the baselines
and fit pulse shapes to the filtered raw data.  The hit-finding algorithms are able to
fit multiple overlapping hits.  The main parameters of the hits are the arrival time, the
integrated charge, and the width.  A raw ADC sum is also retained in the description of a hit, which
often carries a better measurement of the total charge.
The current algorithms are found to perform well in 
ArgoNeuT analyses~\cite{Anderson:2012vc} for the single-phase software and during several phases 
of R$\&$D and prototyping on small-scale dual-phase LAr-LEM-TPC 
setups~\cite{Badertscher:2008rf,Badertscher:2012dq}.
Figure~\ref{fig:lbnoeventdisplay} shows example event displays of the reconstructed hits 
in both real and simulated data. 

The wrapping of induction-plane wires in the single-phase APA design
introduces an additional discrete ambiguity in the data by connecting multiple wire
segments to each DAQ channel. A ``disambiguation'' algorithm is used to break the
ambiguity and determine which wire segment generated the charge on each hit.
The algorithm forms associations between the collection and induction views,
identifying ``triplets'' of hits that have intersecting wire segments
and consistent arrival times. In most events, the majority of hits are
associated with a single wire segment, and can be trivially disambiguated.
The remaining hits are then disambiguated by clustering them with trivially disambiguated hits.

\subsubsection{Photon Detector Signal Reconstruction}

Photon detector signals are processed in similar ways to those on the TPC wires.
Noise is filtered out, and hits are identified as pulses above the pedestal.
Hits are grouped together into clusters in time, called ``flashes,'' for subsequent
association with clusters in the TPC.  Each flash has a time, a total integrated charge, and a position
estimate.  The time of an interaction is important in order to help reject cosmic-ray events
and also to determine the absolute position of an event along the drift direction.  This position
is important in order to correct for finite electron lifetime effects for proper charge measurement,
which is important for particle identification and extraction of physics results.  Signal events which
can be out of time from the beam include atmospheric neutrinos, supernova burst neutrinos, and proton decay
interactions.

\subsubsection{TPC Pattern Recognition}

The reconstruction of particles in 3D can be accomplished either by forming 2D clusters
and associating them between views, or by first associating 2D hits between views
and then clustering the resulting 3D hits. 
The clustering of hits in LArTPC detectors is a challenging task
due to the variety and complexity of event topologies.
However, several automated 2D and 3D pattern-recognition algorithms have been 
implemented using a range of techniques.

One promising suite of reconstruction tools is the 
PANDORA software development kit~\cite{Marshall:2013bda,Marshall:2012hh}
which provides fully automated pattern recognition for both single-phase 
and dual-phase technologies. 
PANDORA implements a modular approach to pattern recognition,
in which events are reconstructed using a large chain of algorithms. 
Several 2D pattern-recognition algorithms are first applied
that cluster together nearby hits based on event topology.
The resulting 2D clusters are then associated between views
and built into 3D tracks and showers, modifying the 2D clustering 
as needed to improve the 3D consistency of the event. 
Vertex-finding algorithms are also applied,
and neutrino events are reconstructed by associating the 
3D particles to the primary interaction vertex.

Figure~\ref{fig:pandoraefficiency} shows the current efficiency for reconstructing
the leading final-state lepton as a function of its momentum
for 5\,GeV $\nu_{e}$ and $\nu_{\mu}$ charged-current interactions
simulated in the MicroBooNE detector; the DUNE single-phase detector is expected to 
perform similarly, although the multiple TPC geometry with wrapped wires requires
additional software effort.
%In both samples, the reconstruction efficiency increases rapidly with momentum,
%rising above 90\% at 500\,MeV and reaching approximately 100\% at 2\,GeV.
Figure~\ref{fig:recoannexpandoravertexresolution} shows the spatial resolution for
reconstructing the primary interaction vertex in these 5-GeV event samples,
projected onto the $x$, $y$ and $z$ axes. An estimate of the overall vertex 
resolution is obtained by taking the 68\% quantile of 3D vertex residuals, 
which yields 2.2\,cm (2.5\,cm) for $\nu_{\mu}$ CC ($\nu_{e}$ CC) events.

\begin{cdrfigure}[PANDORA reconstruction efficiency]{pandoraefficiency}
{Reconstruction efficiency of Pandora pattern recognition algorithms
 for the leading final-state lepton in 5-GeV $\nu_{\mu}$ CC (left) and
 $\nu_{e}$ CC (right) neutrino interactions, plotted as a function of
 the lepton momentum. The reconstruction performance is evaluated
 using the MicroBooNE detector geometry. }
\includegraphics[width=0.49\textwidth]{pandora_uboone_efficiency_5GeV_numucc.pdf}
\includegraphics[width=0.49\textwidth]{pandora_uboone_efficiency_5GeV_nuecc.pdf}
\end{cdrfigure}

\begin{cdrfigure}[PANDORA vertex resolution]{recoannexpandoravertexresolution}
{Distribution of 2D residuals between reconstructed and simulated interaction
 vertex for 5-GeV $\nu_{\mu}$ CC (left) and $\nu_{e}$ CC (right) interactions in the MicroBooNE detector.
 The $x$ axis is oriented along the drift field, the $y$ axis runs parallel 
 to the collection plane wires, and the $z$ axis points along the beam direction.}
\includegraphics[width=0.49\textwidth]{pandora_uboone_vertex_resolution_numucc.pdf}
\includegraphics[width=0.49\textwidth]{pandora_uboone_vertex_resolution_nuecc.pdf}
\end{cdrfigure}

\subsubsection{Track Fitting and Shower Measurement}

After the pattern recognition stage, a series of high-level reconstruction
algorithms is applied to the 2D and 3D clusters,
which fit the trajectories of particle tracks and measure the
spatial and calorimetric properties of electromagnetic and hadronic showers.
Several high-level techniques have been demonstrated 
for use in LArTPC detectors using both real and simulated data.

The Kalman filter technique~\cite{kalman} is well-established in high-energy physics,
and has been applied to 3D track reconstruction in liquid argon by ICARUS~\cite{Ankowski:2006ts}.
The technique incorporates the effects of multiple Coulomb scattering,
enabling a scattering-based measurement of the track momentum,
which is shown by ICARUS to have a resolution as good as $\Delta p/p \approx 10\%$ 
for the most favourable track lengths.
The data from ICARUS have also been used to develop a precise
track reconstruction algorithm, which builds a 3D trajectory for each track by simultaneously
optimizing its 2D projections to match the observed data~\cite{Antonello:2012hu}.
Another promising track reconstruction technique, based on the local principal curve algorithm,
has been implemented for the dual-phase detector, and is shown to provide 
a precise reconstruction of two-body final states~\cite{Back:2013cva,LAGUNA-LBNO-deliv}. 

A full 3D reconstruction of electromagnetic showers is currently in development.
In the present scheme, the first stage is an examination of clusters 
in terms of their 2D parameters, and a selection of shower-like clusters 
for further analysis. The 3D start position, principal axis,
and shower shape variables are then reconstructed by matching up 2D hits between views.
These 3D parameters, combined with calorimetric information, enable a measurement
of the total shower energy, as well discrimination between electrons
and converted photons based on the ionization energy in the initial part of
the shower. The kinematic reconstruction of final-state neutral pions from their
$\pi^{0} \rightarrow \gamma\gamma$ decays can be performed by 
combining together associated pairs of photons.


\subsubsection{Calorimetry and Particle Identification}

The reconstructed energy of hits follows from the measured charge
after corrections are made for sources of charge loss.  The energy of
physics objects can then be reconstructed by summing the energy of the
associated hits and, when this is combined with reconstructed
trajectory, a measurement of the ionization density $dE/dx$ can be
made, which is an important input to particle identification.  In
order to reconstruct this information, the measured charge on each hit
is first obtained from fits to the pulse shapes.  The charge loss due
to the finite electron lifetime is corrected based on the time of the
event, and the path length corresponding to each hit is calculated
based on the event trajectory.  The effects of recombination, known as
``charge quenching'' are corrected using a modified Box
model~\cite{Thomas:1987zz} or Birks Law~\cite{Birks:1964zz}.  The
identity of a particle track that ranges out in the active detector
volume may be ascertained by analyzing the ionization density $dE/dx$
as a function of the range from the end of the track, and comparing
with the predictions for different particle species.

In a liquid argon TPC, electromagnetic showers may be classified as
having been initiated by an electron or a photon using the $dE/dx$ of
the initial $\sim 2.5$~cm of the shower.  Electron-initiated showers
are expected to have $dE/dx$ of one MIP in the initial part, while
photon-initiated showers are expected to have twice that.  Current
algorithms achieve a performance of 80\% electron efficiency with 90\%
photon rejection, with a higher efficiency for fully-reconstructed
showers.

%Do we have any calorimetry plots?  
% not for the main CDR.  Perhaps the annex


\subsubsection{Neutrino Event Reconstruction and Classification}

Once the visible particles in an event have been reconstructed
individually, the combined information is used to reconstruct and
classify the overall event.  The identification of neutrino event
types is based on a multivariate
analysis~\cite{Back:2013cva,WA105_TDR,LAGUNA-LBNO-deliv,LAGUNA-LBNO-EOI},
which constructs a number of characteristic topological and
calorimetric variables, based on the reconstructed final-state
particles. In the present scheme, a Boosted Decision Tree algorithm is
used to calculate signal and background likelihoods for particular
event hypotheses. The current performance has been evaluated using
fully reconstructed $\nu_{e}$ and $\nu_{\mu}$ charged-current
interactions with two-body final states, simulated in the dual-phase
far detector~\cite{LAGUNA-LBNO-deliv}.  The correct hypothesis is
chosen for 92\% (79\%) of $\nu_{\mu}$ ($\nu_{e}$) quasi-elastic
interactions with a lepton and proton in the final state, and 79\%
(71\%) of $\nu_{\mu}$ ($\nu_{e}$) resonance interactions with a lepton
and charged pion in the final state.  For selected events, the
neutrino energy is estimated kinematically for quasi-elastic
interactions using a two-body approximation, or otherwise a
calorimetric energy measurement is applied.  The calorimetric
reconstruction takes into account the quenching factors of the
different particles, assuming that all hits not associated with the
primary lepton are due to hadronic activity.
Figure~\ref{fig:recoenergynue} shows the resulting energy
reconstruction for $\nu_e$ CCQE and CC1$\pi^{+}$ events.

\begin{cdrfigure}[Reconstruction of electron neutrino energy]{recoenergynue}
{Performance of neutrino energy measurement, evaluated using the dual-phase far detector simulation. 
Distributions of reconstructed versus true neutrino are shown for $\nu_{e}$ CCQE events (left),
assuming two-body kinematics, and $\nu_{e}$  CC1$\pi^{+}$ events (right),
using a calorimetric energy estimation.}
\includegraphics[width=0.49\textwidth]{laguna_lbno_recoenergy_nue_ccqe.pdf}
\includegraphics[width=0.49\textwidth]{laguna_lbno_recoenergy_nue_ccres.pdf}
\end{cdrfigure}

%%%% Do we need a summary?
% not really -- 5 pages is a little tight; this is the summary

