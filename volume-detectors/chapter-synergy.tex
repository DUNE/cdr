\chapter{Synergies Between Far Detector Designs}
\label{ch:detectors-synergy}

\section{Overview}

As discussed in Section~\ref{sec:detectors-strategy-FD}, two
technologies for LArTPCs are being pursued. There are a number of
synergies between these development efforts.

Both the reference and alternative designs for the DUNE far detector
are liquid argon TPCs. The designs assume nearly identical cryostats
(with some differences in the cryostat roof) installed in identical chambers and
supported by identical cryogenic systems. The designs differ mostly in
their approaches to collection and readout of the ionization
signals. In the reference design, the ionization charge is measured by
successive wire planes, two induction and one collection, all immersed
in the LAr. In the alternative design, the charge is extracted from
the liquid to the vapor and then amplified and finally collected on a
2D anode, providing two independent views.

%A list of synergies between far detector subsystems designs includes
Several of the far detector subsystems offer great potential for synergy between the reference and alternative designs, including the
\begin{itemize}
\item Interface to the cryogenics system,
\item High voltage, 
\item Photon detection,
\item Calibration,
\item Underground installation strategies,
\item Local computing infrastructure and DAQ, and
\item Detector modeling and simulation.
\end{itemize}

\section{Interface to the Cryogenics System}

In both designs, ionization electrons have drift lengths on the order
of several meters. In order to reach the required millisecond scale for
electron lifetime, the electronegative impurities in the LAr must be
maintained below the ppb level. The contamination will
come primarily from impurities adsorbed onto the tank and detector element surfaces.
Given that the detector modules will be housed in cryostats of the same design,
using the same industrial LNG cryostat technology and the same cryogenics systems,
the process of understanding these sources and minimizing them
is to a great extent the same for either TPC design. This leads to %potential 
expected synergies in the areas of
\begin{itemize}
\item Electronegative contamination mitigation, 
\item Modeling contamination sources,
\item Contamination migration modeling,
\item Material properties,
\item Filtration	,
\item Design of the cryogenics system,
\item Purity monitoring,	
\item Roof interfaces (hatch, feedthrough, mounts),
\item Grounding and shielding, and
\item Installation spaces and cryogenics system needs.	
\end{itemize}

%\subsection{Drift Electron Lifetime}  <--- can't have one subsection in a section; whole sec is about e drift
%\label{sec:detectors-synergy-lifetime}


For detecting interactions of beam neutrinos, the requirement for
electron lifetime of $>$\,3\,ms derives from the minimum signal to noise
ratio (S/N$\,>\,$9) required for MIP signals on induction plane wires from
interactions near the cathode to be above the zero suppression
thresholds.  Initial studies of energy resolution for supernova physics
also requires an electron lifetime above 3\,ms.  The 3\,ms lifetime
detector requirement is the same for both the single-phase reference
design and the dual-phase alternate design.  As the argon purity goal
is similar, work on contamination mitigation can be done jointly.

There is much experience in the community to justify confidence that
high levels of argon purity can be achieved.  Careful design of gas
ullage and the recirculation system is vital to avoid trapped pockets
of gas and to minimize the mixing of the gas and the liquid at the
interface.  ICARUS achieved a lifetime above 15\,ms after modification of
the cryosystem to extend the lifecycle of the recirculation
pump.\cite{Antonello:2014eha} The materials test stand (MTS) at Fermilab has
demonstrated that contaminants in the liquid argon originate from
materials in the gas space in the ullage, where the warmer
temperatures allow for outgassing of exposed surfaces and that
materials immersed in the liquid argon are not a source of
contamination.\cite{andrewsNIM} The MTS has measured the
contamination rate for many materials and is available for
continued testing of additional materials.  The Liquid Argon Purity
Demonstrator (LAPD) acheived lifetimes above 14\,ms without evacuating
the cryostat and with a functioning TPC inside the
cryostat.\cite{Bromberg:2015uia}

The phase-1 run of the 35-t achieved a peak 3\,ms electron lifetime;
however, the purity was still improving when the run ended.  The
engineers and scientists from LBNF and both DUNE detector options will
work together to optimize the cryogenic design for high purity. Two
examples are understanding the sources of contamination in the ullage
and how this contamination migrates to the liquid and developing a
fill procedure that preserves the purity of the incoming liquid.
There are several membrane cryostats that will be designed and built
over the next 10 years by a common engineering team: the
1$\times$1$\times$3\,m$^3$ dual-phase prototype, Short Baseline
Neutrino Detector (SBND), WA105, the Single-Phase CERN Prototype
(SPCP) engineering prototype as well as the DUNE far detector. Each of
these will learn from its predessors and inform its successors. Based
on existing measurements and extrapolations to the \ktadj{10} design a 3\,ms
lifetime should be readily achievable.


\section{High Voltage}

Both LArTPC designs require a large HV to
produce an electric field of the order of 500\,V/cm in the drift
volume.  They both thus require a HV generator, HV
feedthoughs and a field cage to correctly shape the electric
field. While these elements differ in the two designs, they present a common
set of problems to solve, including
\begin{itemize}
\item Design rules for HV,
\item HV generation,
\item HV Feedthroughs,
\item Field cage structure, and
\item Arc mitigation (Stored energy and discharge).
\end{itemize}


 
\section{Photon Detection}

The approaches to photon detection in the two designs is
different.  The reference design uses TPB-coated light guides
instrumented with SiPMs, whereas the alternate design uses large PMTs
(also coated with TPB). Nevertheless, several aspects of and
techniques used in the development of these systems have strong synergies, including
\begin{itemize}
\item Requirements refinement and validation,
\item Development and evaluation of photosensors,
\item Impact of background light,
\item  Surface reflectivity, and
\item Photon detector calibration.
\end{itemize}


\section{Detector Calibration}

The challenging requirement on systematic uncertainties calls for a
robust program of calibration, which may include the use of calibration sources
deployed in the detector, complementary external measurements
and data-driven calibration procedures. It is expected that this effort
will have significant synergies between the two designs, including
\begin{itemize}
\item Active volume,
\item Energy scale,
\item Energy resolution,
\item PID likelihoods, and
\item Absolute light yield.
\end{itemize}


\section{Underground Installation Strategies}

A fundamental aspect of the detector cost optimization is related to
the development of a strategy for underground logistics, safety and detector
installation. Dimensioning of the components to be transported and
assembled underground is a common issue, affording concomitant synergies. 
Strategies for and requirements on implementing the clean
rooms, additional tooling and needs for temporary installations, such
as scaffolding, also present opportunities for potential synergies.

\section{Local Computing Infrastructure and DAQ}

Once the electrical signals from the detector have been processed
(e.g., by front-end preamplifiers), the treatment of the digitized raw data and
their compression can be strongly unified and will therefore provide
synergies. The online computing farm will have a very similar layout
for both reference and alternative designs. The software triggering and
filtering algorithms will be based on similar local computing
architectures, offering strong synergies. Finally, the local data
storage and transmission to offsite tier centers will be common.


\section{Detector Modeling and Simulation}

Accurate and detailed detector modeling is required.  Simulations are
needed for both ionization electrons and scintillation photons. The
%basis of the synergies in this area is the 
common detection medium of
LAr provides a basis for synergies in this area that include
\begin{itemize}
\item Charge generation and transport,
\item Charge diffusion and attenuation studies,
\item Noise and its impact on the detector performance, and
\item Optical model and light propagation.
\end{itemize}

%\section{Further Steps}
\section{Summary}
A large set of possible synergies exists between the reference and
alternative TPC designs. These synergies will be exploited and
developed within the DUNE collaboration and the LBNF team as the
program of prototypes, demonstrators and other development activities
continues and as the detector modules and the accompanying facilities
are constructed. The CERN neutrino platform, in particular, will
provide an excellent opportunity for joint detector development.

