\chapter{Synergies Between FD Designs}
\label{ch:detectors-synergy}

\section{Overview}

The reference and the alternate design for the far detector are both
liquid argon TPCs hosted in identical caverns, with identical
cryogenic systems, and near identical cryostats (details of the
cryostat roof are expected). They differ mostly in their
approach to collection and readout of the ionization signals. In the
reference design, this charge is measured by successive induction on
readout wire plans and eventually collected on a wire plane. In the
alternate design, the charge is extracted from the liquid to the
vapor and then amplified and finally collected on two perpendicular
independent views. While the detailed detector configurations are
based on different technical choices, this basic fact leads to a large
number of potential synergies.

In the following, we list a certain of number of synergies that are
related to the following sectors or subsystems of the far detector:
\begin{itemize}
\item Interface with the cryogenic system
\item Drift high voltage 
\item Photon detection
\item Calibration
\item Underground installation strategies
\item Local computing infrastructure and DAQ
\item Detector modelling and simulations
\end{itemize}

In the following sections we will briefly discuss each of these items. 
  


\section{Interface with the cryogenic system}

Both designs are based on long drifts, of the order of several meters,
of ionization electrons in liquid argon. These drifts require an
excellent purity of the liquid argon, with electronegative impurities
much below the ppb level and electron lifetimes at the ms scale. In
both designs, the detector will be hosted in the same cryostat built
using the industrial LNG cryostat technology. The contamination will
come from impurities adsorbed on the tank and detector surfaces. It is
therefore important to understand these sources and minimize their
impact on the detector.
\begin{itemize}
\item Electronegative contamination mitigation	
\item Modeling contamination sources
\item Contamination migration modeling
\item Material properties
\item Filtration	
\item Design of the cryogenics system
\item Purity monitoring	
\item Roof Interfaces (Hatch, feedthrough, mounts)	
\item Grounding	and Shielding
\item Installation spaces and Cryogenic Facility needs	
\end{itemize}

\subsection{Drift electron lifetime}
\label{sec:detectors-synergy-lifetime}

For detecting interactions of beam neutrinos, the requirement for
electron lifetime of $>$3~ms derives from the minimum signal to noise
ratio (S/N$>$9) required for MIP signals on induction plane wires from
interactions near the cathode to be above the zero suppression
thresholds.  Initial studies of energy resolution for supernova physics
also requires an electron lifetime above 3~ms.  The 3~ms lifetime
detector requirement is the same for both the single-phase reference
design and the dual-phase alternate design.  As the argon purity goal
is similar, work on contamination mitigation can be done jointly.


There is much experience in the community to justify confidence that
high levels of argon purity can be achieved.  Careful design of gas
ullage and the recirculation system is vital to avoid trapped pockets
of gas and to minimize the mixing of the gas and the liquid at the
interface.  ICARUS achieved a lifetime above 15~ms after modification of
the cryosystem to extend the lifecycle of the recirculation
pump.\cite{Antonello:2014eha} The materials test stand (MTS) at Fermilab has
demonstrated that contaminants in the liquid argon originate from
materials in the gas space in the ullage, where the warmer
temperatures allow for outgassing of exposed surfaces and that
materials immersed in the liquid argon are not a source of
contamination.\cite{andrewsNIM} The MTS has measured the
contamination rate for many materials and is available for
continued testing of additional materials.  The Liquid Argon Purity
Demonstrator (LAPD) acheived lifetimes above 14~ms without evacuating
the cryostat and with a functioning TPC inside the
cryostat.\cite{Bromberg:2015uia}


The phase-1 run of the 35-t achieved a peak 3~ms electron lifetime;
however, the purity was still improving when the run ended.  The
engineers and scientists from LBNF and both DUNE detector options will
work together to optimize the cryogenic design for high purity. Two
examples are understanding the sources of contamination in the ullage
and how this contamination migrates to the liquid and developing a
fill procedure that preserves the purity of the incoming liquid.
There are several membrane cryostats that will be designed and built
over the next 10 years by a common engineering team: the
1$\times$1$\times$3~m$^3$ dual phase prototype, Short Baseline
Neutrino Detector (SBND), WA105, the Single-Phase CERN Prototype
(SPCP) engineering prototype as well as the DUNE far detector. Each of
these will learn from its predessors and inform its successors. Based
on existing measurements and extrapolations to the 10~kt design a 3~ms
lifetime should be readily achievable.


\section{Drift high Voltage}

One of the feature of a liquid argon TPC is the need for a large HV to
produce an electric field of the order of 500~V/cm in the drift
volume.  In the two cases this requires then a HV generator, HV
feed-though and a field cage to correctly shape the electric
field. While far from being identical, this set of problems has a
large number of commonalities for the two designs.
\begin{itemize}
\item Design rules for HV
\item HV generation
\item HV Feedthroughs
\item Field cage structure
\item Arc mitigation (Stored energy and discharge)
\end{itemize}


 
\section{Photon Detection}

The photon detection uses very different concepts. For the reference
design it is based on light guide bars with TPB coating and SiPMs.
For the alternate design it is based on large PMTs coated with
TPB. Nevertheless several aspects related to these subsystems can be
further developed through a collaborative effort.  Moreover,
techniques developed in one design could be used in the other design.
\begin{itemize}
\item Requirements refinement and validation
\item Development and evaluation of photon detectors.
\item Impact of background light.
\item  Surface reflectivity.
\item Photon detector calibration
\end{itemize}


\section{Calibration}

The challenging requirement on systematic uncertainties calls for a
full-fledged program of calibration, be it through calibration sources
deployed in the detector, through complementary external measurements
and through data-driven calibration procedures. We think this program
allows for large synergies between the two designs.
\begin{itemize}
\item Active Volume
\item Energy scale
\item Energy resolution
\item PID likelihoods
\item Absolute Light Yield	
\end{itemize}


\section{Underground installation strategies}
A fundamental aspect of the detector cost optimization is related to
the development of an optimized strategy for underground logistics and
detector installation. Dimensioning of the components to be
transported and assembled underground is a common issue where
synergies exist. Also at this level are the strategies for
implementing the clean room. Additional tooling, such as scaffolding,
are also a subject where common developments are expected to take
place.

\section{Local computing infrastructure and DAQ}
Once the electrical signals from the detector have been conditioned
(e.g. F/E preamplifiers), the treatment of the digitized raw-data and
their compression can be strongly unified and will therefore provide
synergies. The online computing farm will have a very similar layout
for both reference and alternate designs. The software triggering and
filtering algorithms will be based on similar local computing
architectures, offering strong synergies. Finally, the local data
storage and transmission to offsite tier-centers will be common.

\section{Detector Modelling and simulations}

An important aspect of the far detector is a correct and detailed
detector modelling to serve as the basis of the simulation of the
signal (ionization electron and photons from argon scintillation)
generated by neutrino interactions. The basis of these synergies is
the fact that liquid argon serves as the TPC detection medium in both
cases. Here the following synergies can be considered:

\begin{itemize}
\item Charge generation and transport
\item Charge diffusion and attenuation studies
\item Noise and impact on the detector performance
\item Optical model and light propagation
\end{itemize}

\section{Further steps}

A large set of possible synergies between the different TPC alternate
designs exists and will be further developed within the DUNE
collaboration.  The program of demonstrators in the CERN neutrino
platform will provide an excellent platform for joint detector
development.
