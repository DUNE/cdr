\chapter{Synergies Between Far Detector Designs}
\label{ch:detectors-synergy}

\section{Overview}

\fixme{Should we start with something about `As discussed in Section~\ref{sec:detectors-strategy-FD}, 
it is envisioned that as technology advances and 
the performance of prototypes is established, the four detector modules, staged for sequential installation, 
may implement different designs, in terms of both single- and dual-phase readout and advances in either
technology as they become available. In this context, it is important to consider the synergies that DUNE can
exploit as it refines the design of each module in turn.'}

Both the reference and alternative designs for the DUNE far detector are 
liquid argon TPCs. The designs assume nearly identical cryostats (with some differences in the
cryostat roof), in identical caverns and supported by identical
cryogenic systems. The designs differ mostly in their
approaches to collection and readout of the ionization signals. In the
reference design, the ionization charge is measured by successive wire planes, two induction and
one collection, all immersed in the LAr. In the
alternative design, the charge is extracted from the liquid to the
vapor and then amplified and finally collected on a 2D anode, providing two %perpendicular
independent views. 

Different technical choices aside, an encouraging set of synergies emerges between the designs in the areas of: \fixme{list}

\fixme{above is proposed rewrite of the following}
The reference and the alternate designs for the far detector are both
liquid argon TPCs hosted in identical caverns, with identical
cryogenic systems and nearly identical cryostats (details of the
cryostat roof are expected). They differ mostly in their
approach to collection and readout of the ionization signals. In the
reference design, this charge is measured by successive induction and
readout wire plans and eventually collected on a wire plane. In the
alternate design, the charge is extracted from the liquid to the
vapor and then amplified and finally collected on two perpendicular
independent views. While the detailed detector configurations are
based on different technical choices, this basic fact leads to a large
number of potential synergies.

A list of synergies related to the following far detector subsystems
follows:

\fixme{start again here}
\begin{itemize}
\item Interface with the cryogenics system
\item Drift high voltage \fixme{I would just call it ``High Voltage''}
\item Photon detection
\item Calibration
\item Underground installation strategies
\item Local computing infrastructure and DAQ
\item Detector modeling and simulation
\end{itemize}

%In the following sections we will briefly discuss each of these items. <--- not needed
  


\section{Interface with the Cryogenics System}

In both designs, ionization electrons have drift lengths on the order
of several meters. In order to reach the required millisecond scale for
electron lifetime, the electronegative impurities in the LAr must be
maintained below the ppb level. The contamination will
come primarily from impurities adsorbed onto the tank and detector element surfaces.
Given that the detector modules will be housed in cryostats of the same design, built
using the industrial LNG cryostat technology, and that the cryogenics system must
be suitable for all the modules, the process of understanding these sources and minimizing them
is to a great extent the same for all the modules, for either TPC design. \fixme{except for detector element sources; need to
address? Or does `to a great extent' cover it?}

\fixme{above is proposed rewrite of the following}
Both designs are based on long drifts, of the order of several meters,
of ionization electrons in liquid argon. These drifts require
excellent purity of the liquid argon, with electronegative impurities
much below the ppb level and electron lifetimes at the ms scale. In
both designs, the detector will be hosted in the same cryostat built
using the industrial LNG cryostat technology. The contamination will
come from impurities adsorbed on the tank and detector surfaces. It is
therefore important to understand these sources and minimize their
impact on the detector.
\begin{itemize}
\item Electronegative contamination mitigation	
\item Modeling contamination sources
\item Contamination migration modeling
\item Material properties
\item Filtration	
\item Design of the cryogenics system
\item Purity monitoring	
\item Roof interfaces (hatch, feedthrough, mounts)	
\item Grounding and Shielding
\item Installation spaces and cryogenics system needs	
\end{itemize}

%\subsection{Drift Electron Lifetime}  <-- can't have just one subsection in a section
%\label{sec:detectors-synergy-lifetime}


\fixme{I don't think this next pgraph is needed; it's described elsewhere that the reason you need
high purity is to get longer lifetime, etc.}
For detecting interactions of beam neutrinos, the requirement for
electron lifetime of $>$3~ms derives from the minimum signal to noise
ratio (S/N$>$9) required for MIP signals on induction plane wires from
interactions near the cathode to be above the zero suppression
thresholds.  Initial studies of energy resolution for supernova physics
also requires an electron lifetime above 3~ms.  The 3~ms lifetime
detector requirement is the same for both the single-phase reference
design and the dual-phase alternate design.  As the argon purity goal
is similar, work on contamination mitigation can be done jointly.

\fixme{I'm not sure this next one is needed either; it's about progress, not the synergy itself}
There is much experience in the community to justify confidence that
high levels of argon purity can be achieved.  Careful design of gas
ullage and the recirculation system is vital to avoid trapped pockets
of gas and to minimize the mixing of the gas and the liquid at the
interface.  ICARUS achieved a lifetime above 15~ms after modification of
the cryosystem to extend the lifecycle of the recirculation
pump.\cite{Antonello:2014eha} The materials test stand (MTS) at Fermilab has
demonstrated that contaminants in the liquid argon originate from
materials in the gas space in the ullage, where the warmer
temperatures allow for outgassing of exposed surfaces and that
materials immersed in the liquid argon are not a source of
contamination.\cite{andrewsNIM} The MTS has measured the
contamination rate for many materials and is available for
continued testing of additional materials.  The Liquid Argon Purity
Demonstrator (LAPD) acheived lifetimes above 14~ms without evacuating
the cryostat and with a functioning TPC inside the
cryostat.\cite{Bromberg:2015uia}


The phase-1 run of the 35-t achieved a peak 3~ms electron lifetime;
however, the purity was still improving when the run ended.  \fixme{I would start again here.} 
The
engineers and scientists from LBNF and both DUNE detector designs will
work together to optimize the cryogenic design to achieve the required high purity. 
In particular, work will proceed in understanding the sources of contamination in the ullage
and how this contamination migrates to the liquid. It will be necessary to develop a
fill procedure that preserves the purity of the incoming liquid.
Several membrane cryostats will be designed and built
over the next 10 years by a common engineering team: the
1$\times$1$\times$3~m$^3$ dual phase prototype, Short Baseline
Neutrino Detector (SBND), WA105, the Single-Phase CERN Prototype
(SPCP) engineering prototype, as well as the first module of the DUNE far detector. 
Each of
these will benefit from lessons learned during construction of its predessors and will in turn inform its successors. 

\fixme{See the above pgraph; proposed rewrite of following} The
engineers and scientists from LBNF and both DUNE detector options will
work together to optimize the cryogenic design for high purity. Two
examples are understanding the sources of contamination in the ullage
and how this contamination migrates to the liquid and developing a
fill procedure that preserves the purity of the incoming liquid.
There are several membrane cryostats that will be designed and built
over the next 10 years by a common engineering team: the
1$\times$1$\times$3~m$^3$ dual phase prototype, Short Baseline
Neutrino Detector (SBND), WA105, the Single-Phase CERN Prototype
(SPCP) engineering prototype as well as the DUNE far detector. Each of
these will learn from its predessors and inform its successors. Based
on existing measurements and extrapolations to the 10~kt design a 3~ms
lifetime should be readily achievable.


%\section{Drift High Voltage}
\section{High Voltage}

Both LArTPC designs require a large HV to
produce an electric field of the order of 500~V/cm in the drift
volume.  They both thus require a HV generator, HV
feedthoughs and a field cage to correctly shape the electric
field. While these elements differ in the two designs, they present a common
set of problems to solve. 
 
\fixme{above is proposed rewrite of the following}
One of the feature of a liquid argon TPC is the need for a large HV to
produce an electric field of the order of 500~V/cm in the drift
volume.  In the two cases this requires then a HV generator, HV
feed-though and a field cage to correctly shape the electric
field. While far from being identical, this set of problems has a
large number of commonalities for the two designs.
\begin{itemize}
\item Design rules for HV
\item HV generation
\item HV Feedthroughs
\item Field cage structure
\item Arc mitigation (Stored energy and discharge)
\end{itemize}


 
\section{Photon Detection}

The photon detection in each of the two designs is quite different.
Despite the dissimilarity between TPB-coated light guides instrumented with SiPMs 
and large PMTs (also coated with TPB), several aspects of and techniques use in these systems can be
further developed through a collaborative effort.  

\fixme{above is proposed rewrite of the following}
The photon detection uses very different concepts. For the reference
design it is based on light guide bars with TPB coating and SiPMs.
For the alternate design it is based on large PMTs coated with
TPB. Nevertheless several aspects related to these subsystems can be
further developed through a collaborative effort.  Moreover,
techniques developed in one design could be used in the other design.
\begin{itemize}
\item Requirements refinement and validation
\item Development and evaluation of photon detectors \fixme{meaning either design might just adopt a new design?}
\item Impact of background light
\item  Surface reflectivity
\item Photon detector calibration
\end{itemize}


\section{Detector Calibration}

The challenging requirement on systematic uncertainties calls for a
robust program of calibration, which may include the use of calibration sources
deployed in the detector, complementary external measurements
and data-driven calibration procedures. It is expected that this effort
allows for significant synergies between the two designs.


\fixme{above is proposed rewrite of the following}
The challenging requirement on systematic uncertainties calls for a
full-fledged program of calibration, be it through calibration sources
deployed in the detector, through complementary external measurements
and through data-driven calibration procedures. We think this program
allows for large synergies between the two designs.
\begin{itemize}
\item Active volume
\item Energy scale
\item Energy resolution
\item PID likelihoods
\item Absolute light yield	
\end{itemize}


\section{Underground Installation Strategies}

A fundamental aspect of the detector cost optimization is related to
the development of a strategy for underground logistics and
detector installation. Dimensioning of the components to be
transported and assembled underground is a common issue where
synergies exist. Strategies for and requirements on 
implementing the clean rooms, additional tooling, and needs for temporary installations, such as scaffolding,
also present opportunities for potential synergies.

\fixme{And maybe safety considerations -- ESH is an area of synergy!}

\fixme{above is proposed rewrite of the following}
A fundamental aspect of the detector cost optimization is related to
the development of an optimized strategy for underground logistics and
detector installation. Dimensioning of the components to be
transported and assembled underground is a common issue where
synergies exist. Also at this level are the strategies for
implementing the clean room. Additional tooling, such as scaffolding,
are also a subject where common developments are expected to take
place.

\section{Local Computing Infrastructure and DAQ}

\fixme{no proposed rewrite, just minor edits}
Once the electrical signals from the detector have been conditioned \fixme{I'm not used to this word in this context; check}
(e.g., by front-end preamplifiers), the treatment of the digitized raw data and
their compression can be strongly unified and will therefore provide
synergies. The online computing farm will have a very similar layout
for both reference and alternative designs. The software triggering and
filtering algorithms will be based on similar local computing
architectures, offering strong synergies. Finally, the local data
storage and transmission to offsite tier-centers will be common.


\section{Detector Modeling and Simulation}

Accurate and detailed detector modeling is required in order to provide a 
basis for neutrino event simulation. It is important to develop simulations for both ionization 
electrons and scintillation photons. The basis of the synergies in this area is
the common detection medium of LAr.

\fixme{above is proposed rewrite of the following}
An important aspect of the far detector is a correct and detailed
detector modelling to serve as the basis of the simulation of the
signal (ionization electron and photons from argon scintillation)
generated by neutrino interactions. The basis of these synergies is
the fact that liquid argon serves as the TPC detection medium in both
cases. Here the following synergies can be considered:

\begin{itemize}
\item Charge generation and transport
\item Charge diffusion and attenuation studies
\item Noise and its impact on the detector performance
\item Optical model and light propagation
\end{itemize}

%\section{Further Steps}
\section{Summary}
A large set of possible synergies exists between the reference and alternative TPC
designs. These synergies will be exploited and developed within the DUNE
Collaboration and the LBNF team as the program of prototypes, demonstrators and other 
development activities continues and as
the detector modules and the accompanying facilities are constructed. The CERN neutrino
platform, in particular, will provide an excellent opportunity for joint detector development.

\fixme{above is proposed rewrite of the following}
A large set of possible synergies  between the different TPC alternate
designs exists and will be further developed within the DUNE
collaboration.  The program of demonstrators in the CERN neutrino
platform will provide an excellent platform for joint detector
development.
