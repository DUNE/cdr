%%%%%%%%%%%%%%%%%%%%%%%%%%%%%%%%
\section{The Photon Detection System}
\label{sec:detectors-fd-ref-pd}

The scope of the photon detector (PD) system for the DUNE far detector
reference design includes design, procurement, fabrication,
testing, delivery and installation of the following components:
\begin{itemize}
\item light collection system including wavelength shifter and light guides,
\item silicon photo-multipliers (SiPMs),
\item readout electronics,
\item calibration system, and
\item related infrastructure (frames, mounting boards, etc.).
\end{itemize}

LAr is an excellent scintillating medium and the photon detection
system will exploit this property in the far detector.  With an
average energy of 19.5~eV needed to produce a photon (at zero field),
a typical particle depositing 1~MeV in LAr will generate
40,000~photons with wavelength of 128~nm. At higher fields this will
be reduced, but at 500~V/cm the yield is still $\sim$20,000~photons
per MeV. Roughly 1/4 of the photons are promptly emitted with a
lifetime of about 6~ns while the rest have a lifetime of
1100--1600~ns. Prompt and delayed photons are detected in
  precisely the same way by the photon detection system. LAr is
highly transparent to the 128-nm VUV photons with a Rayleigh
scattering length of (66~$\pm$~3)~cm~\cite{Rayleigh} and absorption
length of $>$200~cm; this attenuation length requires a LN2
  content of less than 20~ppm. The relatively large light yield makes
the scintillation process an excellent candidate for determination of
$t_0$ for non-beam related events. Detection of the scintillation
light may also be helpful in background rejection and triggering on
non-beam events.

The photon detection system reference design described in this section
meets the required performance for light collection for the DUNE far
detector. This includes detection of light from proton decay
candidates (as well as beam neutrino events) with high efficiency to
enable 3D spatial localization of candidate events. The TPC will
provide supernova neutrino detection. 
The photon system will provide the $t_0$ timing of
events relative to TPC timing with a resolution better than 1~$\mu$s
(providing position resolution along drift direction of a couple of mm). 

Alternative photon detector designs are under investigation to improve
both light detection for the low-energy supernova events and their
momentum resolution through determination of the event t$_0$.

Figure~\ref{fig:PD_overview} shows the layout for the photon detector
system, which will be described in the following sections.
\begin{cdrfigure}[Photon detection system overview]{PD_overview}{Overview of the PD
    system showing a cartoon schematic (a) of a single PD module
    in the LAr and the channel ganging scheme used to reduce the
    number of readout channels. Panel (b) shows how each PD module
    will be inserted into an APA frame. There will be 10 PDs instered
    into an APA frame.}
\includegraphics[width=1.0\linewidth]{pd_schem.png}
\end{cdrfigure}

\subsection{Reference Design}
\label{sec:detectors-fd-ref-pd-refsystem} 

The PD system is mounted as modules on the APA frames.  A PD module is
the combination of one light-guide (also called a ``bar'' due to its
shape) and 12 SiPMs, as shown in Figure~\ref{fig:PD_overview}~(a).  To
enable this, the the reference design for mounting the PDs onto the
APA frames calls for ten PD modules per APA, approximately 2.2-m long,
83-mm wide and 6-mm thick, equally spaced along the full length of the
APA frame, as shown in Figure~\ref{fig:PD_overview}~(b). 

The 128-nm scintillation photons from LAr interact with the wavelength
shifter on the surface of the bar, and the wavelength-shifted light,
with a peak intensity around 430~nm, is re-emitted inside the bar and
transported through the light-guide to 12 silicon photo-multipliers
(SiPMs) mounted at one end of the bar.

The wavelength shifter converts the scintillation photons striking the
bar surface and directs them into the bar bulk with an efficiency of
$\sim$50\%.  A fraction of the wavelength-shifted optical photons are
internally reflected to the bar's end where they are detected by SiPMs
with quantum efficiency well matched to the wavelength-shifted
photons. The light guides are coated with TPB
(1,1,4,4-tetraphenyl-1,3-butadiene). A testing program is currently
underway to determine the absolute performance of the light guides in
LAr.

The SiPMs used in the reference design are SensL C-Series 6~mm$^2$
(MicroFB-60035-SMT) devices. These SiPMs have detection efficiency of
41\%; the detection efficiency combines QE and effective area
  coverage accounting for dead space between pixels. While the
C-Series SensL SiPMs are not rated for operation below
$-$40$^{\circ}$~C their performance has been excellent for this
application. At LAr temperature (89~K) the dark rate is of order 10~Hz
(0.5 p.e. threshold) while after-pulsing has not been an
issue. Extensive testing is underway to ensure that the SiPMs can
reliably survive the stresses associated with thermal cycling in LAr
and long-term operation at LAr temperature.

The SiPMs are read out using shielded twisted-pair cable, one per SiPM,
but the expected final design will have three SiPMs ganged together and
each readout cable will contain four individual channel cables to keep
the cost and cable packing density down. During the R\&D phase of the
project each SiPM was read out individually in order to maximize the information
gathered.  

The front-end electronics reside outside of the cryostat in
instrumentation racks. A custom module for receiving SiPM signals has
been designed and built. The module also performs signal processing in
the front-end as preprocessing for trigger and DAQ.  The module is
called the SiPM Signal Processor (SSP) and consists of 12 readout
channels packaged in a self-contained 1U module.  Each channel
contains a fully-differential voltage amplifier and a 14-bit, 150-MSPS
analog-to-digital converter (ADC) that digitizes the waveforms
received from the SiPMs. There is no shaping of the signal, since the
SiPM response is slow enough relative to the speed of the digitization
to obtain several digitized samples of the leading edge of the pulse
for the determination of signal timing. Digitized data is processed by
a Xilinx Artix-7 Field-Programmable Gate Array (FPGA).  The use of the
FPGA processing allows for a significant amount of customization of
the SSP operation. 

Once the DUNE collaboration arrives at a refined set of physics
requirements for the photon detection system a set of criteria for a
calibration system will be determined. In the absence of such criteria
two calibration systems are being explored, and will be tested in
the 35-t phase-II test. The first system, developed by ANL, utilizes five
fiber-fed diffusers mounted on the TPC CPA which uniformly illuminate
the photon detectors. An alternative design is employed on the IU
prototypes and uses LED-driven fibers mounted alongside the
waveguides. 

\subsection{Alternative Designs} 

Three alternative photon detector designs are currently being considered for the PD
system. More extensive descriptions of the alternative designs can be found in \anxlbnefd, Section 5.3.

The first alternative is based on a TPB-coated acrylic panel with an
embedded S-shaped wavelength-shifting fiber. The Louisiana State
University (LSU) group has developed prototypes based on this design
in an attempt to allow an increase in detector size and hence increase
geometric acceptance of the PD system and reduce overall system
cost.

In this design, a single acrylic panel PD module has the same
dimensions as the reference design and consists of a TPB-coated
acrylic panel with an embedded multi-lobed ``S-shaped'' wavelength
shifting (Y11) fiber. The fiber is read out by two SiPMs (one on the
top edge, and the other on the bottom edge of the plate), which are
coupled to each end of the fiber and serve to transport the light over
long distances with minimal attenuation. The double-ended fiber
readout has the added benefit of providing some position dependence to
the light generation along the panel by comparing relative signal
sizes and arrival times in the two SiPMs. The WLS fiber converts the
430-nm light from the TPB to light with a peak intensity of
480--500~nm, which is well-matched to the peak photon-detection
efficiency of typical SiPMs.

A prototype of a second alternative, under investigation by the Colorado
State University (CSU) group, is intented to address an issue with the reference
design, in which the application of TPB to acrylic, or other base
materials, has been found to cause a significant decrease in
attenuation length (down to about 30~cm) of the light guide.
%
This prototype has a thin TPB-coated acrylic radiator located in front
of a close-packed array of blue-green (Y11) WLS fibers.  The prototype
is two-sided and has two identical fiber arrays and radiators mounted
back-to-back with a tyvek reflector between. This design allows for a
reduction in the number of SiPMs required per PD module. Three SiPMs
per side are needed per PD module (again the same dimensions in length
and width as the reference design) for a total of six SiPMs per PD
module.

The Indiana University (IU) group has advanced the CSU design and
arrived at the third alternative design by replacing the Y11 fiber
with a reference-design-dimensioned cast bar doped (by Eljen
Technology) with the same wavelength shifter as the Y11 fiber. The
TPB-coated acrylic radiator has also been replaced with a thin-fused
silica plate coated with TPB. This prototype has demonstrated an
attenuation length greater than 2.5~m and early indications point to
it meeting the SN neutrino energy requirement. It has the highest
light collection efficiency of any prototype tested so far.

\subsection{Technology Selection}

The alternative designs have all demonstrated the ability to detect
LAr scintillation light, and development work is continuing. A testing
program is underway comparing the various alternatives against the
reference design.  The Tall Bo large dewar at Fermilab and the
cryogenic detector development facility at CSU are being used to
compare the performance of full-scale and near-full-scale prototypes
in LAr utilizing alpha sources and cosmic muons. Data from the 35-t
prototype will also provide input into the technology decision. %In Fall 2015 a decision will be made regarding which design to adopt and optimize for the first 10-kt far detector module.
