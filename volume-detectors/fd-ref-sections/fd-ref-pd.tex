%%%%%%%%%%%%%%%%%%%%%%%%%%%%%%%%
\section{The Photon Detection System}
\label{sec:detectors-fd-ref-pd}

The scope of the photon detector section includes the design,
procurement, fabrication, testing, delivery and installation of all
the systems and components that comprise it:  

\begin{itemize}
\item wavelength-shifting wave-guides
\item silicon photo-multipliers (SiPMs)
\item readout electronics
\item related infrastructure (frames, mounting boards, etc...)
\end{itemize}

The design presented here meets the required performance for light
collection for the DUNE far detector. The most significant
requirements include determination of t$_0$ of an event relative to
the TPC timing - resolution better than 30~ns, and sufficient light
collection efficiency to allow triggering on SN neutrinos with
energies as low as 20~MeV.

Liquid argon is an excellent scintillating medium. With an average
energy needed to produce a photon of 19.5~eV (at zero field) a typical
particle depositing 1~MeV in liquid argon will generate 40,000~photons
with wavelength of 128~nm. At higher fields this will be reduced but
at 500~V/cm the yield is still about $\sim$20,000~photons per
MeV. Roughly 1/3 of the photons are promptly emitted after about 6~ns
while the rest are are emitted with a delay of 1100-1600~ns. LAr
is highly transparent to the 128~VUV photons with a Rayleigh
scattering length and absorption length of 95~cm and >200~cm
respectively. The relatively large light yield makes the scintillation
process an excellent candidate for determination of $t_{0}$ for
non-beam related events. Detection of the scintillation light may also
be helpful in background rejection and triggering on non-beam events.  

Figure~\ref{PD_overview} shows the system layout for the photon
detector system, which will be described in the following sections. 

\begin{cdrfigure}[PD Overview]{PD_overview}{Overview of the photon detector
    system showing the a cartoon schematic (a) of a single PD module
    in the LAr and the channel ganging scheme used to reduce the
    number of readout channels. Panel (b) shows how each PD module
    will be inserted into an APA frame. }
\includegraphics[width=1.0\linewidth]{pd_schem.png}
\end{cdrfigure}

\subsection{Reference System Design}

The baseline design for mounting the PDs into the APA frames calls for
ten PD modules, approximately 2.2m long, 83~mm wide, and 6~mm thick,
equally spaced along the full length of the APA frame
(Figure~\ref{PD_overview}~(b)). 

The 128~nm scintillation photons from liquid argon interact with the
wavelength shifter on the light guide surface and 430~nm light is
re-emitted in the bar. The light guide channels the light to
12 silicon photo-multipliers (SiPMs) mounted at one end of the bar.

The wavelength shifter converts the scintillation photons striking the
bar surface and directs them into the bar bulk with an efficiency of
$\sim$50\%.  A fraction of the waveshifted optical photons are
internally reflected to the bar's end where they are detected by SiPMs
with quantum efficiency well matched to the wavelength-shifted
photons. The light guides are made with a coating of TPB
(1,1,4,4-tetraphenyl-1,3-butadiene). A testing program is currently
underway to determine the absolute performance of the light-guides in
in liquid argon.

The SiPMs used in the reference design are SensL C-Series 6~mm$^2$
(MicroFB-60035-SMT) devices. These SiPMs have detection efficiency of
41\%~\footnote{The detector efficiency combines QE and effective areal
  coverage due to dead space between pixels.} While the C-Series SensL
SiPMs are not rated for operation below -40$^{\circ}{\rm C}$ their
performance has been excellent for this application. At LAr
temperature (88K) the dark rate is on order 50~Hz while after-pulsing
has not been an issue. Extensive testing is underway to ensure that
the SiPMs can reliably survive the stresses associated with thermal
cycling in LAr and long term operation at LAr temperature.

The SiPMs are read out using shielded twisted pair cable, one per SiPM
but the expected final design will have 3 SiPMs ganged together and
each readout cable with contain 4 individual channel cables to keep
the cost and cable packing desity down.  

The front-end electronics reside outside of the cryostat in
instrumentation racks. A custom module for receiving SiPM signals has
been designed and built. The module also performs signal processing in
the front-end as preprocessing for trigger and DAQ.  The module is
called the SiPM Signal Processor (SSP) and consists of 12 readout
channels packaged in a self-contained 1U module.  Each channel
contains a fully-differential voltage amplifier and a 14-bit, 150 MSPS
analog-to-digital converter (ADC) that digitizes the waveforms
received from the SiPMs. There is no shaping of the signal, since the
SiPM response is slow enough relative to the speed of the digitization
to obtain several digitized samples of the leading edge of the pulse
for the determination of signal timing. Digitized data is processed by
a Xilinx Artix-7 Field-Programmable Gate Array (FPGA).  The use of the
FPGA processing allows for a significant amount of customization of
the SSP operation. 

\subsection{Alternative Designs} 



\subsection{Technology Selection}

\subsection{Production and Installation}


These cables (120 of them in
the original baseline design) are routed through the APA side tubes to
a connector at the cold electronics readout end of the APA.  Initially
it was decided that it would be too complicated to design the APA
frames to allow PD module installation following APA wire-wrapping, so
the PD modules were installed prior to this step in the installation
process.  The Wavelength shifting elements in the light collectors of
the PD modules are sensitive to heat, humidity and most critically to
exposure to ambient light, which places significant requirements on
the environment the APAs are assembled and stored in this way, but
initially it was decided this was the best option.




Here is a sample table:


\begin{cdrtable}[short]{cc}{label}{long}  %The third argument (reads {cc}) can use c, l, r or p{some length} 
% but please do not include lines like “|c|l|l|”. It CAN look like {cll} or {llp{3cm}}, for instance.
Header Column1 & Header Column 2 \\ \toprowrule
Row 1 & First \\ \colhline
Row 2 & Second \\ \colhline
Row 3 & Third \\
\end{cdrtable}

Here is a sample reference to a table Table~\ref{tab:label}).  Notice: ``tab:'' is not present in the label as written in the table code itself.
