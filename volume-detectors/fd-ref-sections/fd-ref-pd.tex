%%%%%%%%%%%%%%%%%%%%%%%%%%%%%%%%
\section{The Photon Detection System}
\label{sec:detectors-fd-ref-pd}

The scope of the photon detector (PD) system for the DUNE far detector
reference design includes the design, procurement, fabrication,
testing, delivery and installation into the far detector of all the
systems and components that comprise it:

\begin{itemize}
\item light collection system including wavelength shifter and light guide
\item silicon photo-multipliers (SiPMs)
\item readout electronics
\item calibration system
\item related infrastructure (frames, mounting boards, etc...)
\end{itemize}

Liquid argon is an excellent scintillating medium and the photon
detection system will exploit this property in the far detector.  With
an average energy needed to produce a photon of 19.5~eV (at zero
field) a typical particle depositing 1~MeV in liquid argon will
generate 40,000~photons with wavelength of 128~nm. At higher fields
this will be reduced but at 500~V/cm the yield is still
$\sim$20,000~photons per MeV. Roughly 1/4 of the photons are promptly
emitted with a lifetime of about 6~ns while the rest have a lifetime
of 1100 $-$ 1600~ns. LAr is highly transparent to the 128~VUV photons
with a Rayleigh scattering length of 95~cm and absorption length of
>200~cm.  The relatively large light yield makes the scintillation
process an excellent candidate for determination of $t_{0}$ for
non-beam related events. Detection of the scintillation light may also
be helpful in background rejection and triggering on non-beam events.

The photon detection system design described in this section meets the
required performance for light collection for the DUNE far
detector. The most significant requirements include determination of
t$_0$ of an event relative to the TPC timing - resolution better than
1~$\mu$s (event position resolution along drift direction of a few
mm), and sufficient light collection efficiency to allow triggering on
neutrinos coming from a supernova, with the capability of detecting
neutrinos with energies as low as 20~MeV. Effort is underway to
determine if our reference design can achieve the latter goal
above. We are also exploring alternate designs that may provide
greater sensitivity to low energy neutrinos.

Figure~\ref{fig:PD_overview} shows the layout for the photon
detector system. %, which will be described in the following sections. 

\begin{cdrfigure}[PD Overview]{PD_overview}{Overview of the PD
    system showing a cartoon schematic (a) of a single PD module
    in the LAr and the channel ganging scheme used to reduce the
    number of readout channels. Panel (b) shows how each PD module
    will be inserted into an APA frame. }
\includegraphics[width=1.0\linewidth]{pd_schem.png}
\end{cdrfigure}

\subsection{Reference Design}
\label{sec:detectors-fd-ref-pd-refsystem}   % Refer to SSP description in this section from DAQ section

The PD system is mounted on the APA frames.  To enable this, the
reference design for mounting the PDs into the APA frames calls for
ten PD modules~\footnote{A PD module is the combination of light-guide
and SiPMs.} (Figure~\ref{fig:PD_overview}~(a)), approximately 2.2-m
long, 83-mm wide, and 6-mm thick, equally spaced along the full length
of the APA frame (Figure~\ref{fig:PD_overview}~(b)).

The 128-nm scintillation photons from liquid argon interact with the
wavelength shifter on the light guide, or bar, surface and 430-nm light is
re-emitted in the bar. The light guide channels the light to
12 silicon photo-multipliers (SiPMs) mounted at one end of the bar.

The wavelength shifter converts the scintillation photons striking the
bar surface and directs them into the bar bulk with an efficiency of
$\sim$50\%.  A fraction of the waveshifted optical photons are
internally reflected to the bar's end where they are detected by SiPMs
with quantum efficiency well matched to the wavelength-shifted
photons. The light guides are made with a coating of TPB
(1,1,4,4-tetraphenyl-1,3-butadiene). A testing program is currently
underway to determine the absolute performance of the light guides in
liquid argon.

The SiPMs used in the reference design are SensL C-Series 6~mm$^2$
(MicroFB-60035-SMT) devices. These SiPMs have detection efficiency of
41\%~\footnote{The detector efficiency combines QE and effective areal
  coverage accounting for dead space between pixels.} While the
C-Series SensL SiPMs are not rated for operation below
$-40^{\circ}{\rm C}$ their performance has been excellent for this
application. At LAr temperature (89~K) the dark rate is of order 10~Hz
(0.5 p.e. threshold) while after-pulsing has not been an
issue. Extensive testing is underway to ensure that the SiPMs can
reliably survive the stresses associated with thermal cycling in LAr
and long-term operation at LAr temperature.

The SiPMs are read out using shielded twisted-pair cable, one per SiPM,
but the expected final design will have three SiPMs ganged together and
each readout cable with contain four individual channel cables to keep
the cost and cable packing desity down. During the R\&D phase of the
project each SiPM was read out in order to ensure maximum information
was gathered during tests.  

The front-end electronics reside outside of the cryostat in
instrumentation racks. A custom module for receiving SiPM signals has
been designed and built. The module also performs signal processing in
the front-end as preprocessing for trigger and DAQ.  The module is
called the SiPM Signal Processor (SSP) and consists of 12 readout
channels packaged in a self-contained 1U module.  Each channel
contains a fully-differential voltage amplifier and a 14-bit, 150-MSPS
analog-to-digital converter (ADC) that digitizes the waveforms
received from the SiPMs. There is no shaping of the signal, since the
SiPM response is slow enough relative to the speed of the digitization
to obtain several digitized samples of the leading edge of the pulse
for the determination of signal timing. Digitized data is processed by
a Xilinx Artix-7 Field-Programmable Gate Array (FPGA).  The use of the
FPGA processing allows for a significant amount of customization of
the SSP operation. 

Once the DUNE collaboration arrives at a refined set of physics
requirements for the photon detection system a set of criteria for a
calibration system will be determined. In the absence of such criteria
two calibration systems have are being explored, and will be tested in
the 35t phase II test. The first system, developed by ANL, utilizes 5
fiber-fed diffusers mounted on the TPC CPA which uniformly illuminate
the photon detectors. An alternate design is employed on the IU
prototypes and uses LED-driven fibers mounted alongside the
waveguides. 

\subsection{Alternative Designs} 

Three alternative designs are currently being considered for the PD
system.

The first alternative is based on a TPB-coated acrylic
panel with an embedded S-shaped wavelength-shifting fiber. The LSU
group has developed prototypes based on this design in an attempt to
increase the geometric acceptance of the PD system and hence reduce
overall system cost. 
%
In this design, a single acrylic panel PD module has the same
dimensions as the reference design and consists of a TPB-coated
acrylic panel with an embedded S-shaped wavelength shifting (Y11)
fiber. The fiber is read out by two SiPMs, which are coupled to either
end of the fiber and serve to transport the light over long distances
with minimal attenuation. The double-ended fiber readout has the added
benefit of providing some position dependence to the light generation
along the panel by comparing relative signal sizes and arrival times
in the two SiPMs. The WLS fiber converts the 430-nm light from the TPB
to light with a peak intensity of about 480 $-$ 500~nm, which is
well-matched to the peak photon-detection efficiency of typical SiPMs.

A second alternative, under investigation by the CSU group, utilizes
blue-green (Y11) fibers that have not been treated with TPB. A thin
TPB-coated acrylic radiator is located in front of a close-packed
array of WLS fibers. The motivation for this design comes from %was to
address an issue with the reference design, in which the application
of TPB to acrylic, or other base materials, has been found to cause a
significant decrease in attenuation length (down to about 30~cm) of
the light guide. \fixme{Need to introduce the prototype} %The
prototype is The CSU group has made a prototype; it is two-sided and
has two identical fiber arrays and radiators mounted back-to-back with
a tyvek refelctor between. %There is also This design allows for a
reduction in the number of SiPMs required per PD module. Three SiPMs
per side are needed per PD module (again the same dimensions in length
and width as the reference design) for a total of six SiPMs per PD
module.

The IU group has advanced the radiator and fiber design \fixme{of the second alternative design?} by replacing
the Y11 fiber with a reference-design-dimensioned cast bar doped (by
Eljen Technology) with the same wavelength shifter as the Y11
fiber. The TPB-coated acrylic radiator has also been replaced with a
thin-fused silica plate coated with TPB. This prototype has
demonstrated an attenuation length greater than 2.5~m and early
indications %seem to 
indicate that it could meet the SN neutrino energy
requirement. It is the ``brightest'' prototpye tested so far.

\subsection{Technology Selection}

The alternative designs have all \fixme{both?} demonstrated the ability to detect
LAr scintillation light, and development work is continuing. A testing program
is underway comparing the various alternatives against the reference
design.  The Tallbo large dewar at Fermilab and the cryogenic detector
development facility at CSU are being used to compare the performance
of full-scale and near-full-scale prototypes in LAr utilizing alpha
sources and cosmic muons. %The DUNE technical board, or equivalent
%body, will review the designs using data from the comparison tests and
%recommend the design we will optimize for the first 10~kt LAr TPC.
 In Fall 2015 a decision will be made regarding which design to adopt and optimize for the first 10-kt far detector module.
%whether to proceed with the current reference design or move to one of the alternatives.

