%%%%%%%%%%%%%%%%%%%%%%%%%%%%%%%%
\section{Overview}
\label{sec:detectors-fd-ref-ov}


This chapter describes the reference design of the DUNE far detector.
The detector will consist of four nominal 10-kt fiducial mass,
single-phase liquid argon time projection chambers (TPC) augmented
with photon detection systems.  A ``Single-phase'' detector is one in
which the charge generation, drift and collection all occur in liquid
argon (LAr). The scope of the far detector includes the design,
procurement, fabrication, testing, delivery, installation and
commissioning of the detector components:
\begin{itemize}
\item Time Projection Chamber (TPC)
\item Data Acquisition System (DAQ)  
\item Cold Electronics (CE)
\item Photon Detector System (PD)
\end{itemize}
The TPCs will be housed in cryostats provided by LBNF, described in
\vollbnf. The reference design is based largely on the LBNE far
detector design as of January 2015, documented in \anxlbnefd. This
annex provides the detailed descriptions of the systems and components
that the DUNE reference design incorporates; the differences between
the DUNE and LBNE designs are clearly indicated in this
chapter. Important differences include detector size, APA and CPA
placement, and small changes to the APA dimensions.


The detector modules will be constructed sequentially
with the first module coming online as soon as possible and the rest
at a regular pace. A model of the underground experimental area with
the four 10-kt LArTPCs is shown in
Figure~\ref{fig:FarDet-overview-SP}. 
\begin{cdrfigure}[FD reference design]{FarDet-overview-SP}{Left: 3D model of the reference design for the DUNE far detector to be located at the 4850L. Right: Schematic view of the active detector elements showing the plane ordering of the TPC inside the detector.}
\centering
\begin{minipage}[b]{1.0\textwidth}
\begin{center}
\includegraphics[width=.58\textwidth]{FarDet-3D-SP.jpg}
\includegraphics[width=0.38\textwidth]{FarDet-endview-SP.jpg}
\end{center}
\end{minipage}
\end{cdrfigure}
Planning for the conventional facilities calls for construction of the
second cryostat to be completed prior to filling the first so that it
may serve initially as a liquid storage facility.  The detector
technology is expected to improve in the coming years with MicroBooNE,
the SBN experiment and the CERN neutrino-platform development
program. DUNE's staged program allows selection of the optimal design
for each module as the technology evolves.  %detector module at each step.


The reference design, presented in this chapter and documented in the
project cost and schedule, is patterned after the successful ICARUS
experiment, but adapted to the local site requirements at SURF and the
need to scale up the detector size. The configuration of the TPC
detector is shown on the right in Figure~\ref{fig:FarDet-overview-SP}.
The TPC, described in Section~\ref{sec:detectors-fd-ref-tpc}, is
constructed by placing alternating high-voltage cathode planes and
anode readout planes in a bath of ultra-pure liquid argon. Particles
interacting in the argon generate charge and vacuum ultra-violet (VUV)
photons.


The single-phase design offers the advantage that the charge is
collected directly, enabling precision charge calibration. However,
signal levels are low, requiring the use of cold electronics
(Section~\ref{sec:detectors-fd-ref-ce}), and the readout is based on
stereo induction planes, requiring a deconvolution of the induced
signal. A photon detection system
(Section~\ref{sec:detectors-fd-ref-pd}) provides the $t_0$ or event
time for physics processes uncorrelated with the FNAL neutrino beam.
