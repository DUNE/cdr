%%%%%%%%%%%%%%%%%%%%%%%%%%%%%%%%
\section{Overview}
\label{sec:detectors-fd-ref-ov}

The scope of this chapter includes the design, procurement, fabrication, testing, delivery and installation of the mechanical and high-voltage components of the single-phase liquid argon TPC far detector: 

\begin{itemize}
\item Time Projection Chamber (TPC)
\item Data Aquisition (DAQ) and Monitoring
\item Cold Electronics (CE)
\item Photon Detection (PD) System
\item Installation and Commissioning
\end{itemize}

This design meets the required performance for \fixme{whatever it turns out to be}, developed for the DUNE experiment.

A few hints for author:

Please include at least one figure that illustrates the overall FD.

Here is a sample figure: 

\begin{cdrfigure}[short]{label}{long}
%\includegraphics[width=\linewidth]{file}
\end{cdrfigure}

Here is a sample reference to a figure (Figure~\ref{fig:label}).  Notice: ``fig:'' is not present in the label as written in the figure code itself.

Here is a sample table:

\begin{cdrtable}[short]{cc}{label}{long}  %The third argument (reads {cc}) can use c, l, r or p{some length} 
% but please do not include lines like “|c|l|l|”. It CAN look like {cll} or {llp{3cm}}, for instance.
Header Column1 & Header Column 2 \\ \toprowrule
Row 1 & First \\ \colhline
Row 2 & Second \\ \colhline
Row 3 & Third \\
\end{cdrtable}

Here is a sample reference to a table Table~\ref{tab:label}).  Notice: ``tab:'' is not present in the label as written in the table code itself.
