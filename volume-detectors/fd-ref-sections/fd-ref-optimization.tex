\subsection{Reference Design Optimization}
\label{sec:detectors-fd-ref-optimization}

The reference design for the DUNE far detector is designed to meet the
performance requirements; however, the process of optimizing the
design parameters is not complete.  Considerations of physics reach as
well as cost, schedule and risk enter the process of optimizing the
values for the far detector design parameters.  At this 
conceptual design and not a technical design phase the detailed process of
optimizing the far detector continues.  Furthermore, as the
optimization process weighs different values against one another ---
physics reach vs. cost, for example. The optimal choices depend both
on prevailing boundary conditions and the relative values the DUNE
Collaboration, LBNF Facility, host laboratory and funding
agencies assign to the input variables. The DUNE collaboration plans
to exstablish a detector performance optimization task force to review
various possible detector optimizations in light of the new
collaboration and project organizations. This section briefly outlines
the considerations and procedures that have been and will continue to
be used to optimize these parameters: wire spacing,
APA dimensions, wire angle and maximum drift length.

Ideally one would like to estimate the asymptotic far-future
performance of a full simulation, reconstruction and analysis chain, replicated 
for each design parameter choice and then choose the values that maximize sensitivity
while minimizing the cost and producing timely physics results.
More realistically, the procedure is factorized into separate optimization tasks, with
care taken in cases that a shift in one parameter is expected to alter the optimum value
of another.  Frequently, approximations are used when effects are expected to be small.
The status of the far detector simulation and reconstruction tools is given in 
Sec.~\ref{sec:detectors-sc-physics-software}.  While the GEANT4 simulation is fairly mature,
it needs to be tuned to data from the 35-ton prototype and MicroBooNE so that realistic
signal and noise modeling, which are inputs to the optimization procedures below, can be obtained.
The reconstruction tools are under development and thus estimates of asymptotic performance
are input to the Fast Monte Carlo at the moment, though the full simulation will be needed
to estimate the impact of detector design parameters on the performance.

The DUNE far detector has relatively few parameters that affect its highly repetetive
structure and these are treated in turn below.

\subsubsection{Wire Pitch}

The spacing between neighboring sense wires in the APA's is an optimizable parameter.  In principle
it is freely adjustable for all three wire planes, though to minimize the anisotropy of the
detector response, similar wire pitch should be chosen in all three planes.  The choice of
a $\sim$5~mm pitch is documented in Ref.~\cite{docdb-3407}.  The pitch of the
grid plane wires is less important as they are not instrumented,
though the grid wires do shadow the optical detectors and so
they should not be made too fine.

The signal to noise ratio is expected to be proportional to the wire
spacing, assuming that the noise on a channel is not impacted by the
presence of nearby wires, and that the signal is divided among the
available channels.  Thermal noise and uncorrelated electronics noise
satisfies these conditions.  Coherent noise is a special case ---
filters may be applied either online or offline to reduce its impact.
The signal-to-noise ratio requirement is set so that zero-suppression
can function without elaborate noise filtering.  A high signal to
noise ratio has the added benefit of improving pattern-recognition
performance, calorimetric PID performance and $dE/dx$-based
$e-\gamma$ separation.  A large signal to noise ratio is important in
detecting sub-MIP signals, such as nuclear de-excitation photons, and
it is important when adding up the energy of hits on the edges of
showers or on the ends of stubs initiated by supernova neutrino
interactions.  The signal to noise ratio is expected to be higher in
the collection plane than in the two induction planes.  The need to
deconvolute the bipolar signals while filtering noise in the induction
planes means that the collection plane will be the most reliable in
performing $dE/dx$ measurements, though for tracks that travel in a
plane containing a collection wire and the electric field, the
induction planes will be critical for recovering PID efficiency.
Reducing the spacing between wires will have an adverse impact on the
detector performance parameters that depend on the signal to noise,
with the effect seen more prominently in the induction-plane data.

In a detector with fine wire pitch,
one can of course approximate a coarser wire spacing by summing the responses on neighboring
wires.  The random components of the noise will
average with an RMS that grows as the square root of the number of wires instead of linearly.
Instead of summing signals, a more optimized approach is to deconvolve the signal in both
time and wire number, as diffusion and the field response of the detector will cause
signals to be shared among neighboring wires.  If the diffusion characteristics are known,
either a deconvolution may be attempted, or fits to expected distributions with known
smeared widths can be performed, recovering some of the spatial resolution lost to diffusion,
limited by noise.

The main benefit of a finer wire pitch is to obtain higher resolution measurements of the
ionization density left by events in the detector.  The separation of electrons from
photons using the $dE/dx$ measured in the initial part of an electromagnetic shower, is
described in \anxreco.  The first 2.5~mm of a shower is the most important as subsequent
showering stages have not yet taken place, leaving one MIP for an electron and two for a photon
conversion to two electrons, though sometimes the subsequent showering starts earlier.
As the first hit cannot be used to measure $dE/dx$ since it is not known where in the volume
of argon viewed by that wire the track started, the second and subsequent hits must be used.
But the shower can be aligned unfavorably along the wires of one view or another, resulting
in few usable hits.  If no hits are useful, then the $dE/dx$ method cannot be used.  Reducing
the spacing between wires in all three views increases the precision of the measurement of the
initial part of the shower for $e-\gamma$ separation purposes.  The current study described
in \anxreco however only uses the collection plane wires and thus with a more optimal strategy,
some of the efficiency that is lost with 5~mm wire spacing compared with 3~mm can be recovered
by examining the other two views.

Separation of multiple close tracks
is improved with more closely spaced wires.  The position resolution of hits
is expected to be much better along the drift direction than in either of the axes perpendicular
to it, as the readout digitization sample time spacing times the drift velocity is much smaller
than the wire spacing.  As long as the tracks which are desired to be separated from one another
travel with an angle with respect to the plane parallel to the wires, then the fine time resolution
will help with the pattern recognition even if the wire pitch is broad.

The reconstruction of short tracks, such as low-energy
protons ejected by the struck nucleus at the primary vertex of a neutrino scattering event
is improved with higher spatial resolution.  Finer wires also allows for a more precise measurement
of the distances between the primary vertex and photon conversion points, which is the other
component of $\pi^0\rightarrow\gamma\gamma$ separation from electrons.  Topological
identification of two EM showers and their displacement from the primary vertex is expected
to provide a factor of ten to twenty in NC background rejection while retaining at least 90\%
efficiency for CC-$\nu_e$ events.  A hand-scan study comparing liquid argon TPC detector performance
in topological separation of NC events from CC-$\nu_e$ events in a detector with 5~mm wire pitch
and a detector with 10~mm wire pitch~\cite{2008-hand-scan} showed no degradation in performance with
the wider wire pitch.  Automated topological selection has yet to be developed, though it is
anticipated that this finding will remain true for automated selection as well, given that the
radiation length in liquid argon is typically $\sim$30 times the typical wire spacing.

To balance against the gains in physics is the increased cost of the electronics and
online computing resources needed to read out the additional wires if a finer pitch is chosen.
A higher heat load in the liquid is anticipated from the additional cold electronics components.
Manufacturing the APA's is expected to take longer if more wires are to be installed and wrapped.

The spacing between the planes is customarily chosen to be similar to the spacing between
the wires within the planes, though this too is an optimizable parameter.  Narrowing this
spacing improves the sharpness of the signals in time in both the induction and collection planes,
though electronics shaping and diffusion will limit how sharp the signals eventually end up being.
These functions can be deconvolved, though deconvolution and noise filtering produces 
artifacts in the signals.  Studies varying the spacing between the planes can be performed
to estimate the impact on two-hit resolution in time.

Future studies will include the estimation of the difference in performance of $dE/dx$-based
$e-gamma$ separation for events in which the showers are not identified topologically as
candidates for having come from $\pi^0$ decay using automated tools.  Kinematic properties
of the events such as the energies and angles of the photons will correlate the performance
of $dE/dx$-based separation and topological separation, so a full study using both techniques
together will provide the most information.  Ideally an end-to-end analysis of neutrino
scatters and the impact of particle mis-ID rates and detection efficiencies for short tracks
can be propagated to the oscillation parameter sensitivities, but more realistically, the
effect of changing the wire pitch will be most visible in the 
tracking and particle ID efficiencies.

\subsubsection{Wire Angle}

Like the wire pitch, the choice of the angles of the induction-plane wires affects the
physics performance of the detector.  Because the wires wrap from one side of each APA
to the other and back, a discrete ambiguity is added to the continuous one of not knowing
where along a wire charge was deposited corresponding to a measured hit on a channel.
The more times a wire wraps, the more choices arise for this ambiguity.

Reconstruction of three-dimensional objects based on two-dimensional (channel number vs. time)
data requires associating hits in one plane with those in another, or ideally, all three.
If two wires only cross in one place, then there is no ambiguity once the hit is associated
in two views.  Three are required if the wires cross more than once, to break the ambiguity
of even isolated hits.

This association can most easily be done using the arrival time of the hits.  If the time of
a hit is different from that of all other hits in the event, then the association is easy.
In more complex cases, where dense showers produce many hits at similar drift distances,
misassociation can happen.  If misassociation occurs, the wrong choice can be made for
the discrete ambiguity, displacing a reconstructed charge deposition
by multiple meters from its true location.  In the case that the $U$ and $V$ angles are the
same and the number of times a wire wraps around an APA exceeds one, then even a single isolated
hit can be ambiguous.  A small difference in the $U$ and $V$ angles breaks this ambiguity,
though misassociation still occurs in events with multiple nearby hits close in time.
Using clustering methods assists in obtaining the correct ambiguity choices for hits in
dense environments.

Reducing the wire angle reduces the number of crossings.  The angle chosen for the DUNE
far detector reference design ensures that no induction wire crosses any collection wire
more than once.  Misassociation of hits can still happen, though, in events with multiple
hits simultaneously arriving.  Reducing the angle also makes the features in the views
match each other more easily as their shapes will be more similar.

On the other hand, reducing the wire angle worsens the resolution of three-dimensional
reconstruction of hits in the vertical direction.  It also worsens the resolution on the
measured separation between the primary vertex and photon conversion points, though as pointed
out above, the radiation length is much longer than the wire spacing.  The number of hits
available for $dE/dx$ separation of electrons from photons degrades for vertically-going
showers if the wire angle is reduced.  Ref.~\cite{wire-orientation} is a parametric
study of a figure of merit based on the measurability of the two photon conversion
lengths as a function of the induction-plane wire angles.

It was estimated that the loss in resolution from even a small
fraction of hits mis-associated and placed meters away from their true
locations was worse than the resolution loss in one
dimension\cite{docdb-8981}, though studies carried through to
estimate the impact on event detection efficiency, particle ID
performance, and energy resolution are more tied to the final physics
sensitivity of the DUNE far detector.  Work on algorithms to break
ambiguities in complex environments can also allow for steeper wire
angles.

\subsubsection{Wire Length}

The length of the collection-plane wires is determined by the APA dimensions (or vice-versa),
and the length of the induction-plane wires is determined by their angle and the APA
dimensions.  The APA dimensions have other constraints on them, such as the need to transport
them over the ground, lower them down the shaft and handle them underground.  Long APA frames
also suffer from stiffness issues and may require additional cost in order to procure the
members of the required length and stiffness.  Longer wires have higher capacitance than shorter
wires and thus have more noise.  Cold electronics is expected to have very low noise and thus
this effect is not expected to be masked by electronics noise.  In order to meet the signal-to-noise
requirement with a finer wire spacing, the wires may need to be made shorter.

Longer wires also lower the cost of the detector, as fewer electronics channels and APA frames
and winding time and materials are needed to instrument a given volume of liquid argon.

It is anticipated that much of the work needed to study the impact of the wire pitch will inform
the wire length choice due to its impact on the signal to noise ratio.

\subsubsection{Maximum Drift Length}

The maximum drift length is another parameter that is optimizable.  In
this case, the driver for longer lengths is the cost of the detector.
A longer drift length assigns more liquid argon to be read out by any
given channel, reducing the APA count and the channel count.  A longer
drift length also increases the fraction of liquid argon that is
fiducial.  Fiduciality cuts will need to be made around the APA planes
to ensure containment and that the impact of dead argon inside the APA
planes is minimized.  Fewer APA's reduces the amount of non-fiducial
liquid argon.  The detector can be sized to have the desired fiducial
volume, but it must be larger if the maximum drift length is smaller.

A longer drift length however increases the impact of the electron
lifetime, as charge will attach on impurities as it drifts towards the
anode plane.  This couples the requirement on the signal to noise
ratio with the electron lifetime requirement and the dynamic range
requirement, given the drift length.  If the drift length were
shorter, then the electron lifetime would be less critical.

Increasing the drift length also degrades position resolution due to diffusion, where the
spread of a drifting packet of charge increases proportional to the square root of the
drift time.  Charge deposited near the APA's remains well measured, though charge deposited
near the CPA's will suffer from both attenuation and diffusion, lowering the signal to noise
ratio.  Small-signal detection efficiencies and PID performance may decrease for events
near the CPA's.

Sophisticated reconstruction and analysis algorithms can be used to recover resolution that
is thus lost, but these are limited by noise.  Simulation studies in advance of CD-2 will
study the impact of diffusion and noise on the particle reconstruction performance.
