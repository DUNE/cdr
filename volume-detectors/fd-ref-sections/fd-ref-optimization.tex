\subsection{Reference Design Optimization}
\label{sec:detectors-fd-ref-optimization}

Considerations of physics reach as well as cost, schedule and risk
enter the optimization of the detector design parameters.  Ideally one
would like to estimate the asymptotic far-future performance of a full
simulation, reconstruction and analysis chain, replicated for each
design parameter choice and then choose the values that maximize
sensitivity while minimizing the cost and producing timely physics
results. While the GEANT4 simulation is fairly mature, it needs to be
tuned to data from the 35-t prototype and MicroBooNE so that realistic
signal and noise modeling, which are inputs to the optimization
procedures, can improve the performance modelling.  The reconstruction
tools are under development (see
Section~\ref{sec:detectors-sc-physics-software}) and thus, physics
sensitivity is currently optimized using estimates of detector
performance that are input to the Fast Monte Carlo. The DUNE
collaboration plans to establish a detector performance optimization
task force to review various possible detector optimizations in light
of the new collaboration and project organizations.  This section
briefly outlines the considerations and procedures that have been and
will be used to optimize these parameters: the wire pitch, wire angle,
wire length and maximum drift length.  The wire length, angle and
pitch are directly related to the APA dimensions, as discussed in
Section~\ref{sec:detectors-fd-ref-tpc}, and the APA dimensions are
constrained by the needs of manufacturing, storage, transport and
assembly.

\subsubsection{Wire Pitch}

The spacing between neighboring sense wires in the APAs is an
optimizable parameter.  In principle it is freely adjustable for all
three wire planes, though to minimize the anisotropy of the detector
response, similar wire pitch should be chosen in all three planes.
The choice of a $\sim$5~mm pitch is documented in\cite{docdb-3407}.
The pitch of the grid plane wires is less important as they are not
instrumented, though the grid wires do shadow the optical detectors
and therefore should not be made with too fine a pitch.

The signal-to-noise ratio is expected to be proportional to the wire
spacing, assuming that the noise on a channel is not impacted by the
presence of nearby wires, and that the signal is divided among the
available channels.  Thermal noise and uncorrelated electronics noise
satisfy these conditions.  Coherent noise is a special case ---
filters may be applied either online or offline to reduce its impact.

The signal-to-noise ratio requirement is set so that zero-suppression
can function without elaborate noise filtering.  A high
signal-to-noise ratio improves pattern-recognition performance,
calorimetric PID performance and $dE/dx$-based $e-\gamma$ separation.
It is also important for detecting sub-MIP signals, such as nuclear
de-excitation photons, and it is important when adding up the energy
of hits on the edges of showers or on the ends of stubs initiated by
supernova neutrino interactions.

The signal-to-noise ratio is expected to be higher in the collection
plane than in the two induction planes.  The need to deconvolve the
bipolar signals while filtering noise in the induction planes means
that the collection plane will be the most reliable in performing
$dE/dx$ measurements, though for tracks that travel in a plane
containing a collection wire and the electric field, the induction
planes will be critical for recovering PID efficiency.  Reducing the
spacing between wires will have an adverse impact on the detector
performance parameters that depend on the signal to noise, with the
effect seen more prominently in the induction-plane data.

In a detector with fine wire pitch, one can approximate
coarser wire spacing by summing the responses on neighboring wires.
The random components of the noise will average with an RMS that grows 
as the square root of the number of wires. 
Instead of summing signals, a more optimized approach is to deconvolve
the signal in both time and wire number, since 
diffusion and the field
response of the detector will cause signals to be shared among
neighboring wires.  If the diffusion characteristics are known, it is possible to either attempt
a deconvolution
or perform fits with known smeared widths to expected distributions,
thereby recovering some of the
spatial resolution lost to diffusion, \fixme{where the recovery will be?} limited by noise.
\fixme{check above edit; I think the fit is smeared not the distribution; was ambiguous}

The main benefit of a finer wire pitch is the ability to obtain higher resolution
measurements of the ionization density left by events in the detector.
The separation of electrons from photons using the $dE/dx$ measured in
the initial part of an electromagnetic shower is described in
\anxreco.  The first 2.5~cm of a shower is the most important, since 
subsequent showering stages have not yet taken place, leaving one MIP
for an electron and two for a photon conversion to two electrons,
though sometimes the subsequent showering starts earlier.  As the
first hit cannot be used to measure $dE/dx$ (since it is not known
where in the volume of argon viewed by that wire the track started,
the second and subsequent hits must be used.  But the shower can be
aligned unfavorably along the wires of one view or another, resulting
in few usable hits.  If no hits are useful, then the $dE/dx$ method
cannot be used.  Reducing the spacing between wires in all three views
increases the precision of the measurement of the initial part of the
shower for $e-\gamma$ separation purposes.  The current study
described in \anxreco\ only uses the collection plane wires; 
thus with a more optimal strategy, some of the efficiency that is lost
with 5-mm wire spacing compared with 3-mm spacing can be recovered by
examining the other two views.

Separation of multiple close tracks is improved with more closely
spaced wires.  The position resolution of hits is expected to be much
better along the drift direction than in either of the axes
perpendicular to it as the sampling frequency times drift velocity is
much smaller than the wire spacing.  As long as the tracks that should
be separated from one another travel at an angle with respect to the
APA plane, then the fine time resolution will help with the pattern
recognition even if the wire pitch is large.

The reconstruction of short tracks, such as low-energy protons ejected
by the struck nucleus at the primary vertex of a neutrino scattering
event, is improved with higher spatial resolution.  Finer wires also
allows for a more precise measurement of the distances between the
primary vertex and photon conversion points, which is the other
component of $\pi^0\rightarrow\gamma\gamma$ separation from electrons.
Topological identification of two EM showers and their displacement
from the primary vertex is expected to provide a factor of ten to
twenty in NC background rejection while retaining at least 90\%
efficiency for $\nu_e$CC events.  A hand-scan study comparing liquid
argon TPC detector performance in topological separation of NC events
from $\nu_e$CC events in a detector with 5-mm wire pitch and a
detector with 10-mm wire pitch\cite{2008-hand-scan} showed no
degradation in performance with the coarser wire pitch.  Automated
topological selection has yet to be developed, though it is
anticipated that this finding will remain true, given that the
radiation length in liquid argon is typically $\sim$30 times the
typical wire spacing.

The gains in physics from a finer wire pitch must be balanced against
the increased cost of the electronics and online computing resources
needed to read out the additional wires.  In addition, the additional
cold electronics components would likely create a higher heat load in
the liquid and manufacturing the APAs would likely take longer due to
the higher density of wires.

The spacing between the planes is customarily chosen to be similar to
the spacing between the wires within the planes, though this, too, is
an optimizable parameter.  Narrowing this spacing improves the
sharpness of the signals (in time) in both the induction and
collection planes, though electronics shaping and diffusion will limit
the signals' ultimate sharpness.  These functions can be deconvolved,
though deconvolution and noise filtering produces artifacts in the
signals.  Studies varying the spacing between the planes can be
performed to estimate the impact on two-hit time resolution.
%%%%%%%%%%%  HERE!!!!!


Future studies will estimate the difference in
performance of $dE/dx$-based $e-\gamma$ separation for events in which
the showers are not identified topologically as candidates for having
come from $\pi^0$ decay using automated tools.  
\fixme{I can't parse this: is it the difference in the ability to identify (blah) using hand-scans versus
automated tools? No, it's the diff betw dE/dx vs topo... Please clarify}
Kinematic properties
of the events, such as the energies and angles of the photons, will 
correlate the performance of $dE/dx$-based separation and topological
separation, so a full study using both techniques together will
provide the most information.  Ideally an end-to-end analysis of
neutrino scatters and the impact of particle mis-ID rates and
detection efficiencies for short tracks can be propagated to the
oscillation parameter sensitivities, but more realistically, the
effect of changing the wire pitch will be most visible in the tracking
and particle ID efficiencies.

\fixme{It's not efficient for me to spend the time to try to parse the previous pgraph. Please clarify! Anne}

\subsubsection{Wire Angle}
\label{v4:fd-ref-wireangle}

Like the wire pitch, the choice of the angles of the induction-plane
wires relative to the collection plane wires affects the physics
performance of the detector.  Because the wires wrap from one side of
each APA to the other, a discrete ambiguity is added to the
continuous ambiguity of identifying where along the wire the charge was deposited.
% of not knowing where along a wire charge was deposited  corresponding to a measured hit on a
%channel.  The more times a wire wraps, the more choices arise for this
%ambiguity.
%Given a measured hit on a channel, it is not known at what point on the wire the charge was
%deposited; he more times a wire wraps, the more choices arise for this
%ambiguity.

Reconstruction of 3D objects based on 2D data (channel number
vs. time) requires associating hits in one view with those in at least
one other.  If two wires cross only in one place, the ambiguity is
removed once the hit is associated in the two views.  If the wires
cross more than once, three views are required in order to break the
ambiguity of even isolated hits.

This association can most easily be done using the arrival time of the
hits.  If the time of a hit is different from that of all other hits
in the event, then the association is easy.  In more complex cases,
where dense showers produce many hits at similar drift distances,
misassociation can happen.  In this case, the discrete ambiguity makes
it possible to displace a reconstructed charge deposition by multiple
meters from its true location.  In the case that the $U$ and $V$
angles are the same and the number of times a wire wraps around an APA
exceeds one, then even a single isolated hit can be ambiguous.  A
small difference in the $U$ and $V$ angles breaks this ambiguity,
though misassociation still occurs in events with multiple nearby hits
close in time.  The use of clustering methods assists in obtaining the
correct ambiguity choices for hits in dense environments.

Reducing the wire angle reduces the number of crossings, but does not
eliminate the possibility of misassociating hits in events with
multiple hits simultaneously arriving.  
Reducing the angle aligns the
  shapes of features in the different views making it easier to
  correlate them.
The angle chosen for the DUNE far detector
reference design ensures that no induction wire crosses any collection
wire more than once.

On the other hand, reducing the wire angle worsens the resolution of
3D reconstruction of hits in the vertical direction. \fixme{because the
number of hits available for $dE/dx$ separation of electrons from
photons degrades for vertically-going showers?}
It also worsens the resolution on the measured separation between the
primary vertex and photon conversion points, though as pointed out
above, the radiation length is much longer than the wire spacing.  
\fixme{can you get rid of repetitive next sentence?} 
The
number of hits available for $dE/dx$ separation of electrons from
photons degrades for vertically-going showers if the wire angle is
reduced.  A parametric study of a figure-of-merit based on the
measurability of the two photon-conversion lengths as a function of
the induction-plane wire angles is provided
in~\cite{wire-orientation}.

It was estimated that the loss in resolution from even a small fraction
of grossly misplaced hits %misassociated and placed meters away from their true locations 
was worse than the resolution loss in one
dimension\cite{docdb-8981}, though studies have yet to be performed
to estimate the impact on event-detection efficiency,
particle ID performance, and energy resolution are more tied to the
final physics sensitivity of the DUNE far detector.  \fixme{can't parse previous sentence} 
Advances in algorithms to break ambiguities in complex environments can also allow
for steeper wire angles.

The fact that the channel count must be an integer multiple of 128 in
an APA also constrains the wire angles as functions of the APA frame
dimensions, though a procedure to find the proper arrangement of
channels that most closely approximates the optimized wire pitches and
angles and meets the channel count constraint may cause small
deviations in the parameters.

\subsubsection{Wire Length}

The length of the collection-plane wires is determined by the APA
dimensions (or vice-versa), and the length of the induction-plane
wires is determined by their angle and the APA dimensions.  The APA
dimensions are largely constrained by transport and handling needs as
well as stiffness and production cost issues as they get larger.
Capacitance and noise increase with wire length; this effect would
likely not be masked by electronics noise since the cold electronics
is expected to have very low noise.  Therefore, in order to meet the
signal-to-noise requirement with a finer wire spacing, the wires may
need to be made shorter.

On the other hand, longer wires lower the cost of the detector, as
fewer electronics channels and APA frames --- and winding time and
materials --- are needed to instrument a given volume of liquid argon.

It is anticipated that much of the work needed to study the impact of
the wire pitch will inform the wire length choice due to its impact on
the signal-to-noise ratio.

\subsubsection{Maximum Drift Length}

A longer drift length is advantageous for several reasons:
\begin{itemize}
\item It reduces detector cost (this is the driving consideration).
\item It assigns more liquid argon to be read out per channel, reducing the APA count and the channel count.
\item It increases the fraction of liquid argon that is fiducial. Fiduciality cuts need to be made around the APA planes to ensure containment and to minimize the impact of dead argon inside the APA
planes; fewer APAs leads to fewer fiduciality cuts and thus to higher fiducial volume.
\end{itemize}

\fixme{The above is intended to replace next pgraph; see what you think}
The maximum drift length is another parameter that is optimizable.  In
this case, the driver for longer lengths is the cost of the detector.
A longer drift length assigns more liquid argon to be read out by any
given channel, reducing the APA count and the channel count.  A longer
drift length also increases the fraction of liquid argon that is
fiducial.  Fiduciality cuts will need to be made around the APA planes
to ensure containment and that the impact of dead argon inside the APA
planes is minimized.  Fewer APAs reduces the amount of non-fiducial
liquid argon.  The detector can be sized to have the desired fiducial
volume, but it must be larger if the maximum drift length is smaller.

A longer drift length however increases the impact of the electron
lifetime, 
\fixme{increases the NEGATIVE impact of longer e lifetime?}
as charge \fixme{electrons (only neg charge)} will attach on impurities as it drifts towards the
anode plane.  This couples the requirement on the signal-to-noise
ratio with the electron lifetime requirement and the dynamic range
requirement, given the drift length.  If the drift length were
shorter, then the electron lifetime would be less critical. \fixme{This makes it sound like
you want a shorter e lifetime. It looks like we want greater than 3 ms, not less. Please clarify}

Increasing the drift length also degrades position resolution due to
diffusion, where the spread of a drifting packet of charge increases in
proportion to the square root of the drift time.  Charge deposited
near the APAs remains well measured, though charge deposited near the
CPAs will suffer from both attenuation and diffusion, lowering the
signal-to-noise ratio.  Small-signal detection efficiencies and PID
performance may decrease for events near the CPAs.
%
Sophisticated reconstruction and analysis algorithms can be used to
recover resolution that is thus lost, but these are limited by noise.\fixme{the algorithms are limited
by noise? Or the final recovery of resolution is limited by noise?}
Simulation studies in advance of CD-2 will
address the impact of
diffusion and noise on the particle reconstruction performance.
