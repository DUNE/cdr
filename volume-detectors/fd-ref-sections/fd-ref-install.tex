%%%%%%%%%%%%%%%%%%%%%%%%%%%%%%%%
\section{Installation and Commissioning}
\label{sec:detectors-fd-ref-install}

The scope of the Installation and Commissioning task includes the design, procurement, fabrication, testing and delivery of equipment and infrastructure to support installation and commissioning of the detector at the Far site. The following are included in the scope.
\begin{itemize}
\item Detector installation planning
\item Installation equipment design and procurement
\item A full scale mockup to test installation operations and equipment
\item Procurement of support rails for the TPC
\item Procurement and installation Relay rack for the detector systems
\item Material receipt, storage and transport to underground at the far site.
\item Installation of the TPC, Photon Detection and DAQ systems at the far site
\item Commissioning of the detector systems
\end{itemize}

The Installation and Commissioning will have many interfaces with LBNF however will not be responsible for the following installation activities at the far site:
\begin{itemize}
\item Excavation and outfitting of the cavern is the responsibility of the Conventional Facilities (CF), subproject of LBNF
\item Construction and installation of the cryogenics system and cryostats is the responsibility of the cryogenics system subproject of LBNF
\end{itemize}

The design presented here meets the required performance for the Installation and Commissioning of the DUNE far detector. 

The Installation and Commissioning system will provide some permanently installed equipment that is 
used by multiple Detector systems or is integral to the installation process. This equipment includes the 
relay racks, cable management, support rails for the TPC and the remaining outfitting of the detector 
cavern that was too detector specific to be included with the conventional facilities work. The cavern 
outfitting includes a clean area enclosure near the entrance to the cryostat to keep the open hatch and the 
TPC components isolated from the cavern environment. The TPC elements will be supported by a set of 
five support rails permanently mounted at the top of the cryostat. The DUNE TPC arrangement will have 3 
rows of APAs with two of the rows near the cryostat walls. This represents an increase of 50\% in the 
number of APAs and a decrease of the number of CPAs relative to the LBNE (?) configuration. The rails will 
be supported by rods spaced at 5-m intervals from anchor points at the cryostat roof. The rods will be 
installed with and angle bias that allows the rails to return to a level condition after the cryostat and TPC is 
cooled. 

The TPC and photon detectors will require various electrical services for operation including bias voltages, 
power and control signal. The data will be carried out by signal cables. The electrical services will pass 
through ports located on top of the cryostat. Ports will be located above every other APA junction and so 
there will be 78 feedthrough ports. A relay rack will be located adjacent to each port. The rack space will 
be shared between the TPC and photon detection system readout and power supplies. For the upper APA, 
the cables will be preinstalled and routed from the cold side of the feedthrough down and along the 
support rail to the location where they will connect to their corresponding APA. For the lower APAs with 
the electronics near the floor of the cryostat, the cables will be routed from the cold side of the 
feedthrough down the sides of the cryostat. The lower APA cables will be routed in cable trays supported 
from the cryostat.

The installation and commissioning group will develop a detector grounding plan the will involve the 
conventional facilities power distribution and the cryostat design along with the detector systems.  The 
detector will have approximately 300,000 channels of electronics with an intrinsic noise level less than 
1,000 electrons. The channels will be connected to wires that are up to 7 m long. Thus grounding 
shielding and power distribution are critical to the success of the experiment. The grounding will be 
configured to that each detector is on an isolated and separate detector ground which is referenced to 
building ground through a safety saturable inductor.  Dielectric breaks will be used on all conductive 
piping/services which penetrate the cryostat.  A copper ground plane under the steel top plate of each 
cryostat will be used to serve as a central star ground point. 

Local storage in the far site region will be required because detector installation is a much shorter duration 
than the detector components fabrication/assembly time. The local storage facility will also be used for 
checkout and testing of detector components after transportation from the production sites. The 
installation space in the cavern is very limited and so material will be moved from the local storage to the 
cavern at the rate of installation. The Yates shaft will be used to move material underground. Most items 
can be moved inside the cage of the Yates shaft. However, the APAs are too long to fit in the cage and the 
APAs must be slung from underneath the Yates cage. A special APA transportation container will hold 4 
APA in horizontal and vertical orientation for movement in the Yates shaft.  The APA transportation 
container includes and internal rack that can be extracted from the outer container thus avoiding moving 
the outer container into the clean area used for installation.

A clean area enclosure in the range of class 100,000 (ISO 8 equivalent) will be constructed near the 
entrance to the cryostat. The enclosure will have an area for personnel to gown with the appropriate 
clean-room clothing and safety shoes. A large closable door will be located at the drift junction where 
TPC storage containers can be parked to allow unloading of TPC components directly from the container 
into the clean area. The TPC components will be cleaned and protected to a level suitable for installation 
into the cryostat as part of the TPC production process and will be delivered to the far site in clean 
containers.

Several items will be installed in the cryostat for use during TPC installation and removed before the 
cryostat is filled with argon. A temporary lighting system with emergency backup lighting will be in place 
inside the cryostat for TPC installation and removed in section as the TPC installation proceeds. The 
lighting will also be filtered to the appropriate spectrum to protect the photon detection system installed 
in the APAs. A temporary filtered air ventilation system with air monitoring sensors and alarms will assure 
adequate air quality for work inside the cryostat. The system will also include a high sensitivity smoke 
detection system that is interlocked to the power for all devices inside the cryostat. A raised floor would 
be installed at the bottom of the cryostat to protect the cryostat membrane and provide a flat surface 
above the corrugations of the cryostat. The raised floor would be modular so that it can be removed in 
sections as the TPC installation progresses.

A combination of commercial and special design tooling will be required for TPC installation. All of the 
detector components and equipment inside the cryostat will be installed through hatches located at one 
end of the cryostat.  Fixed scaffolds with integral stair towers will be installed temporarily to provide 
personnel access into the cryostat. A rolling scaffold with and integral stair tower would move on the 
cryostat floor to allow personnel access to the top of the cryostat and the TPC connection near the top. 
Special fixtures and commercial gantry hoists would be required to move APAs  from a horizontal 
orientation in a storage racks to a vertical orientation at the cryostat hatches. Special platforms would be 
located at the cryostat hatches to support the lower APA section while the upper APA section is 
connected. The platform would have multiple levels to allow personnel to access the connection points 
and the top of an APA and the junction between an upper and lower APA. The installation equipment and 
installation procedures will be tested with a full-height mock TPC section at a suitable location at 
Fermilab. 

The TPC installation is a highly repetitive process with most steps repeated many times. Prior to the start 
of the TPC installation the DAQ, including relay racks and TPC cables, will be installed so that the APAs 
can be tested immediately after they are located in their final position inside the cryostat. The TPC 
installation will start by installing the two rows of CPA assemblies. This will be accomplished one 
horizontal strip at a time. Once a strip is assembled, it is lifted and another strip is attached below until 
the entire height is reached. Next the end wall field cage will be installed at the end of the cryostat away 
from the end with the hatches. The first step of installing an APA pair is to move an APA with electronics 
towards the bottom into the cryostat and hold it temporarily in the area of the hatch. Next an APA with 
electronics towards the top is positioned above the other APA. The two APAs are joined at the center and 
then lifted on to the support rail. The connected pair of APAs are moved along the support rail to the 
final position at connected to adjacent stacked pair of APAs.  The power and signal cables are connected 
to the APAs and testing of the APAs begins. After it is confirmed that the APAs are functioning properly, 
the field cage sections between APA and CPA are installed. The removable floor section will be removed 
while there are still accessible under the TPC drift cell.  The APA and field cage installation process is 
repeated as all of the APAs are installed in a row progressing towards the end of the cryostat with the 
hatches. The APA and field cage installation process is the same for each of the three rows of APA. After 
the last APA in a row is installed the end field cage will be installed. After the TPC is completely installed 
the installation equipment including the scaffolding will be removed from the cryostat.

After the TPC is installed and all temporary equipment is removed from the cryostat the hatches will be 
closed and all channels of the detector will be tested for expected electronics noise. After successful 
testing the cryostat hatches will be sealed and the purge will proceed followed by a cool-down of the 
cryostat and detector.  Extensive detector testing will be conducted after cool-down before filling with 
liquid argon begins. Filling each 10kt cryostat will require approximately 6 months after which a several 
month long detector commissioning phase will begin.

