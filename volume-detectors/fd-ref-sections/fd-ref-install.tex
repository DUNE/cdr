%%%%%%%%%%%%%%%%%%%%%%%%%%%%%%%%
\section{Installation and Commissioning}
\label{sec:detectors-fd-ref-install}

The scope of the Installation and Commissioning  (I\&C) task includes the
design, procurement, fabrication, labor, testing and delivery of
equipment and infrastructure to support installation and commissioning
of the detector at the far site. The following are included in the
scope:
\begin{itemize}
\item detector installation planning;
\item installation equipment design and procurement;
\item construction of a full scale mockup, consisting of four APAs, two CPAs and associated
  field cage, to test installation operations and
  equipment;
\item procurement of support rails for the TPC;
\item procurement and installation of relay racks to house the
  electronics provided by other subsystems for detector operation;
\item material receipt, storage and transport to underground at the far site;
\item installation of the TPC, photon detection and DAQ systems at the
  far site (support for detector checkout will be provided by the
  subsystems); and
\item coordination of the commissioning for the detector and personnel
  to support detector operations.
\end{itemize}

I\&C will have many interfaces with LBNF, and LBNF
has certain responsibilities of its own, including the following.
%however, it will not be responsible for the following installation activities at the far site:
\begin{itemize}
\item Excavation and outfitting of the cavern is the responsibility of
  the Conventional Facilities (CF) subproject of LBNF.
\item Construction and installation of the cryogenics system and
  cryostats is the responsibility of the cryogenics system subproject
  of LBNF.
\end{itemize}
%The design presented here meets the required performance for the
%Installation and Commissioning of the DUNE far detector.

\subsection{Equipment and Services}

The I\&C system provides 
permanently installed equipment that is used by multiple detector
systems and/or is integral to the installation process. This 
includes the relay racks, cable management and support rails for the TPC.
I\&C is also responsible for several detector-specific aspects of the cavern outfitting, including a clean area enclosure near the cryostat hatch to 
isolate  the open hatch and the TPC components
from the cavern environment. 

The TPC elements (APAs and CPAs, described in Section~\ref{sec:detectors-fd-ref-tpc}), are
supported by a set of five support rails permanently mounted at the
top of the cryostat. The rails are supported
by rods spaced at 5-m intervals from anchor points at the cryostat
roof. The rods are installed with an angle bias that allows the
rails to return to a level condition after the cryostat and TPC are
cooled. 
%\fixme{Is the following relevant? It's more for the TPC section}  <--- Jack said ok to remove
%The APAs are arranged in three rows
%with two of the rows near the cryostat walls. Relative to the LBNE configuration, 
%this represents an
%increase of 50\% in the number of APAs and a corresponding decrease of the number of
%CPAs (see \anxlbnefd). 

The TPC and photon detectors require various electrical services
for operation, including bias voltages, power and control signal. These services
follow a grounding plan (still under development) 
discussed in Section~\ref{sec:detectors-fd-ref-install-ground}.  The
electrical services pass through a total of 78 feedthrough ports located on top of the
cryostat, above every other APA junction.  A relay rack is located
adjacent to each port. The rack space is shared between the TPC,
photon detection system readout and power supplies. For the upper APA of each pair,
the cables are pre-installed and routed from the cold side of the
feedthrough down and along the support rail to the location where they
connect to their corresponding APA. For the lower APAs with the
electronics near the floor of the cryostat, the cables are routed
from the cold side of the feedthrough down the sides of the
cryostat. The lower APA cables are routed in cable trays supported
from the cryostat walls.

\subsection{TPC Installation Process}

Once detector components arrive at the far site they are put in a
facility used for both storage and testing/checkout, called an integration facility.  
Material is moved from the
integration facility to the cavern (after undergoing checkout)  via the Yates shaft. 
As installation space in
the cavern is very limited, the moves will take place at the rate of installation. 
 Most items can be
transported inside the Yates shaft cage, however, the APAs are too
long to fit in the cage and therefore are slung underneath it in a special %APA transportation 
container that holds four APAs 
in an internal rack. % for movement in the Yates shaft.The APA transportation container includes 
This cleaner rack %inside the container 
can be extracted from the outer container into the clean area used for installation.
%thus avoiding moving the outer
%container into .
The TPC components will have been cleaned and
protected to a level suitable for installation into the cryostat as
part of the TPC production process, and will %be delivered to 
have arrived at the far
site in clean containers.

The clean area enclosure, in the range of class 100,000 (ISO 8
equivalent), %will be constructed near the entrance to the cryostat. The enclosure will have 
provides an area for personnel to gown with the appropriate
clean-room clothing and safety shoes. A large closable door is
located at the drift junction where TPC storage containers can be
parked to allow unloading of the TPC components 
directly %from the container 
into the clean area. 

The TPC installation process requires the temporary installation of several 
items in the cryostat before it is filled with argon.
\begin{itemize}
\item A lighting system with emergency backup lighting will be installed and then removed in sections
as the TPC installation proceeds. (This lighting will be filtered
to the appropriate spectrum to protect the photon detection system
installed in the APAs.) 
\item A filtered air ventilation system
with air-monitoring sensors and alarms will be installed to ensure adequate air
quality for work inside the cryostat. The system will also include a
high-sensitivity smoke-detection system that is interlocked to the
power for all devices inside the cryostat. 
\item A raised floor will be
installed at the bottom of the cryostat to protect the cryostat
membrane and provide a flat surface above the corrugations of the
cryostat. A modular design will allow it to be removed
in sections as the TPC installation progresses.
\end{itemize}



A combination of commercial and specially designed tooling will be
required for TPC installation. All of the detector components and
equipment inside the cryostat will be inserted through hatches
located at one end of the cryostat.  Temporary fixed scaffolds with integral
stair towers will  provide personnel access
into the cryostat. A rolling scaffold, on the cryostat raised floor, also with an integral stair tower,
will provide access to the top
of the cryostat where the TPC connections are made.  Special fixtures
and commercial gantry hoists are required to move APAs from a
horizontal orientation in storage racks to a vertical orientation at
the cryostat hatches. Special platforms located at the
cryostat hatches will support each lower APA section while its upper APA
section is connected. The platform will have multiple levels to allow
personnel to access the connection points at the top of an APA and at the
junction between an upper and lower APA. The installation equipment
and installation procedures will be tested with a full-height mock TPC
section at a suitable location at Fermilab.

%The TPC installation is a highly repetitive process. % with most steps repeated many times. 
%
Installation of the TPC is preceded by installation of the DAQ, including relay racks and TPC cables, 
 in order to allow immediate testing of APAs upon their placement in the cryostat.

The TPC installation starts with installation of the cathode planes, one 
row at a time, starting with the top row of a plane, and progressing, one CPA at a time, 
from the far end of the
cryostat to the hatch end.  As each cryostat-length row is completed, 
it is lifted, and the next row
is attached below it in the same manner; this is repeated until all four rows of the cathode plane are in place.
At this point, the end-wall field cage is installed at the non-hatch end of
the cathode plane. 

APA installation begins next. %The first step of installing an APA pair is to move an 
An APA, electronics side down, is first moved %towards the bottom 
into the cryostat and held temporarily in the area of the
hatch. A second APA, its pair, is positioned above 
the first, electronics side up. The two APAs are joined at the
center, lifted and attached to the support rail. The connected pair is 
moved along the support rail to its designated position and, except for the first pair,
connected to the previously installed adjacent stacked pair.  
%
At this point the power and signal
cables are connected to the APAs for testing. After proper functioning
is confirmed, the field cage
sections between the APAs and their facing CPAs are installed and the raised floor
sections in that area are removed.  This APA and field cage installation process is repeated progressing towards 
the hatch end of the cryostat until the entire anode plane is in place; the field cage is then
installed. The process is then repeated
for the other anode planes. Once TPC installation is complete, the installation equipment and
the scaffolding is removed from the cryostat.


Once the TPC is installed and all temporary equipment is removed from
the cryostat, the hatches are closed and all channels of the
detector are tested for expected electronics noise. After
successful testing, the cryostat hatches are sealed and the purge
proceeds, followed by a cool-down of the cryostat and detector.
At this point extensive detector testing will be conducted prior to 
filling with LAr. Filling each \ktadj{10} cryostat 
requires approximately six months, after which a several-month-long
detector commissioning phase begins.

\subsection{Grounding}
\label{sec:detectors-fd-ref-install-ground}

The detector will have approximately 300,000 channels of electronics with
an intrinsic noise level less than 1,000 electrons. The channels will
be connected to signal collection wires that are up to 7~m long, thus
grounding, shielding and power distribution are critical to the success
of the experiment.
The installation and commissioning group will develop a detector
grounding plan that coordinates between the CF
power distribution, cryostat design and the detector systems.   The grounding will be configured such that each
detector is on an isolated and separate detector ground that is
referenced to building ground through a safety saturable inductor.
Dielectric breaks will be used on all conductive piping/services that
penetrate the cryostat.  A copper ground plate under the steel top
plate of each cryostat will be provided as part of the cryostat and
used to serve as a central star ground point.


