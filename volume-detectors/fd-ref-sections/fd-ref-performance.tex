
\section{Reference Design Expected Performance}
\label{sec:detectors-fd-ref-perf}

The physics programmatic requirements are described in \volphys,
separately for the long-baseline oscillation, atmospheric, supernova,
and nucleon decay physics programs.  This section briefly outlines the
numerical detector performance parameters needed to meet the
programmatic requirements, and also the ability of the reference
design DUNE far detector to achieve these performance parameters.  The
expected performance of the DUNE far detector is based on the measured
performance of the ICARUS\cite{Amerio:2004ze} and
ArgoNeuT\cite{Anderson:2012vc} detectors, on scanned Monte Carlo
events\cite{docdb-6954}, and on newer studies with automated
reconstruction, which are described in
Section~\ref{sec:detectors-sc-physics-software-simulation-fd}
and~\ref{sec:detectors-sc-physics-software-reconstruction-fd}, and in
\anxreco.  Simulation and reconstruction studies are ongoing.  While
many components are in place, a full end-to-end simulation,
reconstruction, and analysis chain does not yet exist, and thus many
of the numerical performance requirements are estimates.  Some of the
numerical requirements correspond to achievements by ICARUS and
ArgoNeuT, although these experiments differ somewhat from the DUNE far
detector.  Additional parameters will be calibrated using the data
from LArIAT and the two CERN prototypes, the \cernsingleproto{} and
the \cerndualproto.

The {\bf PRELIMINARY} numerical detector performance parameters are
listed in Table~\ref{tab:TPC-metric}.  The required performance values
are given, as well as achieved values (if any) and the expected DUNE
performance.  The rest of this section describes each parameter and
its connection to the detector design and the physics goals.
\begin{cdrtable}[{\bf PRELIMINARY} Far Detector Performance Expectations]{llll}{TPC-metric}{({\bf PRELIMINARY}) Summary of the most 
important performance parameters of the DUNE reference far detector. Included are the parameters, 
previous detector performance, and projected performance with references to relevant studies.  Notes:
$^1$For a MIP at the CPA, minimum in all three views, for any track angle;
$^2$Achieved for the collection view;
$^3$In order for the fiducial volume to be known to $\pm 1\%$;
$^4$For a sample of stopping muons;
$^5$For electron stubs with $E>5$~MeV.
} 
%The third argument (reads {cc}) can use c, l, r or p{some length}  e.g. {clll} or {llp{3cm}}, for instance.
Parameter & Reference Perfromance & Achieved Elsewhere & Expected Performance \\ \toprowrule
Signal/Noise Ratio$^1$ & 9:1 & 10:1~\cite{Antonello:2015zea,Antonello:2014eha}$^2$ & 9:1 \\ \colhline
Electron Lifetime & 3~ms & $>15$~ms~\cite{Antonello:2014eha} & $>3$~ms \\ \colhline
Uncertainty on Charge & & & \\
Loss due to Lifetime  &   $<1\%$  & $<1\%$~\cite{Antonello:2014eha} & $<1\%$ \\ \colhline
Dynamic Range of Hit & & & \\
Charge Measurement & 15 MIP & & 15 MIP \\ \colhline
% two-hit resolution needs more study.  And likely a different resolution along the drift axis than in the other two directions
% Two-Hit Resolution & 2~mm & & 2~mm \\ \colhline
% Track-finding efficiency is expected to be high.  Needs study to connect it to physics performance
%Track-Finding Efficiency\footnote{For tracks with $L>5$~cm} & $>98\%$ & & $>98\%$ \\ \colhline
Vertex Position Resolution$^3$ & (2.5,2.5,2.5)~cm & & (0.5,0.8,2.0)~cm~\cite{Marshall:2013bda,Marshall:2012hh}\\ \colhline
$e-\gamma$ separation $\epsilon_e$ & 0.9 & & 0.9 \\ \colhline
$e-\gamma$ separation $\gamma$ rejection & 0.99 & & 0.99 \\ \colhline
Multiple Scattering Resolution & & & \\
on muon momentum$^4$ & $\sim$18\% & $\sim$18\%~\cite{gibinmuon,Ankowski:2006ts} & $\sim$18\% \\ \colhline
% electron energy scale uncertainty requirement from LBNE DocDB 8741
Electron Energy Scale & & & From LArIAT \\
Uncertainty & 5\% & 2.2\%\cite{ICARUS-pizero} &  and CERN Prototype \\ \colhline
Electron Energy Resolution & $0.15/\sqrt{E{\rm (MeV)}}$ &$0.33/\sqrt{E{\rm (MeV)}}$  \cite{ICARUS-pizero} & From LArIAT \\
 & $\oplus 1\%$ &  +1\% & and CERN Prototype \\ \colhline
Energy Resolution for & & & From LArIAT\\
Stopping Hadrons & 1-5\% & & and CERN Prototype \\ \colhline
Stub-Finding Efficiency$^5$ & 90\% & & $>90\%$ \\ \colhline
%Stub Arrival Time Resolution & 0.1~ms & & \\
%Efficiency for & & & \\
%finding $t_0$ for & & & \\
%a contained & & & \\
%100 MeV $K^\pm$ & 99\% & & 99\% \\
\end{cdrtable}


The signal-to-noise ratio requirement is motivated by the need to
detect weak signals in a large detector that has a low signal rate,
while limiting the required output data volume.  It is set at 9:1 for
a minimum-ionizing particle in all three views, for any orientation of
the track.  We require this ratio for all particles in the detector,
specifically those ionizing the liquid argon close to the CPA, where
the reattachment effects are greatest.  Since the strategy is to
zero-suppress the data, we would like the data volume from random
excursions of the noise over the zero-suppression threshold to be a
vanishingly small fraction of all ADC samples, while preserving the
ability to detect sub-MIP signals, such as those from nuclear
de-excitation photons, or isolated hits on the edges of
electromagnetic showers.  Since the noise in the detector may vary by
channel and by time, and may include, in addition to thermal noise
from the wires and the electronics, coherent noise sources from
electromagnetic pickup and acoustical vibrations of the wires among
other sources, we would like sufficient contingency on the
signal-to-noise ratio to ensure that the detector meets the
programmatic physics requirements.  A value of 10:1 was achieved by
ICARUS\cite{Antonello:2015zea,Antonello:2014eha}, and even higher
values by Long Bo\cite{Bromberg:2015uia}, which benefits from the use
of cold electronics as will the DUNE far detector, though Long Bo had
much shorter wires than DUNE.

The electron lifetime requirement of 3~ms is placed in order to
preserve the signal-to-noise ratio across the entire detector volume
in the presence of noise sources that have not yet been foreseen.  A
shorter lifetime also places demands on the dynamic range of the
ADC's, as a sufficiently large gain will be necessary to see weak
signals at the CPA, while at the same time, strong signals near the
APA's should be recorded without saturation if possible.  The
calorimetric energy resolution of low-energy electrons is also highly
sensitive to the lifetime for electrons that do not record flashes in
the photon-detection system.  The energy resolution is approximately
20\% for electrons of energy below 50~MeV but without corresponding
photon flashes in a detector with a 2.5~m maximum drift length,
assuming only an average correction is applied for the lifetime.  This
resolution rapidly degrades for lower lifetimes and longer drift
lengths.  For a maximum drift length of 3.6~m and an electron lifetime
of 1.5~ms, the estimate is that the energy resolution degrades to
44\%.  A lifetime of 3~ms is consistent with that achieved by the
35-ton prototype.  The ICARUS collaboration has reported a much longer
lifetime, of $>$15~ms\cite{Antonello:2014eha}.

The charge loss due to lifetime effects is expected to be well
measured in the DUNE far detector.  In addition to the cosmic-ray
muons which accumulate at a rate of 0.259~Hz (see \anxrates), the
laser calibration system and purity monitors will provide detailed
time-dependent measurements of the electron lifetime.  This
requirement is placed in order to meet the energy scale and resolution
requirements for electrons, and to a lesser extent, other particles
that rely on $dE/dx$ measurements to perform particle identification.

The dynamic range requirement is placed in order to particle
ionization densities from one MIP up to the last hit on a track before
a particle stops.  The typical application is for protons, where data
from ArgoNeuT show roughly a factor of 15 between the lowest-charge
hit and the highest.  Nonetheless, particles also travel along wires
and dense showers may require even more dynamic range before
saturation.  The desire to measure sub-MIP signals also expands the
desired dynamic range.  MicroBooNE set a requirement of 50 on the
signal dynamic range\cite{microboonetdr}.  The dynamic range
requirement is effectively a compound requirement on the noise level,
the electron lifetime, and the number of bits in the ADC.

The primary vertex position resolution requirement is placed in order
to keep it from being a significant source of uncertainty on the
fiducial volume determination, though this effect is mitigated if the
resolution is well known.  The current resoluton from PANDORA easily
meets this requirement.  The axis along which the resolution is the
weakest is along the neutrino beam direction, and the asymmetry in the
achieved resolution is not a result of detector anisotropy.  Tighter
demands on the primary vertex position resolution will be made by
topological selection of $\pi^0\rightarrow\gamma\gamma$ decays, which
require pointing of the photon-induced showers back to the primary
vertex.

In order to reduce the neutral-current background to $\nu_e$CC events
by a factor of roughly 100, information from the $dE/dx$ of the
initial $\approx 2.5$~cm of an electromagnetic shower must be combined
with the topological $\pi^0\rightarrow\gamma\gamma$
selection\cite{docdb-6954}.  Current Monte Carlo studies indicate
that, for showers with enough hits in the initial part to measure
$dE/dx$, the performance of the ionization method is roughly 90\%
electron efficiency with a 90\% rejection factor for single photons.
A topological hand-scan indicates that a signal $\nu_e$CC signal
efficiency of $\approx 80$\% with a 95\% rejection of neutral-current
background can be obtained.  with optimizations to the $dE/dx$
analysis and automating the pattern-recognition identification of
$\pi^0$ decays by topology, it is anticipated that the requested level
of 99\% $\pi^0$ rejection can be obtained at 90\% signal efficiency.

The momentum of muons in $\nu_\mu$CC events is an important ingredient
in measuring the $\nu_\mu$ energy spectrum in the far detector, which
is one of the inputs to the oscillation parameter fits.  Muons which
stop in the detector volume will be well-measured using their ranges,
though many will not be contained, though the distribution of
deviations of the muon track from a straight line is a function of the
muon momentum.  The expected performance of $\pm$18\% on the muon
momentum is achieved by ICARUS for a sample of stopping muons, where
the momentum measured by multiple scattering can be compared against
that obtained from the range.  It is anticipated that the resolution
will deteriorate for higher-energy muons because they scatter less.

The requirements on the electron energy scale uncertainty and the
resolution are driven by the need to analyze the reconstructed $\nu_e$
energy spectrum to extract oscillation parameters in the
high-statistics phase of DUNE.  A fraction of the energy of an
electromagnetic shower escapes in undetected low-energy photons which
can be simulated, but must be calibrated in data in order to give
confidence in the uncertainty.  An absolute energy scale will need to
come from test-beam data -- LArIAT and the CERN prototypes, the
\cernsingleproto{} and the \cerndualproto.  Analyzing
$\pi^0\rightarrow\gamma\gamma$ decays in ICARUS\cite{ICARUS-pizero}
gives an achieved $\pm 2.2$\% uncertainty on the electromagnetic
energy scale.  The same data also constrain the energy resolution.
The proposed data sets from the test-beam experiments will easily
measure these, though the results will need to be extrapolated to the
DUNE far detector geometry and readout details using a full
simulation.  Similarly, the detector response to hadrons -- protons,
charged pions and kaons, will be calibrated to the necessary precision
by LArIAT and the CERN prototypes.

Many low-energy (5-50~MeV) electron-type neutrino interactions
appearing close together in time is the signature for a supernova
event.  A 5~MeV electron is expected to hit four wires in the DUNE far
detector, and given the signal-to-noise requirement above, it is
anticipated that this signal will be easy to separate from noise with
the required 90\% efficiency.  Similarly, the photon-detection system
is expected to see the energy from a proton-decay event resulting in a
100~MeV kaon without loss of efficiency.

Absent from the list is a requirement on the two-hit resolution In the
direction parallel to the drift field, this resolution is expected to
be very good, of the order of 2~mm, given existing ArgoNeuT data and
simulations.  The resolution in the other two dimensions is governed
by the wire spacing.  Separation of hits is important for pattern
recognition, for counting tracks near the primary vertex which is
important for classifying neutrino scatters as quasi-elastic,
resonant, or DIS, and for associating dense groups of tracks in
showers between views.  More study is required in order to determine
the needed two-hit resolution.

It is expected that as the software tools improve and as measurements
from MicroBooNE and other dedicated test-beam programs become
available the uncertainties on the projected performance will become
smaller.
