%%%%%%%%%%%%%%%%%%%%%%%%%%%%%%%%
\section{Overview} %Near Detector Systems}
\label{sec:detectors-nd-ref-ov}

This chapter describes the reference design of the DUNE Near Detector
Systems (NDS). The scope includes the design, procurement,
fabrication, testing, delivery and installation of the NDS components:
\begin{itemize}
\item Fine-Grained Tracker (FGT) near neutrino detector (NND)
\item Beamline Measurement System (BLM)
\item NDS Data Acquisition system (DAQ)  
\end{itemize}
Detailed descriptions of the NDS subsystems are provided in \anxndref. 


The concept and design of the reference DUNE-ND evolved from the
experience gained from MINOS, the first generation of long-baseline
neutrino experiment at Fermilab, NOvA, the second generation
experiment, the high resolution NOMAD detector at CERN and the T2K
detector at JPARC. MINOS and NOvA employ functionally `identical'
detectors which fulfill the mission of these experiments, given the
statistics and resolution of the respective far detectors.  DUNE, the
third generation experiment at Fermilab, has more ambitious goals:
discovery of CP-violation, discovery of mass hierarchy, and a search
for physics beyond PMNS with unparalleled precision. DUNE will have a
more intense neutrino source and a higher resolution massive FD.  To
meet the ultimate systematic precision needed to fulfill these goals,
the ND must thoroughly characterize the beam composed of muon
neutrinos/antineutrinos and electron neutrinos/antineutrinos. It must
precisely measure the cross-sections and particle-yields of various
neutrino processes.  The particle-yields include the multiplicity and
momentum distributions of pions, kaons, protons, ... produced in the
hadronic jet that constitute backgrounds to the appearance or disappearance oscillation signals.
%An ``identical ND", like those in MINOS or NOvA with the experience gained therein, cannot accomplish these requirements. 

As pointed out in Chapter 6 of CDR \volphys %Chapter~\ref{ch:physics-nd} of Volume-2
%, because not a single spectrum at FD is identical to that at the ND 
the concept of `identical' detectors is an over-simplification.
Furthermore the need to precisely quantify the neutrino source and
cross-sections, including the hadronic composition, motivates
a high resolution ND. The NOMAD experience suggests that a high
resolution detector, capable of measuring $e^{\pm}$, $\mu^{\pm}$,
$\pi^{-,+,0}$, proton and $K^{0}$, would partially meet the
challenges of DUNE --- the detector must be augmented to measure, and
thereby precisely model, the neutrino-nuclear effects. The reference
DUNE-ND is a next generation of near-detector concept
compared to the T2K experiment. Such a detector will
enrich the physics potential of the DUNE/LBNF program.  Complementary
LAr-detector(s) upstream of the high resolution ND will enhance the
capability of the ND complex.


The reference DUNE-ND, a Fine-Grained Tracker (FGT) near detector,
consists of a straw-tube tracking detector (STT) and electromagnetic
calorimeter (ECAL) inside of a 0.4-T dipole magnet. In addition, Muon
Identifiers (MuIDs) are located in the steel of the magnet, as well as
upstream and downstream of the STT. The FGT is designed to make
precision measurements of the neutrino fluxes, cross sections, signal
and background rates at the percent level.

The Beamline Measurement System (BLM) will be located in the region of
the Absorber Complex at the downstream end of the decay region to
measure the muon fluxes from hadron decay. The absorber itself is part
of the LBNF Beamline.  The BLM is intended to determine the neutrino
fluxes and spectra and to monitor the beam profile on a spill-by-spill
basis, and will operate for the life of the experiment.

The Near Detector System Data Acquisition system (NDS-DAQ) collects
raw data from each NDS detector's individual DAQ, issues triggers,
adds precision timing data from a global positioning system (GPS), and
builds events.  The NDS-DAQ is made up of three parts: NDS Master DAQ
(NDS-MDAQ), the Beamline Measurements DAQ (BLM-DAQ) and the Near
Neutrino Detector DAQ (NND-DAQ).
