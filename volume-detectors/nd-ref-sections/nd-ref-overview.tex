%%%%%%%%%%%%%%%%%%%%%%%%%%%%%%%%
\section{Near Detector Systems}
\label{sec:detectors-nd-ref-ov}

The scope of the Near Detector Systems (NDS) includes the design, procurement, fabrication, testing, delivery and installation of the components of the DUNE near detector reference design: 

\begin{itemize}
\item Fine-Grained Tracker (FGT) near neutrino detector (NND);
\item Beamline Measurement System (BLM);  
\item NDS Data Acquisition system (DAQ).  
\end{itemize}

The DUNE Fine-Grained Tracker (FGT) near detector consists of a straw-tube
tracking detector (STT) and electromagnetic calorimeter (ECAL) inside of a 0.4-T
dipole magnet. In addition, Muon Identifiers (MuIDs) are located in the
steel of the magnet, as well as upstream and downstream of the STT. The FGT
is designed to make precision measurements of the neutrino fluxes, 
cross sections, signal rates and background rates. 

The Beamline Measurement System (BLM) will be located in the region of the Absorber Complex at 
the downstream end of the decay region to measure the muon fluxes from hadron decay. The 
absorber itself is part of the LBNF Beamline. 
The BLM is intended to determine the neutrino fluxes and spectra
and to monitor the beam profile on a spill-by-spill basis, and will operate for the life of the
experiment. 

The Near Detector System Data Acquisition system (NDS-DAQ) collects raw data from each NDS detector's
individual DAQ, issues triggers, adds precision timing 
data from a global positioning system (GPS), and builds events. 
The NDS-DAQ is made up of three parts: NDS Master DAQ (NDS-MDAQ), the Beamline Measurements 
DAQ (BLM-DAQ) and the Near Neutrino Detector DAQ (NND-DAQ).



%\begin{cdrtable}[short]{cc}{label}{long}  %The third argument (reads {cc}) can use c, l, r or p{some length} 
%%%% but please do not include lines like “|c|l|l|”. It CAN look like {cll} or {llp{3cm}}, for instance.
%Header Column1 & Header Column 2 \\ \toprowrule
%Row 1 & First \\ \colhline
%Row 2 & Second \\ \colhline
%Row 3 & Third \\
%\end{cdrtable}


