
%%%%%%%%%%%%%%%%%%%%%%%%%%%%%%%% 
\section{Matching the ND Requirements for DUNE/LBNF} 
\label{cdrsec:detectors-nd-ref-fgt-req}

The scope of the this section addresses how the the reference NND ~\ref{cdrsec:detectors-nd-ref}, detailed 
in ~\ref{cdrsec:detectors-nd-ref-fgt}, meets the requirements  to accomplish the oscillation studies 
~\ref{sec:physics-lbnosc-beamnd-req} and the short baseline precision measurements ~\ref{sec-nd-sbp} 
and searches ~\ref{sec-nd-np}. First, we present the oscillation related systematics; 
the systematics affecting the  precision measurements and new-physics search program follows. 


\subsection{Matching ND Requirements for the Oscillation Analyses} 
\label{cdrsec:detectors-nd-ref-fgt-req-oscl}

The Table~\ref{tab:nuesysts} in Section~\ref{sec:physics-lbnosc-beamnd-req} presents a conservative 
projection of systematic errors affecting the $\nu_e$ appearance. The FGT measurement alleviates the 
systematic errors as enumerated below. 

\begin{itemize}
    \item {\bf Beam $\nu_e$:} The FGT will offer an event-by-event  measurement of the beam $\nu_e$ via the  identification 
    of the emergent $e^-$ in STT while rejecting the $\pi^0 \rightarrow \gamma \rightarrow e$  
    background via the determination of the missing-$P_T$ vector to a high degree.  In a 5-year neutrino-run (focus-positive) 
    the FGT will accumulate $500,000$ $\nu_e$ sample with $\simeq 55\%$ efficiency and  $\geq 95\%$ purity.  
    The resulting $\nu_e / \nu_\mu$ ratio will be determined to  $\leq 1\%$ precision. Furthermore, 
    by constraining the sources of $\nu_e$ ($\mu^+$, $K^+$, and $K^0_L$),   the FGT will predict the the ratio of  
    $\nu_e$ and $\nu_\mu$ spectra at the far detector (FD) with respect to the near detector (ND), as a function 
    of the neutrino energy, FD/ND ($E_\nu$).    

    \item {\bf  Beam $\bar\nu_e$:} Although the FD does not distinguish $\nu_e$ from $\bar\nu_e$, the FGT would 
    accurately measure the beam $\bar\nu_e$ by identifying the emergent $e^+$ in the STT with efficiency and purity 
    similar to those for $\nu_e$.  (We point out that the dominant kaonic source of $\bar\nu_e$ is $K^0_L$; 
    the neutrino spectra from $K^0_L$ are different from those of $K^+$ at the FD.)  In a 5-year neutrino-run, the FGT 
    would accumulate 40,000 $\bar\nu_e$ and $5\times 10^6$ $\bar\nu_\mu$ samples, providing a precise FD/ND 
    prediction for the anti-neutrino to $\simeq 1\%$. 
    These constraints will be even more valuable during the anti-neutrino run (focus-negative) where the wrong-sign 
    backgrounds are larger.  

    \item {\bf  Cross Sections:} First, the FGT would conduct an in situ measurement of  the absolute flux, via $\nu$-e 
    scattering, to $\simeq 2.5\%$ precision. Second, in the radiator targets, the FGT would measure the exclusive channels, 
    such as quasi-elastic, resonance, coherent-mesons, and the inclusive DIS channel, with unparalleled precision. Since 
    the Argon is the nuclear target of FD, a set of various nuclear targets will allow to translate the cross-section   
    measurements to $\nu$-Ar scattering. 

    \item {\bf  Nuclear Effects:} The FGT would employ a suite of nuclear targets including Ar-gas in pressurized tubes, 
    a thin solid calcium target (which has the same A=40 at Ar), a C-target, etc (see Section~\ref{cdrsec:detectors-nd-ref-fgt}). 
    Specifically, the number of $\nu$-Ar interaction will be $ 10$ times larger than that expected in a 40 kt FD, without 
    oscillations. Additionally, comparisons of calculations of elastic and  inelastic  interactions in  Ar versus Ca, 
    including the FSI effect, indicate negligible differences between the two targets. Thus, the combination of the 
    Ar and Ca targets would provide a strong constraint on the nuclear effects from both initial and final state interactions. 
    Finally, FGT's ability to isolate $\nu$ ($\bar\nu$) off free-hydrogen, via subtraction of hydrocarbon and carbon targets, 
    would provide a {\em model-independent} measurement of nuclear effects. 

    \item {\bf  Hadronization:} A notable strength of FGT is to identify the yield of $\pi^0$ $separately$ in neutral-current 
    (NC) and charged-current (CC) interactions. (The estimated $\pi^0$ detection efficiency is $\simeq 50\%$.) In addition, 
    FGT would determine the yields of $\pi^-$ and $\pi^+$, the dominant backgrounds to the $\nu_\mu$- and 
    $\bar\nu_\mu$-disappearance. Finally, the measurement of the composition, energy, and angle of the hadronic jet 
    would provide a tight constraint on the overall hadronization models. 

    \item {\bf Energy Scale:}  Because the average density of FGT, 0.1 gm/cm$^3$, is close to that of liquid hydrogen,  
    we will be able to measure the missing-$P_T$ vector in the CC processes, besides accurately measuring the lepton 
    and hadron energies. This redundant missing-$P_T$ vector measurement provides  a most important constraint 
    on the neutrino and antineutrino energy scales. Measurements of exclusive topologies offer additional constraints 
    on the neutrino energy scale. 
                 
\end{itemize}

In summary FGT would accurately quantify all four neutrino species and predict the ratio FD/ND for them. 
It will measure the 4-momenta of the outgoing hadrons composing the hadronic jets in a variety of 
nuclear targets, in essence proving a data-driven `event generator' which can be applied to the FD. 

One notable lacuna of the FGT is that it cannot `cancel' the reconstruction inefficiencies in the FD versus ND. 
Only an identical ND can effect such a cancellation. However, given the detailed programs to calibrate the LAr-detector 
in a test beam and the multiple efforts with neutrino experiments employing LAr detectors, by the time DUNE 
becomes operational the reconstruction of particles in LAr will likely be well understood. Finally in 
Section~\ref{sec:detectors-nd-alt} we outline the enhancement of the ND complex by placing 
complementary LAr-detector(s) upstream of FGT. 



\subsection{Matching ND Requirements for the Short Baseline Precision Measurements and Searches} 
\label{cdrsec:detectors-nd-ref-fgt-req-sbp}


Sections~\ref{sec-nd-sbp} and ~\ref{sec-nd-np} summarize a rich physics program at the near site 
providing a generational advance in precision measurements and sensitive searches. 
This short-baseline physics program and the long-baseline oscillation analyses  
share similar detector requirements and offer a deep sinergy and mutual feedback.  
The reference FGT meets the requirements of the short-baseline studies as briefly outlined below. 

\begin{itemize}
    \item {\bf Resolution:} The FGT is designed to have an order of magnitude higher granularity than NOMAD, 
    the most precise, high statistics neutrino experiment. The corresponding improvements include better tracking, 
    continuous electron/positron ID, dE/dx measurement providing hadron-ID, 4$\pi$ calorimetry,
     4$\pi$ muon coverage, and a larger transverse area for event containment. 
    
    \item {\bf Statistics:} The 1.2MW neutrino source at LBNF would offer a factor of 100 enhancement in statistics 
    compared to NOMAD. Additionally, the program of measuring $\nu$ and $\bar\nu$ interactions 
    in a set of crucial nuclear targets, including Ar and H, would effectively enhance the physics potential 
    of precision measurements and searches.  

                 
\end{itemize}

Sensitivity studies to the salient precision measurements and searches can be found in 
~\cite{Adams:2013qkq, DPR}. 


\subsection{Future Tasks to Quantitate the Systematic Errors}
\label{cdrsec:detectors-nd-ref-fgt-req-future} 

We need to undertake two outstanding tasks to quantify the systematic errors in oscillation studies 
and precision measurement program. They are: 

\begin{itemize}
    \item {\bf Geant4 Simulation:} A Geant4 simulation of the FGT is needed to confirm and correct 
    the projected systematic errors and  the salient sensitivity studies. 
    
     \item {\bf Translating ND-Measurements to FD:} A concerted effort needs to be mounted to transfer 
     the precision measurements in ND to the LAr-FD. 
     
\end{itemize}    

The DUNE collaboration plan to pursue these two issues with high priority in the coming years before CD2. 
