%%%%%%%%%%%%%%%%%%%%%%%%%%%%%%%% 
\section{Addition of a Liquid Argon Detector to the NND}
\label{sec:detectors-nd-alt}

Of note for the reference ND concept is that since the FGT is not identical to the far detector it is not possible in 
long-baseline analyses to `cancel' the event reconstruction errors in a near to far ratio, as it is done for example in MINOS. 
The extent to which such a cancellation will limit the ultimate precision of the experiment has yet to be fully explored. 
However, at the international Near Neutrino Detector workshop held at Fermilab (July 2014) it was accepted that the FGT offers 
a sound basis for moving forward but also that a LAr TPC or a high-pressure gaseous-argon TPC placed upstream of the FGT 
would enhance the near detector capability.


We note that for LBNF/DUNE, an `identical near detector' concept neither works nor is sufficient to fulfill the 
requirements for ND (see Section ~\ref{sec-nd-oscl}). The principal impediment to an identical ND -- a liquid argon (LAr) ND -- 
is the event-pileup problem due to the high intensity of LBNF. Nevertheless, during the operation of LBNF/DUNE there will be 
periods when the accelerator is running at low intensity, for example during the initial ramp-up and during the periodic 
shut-downs and accelerator upgrades. A $\sim$ 100t LAr stationed upstream of the FGT (the reference ND) would be able to accumulate 
tens of thousands of neutrino interactions during the low-intensity runs over the lifetime of the experiment. 
The FGT will act as a spectrometer for the emerging muons. In conjunction with the measurements with nuclear targets in FGT, 
including Ar-gas, such an LAr-ND would provide a mean to accurately validate {\em in-situ} the FGT predictions for a LAr detector 
before the extrapolation to the FD, thus providing a valuable redundancy check. 

Furthermore, for special neutrino interaction topologies, such as neutrino-electron scattering, where there is a single 
electron or muon in the final state, the combined LAr-ND \& FGT configuration could provide unique precision measurements. 

Conceptual designs for a standalone LAr TPC near detector and a standalone gaseous argon TPC near detector are 
under consideration and could serve as starting points for the design of this addition to the FGT. 
Significant simulation and engineering studies are required to understand whether a liquid or gas argon TPC is 
optimal for minimizing systematic errors in the long-baseline measurements and to integrate the additional detector 
system with the FGT design to make a coherent near detector system.

%In the coming year, the DUNE near detector working group would explore the physics implication and cost estimates of such alternatives. 


