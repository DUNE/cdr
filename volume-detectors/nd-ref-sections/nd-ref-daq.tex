%%%%%%%%%%%%%%%%%%%%%%%%%%%%%%%% 
\section{The Data Acquisition System (DAQ) and Computing}
\label{cdrsec:detectors-nd-ref-daq-comp}

The scope of the Near Detector System DAQ (NDS-DAQ) and computing includes the
design, procurement, fabrication, testing, delivery and installation
of all the NDS-DAQ subsystems: 
\begin{itemize}
\item NDS Master DAQ (NND-MDAQ)
\item Near Neutrino Detector DAQ (NND-DAQ)
\item Beamline Measurements DAQ (BLM-DAQ)
\item NDS Computing
\end{itemize}



\subsection{NDS DAQ}
\label{cdrsec:nd-gdaq-intro}

The Near Detector System (NDS) Data Acquisition system (NDS-DAQ)
collects raw data from each NDS individual DAQ, issues
triggers, adds precision timing data from a global positioning system
(GPS), and builds events.  The NDS-DAQ is made up of three parts, as
shown in the block diagram of Figure~\ref{fig:DAQ_Block}, a master DAQ
and one each for the near neutrino detector (NND, which is the FGT)
and the BLM systems. The names for these are, respectively, NDS-MDAQ,
NND-DAQ and BLM-DAQ.
\begin{cdrfigure}[Near Detector System DAQ block diagram]{DAQ_Block}
{Near Detector System DAQ block diagram: The NDS-DAQ consists 
of the NDS Master DAQ (green blocks), the Beamline Measurement DAQ (yellow summary 
block) and the Near Neutrino Detectors DAQ (orange summary block).  The 
NDS-DAQ connects to other portions of DUNE and LBNF, shown here in other colors (blue, 
light red, tan).}
\includegraphics[width=6in,angle=0]{DAQ_Block}
\end{cdrfigure}

\subsubsection{NDS Master DAQ} 
\label{cdrsec:nd-master-daq}

The NDS Master DAQ (NDS-MDAQ) is designed to provide a high-level user
interface for local run control and data taking, as well as for secure
remote control and monitoring.  It will serve as the primary interface
to the NND-DAQ and BLM-DAQ and will include the following:
\begin{itemize}
\item slow-control system 
\item online data and DAQ performance monitoring  
\item raw data collection
\item building of events
\item data storage.   
\end{itemize}
The NDS-MDAQ includes hardware two-way triggering for both the NND-DAQ
and BLM-DAQ, and GPS hardware for precision time-stamping and global
clock synchronization.  The design is currently based on a channel
count estimate of approximately 433,000 from the near neutrino
detector, plus $<$1,000 from the beamline detectors.  Custom
electronic components for the NDS-DAQ are based on existing custom
designs from other experiments, e.g., T2K and ATLAS, and implement
commercial components for the trigger modules, clock and timing
synchronization, GPS and environmental monitoring.


\subsubsection{Near Neutrino Detector DAQ (NND-DAQ)} 
\label{cdrsec:nd:nnd:daq}


The Near Neutrino Detector Data Acquisition system (NND-DAQ) collects
raw data from the DAQ in each NND subdetector and connects to the NDS
Master DAQ via Gigabit Ethernet. A block diagram of the NND-DAQ is
shown in Figure~\ref{fig:DAQ_NND}. 
\begin{cdrfigure}[A block diagram of the Near Neutrino Detector DAQ]
{DAQ_NND}{A block diagram of the Near Neutrino Detector DAQ (NND-DAQ).}
\includegraphics[width=5in,angle=0]{DAQ_NND_trim}
\end{cdrfigure}
The NND-DAQ will mainly consist of a scalable back-end computer array,
interconnected to the individual subdetector DAQs via Gigabit
Ethernet and specialized electronics modules for trigger processing
and clock synchronization. It interfaces to the NDS-MDAQ for run
control and data collection. The NND-DAQ will also have its own local
run-control setup, consisting of a number of desktop workstations to
allow independent local runs that include NND subdetectors only; this
is useful during detector commissioning, calibration runs, stand-alone
cosmic runs, or other runs where the beam is stopped or not needed.

The quantity of computers required for the NND-DAQ back-end system is
highly dependent on the number of channels and expected data rates of
the individual neutrino detectors.  One back-end computer should be
able to handle approximately 3,000 channels for sustainable and
continuous runs. Assuming a total of 433,000 channels for all NND
subdetectors combined, about 150 back-end computers would be needed.

Trigger signals from each subdetector will be collected and
pre-processed by a trigger electronics module, similar in design to
the NDS trigger or master-clock modules of the NDS-MDAQ
design. Depending on the run mode, this module could feed local
trigger decisions to the detector DAQs for data collection, or it
could forward NDS triggers from the NDS-MDAQ or higher levels to the
NND subdetector DAQs.  A slave-clock electronics module, similar to
the master-clock module in the NDS-MDAQ, distributes clock- and
time-synchronization signals from the NDS-MDAQ to all NND
subdetectors.




\subsubsection{Beamline Measurements DAQ (BLM-DAQ)}

The BLM-DAQ will mainly consist of a scalable back-end computer array,
inter-connected to the individual beamline measurement detector DAQs
via Gigabit Ethernet and specialized electronics modules for trigger
processing and clock synchronization. It interfaces to the NDS-MDAQ
for run control and data collection. It will also have its own local
run-control setup, consisting of a number of desktop workstations to
allow independent local runs that include beamline measurement
detectors only; this is useful during detector commissioning,
calibration runs, stand-alone cosmic runs or other runs where the beam
is stopped or not needed.


\subsection{NDS Computing}
\label{cdrsec:nd-gdaq-global-computing}

The computing system encompasses two major activities: online
computing with required slow-control systems, and offline computing
for data analysis and event simulation.  The computing components are
based on currently available commercial computing and gigabit
networking technology, which is likely to improve over the next years
without driving costs up for the final design.

