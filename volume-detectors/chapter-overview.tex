\chapter{Overview}
\label{ch:detectors-overview}

% Intro shared by all subsections

\section{An International Physics Program}

The global neutrino physics community is developing a multi-decade
physics program to measure unknown parameters of the Standard Model of
particle physics and search for new phenomena.  The program will be carried out as an international,
leading-edge, dual-site experiment for neutrino science and proton decay studies, which 
is known as the Deep Underground Neutrino Experiment (DUNE).
The detectors for this experiment will be designed, built, commissioned and operated by the international DUNE Collaboration. The facility required to support this experiment, the Long-Baseline Neutrino Facility (LBNF), is hosted by Fermilab and its design and construction is organized as a DOE/Fermilab project incorporating international partners. Together LBNF and DUNE will comprise the world's highest-intensity neutrino beam at Fermilab, in Batavia, IL, a high-precision near detector on the Fermilab site, a massive liquid argon time-projection chamber (LArTPC) far detector installed deep underground at the Sanford Underground Research Facility (SURF) \SI{1300}{\km} away in Lead, SD, and all of the conventional and technical facilities necessary to support the beamline and detector systems. 


The strategy for executing the experimental program presented in this Conceptual 
Design Report (CDR) has been developed to meet the requirements 
set out in the P5 report~\cite{p5report} and takes into account the recommendations of the European Strategy for Particle Physics~\cite{ESPP-2012}. It adopts a model where U.S. and international funding agencies 
share costs on the DUNE detectors, and CERN and other participants provide in-kind contributions 
to the supporting infrastructure of LBNF. LBNF and DUNE will be tightly coordinated as DUNE collaborators 
design the detectors and infrastructure that will carry out the scientific program.
  
The scope of LBNF is
\begin{itemize}
\item an intense neutrino beam aimed at the far site
\item conventional facilities at both the near and far sites
\item cryogenics infrastructure to support the DUNE
  liquid argon time-projection chamber (LArTPC) detectors at SURF
\end{itemize}

The DUNE detectors include
\begin{itemize}
\item a high-performance neutrino detector and beamline monitoring system
located a few hundred meters downstream of the neutrino source
\item a massive LArTPC neutrino detector located deep underground at the far site
\end{itemize}

With the facilities provided by LBNF and the detectors
provided by DUNE, the DUNE Collaboration proposes to mount a focused
attack on the puzzle of neutrinos with broad sensitivity to neutrino
oscillation parameters in a single experiment.  The focus of the scientific program is the determination of the neutrino mass hierarchy and the explicit demonstration of leptonic CP violation, if it exists, by precisely measuring differences between the oscillations of muon-type neutrinos and antineutrinos into respectively electron-type neutrinos and antineutrinos. Siting the far detector deep underground will provide exciting additional research opportunities in nucleon decay, studies utilizing atmospheric neutrinos, and neutrino astrophysics, including measurements of neutrinos from a core-collapse supernova should such an event occur in our galaxy during the experiment’s lifetime.

\section{A Roadmap of the Conceptual Design Report}

The LBNF/DUNE CDR describes the proposed physics program and 
technical designs at the conceptual design stage.  At this stage, the design is
still undergoing development and the CDR therefore presents a \textit{reference design} 
for each element as well as any \textit{alternative designs} that are under consideration.

The CDR is composed of four volumes and is supplemented by several annexes that 
provide details on the physics program and technical designs. The volumes are as follows

\begin{itemize}
\item \volintro provides an executive summary of and strategy for the experimental 
program and of the CDR as a whole.
\item \volphys outlines the scientific objectives and describes the physics studies that 
the DUNE Collaboration will undertake to address them.
\item \vollbnf describes the LBNF Project, which includes design and construction of the 
beamline at Fermilab, the conventional facilities at both Fermilab and SURF, and the cryostat
 and cryogenics infrastructure required for the DUNE far detector.
\item \voldune describes the DUNE Project, which includes the design, construction and 
commissioning of the near and far detectors. 
\end{itemize}

Annexes to these volumes are listed at \fixme{provide URL}:




%%%%%%%%%%%%%%%%%%%%%%%%%%%%%%%
\subsection{About this Volume}

The first part of \voldune{} of the CDR describes the strategies for
implementing the near and far detectors
(Chapter~\ref{ch:detectors-strategy}) and outlines the DUNE management
structure (Chapter~\ref{ch:detectors-pm}). The next part describes the
technical designs: the reference and alternative designs for the far
detector and the synergies between them
(Chapters~\ref{ch:detectors-fd-ref},~\ref{ch:detectors-fd-alt}
and~\ref{ch:detectors-synergy}), and the near detector systems design
(Chapter~\ref{ch:detectors-nd-ref}).  Following this,
Chapter~\ref{ch:detectors-sc} describes the designs for the computing
infrastructure and physics software and
Chapter~\ref{ch:detectors-proto} provides an overview of the ongoing
and planned prototyping effort.  The software and computing efforts,
as well as some of the prototyping activities are
off-project. Chapter~\ref{ch:detectors-summary} summarizes and
concludes the volume.
 
%%%%%%%%%%%%%%%%%%%%%%%%%%%%%%%%%%%%%%%%%%%%%%%%%%%%%%%%%%%%%%
\section{Introduction to the DUNE Detectors}
\label{sec:intro-dune-det}

%%%%%%%%%%%%%%%%%%%%%%%%%%%%%%%
\subsection{Far Detector}
\label{sec:intro-dune-far-det}

The proposed far detector (FD) will be located deep underground at the
SURF 4850L with a fiducial mass of 40~kt. It consists of four %identical 
cryostats instrumented with Liquid Argon Time Projection
Chambers (LArTPCs). 
It is assumed that all four detector modules will be similar but not
necessarily identical, allowing for evolution of the LArTPC
technology to be implemented.  % Anne moved this up from end of section

LArTPC technology provides excellent tracking and calorimetry performance. It is
ideal for massive neutrino detectors that require high signal
efficiency and effective background discrimination, %excellent
the capability to identify and precisely measure neutrino events over a
wide range of energies, and %excellent  (I think `precisely' and `high res' cover the `excellent' aspects here}
reconstruction of the kinematic
properties with  high resolution. The full imaging of events in the DUNE detector will
allow study of neutrino interactions and other rare events with
unprecedented detail. The detector's huge mass will %enable 
result in data sets %for
large enough to enable precision studies and the search for CP violation.

The mature LArTPC technology, pioneered by ICARUS, is the result of several decades of worldwide R\&D.
%The LArTPC concept, pioneered in the context of the ICARUS project, is
%a mature technology with several decades of worldwide R\&D.
Nonetheless, the size of a single \ktadj{10} DUNE detector module represents an
extrapolation by %approximately 
over one order of magnitude relative to  %compared to the
the ICARUS~T600, which is the largest detector of this kind operated to date. To address this challenge,
DUNE is developing both a reference and an
alternative design (see Figure~\ref{fig:FarDet-overview-SPDP}), and is engaged in a comprehensive prototyping
effort. %At this stage, t
A list of synergies
between the reference and alternative designs has been identified and is
summarized in Chapter~\ref{ch:detectors-synergy}. Common solutions for
DAQ, electronics, HV feedthroughs, and so on, will be pursued and
implemented, independent of the details of the TPC design choice. The development of the two detector module designs is %a strengthand an added value 
a considerable advantage, and it is made possible by the %efforts of the international DUNE collaboration.
convergence of previously separate international neutrino efforts into the DUNE Collaboration.

\begin{cdrfigure}[3D models of the DUNE far detector designs]{FarDet-overview-SPDP}
{3D models of two 10-kt detectors using the single-phase reference design (left) 
and the dual-phase alternate design (right) for the DUNE far detector to be 
located at 4850L.}
\centering
\begin{minipage}[b]{1.0\textwidth}
\begin{center}
\includegraphics[width=.5\textwidth]{FarDet-3D-SP.jpg}
\includegraphics[width=0.46\textwidth]{DP_det2.jpg}
\end{center}
\end{minipage}
\end{cdrfigure}

Interactions in liquid argon (LAr) produce ionization charge and
scintillation light.  The %charge drifts 
electrons drift in a constant electric field
away from the cathode plane towards the segmented anode plane.  The
prompt scintillation light is observed by photodetectors that provide
the absolute time of the event.  The reference design, described in
Chapter~\ref{ch:detectors-fd-ref}, adopts a single-phase readout,
in which the readout anode is composed of wire planes in the LAr
volume.  The alternate design, discussed in
Chapter~\ref{ch:detectors-fd-alt}, considers the dual-phase approach,
where ionization charge is extracted, amplified and detected in
gaseous argon above the liquid surface.  The dual-phase design 
allows a finer readout pitch (3~mm), a lower detection-energy threshold,
and better pattern reconstruction of the events.  
Both the reference and alternate designs include systems to collect the
scintillation light.


A comprehensive prototyping strategy for both designs is being actively
pursued, as described in Chapter~\ref{ch:detectors-proto}.  The
reference design, closer to the original ICARUS design, is currently
being validated in the 35-t prototype LAr detector at Fermilab (see Section~\ref{sec:proto-35t}).  The
novel alternative design approach has been proven on
several small-scale prototypes, and a 20-t dual-phase
prototype %(WA105 3$\times$1$\times$1) 
is being constructed at CERN, intended for
operation in 2016.  Full-scale engineering prototypes will be
assembled and commissioned at the CERN neutrino platform~\footnote{See CERN Bulletin article at \href{http://cds.cern.ch/journal/CERNBulletin/2014/51/News\%20Articles/1975980?ln=en}{http://cds.cern.ch/journal/CERNBulletin/2014/51/News\%20Articles/1975980?ln=en}.}; they are 
expected to provide the ultimate validation of the engineered solutions
for both far detector designs around the year 2018. 

A test-beam data
campaign will be executed in the following years to collect a large
sample of charged particle interactions to study the detector response
with high precision.  %These ongoing
%efforts including those at the CERN Neutrino Platform will provide the ideal
%environment to exploit these synergies and implement common solutions.  <--- Not needed (Anne)

The deployment of the four 10-kt modules at SURF will take several
years % with a vision 
and be guided by principles detailed in
Chapter~\ref{ch:detectors-strategy}. According to this strategy, DUNE
 adopts the lowest-risk design that satisfies the
physics and detector requirements %in order to start the
and allows starting installation of the first 10-kt detector module as early as possible.
Accordingly, the first 10-kt module will implement the reference design.   
Installation of the second 10-kt module will commence
soon after the first one is implemented \fixme{right? It used to say `soon thereafter' (Actually, I think we could get rid of this whole sentence. Anne)}  %.  Hence, the second cryostat and should be
and will be instrumented as soon as possible. %There is recognition that

%LArTPC technology %will 
%is expected to continue to evolve with (1) the large-scale
%prototypes at the CERN neutrino platform and the experience from the
%Fermilab SBN program, and (2) the experience gained during the
%construction and commissioning of the first %10-kt 
%modules.    <----- this is not well placed and doesn't add anything new. Anne removed

%% moved sentence up
%There will be a
A clear and transparent decision process will be adopted for
determining the design of the second and subsequent modules.  The
decision will be based on physics performance, technical and schedule
risks, costs, and funding opportunities.  Besides taking advantage of
technological developments, a flexible approach to the far detector
design acknowledges the diversity of DUNE and offers the potential to
attract additional interest and resources into the Collaboration. A
staged approach provides access to an early science program while
allowing for new developments to be implemented over the relatively
long installation period of the experiment.

%%%%%%%%%%%%%%%%%%%%%%%%%%%%%%%
\subsection{Near Detector Systems}
\label{sec:intro-dune-near-det}

DUNE will install a near neutrino detector (NND) $\sim$0.5~km
downstream of the target and a Beamline Measurement System (BLM)
$\sim$300~m upstream of the NND. These are collectively called the
Near Detector Systems (NDS).  The NDS will allow DUNE to reduce
systematic errors to match the high-statistics precision sensitivity
for the long-baseline neutrino oscillation studies.  The primary role
of the neutrino detector is to measure the spectrum and flavor
composition of the beam to high precision. This detector will be
magnetized so that it can charge-discriminate electrons and muons
produced in the neutrino charged current interactions; it will
therefore be capable of making separate measurements of the neutrino
and antineutrino fluxes.
%
%In order to reach the ultimate
%sensitivity for the long-baseline neutrino oscillation studies, the neutrino detector will
%measure the spectrum and flavor composition of the (neutrino beam to high precision.  
%Separate measurements of the fluxes of neutrinos and antineutrinos requires a
%magnetized neutrino detector to charge-discriminate electrons and
%muons produced in the neutrino charged current interactions.  This is
%the primary role of the DUNE near detector; % system; 
%however, 

In addition, exposure
to the intense neutrino flux provides the opportunity to collect
neutrino interaction data sets of unprecedented size, enabling an extended
science program.  The near detector therefore provides an
opportunity for a wealth of fundamental neutrino interaction
measurements, which are an important part of the ancillary %secondary 
scientific
goals of the DUNE collaboration.  

The reference design for the
neutrino near detector (NND) design is the NOMAD-inspired fine-grained
tracker (FGT) and is described in
Chapter~\ref{ch:detectors-nd-ref}. The NND subsystems include a
central straw-tube tracker and an electromagnetic calorimeter embedded
in a 0.4-T dipole field. The magnet yoke steel will be instrumented
with muon identifiers.

%The DUNE near neutrino detector %system 
%will be complemented with a 
The Beamline
Measurement System (BLM), designed to measure the muon
flux from hadron decay, is located in the region of the beam absorber at
the downstream end of the decay region. It is intended to monitor the beam
profile on a spill-by-spill basis and will operate for the life of
the experiment.
