%%%%%%%%%%%%%%%%%%%%%%%%%%%%%%%% 
\section{The WA105 Prototype} 
\label{sec:proto-cern-double}

 In the recent years, two consecutive FP7 Design Studies (LAGUNA/LAGUNA-LBNO) have led to the development of a conceptual design (fully engineered and costed) for a 20kton/50kton GLACIER-type underground neutrino detector. In these studies, an underground implementation has been assumed {\it ab initio}
and such constraints have been important and taken into account in design choices. The LAGUNA-LBNO design study, completed in August 2014, has produced many technological developments focused on the construction of  large and affordable liquid argon underground detectors addressing the complete investigation of 3-flavours neutrino oscillations and the determination of their still unknown parameters. These detectors will be as well very powerful for important non-beam studies such as proton decay, atmospheric neutrinos and supernovae neutrinos. The WA105 experiment, approved in 2013, is designed in order to provide a full scale demonstration of these technological developments and also to characterize the detector response to hadronic and electromagnetic showers, being the detector exposed to a charged hadrons/electrons/muons beam-line from 0.5 to 20 GeV/c. A detailed description of the experiment is available in the Technical Design Report of 2014  \cite{WA105_TDR} and an up to date picture of the technical developments in WA105 can be found in the Status Report Document submitted to the SPSC CERN committee in  March 2015 \cite{WA105_SREP}.These developments are also at the basis of the Alternative Far Detector design for DUNE and are consequently described in the related sections.

The WA105 demonstrator is a double-phase liquid argon TPC with an active volume of $6\times 6\times 6$ $m^3$. These dimensions are motivated by the fact that the  basic readout component of the large-scale LAGUNA/LBNO 20-50 kton detectors are $4\times 4$~m$^2$ Charge Readout Plane units. 
The $6\times6$~m$^2$ is consistent with having a fiducial volume corresponding to that readout unit and with a full containment of hadronic showers. 
Surface operation prohibits drift lengths above 6~m. The footprint of the active volume corresponds to 1:20 of the surface of the LBNO 20 kton detector. The active volume contains about 300 tons of liquid argon. The basic parameters of the detector are presented in Table~\ref{tab:demo_para} and a 3D drawing and two cut views are available in Figure~\ref{fig:6by6_open},  Figure~\ref{fig:6by6_plan} and  Figure~\ref{fig:6by6_vert}.

\begin{cdrtable}[Main parameters of the WA105 demonstrator]{lcc}{demo_para}{Main parameters of the WA105 demonstrator} 
Liquid argon density & T/m$^3$& 1.38 \\ \toprowrule
Liquid argon volume height & m& 7.6 \\ \colhline
Active liquid argon height& m  & 5.99 \\ \colhline
Hydrostatic pressure at the bottom& bar & 1.03 \\ \colhline
Inner vessel size (WxLxH) &m$^3$ & 8.3 $\times$ 8.3 $\times$ 8.1\\ \colhline
Inner vessel base surface& m$^2$& 67.6 \\ \colhline
Total liquid argon volume& m$^3$ & 509.6 \\ \colhline
Total liquid argon mass & t & 705 \\ \colhline
Active LAr area & m$^2$& 36 \\ \colhline
Charge readout module (0.5 x0.5 m$^2$) & & 36\\ \colhline
N of signal feedthrough & & 12 \\ \colhline
N of readout channels & & 7680\\ \colhline
N of PMT & & 36 \\ 
\end{cdrtable}

The double-phase liquid argon TPC is based on the vertical drift of the ionization electrons in LAr in a uniform electric field up to the liquid-vapor interface, where they are extracted from the liquid into the gas phase thanks to a higher field made with a submersed grid. The electrons are then amplified in the GAr thanks to avalanches occurring in micropattern detector called LEM (Large Electron Multiplier) and collected on a 2D anode made withe a printed circuit board. The sandwich structure composed by the LEM the Anode and the extraction grid is the so called Charge Readout Plane. The gain provided by the amplification in gas allows for the compensation of the charge attenuation along long drift paths and to achieve a S/N greater than 100 for minimum ionizing particles over 12m drift path.The drift path in the WA105 demonstrator reaches 6 m, the detector is foreseen to work with a drift field of 0.5 kV/cm and 1 kV/cm, corresponding to a voltage applied to the cathode respectively of -300kV and -600 kV. The CRP has an active surface of 36 $m^2$ subdivided in strips of 3.125 mm pitch and 3 m length for a total of 7680 readout channels.

The WA105 detector is supposed to demonstrate all the techniques developed for the 20 and 50 kton LBNO detectors:

\begin{itemize}
\item{Tank construction technique based on the LNG industry with non evacuated vessel}
\item{Purification system}
\item{Long drift}
\item{HV system 300-600 kV, large hanging field cage}
\item{Large area double-phase charge readout}
\item{Accessible cryogenic front end electronics and cheap data acquistion electronics}
\item{Long term stability of UV light readout}
\end{itemize}

At the same time the $6\times 6\times 6$ $m^3$ exposed to the test-beam line has a rich physics programme:

\begin{itemize}
\item{Assessing the detector performance in reconstructing hadronic showers; most demanding task in neutrino interactions}
\item{ Measurements in hadronic and electromagnetic calorimetry and PID performance}
\item{Full-scale software development, simulation and reconstruction}
\item{Collection of high statistic hadronic interaction samples  unprecedented granularity and resolution for the study of hadronic interactions and nuclear effects}
\item{Assessment of the impact on the physics capabilities of a better detector performance wrt  single phase LAr TPC: high S/N ratio, 3 mm pitch, absence of materials in long drift space, two collection views, no ambiguities}
\item{Study of the systematics for the long baseline experiment related to the reconstruction of the hadronic system (resolution and energy scale), electron identification efficiencies and pi0 rejection. Particles dE/dx identification for proton decay}
\end{itemize}


\begin{cdrfigure}[Illustration of the  $6\times 6\times 6$ $m^3$  with the inner detector inside the cryostat]{6by6_open}{Illustration of the  $6\times 6\times 6$ $m^3$  with the inner detector inside the cryostat}
\includegraphics[width=0.95\linewidth]{130618_6x6x6m303v2}
\end{cdrfigure}

\begin{cdrfigure}[Plan view section of the $6\times 6\times 6$ $m^3$]{6by6_plan}{\small Plan view section of the $6\times 6\times 6$ $m^3$ }
\includegraphics[width=0.95\linewidth]{Detector_overview_horizcross}
\end{cdrfigure}

\begin{cdrfigure}[\small Vertical cross section of the $6\times 6\times 6$ $m^3$]{6by6_vert}{\small Vertical cross section of the $6\times 6\times 6$ $m^3$}
\includegraphics[width=0.9\linewidth]{Detector_overview_verticalcross}
\end{cdrfigure}


The $6\times 6\times 6$ $m^3$  detector is foreseen to start data taking in 2018 in the extension of the EHN1 Hall, actually under construction. All its components are in an advanced state of design/prototyping or pre-production.Since the submission of the TDR, the completion of the WA105 detector design and the preparation of its construction have been progressing very quickly during the last year. Many technical aspects of the design have largely benefited of the possibility of performing a pre-production and direct practical implementation on a  $3 \times 3 \times 1$ $m^3$  setup which has the minimal size of a readout unit in the final detector.  This allowed to have a first overview of the complete system integration; to produce a fully engineered prototype version of many detector parts including all their installation details and ancillary services; to set up full Quality Assessment (QA), construction, installation and commissioning chains, to anticipate legal and practical aspects related to the procurement of the different components and to validate the cost estimations and time schedule for WA105.  The  $3 \times 3 \times 1$ $m^3$   represents a technical playground and integration exercise to speed up the design, procurement, QA and commissioning activities needed for the  $6\times 6\times 6$ $m^3$ detector.  In particular a complete procedure for the construction of tanks based on the GTT licenced corrugated membrane technology has been set up with CERN and a full chain for the procurement, processing, assembly and commissioning of the LEM detectors and of the anodes had been also implemented.

\begin{cdrfigure}[Exploded view of the  $3\times 3\times 1$ $m^3$  prototype]{3by1}{Exploded view of the  $3\times 3\times 1$  $m^3$  prototype}
\includegraphics[width=0.95\linewidth]{3by1}
\end{cdrfigure}