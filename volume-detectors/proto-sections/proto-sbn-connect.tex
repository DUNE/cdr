%%%%%%%%%%%%%%%%%%%%%%%%%%%%%%%% 
\section{Connections to the Short-Baseline Program at Fermilab} 
\label{sec:proto-cern-single}


The prototyping plans made by LBNE were in turn a part of a larger LArTPC detector development program at Fermilab and elsewhere in the U.S., which are summarized in Chapter 7 of \anxlbnefd and fully described in the \textit{Integrated Plan for LArTPC Neutrino Detectors in the U.S}.
% reference is ?? FNAL Tech Report xxx??  LBNE:DocDB-2113
This detector development plan has largely been fulfilled, with an endpoint represented by the experiments which comprise the short-baseline program at Fermilab on the Booster Neutrino Beamline : MicroBooNE, the Short-Baseline Near Detector, and ICARUS.
 % probably want a reference for each of those....
 Each of these detectors shares some technical elements with each other and/or with the DUNE far detector prototypes. Elements such as :
\begin{itemize}
\item cryogenic system design
\item argon purification techniques
\item cold electronics
\end{itemize}
In other aspects however the detectors are very different -- in the design details of the anode wire planes for example.  The Short-Baseline Near Detector is most similar to DUNE detectors, having adopted the 35-ton APA-CPA-Field Cage design.

The Fermilab LArTPC detector development program also included a calibration detector and reconstruction software development.  The planned calibration detector has become the Liquid Argon In A Test Beam program (LArIAT) which has recently begun operations in the Meson Center beamline.  The LArSoft software is a set of configurable packages designed to reconstruct the data from any liquid argon detector.  It is supported by the Fermilab Scientific Computing Division and has contributors from all of the operating and planned LArTPC experiments at Fermilab.