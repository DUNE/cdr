%%%%%%%%%%%%%%%%%%%%%%%%%%%%%%%%
\section{Connections to the Short-Baseline Program at Fermilab}
\label{sec:sbn_connect}

% Top part moved to overview of chapter. Anne 5/22
 
%Not all prototypes must be executed by DUNE in order for DUNE to benefit.  
DUNE will benefit from a range of past and ongoing efforts at
Fermilab. Some have been evolving in tandem with the former LBNE and
present DUNE efforts. The strategy behind MicroBooNE --- to
incorporate some detector development aspects in an experiment with
goals to investigate short-baseline neutrino physics --- expanded to
include detectors upstream and downstream.  The Fermilab
Short-Baseline Program on the Booster Neutrino Beamline now consists
of the Short-Baseline Near Detector (SBND), MicroBooNE, and the ICARUS
T-600; the program is fully described in a recent
proposal~\cite{Antonello:2015lea}.  There is significant overlap in
the collaboration membership of DUNE and the three short-baseline
detectors.
 
Each of the short-baseline detectors shares some technical elements with each other and/or with the DUNE far detector prototypes e.g.,   
 cryogenic system design,
 argon purification techniques and
 cold electronics.

In other aspects, e.g., the design details of the anode wire planes, %however, 
the detectors are very different.  The SBND is most similar to the DUNE single-phase detector design, having adopted the 35-t APA-CPA-field cage design, while the MicroBooNE TPC field cage follows the ICARUS design.  The cold electronics installed on the MicroBooNE TPC represent an initial step in an ongoing program; the 35-t and the SBND 
implement subsequent outgrowths of
%utilize subsequent steps in the 
cold electronics development.  While commissioning its cryogenics system, MicroBooNE conducted investigations of the voltage breakdown in high-purity argon; the results prompted some design adjustments to the field cage adopted by the 35-t Phase-2 and the SBND, demonstrating the sharing and feedback of technical developments.  

Coordinated development of reconstruction software for LArTPC detectors is a major outcome of the 2009 \textit{Integrated Plan}.  LArSoft is fully supported by the Fermilab Scientific Computing Division and has contributors from all of the operating and planned LArTPC experiments at Fermilab.  Track and shower reconstruction methods, and particle identification techniques , are already shared between ArgoNeuT, MicroBooNE, LArIAT and the 35-t.  Real data from these detectors is assisting DUNE simulation efforts.  The Short-Baseline experiments, starting with MicroBooNE, will develop neutrino interaction classification techniques based on the details revealed by their fine-grained tracking capabilities, and are likely to exert a strong influence on DUNE oscillation analyses.



 
