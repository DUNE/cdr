%%%%%%%%%%%%%%%%%%%%%%%%%%%%%%%% 
\section{The CERN Single-Phase Prototype} 
\label{sec:proto-cern-single}


The CERN single-phase prototype detector is a crucial milestone towards construction and operation of the 
first 10~kton DUNE far detector module. The prototype detector and beam test serves two principal functions. 
The first is to serve as an engineering prototype to validate the performance of all detector components,
establish and commission production sites and to test the installation procedure. 
The second is to collect and study physics data in response to charged particles of different type and energy. 
Results from these measurements serve to validate MC simulations, serve as data input to sensitivity studies 
of the DUNE experiment  and allow to validate and tune event reconstruction and particle identification tools.\\
To mitigate the risks associated with extrapolating small scale versions of the single-phase LAr TPC technology 
to a full-scale detector element, it is essential to benchmark the operation of full-scale detector elements
and perform measurements in a well characterized charged particle beam.  


The basic detector performance can be established with cosmic ray muons and the results are critical to inform 
the production of the first 10~kton DUNE far detector components.
It allows to identify potentially problematic components and lead to future improvements and optimizations of the detector design.
A well defined charged particle test beam can further enhance the detector performance measurements.
In particular the following checks are anticipated:
\begin{enumerate}
 \item characterize performance of full scale TPC module
 \item verify functionality of cold TPC electronics under LAr cryogenic conditions
 \item perform full-scale structural test under LAr cryogenic conditions
 \item study performance of the photon detection system
 \item verify argon contamination levels and associated mitigation procedures
 \item develop and test installation procedures for full-scale detector components
 \item test and evaluate the performance of detector calibration tools (e.g. laser system)
 \item identify flaws and inefficiencies in the manufacturing process
\end{enumerate}

The physics sensitivity of the DUNE experiment has been estimated based on detector performance characteristics published in the literature, simulation based estimates as well as a variety of assumptions about the anticipated performance of the future detector and event reconstruction and particle identification algorithms.
The proposed single-phase LAr prototype detector and CERN beam test aim to replace these assumptions with measurements for the full scale DUNE detector components and the presently available algorithms. Thereby the measurements will allow to enhance the accuracy and reliability of the DUNE physics sensitivity projections. 
The beam measurements will serve as a calibration data set to tune the Monte Carlo simulations and serve as a reference data set for measurements of the future DUNE detector. \\
%
In order to make such precise measurements, the detector will need to accurately identify and measure the energy of the particles produced in the neutrino interaction with Argon which will range from hundreds of MeV to several GeV.

More specifically, the goals of the prototype detector beam test measurements include the
the use of a charged particle beam to:
\begin{enumerate}
\item measure the detector calorimetric response for
\begin{enumerate}
	\item hadronic showers
	\item electromagnetic showers
\end{enumerate}
\item study e/$\gamma$-separation capabilities
\item measure event reconstruction efficiencies as function of energy and particle type based on experimental data
\item measure performance of particle identification algorithms as function of energy and for realistic detector conditions
\item assess single particle track calibration and reconstruction
%		-- characterize performance of algorithms
\item validate accuracy of Monte Carlo simulations for relevant energy ranges as well as directions

%  \item secondary hadron interactions in detector
\item study other topics with the collected data sets
 \begin{enumerate}
    \item pion interaction kinematics and cross sections
    \item kaon interaction cross section to characterize proton decay backgrounds 
    \item muon capture for charge identification
 \end{enumerate}
\end{enumerate}

The CERN single-phase detector and beam test program is in preparation and an invited technical proposal 
\cite{CERN_single-phase_proposal} to the CERN SPSC will be submitted in June 2015. The planing foresees to 
take a first beam data run in 2018 before the long shutdown of the LHC. Experiences gained from construction, installation and commissioning of the CERN single-phase prototype detector as well as performance tests with 
cosmic ray data are expected to lead to an optimization of equivalent phases of the DUNE far detector. 

\subsection{Detector Configuration and Components}

Since the CERN single-phase prototype detector serves as an engineering prototype detector for the 
first 10~kton module of the DUNE far detector all components have exactly the same dimensions
and features as those described in section \ref{CDR_far_detector}. That is all TPC and photon detector components, 
as well as their positioning and spacing within their cryostat are exactly as planned for the DUNE far detector modules.

\subsubsection{TPC configuration}

The size of the CERN single-phase prototype detector is determined by the physics requirement of the previously 
listed physics measurements. The particle containment of hadronic showers initiated by charged pions or protons
is a critical feature for calorimetric measurements. Simulation studies indicate that showers initiated by 10 GeV
primary pions and protons are contained within a volume measuring 6~m in the longitudinal and 5$\times$5~m$^2$
in the transverse directions. With the basic APA unit measuring 6$\times$2.3~m$^2$ an arrangement of two times 
3 APAs  side-by-side, a central cathode and two drift volumes with 3.6~drift length each was identified as the arrangement which satisfies the requirements. Figure~\ref{fig:CERN_single_TPC} shows a view of the CERN single phase TPC along with the field cage and also a view of the TPC within the cryostat.
%
\begin{cdrfigure}[Cutaway view]{CERN_single_TPC}{View of the CERN single-phase detector TPC (left) and inserted in the cryostat (right). }
\includegraphics[width=0.40\textwidth]{CERN_single_TPC}
\includegraphics[width=0.59\textwidth]{TPC-3D-section}
\end{cdrfigure}
%
For descriptions of the TPC readout, photon-detection system, DAQ, slow control and monitoring as well as 
the key issues of the installation procedure we refer to the corresponding sections of the DUNE far 
detector \ref{DUNE-far-detector}.

\subsubsection{Cryostat}

The Single-phase TPC test at CERN will use a membrane tank technology with internal dimensions of 
7.8~m (tranverse)$\times$ 8.9~m (parallel)$\times$8.1~m (height). 
It can contain  725 tons of LAr, equivalent to about 520 m$^3$. The active (fiducial) detector mass of liquid argon amounts to 400~tons (300~tons).\\
The cryostat design is based on a scaled up version of the LBNE 35-ton Prototype\cite{montanari_35ton}. 
The cryostat will use a steel outer supporting structure with a metal liner inside to isolate the insulation volume, similar to the one of the dual phase detector prototype WA105 $1\times1\times3$ and to the Fermilab Short-Baseline Near Detector. The support structure will rest on I-beams to allow for air circulation underneath in order to maintain the temperature within the allowable limits.
In this vessel a stainless steel membrane contains the liquid cryogen. The pressure loading of the liquid cryogen is transmitted through rigid foam insulation to the surrounding outer support structure, which provides external support. The membrane is corrugated to provide strain relief resulting from temperature related expansion and contraction. The vessel is completed with a top cap that uses the same technology.
The external cryostat dimensions are 10.6~m (tranverse)$\times$ 11.7~m (parallel)$\times$10.9~m (height). 


For the cryostat top cap
several steel reinforced plates welded together. The stainless steel primary 
membrane, intermediate insulation layers and vapor barrier continue across the top of the detector, 
providing a leak tight seal. The secondary barrier is not used nor required at the top. The cryostat roof is 
a removable steel truss structure that also supports the detector. Stiffened steel plates are welded to the 
underside of the truss to form a flat vapor barrier surface onto which the roof insulation attaches directly. 
The penetrations will be clustered in the back region. The top cap will have two large openings for TPC 
installation, and a manhole to enter the tank  after the 
hatches have been closed.

The truss structure rests on the top of the supporting structure where a positive structural connection 
between the two is made to resist the upward force caused by the slightly pressurized argon in the ullage 
space. The hydrostatic load of the LAr in the cryostat is carried by the floor and the sidewalls. In order to meet the maximum deflection between APA and CPA and to decouple the detector form possible sources of vibrations, the TPCs will be connected to an external bridge over the top of the plate supported on the floor of the building. Everything else within the cryostat 
(electronics, sensors, cryogenic and gas plumbing connections) is 
supported by the steel plates under the truss structure. All piping and electrical penetration into the 
interior of the cryostat are made through this top plate, primarily in the region of the penetrations to 
minimize the potential for leaks. 

\subsubsection{Cryogenic System}

Main goal of the LAr system is to purge the cryostat prior to the start of the operations (with Gar in open 
and closed loop), cool down the cryostat and fill it with LAr. Then continuously purify the LAr and the boil 
off Gar to maintain the required purity. The design requirement calls for a 10~ms electron lifetime (30 ppt O$_2$
equivalent) which is measured by the detector.

The LAr receiving facility includes a storage dewar and an ambient vaporizer to deliver LAr and Gar to the 
cryostat. The LAr goes through the liquid argon handling and purification system, whereas the Gar 
through the gaseous argon purification before entering the vessel.
Studies are ongoing to standardize the filtration scheme and select the optimal filter medium for all 
future generation detectors, including this test prototype. 

During operation, an external LAr pump circulates the bulk of the cryogen through the LAr purification 
system. The nominal LAr purification flow rate allows for 5.5 days for a full volume exchange.
The boil off gas is first recondensed and then is sent to the LAr purification system before re-
entering the vessel.

The proposed liquid argon system is based on the design of the 
LBNE 35 ton prototype, the MicroBooNE detector systems and the current plans for the Long Baseline Far 
Detector.








    
    
