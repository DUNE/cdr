%%%%%%%%%%%%%%%%%%%%%%%%%%%%%%%%
\section{The CERN Single-Phase Prototype}
\label{sec:proto-cern-single}

\fixme{I added a short description of what it is, how big, what goes in it, e.g.,}
A CERN single-phase prototype detector and accompanying beam-test program is in preparation. As an \textit{engineering} prototype, it is intended to validate the performance of the components planned for the first DUNE \ktadj{10} detector module at scale and thereby
mitigate the risks associated with extrapolating small-scale versions of the single-phase LArTPC technology
to a full-scale detector module.
It is also intended to benchmark the operation of full-scale detector elements
and perform measurements in a well characterized charged-particle beam --- an essential step.
%
The prototype will incorporate components with exactly
the same dimensions and features as those for the first 10-kt DUNE far detector module, and its success will be a crucial milestone towards the DUNE far detector construction and operation.

%The prototype detector and beam test serve two principal functions: 
%The first is to serve as an 
Besides validating the performance of the %all 
detector components, planning and constructing the CERN prototype will
establish and commission production sites and test the installation procedure.
Further, before the beam test, many basic detector-performance parameters can be established with cosmic-ray muons.  These data will aid in identification of potentially problematic components, leading to future improvements and optimizations of the detector design.
%
Once it is exposed to a test beam of charged particles of different types and energies it will %The second is to 
collect data that %and study physics data that  % in response to .Results from these measurements 
can be combined with results from %the operation of 
LArIAT and short-baseline
 detectors at Fermilab. Together these measurements will be used to %.  All such measurements serve to 
 validate MC simulations, and they will serve as data input to DUNE
 sensitivity studies  and allow validation and tuning of tools for event reconstruction and particle identification.
 
 The following detector performance measurements are anticipated
 
 \fixme{moved bullet list up}
 \begin{enumerate}
 \item characterize performance of a full-scale TPC module
 \item study performance of the photon detection system
 \item test and evaluate the performance of detector calibration tools (e.g., the laser system)
  \item verify functionality of cold TPC electronics under LAr cryogenic conditions
  \item perform full-scale structural test under LAr cryogenic conditions
  \item verify argon contamination levels and associated mitigation procedures
  \item develop and test installation procedures for full-scale detector components
  \item identify flaws and inefficiencies in the manufacturing process
\end{enumerate}


%Many basic detector-performance parameters can be established with cosmic-ray muons; these results are critical to inform the production of the first 10~kton DUNE far detector components.  These data will aid in identification of potentially problematic components, leading to future improvements and optimizations of the detector design.
%A well defined charged particle test beam can further enhance the detector performance measurements. In particular, the following checks are anticipated:

The physics sensitivity of the DUNE experiment has so far been estimated based on detector performance characteristics published in the literature, simulation-based estimates, and %as well as 
a variety of assumptions about the anticipated performance of the future detector and of the event reconstruction and particle-identification algorithms.
%The proposed single-phase LAr prototype detector and CERN beam test aim 
This engineering prototype and the test beam measurements aim to replace these assumptions with measurements to use for the full-scale DUNE detector components and the algorithms, %currently available algorithms,
and thereby %. Thereby the measurements will allow to 
enhance the accuracy and reliability of the DUNE physics-sensitivity projections.
The collection of beam measurements will serve both as a calibration data set for tuning the MC simulations and %serve 
as a reference data set for %measurements of 
the future DUNE detector. 
%
In order to make %such precise 
precise enough measurements, the \fixme{prototype?} detector will need to accurately identify and measure the energy of the particles produced in the neutrino interaction with argon, which will range from hundreds of MeV to several GeV.

More specifically, the goals of the prototype detector beam-test measurements include
the use of a charged-particle beam to
\begin{enumerate}
\item measure the detector calorimetric response for
\begin{enumerate}
	\item hadronic showers
	\item electromagnetic showers
\end{enumerate}
\item study e/$\gamma$-separation capabilities
\item measure event reconstruction efficiencies as function of energy and particle type based on experimental data
\item measure performance of particle identification algorithms as function of energy and for realistic detector conditions
\item assess single-particle track calibration and reconstruction
%		-- characterize performance of algorithms
\item validate accuracy of MC simulations for relevant energy ranges as well as directions \fixme{directions?}

%  \item secondary hadron interactions in detector
\item study other topics with the collected data sets
 \begin{enumerate}
    \item pion interaction kinematics and cross sections
    \item kaon interaction cross section to characterize proton decay backgrounds
    \item muon capture for charge identification
 \end{enumerate}
\end{enumerate}

A detailed %estimate enumerating 
enumeration of the desired minimum integrated particle counts as a function
of charged-particle species and momentum is nearing completion. %This estimate is also converted into a 
It has led to development of a run plan based on realistic beam composition, particle energies and efficiency information. \fixme{check my edits}

% moved up: The CERN single-phase detector and beam test program is in preparation and 
An invited technical proposal for the CERN single-phase detector and beam-test program
\cite{CERN_single-phase_proposal} will be submitted to the CERN SPSC in June 2015. The plan
includes %ning foresees to
taking a first beam data run in 2018 before the long shutdown of the LHC. Experience gained from construction, installation and commissioning of this prototype, as well as performance tests with
cosmic-ray data are expected to lead to an optimization of corresponding phases of the DUNE 
single-phase far detector module(s).

\subsection{Detector Configuration and Components}

As mentioned above, the prototype detector's components have exactly the same dimensions
and features as those described in section \ref{CDR_far_detector}. This includes the TPC and photon detector components,
as well as their positioning and spacing within the cryostat.% are exactly as planned for the DUNE far detector modules.

\subsubsection{TPC Configuration}

The size of the %CERN single-phase 
prototype %detector is mostly 
is in large part determined by the requirement to fully contain hadronic showers of up to several GeV in energy.
The particle containment of hadronic showers initiated by charged pions or protons
is a critical feature for calorimetric measurements. Simulation studies indicate that showers initiated by \GeVadj{10}
primary pions and protons are contained within a volume measuring 6~m in the longitudinal and 5$\times$5~m$^2$
in the transverse directions. With the basic APA unit measuring 6$\times$2.3~m$^2$, the arrangement 
identified as satisfying the requirement consists of two times
three APAs side-by-side, a central cathode and two drift volumes each with 3.6~m \fixme{m?} drift length. Figure~\ref{fig:CERN_single_TPC} shows a view of the CERN single-phase TPC along with the field cage, and  a view of the TPC within the cryostat.
%
\begin{cdrfigure}[Cutaway view]{CERN_single_TPC}{View of the CERN single-phase detector TPC (left) and inserted in the cryostat (right). }
\includegraphics[width=0.40\textwidth]{CERN_single_TPC}
\includegraphics[width=0.59\textwidth]{TPC-3D-section}
\end{cdrfigure}
%
%For descriptions of the 
The TPC readout, photon-detection system, DAQ, slow control and monitoring, as well as
the key issues of the installation procedure % we refer to the corresponding 
are described in corresponding sections of Chapter~\ref{ch:detectors-fd-ref}.

\subsubsection{Cryostat}

The %Single-phase TPC test at 
CERN prototype uses a membrane tank technology with internal dimensions of
7.8~m (tranverse)$\times$ 8.9~m (parallel)$\times$8.1~m (height).
It contains  725 tons of LAr, equivalent to about 520 m$^3$. The active (fiducial) detector mass of LAr amounts to 400~tons (300~tons). 
The external cryostat dimensions are 10.6~m (tranverse)$\times$ 11.7~m (parallel)$\times$10.9~m (height).

The cryostat design is based on %a scaled up version of 
the 35-t prototype cryostat~\cite{montanari_35ton}, described in Section~\ref{sec:proto-35t}, and scaled up.
Unlike the 35-t cryostat, it uses a steel outer supporting structure with an inside metal liner to 
separate the insulation volume. \fixme{separate insulation layer from what?} It is similar to the WA105 dual-phase prototype detector cryostat and to that for the Fermilab Short-Baseline Near Detector (SBND). \fixme{could use reference} The support structure rests on I-beams to allow for air circulation underneath the cryostat; this maintains the temperature within the allowable limits.
A stainless-steel membrane contains the LAr within the cryostat. The pressure loading of the cryogenic liquid is transmitted through rigid foam insulation to the surrounding outer support structure. The membrane is corrugated to provide strain relief resulting from temperature-related expansion and contraction. The cryostat top cap consists of the same layers as the cryostat walls. %The vessel is completed with a top cap uses the same technology.
%
From the inside out, the layers include 
the stainless-steel primary membrane, intermediate insulation layers and vapor barrier; they all continue across
the top of the detector, thereby providing a leak-tight seal.
The cryostat roof is a removable steel truss structure
to which stiffened steel plates are welded from the
underside. They form a flat vapor-barrier surface onto which the roof insulation attaches directly.


The truss structure rests on the top of the supporting structure where a positive structural connection
between the two is made in order to resist the upward force caused by the slightly pressurized argon in the ullage
space. The hydrostatic load of the LAr in the cryostat is carried by the floor and the sidewalls. In order to meet the maximum deflection of 3~mm between APA and CPA and to decouple the detector form possible sources of vibrations, the TPCs are connected to an external bridge over the top of the plate supported on the floor of the building. Everything else within the cryostat
(electronics, sensors, cryogenic and gas plumbing connections) is
supported by the steel plates under the truss structure.

All piping and electrical penetrations into the interior of the cryostat are made through the top plate.
Penetrations are clustered in one region.
The top cap has two large openings for TPC installation, and a manhole to allow entry into the tank  after the
hatches have been closed.

\subsubsection{Cryogenics System}

The main goals of the %LAr 
cryogenics system are to purge the cryostat prior to the start of the operations (with GAr in an open
and a closed loop), \fixme{Not clear what this means} cool the cryostat and fill it with LAr. %Successively the 
The LAr is continuously purified
and the boil-off GAr is captured  to maintain the required purity.
\fixme{capturing it can't maintain the purity, it can keep it from escaping so that can be reliquified...}  The design requirement calls for a 10-ms electron lifetime (30 ppt O$_2$ equivalent), a quantity that is measured by the detector.

The LAr-receiving facility includes a storage dewar and an ambient vaporizer to deliver LAr and Gar to the
cryostat. The LAr goes through the LAr handling and purification system, whereas the GAr goes 
through the GAr purification system before entering the cryostat.
Studies are ongoing to standardize the filtration scheme and select the optimal filter medium for %all
both the prototype and future %generation 
detectors. %, including this test prototype.

During operation, an external LAr pump circulates the bulk of the cryogen through the LAr purification
system. The nominal LAr purification flow rate allows for 5.5 days for a full volume exchange.
The boil-off gas is first recondensed and then sent to the LAr purification system before
re-entering the vessel.

The proposed LAr cryogenics system is based on that of the 35-ton prototype, the MicroBooNE and Short-Baseline Near Detector systems, and the current plans for the DUNE single-phase far detector module.










