%%%%%%%%%%%%%%%%%%%%%%%%%%%%%%%%
\section{Overview}
\label{sec:proto-overview}


This chapter describes the prototyping strategy for the DUNE  far and near detectors and the efforts that are underway or being planned.  These include:
\begin{itemize}
\item the 35-ton prototype at Fermilab,
\item the single-phase DUNE prototype at CERN,
\item the dual-phase LArTPC prototype at CERN known as WA105, and
\item prototypes for the Near Neutrino Detector (NND) and Beamline Measurement (BLM) systems
\item the LArTPC-based short-baseline neutrino physics program (SBN) at Fermilab.
%\item The near detector Beamline Measurements prototype (for the detector described in \ref{sec:detectors-nd-ref-blm})
\end{itemize}

The single-phase LArTPC prototyping efforts, i.e., the 35-t at
Fermilab and the DUNE prototypes at CERN, have evolved from the plans
made initially for LBNE, as summarized in Chapter 7 of \anxlbnefd~\cite{cdr-annex-lbne-design} 
and fully described in the \textit{Integrated Plan for LArTPC Neutrino
  Detectors in the U.S}, a report which was submitted to the DOE in
2009.  The 2009 report outlined an R\&D program with the goal of
demonstrating a scalable LArTPC far detector design for a
long-baseline neutrino oscillation experiment.  The following list of
the detector development components is taken from the Executive
Summary of the 2009 report (and edited to remove out-of-date
information).
\begin{itemize}
   \item Materials Test Stand (MTS) program
   to address questions pertaining to maintenance of argon purity;
    \item electronics test stands at Fermilab and BNL;
    \item the Liquid Argon Purity Demonstrator (LAPD) at Fermilab;
    \item the ArgoNeuT prototype LArTPC;
    \item the MicroBooNE experiment, a physics experiment that will
      advance our understanding of LArTPC technology;
    \item a software development effort that is well integrated across
      present and planned LArTPC detectors;
    \item a membrane-cryostat mechanical prototype to evaluate this technology;
    \item an installation and integration prototype for LBNE, to
      understand issues pertaining to detector assembly, particularly
      in an underground environment;
    \item $\sim$5\%-scale electronics systems test to understand
      system-wide issues as well as individual component reliability; and
    \item a calibration test stand that would consist of a small TPC
      to be exposed to a test beam for calibration studies, relevant
      for evaluation of physics sensitivities.
\end{itemize}

This detector development plan has largely been enacted.  The MTS is a
standard tool used to assess any materials planned for use in LArTPCs.
LAPD operated successfully, showing the viability of a ``piston purge'' in
place of evacuation to attain the desired argon purity.  The membrane
cryostat mechanical prototype became the DUNE 35-t prototype and
repeated the demonstration of the piston purge in its Phase-1
operation.  ArgoNeuT collected quality data in the NuMI beam and
helped spur the integrated reconstruction software development effort
now known as LArSoft.  The proposed calibration test stand has become
the LArIAT facility at Fermilab.  The development of cold electronics
for LArTPCs continues, with a goal of incorporating most of the signal
processing and formatting into analog and digital ASICs.  The
installation and integration prototype for LBNE has evolved into the
35-t Phase-2 and single-phase CERN prototypes for DUNE, described
in this Chapter.
 
At the time the LBNE document was written, the 35-t membrane cryostat
had not yet been fabricated; Section~\ref{sec:proto-35t} provides a
summary of the recent and near-future 35-t operations.  The 1-kt
prototype described in the LBNE document is no longer planned and has
been replaced by the DUNE single-phase LArTPC prototype at CERN, which
is summarized in Section~\ref{sec:proto-cern-single} and fully
described in \anxcernproto~\cite{cdr-annex-singleph-proto}.  The prototyping plans made for LBNE were
in turn a part of a larger LArTPC detector development program at
Fermilab, which is also linked to the short-baseline program at
Fermilab; these connections are described in
Section~\ref{sec:sbn_connect}.

The dual-phase WA105 prototype detector has evolved from the
European-based design studies performed for LAGUNA-LBNO; the
development of these is summarized in
Section~\ref{sec:detectors-fd-ref-ov}.  The WA105 detector provides a
large-scale implementation of a dual-phase LArTPC as summarized 
in Section~\ref{sec:proto-cern-double} and fully described in
\anxdualtdr~\cite{WA105_TDR}.  A 20-t engineering prototype of WA105, also described in
Section~\ref{sec:detectors-fd-ref-ov}, is planned for the immediate
future.

Both the single-phase and dual-phase prototypes at CERN will be
operated in a test beam and calibrated with charged particles (pions,
protons, muons, electrons). The data from the CERN prototypes will be
combined with data from other operating LArTPCs, such as those in the
Fermilab short-baseline and test beam programs and used to refine the
simulations of the DUNE far detectors.

The near detector prototyping plans described in
Section~\ref{sec:proto-nd} are under development and will
utilize particle test beams at CERN and Fermilab.  Prototypes
of certain short-baseline systems have already been operated in the
NuMI beam at Fermilab.

The prototyping efforts enable the Collaboration
to acquire the knowledge and skills necessary for successful
construction and implementation of the DUNE detectors. They provide
guidance in areas such as procurement, construction and installation
techniques, in addition to refinement of technical designs and validation of
cost and schedule estimates.  Operation of the prototypes will provide
opportunities to test data-taking and data-handling assumptions and
to enhance the development of data analysis tools.  Finally, the
Fermilab short-baseline program offers experience with large neutrino
event samples in LArTPC detectors as well as an opportunity to make
detailed measurements of neutrino-argon interactions that are
important for DUNE physics. 

