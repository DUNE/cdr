%%%%%%%%%%%%%%%%%%%%%%%%%%%%%%%%
\section{Overview}
\label{sec:proto-overview}


This chapter describes the prototyping strategy for the DUNE  far and near detectors and the efforts that are underway or being planned.  The efforts include:

\begin{itemize}
\item The 35-ton prototype at Fermilab
\item The single-phase LArTPC prototype at CERN
\item The dual-phase LArTPC prototype known as WA105 
\item \fixme{Should we mention the SBN synergy (from section~\ref{sec:proto-sbn}) here or past ICARUS experience?}
%%    \item Something for the NND  (described in detail in \fixme{??})    
%%    !! No plans have yet been advanced by any members of the collaboration
\item The near detector Beamline Measurements prototype (for the detector described in \ref{sec:detectors-nd-ref-blm})
% frankly, the beam measurements prototype seems out of place here, given that everything else is about LArTPCs
\end{itemize}

The three far detector prototyping efforts, the 35-ton, the CERN
single-phase, and WA105, have evolved from the plans made initially
for LBNE (as described in Chapter 7 of \anxlbnefd ).  At the time the
LBNE document was written, the 35-ton cryostat was not yet fabricated;
the section below in this chapter provides a summary of the 35-ton
past and near future operations.  The 1-kton prototype for LBNE has
been replaced by the DUNE single-phase LArTPC prototype at CERN; it is
summarized below and fully described in its proposal (\anxcernproto
).  Following the overall DUNE far detector strategy outlined in
Chapter~\ref{ch:detectors-strategy} of this volume, the WA105
dual-phase LArTPC prototype will also operate at CERN; this detector
is summarized below and described in detail in \anxdualtdr .

%  \fixme{Some statement about how together these prototypes will provide guidance on procurement strategies, 
%  vendors, construction and installation techniques, as well as on refinement of the design of various components,
%  procedures and analysis techniques. (or whatever)   Hey, that sounds good - so how about --- }

Each of these prototypes will provide guidance on procurement
strategies, vendors, construction and installation techniques, as well
as on refinement of the designs, operational procedures, data handling,
and data analysis tools.
