%%%%%%%%%%%%%%%%%%%%%%%%%%%%%%%%
\section{Overview}
\label{sec:proto-overview}


This chapter describes the prototyping strategy for the DUNE  far and near detectors and the efforts that are underway or being planned.  The efforts include:

\begin{itemize}
\item The 35-ton prototype at Fermilab
\item The single-phase DUNE prototype at CERN
\item The dual-phase LArTPC prototype known as WA105
\item Prototypes for the Near Neutrino detector and Beamline Measurement systems
%\item The near detector Beamline Measurements prototype (for the detector described in \ref{sec:detectors-nd-ref-blm})
\end{itemize}

The single-phase LArTPC prototyping efforts, being the 35-ton at Fermilab and the DUNE prototype at CERN, have evolved from the plans made initially for LBNE, as described in Chapter 7 of \anxlbnefd.  At the time the LBNE document was written, the 35-ton membrane cryostat had not yet been fabricated; Section~\ref{sec:proto-35t} below provides a summary of the 35-ton operations, recent and near future.  The 1-kton prototype described in the LBNE document has been replaced by the DUNE single-phase LArTPC prototype at CERN; it is summarized in Section~\ref{sec:proto-cern-single} below and fully described in \anxcernproto.  The prototyping plans made for LBNE were in turn a part of a larger LArTPC detector development program at Fermilab which is also linked to the Short-Baseline program at Fermilab; these connections are described below in Section~\ref{sec:sbn_connect}.

The dual phase WA105 prototype detector has evolved from the European-based design studies performed for LAGUNA-LBNO; the development background for these is provided in Section~\ref{sec:detectors-fd-ref-ov}.  The WA105 detector provides a full scale implementation of a dual-phase LArTPC, summarized below in Section~\ref{sec:proto-cern-double} and fully described in  \anxdualtdr.  A 20-ton engineering prototype of WA105, also described below, is planned for the immediate future.

The single-phase and dual-phase prototypes at CERN will both be operated in a test beam, and calibrated with charged particles (pion, muons, electrons). The data from the CERN
prototypes will be combined with data from other operating LArTPCs such as those in the Fermilab short-baseline and test beam programs, and used to refine the simulations of the DUNE far detectors.  

The Near detector prototyping plans described below in Section~\ref{sec:proto-nd} plans are under development and also expect to utilize particle test beams at CERN and Fermilab.  Prototypes of certain Near Beamline systems have already been operated in the NuMI beam at Fermilab.

All of the DUNE prototypes provide guidance on technical concerns such as procurement strategies, vendors, construction and installation techniques, as well as on refinement of the designs and validation of cost and schedule estimates.  Operation of the DUNE far and near detector prototypes provides opportunities to test data taking and data handling assumptions, and enhance the development of data analysis tools.  All the prototyping efforts allow the collaboration to acquire and practice the skills and knowledge needed to implement the final DUNE detectors.


