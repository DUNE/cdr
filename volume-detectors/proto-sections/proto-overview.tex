%%%%%%%%%%%%%%%%%%%%%%%%%%%%%%%%
\section{Overview}
\label{sec:proto-overview}


This chapter describes the prototyping strategy for the DUNE  far and near detectors and the efforts that are underway or being planned.  The efforts include:
\begin{itemize}
\item the 35-ton prototype at Fermilab
\item the single-phase DUNE prototype at CERN
\item the dual-phase LArTPC prototype at CERN known as WA105
\item prototypes for the Near Neutrino Detector (NND) and Beamline Measurement (BLM) systems
\item the short-baseline LArTPC program at Fermilab
%\item The near detector Beamline Measurements prototype (for the detector described in \ref{sec:detectors-nd-ref-blm})
\end{itemize}

The single-phase LArTPC prototyping efforts, i.e., the 35-t at
Fermilab and the DUNE prototypes at CERN, have evolved from the plans
made initially for LBNE, as described in Chapter 7 of \anxlbnefd.  At
the time the LBNE document was written, the 35-t membrane cryostat had
not yet been fabricated; Section~\ref{sec:proto-35t} provides a
summary of the recent and near future 35-t operations.  The 1-kt
prototype described in the LBNE document is no longer planned and has
been replaced by the DUNE single-phase LArTPC prototype at CERN, which
is summarized in Section~\ref{sec:proto-cern-single} and fully
described in \anxcernproto.  The prototyping plans made for LBNE were
in turn a part of a larger LArTPC detector development program at
Fermilab, which is also linked to the short-baseline program at
Fermilab; these connections are described in
Section~\ref{sec:sbn_connect}.

The dual-phase WA105 prototype detector has evolved from the
European-based design studies performed for LAGUNA-LBNO; the
development background for these is summarized in
Section~\ref{sec:detectors-fd-ref-ov}.  The WA105 detector provides a
large-scale implementation of a dual-phase LArTPC 
as summarized below in
Section~\ref{sec:proto-cern-double} and fully described in
\anxdualtdr.  A 20-t engineering prototype of WA105, also described
Section~\ref{sec:detectors-fd-ref-ov}, is planned for the immediate
future.

Both the single-phase and dual-phase prototypes at CERN will be
operated in a test beam and calibrated with charged particles (pion,
muons, electrons). The data from the CERN prototypes will be combined
with data from other operating LArTPCs, such as those in the Fermilab
short-baseline and test beam programs, and used to refine the
simulations of the DUNE far detectors.

The near detector prototyping plans described in
Section~\ref{sec:proto-nd} are under development and also include
utilization of particle test beams at CERN and Fermilab.  Prototypes
of certain beamline systems have already been operated in the NuMI
beam at Fermilab.

The prototyping efforts provide the opportunity for the Collaboration
to acquire the knowledge and skills necessary for successful
construction and implementation of the DUNE detectors. They provide
guidance in areas such as procurement, construction and installation
techniques, in addition to refinement of technical designs and validation of
cost and schedule estimates.  Operation of the prototypes will provide
opportunities to test data-taking and data-handling assumptions, and
to enhance the development of data analysis tools.



