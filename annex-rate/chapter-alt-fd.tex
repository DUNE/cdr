\chapter{Alternative Far Detector Design}
\label{ch:alt-annex-rate}

\section{Data rates tables for the alternate Far Detector Design}

In the following the summary tables for the rates and data volumes, corresponding to the alternate Far Detector design, are reported.
Given the amplification factor of the ionization electrons provided in the gas phase by the LEM detectors the signal to noise ratio is 100:1
for a particle at the ionization minimum. The noise level, given the strips pitch of 3.125 mm, corresponds to about 20 keV. 
A zero-suppressed (ZS) threshold at $\approx 3\sigma$ above the mean noise level of the front-end electronics can be computed similarly to the 
one reported for the reference design, this would correspond to 60 keV.Each 10 kton detector is read out by
80 CRP anode units of $3 \times 3$ m$^2$ with two perpendicular collection views segmented in strips of 3m length and 3.125 mm pitch. 
There are 1920 readout channels per CRP and 153600 channels per detector. Given the drift path of 12m the drift time amounts to 7.5 ms.  The ADC sampling frequency is 2.5 MHz and its resolution is 12 bits. The number of ADC samples per drift window is then 18750.

These fundamental parameters serving as input for the data rate estimations are summarized in table~\ref{tab:ad-fundamental-parameters}

% do not edit, this is generated by dune-params
\begin{tabular}[h]{l|r}
\hline
Full height & \SI[round-mode=places,round-precision=1]{12.0}{\meter} \\
Full width & \SI[round-mode=places,round-precision=1]{12}{\meter} \\
Full length & \SI[round-mode=places,round-precision=1]{60.0}{\meter} \\
Detectors & \num[round-mode=places,round-precision=0]{4.0} \\
\hline
channel/CRP & \num[round-mode=places,round-precision=0]{1920.0} \\
CRP/detector & \num[round-mode=places,round-precision=0]{80.0} \\
Active height & \SI[round-mode=places,round-precision=1]{12.0}{\meter} \\
Active width  & \SI[round-mode=places,round-precision=1]{12.0}{\meter} \\
Drift distance & \SI[round-mode=places,round-precision=2]{12.0}{\meter} \\
\hline
Drift velocity & \SI[round-mode=places,round-precision=1]{1.6}{\milli\meter\per\micro\second} \\
Drift time & \SI{7.5}{\milli\second} \\
\hline
bytes/sample & \SI[round-mode=places,round-precision=1]{1.5}{\byte} \\
sample rate & \SI[round-mode=places,round-precision=1]{2.5}{\mega\hertz} \\
\# drifts/readout & \num[round-mode=places,round-precision=1]{1.0} \\
\hline
readout time & \SI{7.5}{\milli\second} \\
samples/readout & \num[round-mode=places,round-precision=0]{18750.0} \\
\hline
\end{tabular}


The parameters which apply to the full-stream data rates are given in table~\ref{tab:ad-full-stream-parameters}.

\begin{tabular}[h]{l|r}
\hline

Bytes per sample & \SI[round-mode=places,round-precision=1]{1.5}{\byte} \\

DAQ sample rate & \SI[round-mode=places,round-precision=1]{2.5}{\mega\hertz} \\

Channels per CRP & \num[round-mode=places,round-precision=0]{1920.0} \\

Number of CRP per detector & \num[round-mode=places,round-precision=0]{80.0} \\

Number of detectors & \num[round-mode=places,round-precision=0]{4.0} \\

Total channels in DUNE & \num[round-mode=places,round-precision=0]{614400.0} \\

Drifts per readout & \num[round-mode=places,round-precision=1]{1.0} \\

Drift time & \SI{7.5}{\milli\second} \\

Beam spill repetition rate & \SI[round-mode=places,round-precision=1]{1.0}{\hertz} \\

Annual run time fraction & \num[round-mode=places,round-precision=3]{0.667} \\

\hline
\end{tabular}

The expected data rates for two scenarios of FS data are given in table~\ref{tab:ad-full-stream-volume}.

% do not edit, this is generated by dune-params
\begin{tabular}[h]{l|r}
\hline
Full-stream readout size & \SI[round-mode=places,round-precision=1]{16.9}{\giga\byte} \\
\hline
Full-stream in-spill data rate & \SI[round-mode=places,round-precision=0]{339.0}{\peta\byte\per\year} \\
\hline
Full-stream 1 second data volume & \SI[round-mode=places,round-precision=1]{2.2}{\tera\byte} \\
Full-stream 1 minute data volume & \SI[round-mode=places,round-precision=1]{132.0}{\tera\byte} \\
\hline
Full-stream 1 year data volume & \SI[round-mode=places,round-precision=1]{66.18}{\exa\byte} \\
\hline
\end{tabular}
