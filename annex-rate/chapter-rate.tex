\chapter{Characterization of Data Rates}
\label{ch:annex-rate}

\section{The tables}

The Liquid Argon TPC will produce a variety of signals due to various
physics processes taking place in the detector, such as scattering of
the beam neutrinos, cosmic ray muons, radiological backgrounds,
electronics noise and others.
For practical reasons, the signal processing systems will be
incorporate threshold values which will determine which part of data
is rejected and how the data stream needs to be treated at any given
time (cf. search for Supernova Burst events or nucleon decay).

Depending on the origin, these signals will be characterized by
specific spectra, energy scale and topologies.
In combination with the threshold settings and real-time algorithms
this will result in different rates of occurrence and volume of data
produced (either per nominal event or per unit time).

It is important to account for this information for the following reasons:
\begin{itemize}
\item An attempt must be made to understand, quantify and optimize
  systematic uncertainties for each class of measurements
\item There are considerable implications of real-time data processing
  strategies and techniques of all of ``downstream'' data processing
  (e.g. offline), and in particular to the required mass storage
  volume.
  This, in turn, will affect the DUNE computing model and costs
  associated with the DUNE computing effort.
\end{itemize}


We recognize that not all the information exists at this point that is
necessary for exhaustive characterization of the DUNE data.
We will mark as such where appropriate.


The following sources of signal are included and itemized here to
facilitate reference in the rest of the text:
\begin{itemize}
\item ``Full stream'' - effectively continuous stream of digitized
  voltages read out from the TPC, not subject to thresholds.
  This can be likened to a video stream (although at a different
  scale).
\item ``ZS stream'' - the full stream subject to zero-suppression with
  a particular choice of threshold.
  With threshold optimal for the oscillation physics with beam
  neutrinos, this results in an order of magnitude less data than in
  the full stream.
\item ``Beam'' - beam neutrino events
\item ``SNB'' - supernova burst
\item noise - signals due to various electronic-related noises
\item ``rad'' - radiological and cosmogenic backgrounds combined.
  By the latter we mean radioactive decays and other signals caused by
  reaction of cosmic rays in the detector and various materials
  surrounding it.
\item ``cosmic $\mu$ - cosmic muons passing through the detector
  experiencing normal energy loss
\end{itemize}

\subsection{Event rate and sizes}

Table~\ref{tab:rate-summary} summarizes the understanding of the
expected rates and annual data volumes for specific sources of
activity in the detector listed above, independent of threshold
(unless indicated).
An entry in the table does not imply that this rate will be acquired.
The estimation of each row is described in the following sections.

\begin{table}[htbf]
  \centering
  % \begin{tabular}{|r|l|l|l|l|}
  %   \hline
  %   Source & volume & unit & volume/year & comment\\
  %   \hline
  %   \hline
  %   full stream & \SI{24}{\gibi\byte} & readout & \SI{740}{\exbi\byte} & \SI{2.13}{\micro\second} readout\\
  %   \hline
  %   ZS stream &\multicolumn{3}{c|}{undefined, see text} \\
  %   \hline
  %   beam $\nu$-interaction & $10^{3}$/yr.  & year  & 10MB & 100GB \\
  %   \hline
  %   $\nu_{atm}$& \multicolumn{3}{c|}{similar to beam}  & 100GB \\
  %   \hline
  %   SNB  & ZS stream & 60secs/year$\times$?? & 28 TB & 28 TB $\times$??? \\
  %   \hline
  %   rad              & 65kHz/APA($300KeV<E<5MeV$ (0.3 MIPs)\cite{lbne-fd-closeout}) & year & ??? & ??? \\
  %   \hline
  %   cosmic $\mu$     & ??? & ??? & ??? & ??? \\
  %   \hline
  %   sol              & ??? & ??? & ??? & ??? \\
  %   \hline
  %   minbias & \multicolumn{3}{c|}{multiple, variable, threshold dependent} & ??? \\
  %   \hline
  %   beam minbias & \multicolumn{3}{c|}{multiple, variable, threshold dependent} & ??? \\
  %   \hline
  %   calib            & ??? & ??? & ??? & ??? \\
  %   \hline
  % \end{tabular}
  \caption{Rate estimations of hypothetically isolated sources of activity in
    the DUNE far detector.
  The ``???'' indicate input needed is still needed.}
  \label{tab:rate-summary}
\end{table}

\subsubsection{Full Stream}

The \textit{full stream} activity is simply reading out all channels
at the digitizing sampling rate.
One ``readout'' of is a sampling of this stream for a period of time
equal to 2.4 drifts.
The number of samples per readout of one channel is calculated to be
10230 as shown in table~\ref{tab:full-stream}.

\begin{table}[htbp]
  \caption{Relevant values for drift related parameters and
    calculation of number of samples for one readout.}
  \centering
\begin{tabular}[h]{l|r}
\hline
sample rate & \SI{2}{\MHz} \\
drift distance & \SI{3.41}{\meter} \\
drift velocity & \SI{1.6}{\milli\meter/\micro\second} \\
\# drifts & \num{2.4} \\
\hline
readout time & \SI{2.13125}{\milli\second} \\
samples/readout & \num{10230} \\
\hline
\end{tabular}
  \label{tab:full-stream}
\end{table}

\fixme{Explain why 2.4 drifts in a readout.}

The drift velocity of \driftvelocity is for a
\SI{500}{\volt/\centi\meter} electric field.~\cite{docdb3383}.

\begin{table}[htbp]
  \caption{Calculation of \textit{full stream} data volume for a
    \SI{2.13}{\milli\second} readout of the entire 4 detectors (\SI{40}{\kton}).}
  \centering
  \begin{tabular}[h]{l r}
\hline
bytes/sample & \num{12}/\num{8} \\
samples/readout/channel  & \num{10230} \\
channel/APA & \num{2560} \\
APA/detector & \num{150} \\
detectors & \num{4}\\
\hline
% 23.56992 TiB
readout size & \SI{24}{\gibi\byte}\\
1 second & \SI{11}{\tebi\byte}\\
1 minute & \SI{0.7}{\pebi\byte}\\
1 year & \SI{7.4e20}{\byte} (\SI{740}{\exbi\byte})\\
\hline
  \end{tabular}
  \label{tab:full-stream-data-rate}
\end{table}

\subsubsection{ZS Stream}

The \textit{ZS stream} (zero-suppressed stream) contains \textit{full
  stream readouts} as described above but with all samples below some
threshold removed and with an added encoding (number) to indicate
where each run of above-threshold samples begins inside the
readout.\fixme{Is this accurate enough of a description?}

The intention of ZS is to remove electronics noise from what is
ultimately read out in order to reduce data volume with no negative
impact on the physics capabilities.
The requirement~\cite{docdb3383} on the electronic noise and
indirectly on the LAr purity is that one minimum-ionizing particle
(MIP) should produce an amount of charge that will result in a
digitized signal on the wire which is at least $15\times$ greater than
the noise.\fixme{The noise level is RMS?}.
The ZS threshold is taken to be set a level equivalent to the 30\% of
the ADC signal produce on a wire due to a MIP traversing the distance
of one wire pitch.\fixme{Does this make sense?}

The data rate for the \texttt{ZS stream} therefore depends on whatever
particle interactions may have occurred during the readout time.
It is not meaningful to speak about ``the data rate of the \textit{ZS
  stream}'' unless one specifies some other criteria applied to select
the readouts.
The remaining categories then make the assumption that a ZS threshold
of \num{0.3} MIP equivalent ADC is applied.

\subsubsection{Beam Events}

The data rate of events associated in time with spills of the proton
beam on target depend on many things: the beam spill repetition rate,
number of protons on target, annual running time, neutrino flux
spectrum, horn/target configuration, and event selection criteria.
In particular, the last case can be broken into two classes.
The most important contains events with energies consistent with known
and expected interactions by the beam neutrinos.
These are take to be events with energy greater than \SI{100}{\MeV}
and are called simply \textit{high energy beam events}.

The readouts with less than this analysis threshold, still above ZS
threshold and coincident with the beam spill may contain activity from
various sources.
A leading source of data rate is radioactive decay and described in
section~\ref{sec:radrates}.
Cosmic muons will usually be above the analysis threshold but will be
calculated separately on the assumption that reconstruction techniques
can reject them.
They are described in section~\ref{sec:cosmicmurates}.
For this section, both of these sources are estimated to contribute to
the beam event data rate as they will fall accidental in coincidence
with the beam spills.
The parameters assumed and derived rate estimations are given in
table~\ref{tab:beam-nu-rates}.

A third class of activity may be found in coincidence with beam spills
which is due to as yet unknown physics.
Given that detectors of higher energy thresholds and coarser grained
readout have not observed these events they are expected to occur at
low energy, if at all.


\begin{table}[htbp]
  \caption{Assumed parameters and rate estimation for beam events.}
  \centering
  \begin{tabular}[h]{l r}
\hline
beam rep rate & \SI{1}{\Hz}\\
beam run time & \SI{2e7}{\second/\year}\\
average event size & \SI{10}{\mebi\byte}\\
high energy events & \SI{1e4}{/\year}\\
\hline
high energy beam data & \SI{100}{\gibi\byte/\year}\\
rad coincident beam data & (fixme)\\
cosmic-$\mu$ coincident beam data & (fixme)\\
full stream beam data & \SI{471}{\pebi\byte/\year}\\
\hline
  \end{tabular}
  \label{tab:}
\end{table}

\fixme{After calculating the rates due to radioactive decays
  independent of ``trigger'' criteria, fill in how much contributes to
  the beam readouts.}

\fixme{After calculating the rates due to cosmics, multiply by the
  duty factor of the beam window and fill in above.}

\subsubsection{Radioactivity}

\subsubsection{Cosmic Muons}

\subsubsection{Supernova Burst}


\section{Characteristic Energy Thresholds}


Table~\ref{tab:physrates} summarizes the understanding of important
energy thresholds needed to study a particular study topic (physics,
calibration, selection efficiency).
The minimum threshold is that at which any lower threshold has no
additional benefit to the analysis.
The maximum threshold is that at which any higher will negatively
impact the analysis.
The threshold is understood to be interpreted as the amount of
ionizing energy deposited in the active volume of the far detector
and may span APAs.

\begin{table}[htbf]
  \centering
  \begin{tabular}{|l||l|l|l|}
    \hline
    Topic & min thresh & max thresh & trigger \\
    \hline
    \hline
    beam & 50 MeV & 500 MeV & spill \\
    \hline
    pdk & $\sim$few MeV & 100 MeV (see K) & self+see K+nuc. dexcite \\
    \hline
    atm$\nu$ & $\sim$few MeV & 50 MeV &
    self+see Michele $e$ \\
    atm$\nu$ & $\sim$100 MeV & 500 MeV &
    self+no Michele $e$ \\
    \hline
    snb & $\sim$MeV & 5 MeV & self+ZS stream+dump \\
    \hline
    noise & full stream?&$\sim$MeV& min bias?\\
    \hline
    rad &$\sim 0$ & $\sim$MEV & self+minbias\\
    \hline
    calib & ??? & ??? & ??? \\
    \hline
  \end{tabular}
  \caption{Summary thresholds required for different study topics
    considering \textbf{signal only}.
  Below the minimum threshold no further impact on the physics is
  expected.
  Above the maximum threshold the physics is negatively impacted.
  The ``???'' indicate input is still needed.}
  \label{tab:physrates}
\end{table}

\section{Other estimates}

\fixme{rough list for now, these numbers need to be moved into their
  proper subsection}
\begin{itemize}
\item 5 minutes reco CPU / cosmic muon (scale by energy dep to get
  beam event reco CPU?)
  (tj)
\item Want 10-100$\times$ MC events as data for beam events.
  (tj)
  How frequent?
  How long to retain?
\item cosmic muons are about 1M events/year in 30kt (tj)
Can scale in-beam rates by energy deposition to this assuming constant
byte-per-MeV/cm deposited?
\item Square all numbers with those in the science book  
\end{itemize}

