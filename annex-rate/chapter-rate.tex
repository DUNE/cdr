\chapter{Rate Estimates}
\label{ch:annex-rate}

\section{The tables}

\fixme{This table needs input from many people in DUNE.}

Table~\ref{tab:rate-summary} summarizes the understanding of the
expected rates and annual data volumes for specific sources of activity
in the detector independent of threshold (unless indicated).
An entry in the table does not imply that this rate will be acquired.


\begin{table}[htbf]
  \centering
  \begin{tabular}{|r|l|l|l|l|}
    \hline
    Source & rate & time & RO size & volume/year\\
    \hline
    \hline
% 1,536,000 ch * 2e6 Hz * 1e-3s * 12 bit / 8  = 9.2GB
    full stream & 1/2ms & 3e7sec &  9.2GB & 138 EB ($10^{18}B$) \\
    \hline
    ZS stream &  \multicolumn{3}{c|}{10 \% of full stream} & 13.8 EB ($10^{18}B$) \\
    \hline
    beam ($>$500 MeV)  & 10K/yr  & year  & 10MB & 100GB \\
    \hline
    atm$\nu$& \multicolumn{3}{c|}{similar to beam}  & 100GB \\
    \hline
    snb  & ZS stream & 60secs/year$\times$?? & 28 TB & 28 TB $\times$??? \\
    \hline
    noise            & ??? & ??? & ??? & ??? \\
    \hline
    rad              & ??? & ??? & ??? & ??? \\
    \hline
    cosmic $\mu$     & ??? & ??? & ??? & ??? \\
    \hline
    sol              & ??? & ??? & ??? & ??? \\
    \hline
    minbias & \multicolumn{3}{c|}{multiple, variable, threshold dependent} & ??? \\
    \hline
    beam minbias & \multicolumn{3}{c|}{multiple, variable, threshold dependent} & ??? \\
    \hline
    calib            & ??? & ??? & ??? & ??? \\
    \hline
  \end{tabular}
  \caption{Rate estimations of hypothetically isolated sources of activity in
    the DUNE far detector.
  The ``???'' indicate input needed is still needed.}
  \label{tab:rate-summary}
\end{table}

Table~\ref{tab:rate-summary} summarizes the understanding of important
energy thresholds needed to study a particular study topic (physics,
calibration, selection efficiency).
The minimum threshold is that at which any lower threshold has no
additional benefit to the analysis.
The maximum threshold is that at which any higher will negatively
impact the analysis.
The threshold is understood to be interpreted as the amount of
ionizing energy deposited in the active volume of the far detector
and may span APAs.

\begin{table}[htbf]
  \centering
  \begin{tabular}{|l||l|l|l|}
    \hline
    Topic & min thresh & max thresh & trigger \\
    \hline
    \hline
    beam & 50 MeV & 500 MeV & spill \\
    \hline
    pdk & $\sim$few MeV & 100 MeV (see K) & self+see K+nuc. dexcite \\
    \hline
    atm$\nu$ & $\sim$few MeV & 50 MeV &
    self+see Michele $e$ \\
    atm$\nu$ & $\sim$100 MeV & 500 MeV &
    self+no Michele $e$ \\
    \hline
    snb & $\sim$MeV & 5 MeV & self+ZS stream+dump \\
    \hline
    noise & full stream?&$\sim$MeV& min bias?\\
    \hline
    rad &$\sim 0$ & $\sim$MEV & self+minbias\\
    \hline
    calib & ??? & ??? & ??? \\
    \hline
  \end{tabular}
  \caption{Summary thresholds required for different study topics
    considering \textbf{signal only}.
  Below the minimum threshold no further impact on the physics is
  expected.
  Above the maximum threshold the physics is negatively impacted.
  The ``???'' indicate input is still needed.}
  \label{tab:physrates}
\end{table}

\section{Other estimates}

\fixme{rough list for now, these numbers need to be moved into their
  proper subsection}
\begin{itemize}
\item 5 minutes reco CPU / cosmic muon (scale by energy dep to get
  beam event reco CPU?)
  (tj)
\item Want 10-100$\times$ MC events as data for beam events.
  (tj)
  How frequent?
  How long to retain?
\item cosmic muons are about 1M events/year in 30kt (tj)
Can scale in-beam rates by energy deposition to this assuming constant
byte-per-MeV/cm deposited?
\item Square all numbers with those in the science book  
\end{itemize}

