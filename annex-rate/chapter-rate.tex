\chapter{Characterization of Data Rates and Volumes for specific sources of activity in DUNE Detectora}
\label{ch:annex-rate}

\section{The tables}

\fixme{This table needs input from many people in DUNE.}

The Liquid Argon TPC will produce a variety of signals due to various physics processes taking
place in the detector, such as scattering of the beam neutrinos, cosmic ray muons,
radiological backgrounds, electronics noise and others. For practical reasons, the signal
processing systems will be incorporate threshold values which will determine which part of
data is rejected and how the data stream needs to be treated at any given time (cf. search
for Supernova Burst events or nucleon decay).

Depending on the origin,
these signals will be characterized by specific spectra, energy scale and topologies. In combination
with the threshold settings and real-time algorithms this will result in different rates
of occurrence and volume of data produced (either per nominal event or per unit time).

It is important to account for this information for the following reasons:
\begin{itemize}
	\item An attempt must be made to understand, quantify and optimize
systematic uncertainties for each class of measurements
	\item There are considerable implications of real-time data processing
	strategies and techniques of all of ``downstream'' data processing (e.g. offline),
	and in particular to the required mass storage volume. This, in turn, will affect
	the DUNE computing model and costs associated with the DUNE computing effort.
\end{itemize}


We recognize that not all the information exists at this point that is necessary for
exhaustive characterization of the DUNE data. We will mark as such where appropriate.

The following sources of signal are included and itemized here to facilitate reference in
the rest of the text:
\begin{itemize}
	\item ``Full stream'' - effectively continuous stream of digitized voltages read out from the TPC,
	not subject to thresholds. This can be likened to a video stream (although at a different scale).
	\item ``ZS stream'' - the full stream subject to zero-suppression with a particular choice of threshold.
	With threshold optimal for the oscillation physics with beam neutrinos, this results in an order of magnitude less data than in the full stream.
	\item ``Beam'' - beam neutrino events
	\item ``SNB'' - supernova burst
	\item noise - signals due to various electronic-related noises
	\item ``rad'' - radiological and cosmogenic backgrounds combined. By the latter we mean
	radioactive decays and other signals caused by reaction of cosmic rays in the detector and
	various materials surrounding it.
	\item ``cosmic $\mu$ - cosmic muons passing through the detector experiencing normal energy loss
\end{itemize}

Table~\ref{tab:rate-summary} summarizes the understanding of the
expected rates and annual data volumes for specific sources of activity in the detector
listed above, independent of threshold (unless indicated).
An entry in the table does not imply that this rate will be acquired.


\begin{table}[htbf]
  \centering
  \begin{tabular}{|r|l|l|l|l|}
    \hline
    Source & rate & time & RO size & volume/year\\
    \hline
    \hline
% 1,536,000 ch * 2e6 Hz * 1e-3s * 12 bit / 8  = 9.2GB
    full stream & 1/2ms & $3*10^{7}$sec &  9.2GB & 138 EB ($10^{18}B$) \\
    \hline
    ZS stream &  \multicolumn{3}{c|}{10 \% of full stream} & 13.8 EB ($10^{18}B$) \\
    \hline
    beam ($>$500 MeV)  & $10^{3}$/yr.  & year  & 10MB & 100GB \\
    \hline
    $\nu_{atm}$& \multicolumn{3}{c|}{similar to beam}  & 100GB \\
    \hline
    SNB  & ZS stream & 60secs/year$\times$?? & 28 TB & 28 TB $\times$??? \\
    \hline
    noise            & ??? & ??? & ??? & ??? \\
    \hline
    rad              & ??? & ??? & ??? & ??? \\
    \hline
    cosmic $\mu$     & ??? & ??? & ??? & ??? \\
    \hline
    sol              & ??? & ??? & ??? & ??? \\
    \hline
    minbias & \multicolumn{3}{c|}{multiple, variable, threshold dependent} & ??? \\
    \hline
    beam minbias & \multicolumn{3}{c|}{multiple, variable, threshold dependent} & ??? \\
    \hline
    calib            & ??? & ??? & ??? & ??? \\
    \hline
  \end{tabular}
  \caption{Rate estimations of hypothetically isolated sources of activity in
    the DUNE far detector.
  The ``???'' indicate input needed is still needed.}
  \label{tab:rate-summary}
\end{table}

Table~\ref{tab:physrates} summarizes the understanding of important
energy thresholds needed to study a particular study topic (physics,
calibration, selection efficiency).
The minimum threshold is that at which any lower threshold has no
additional benefit to the analysis.
The maximum threshold is that at which any higher will negatively
impact the analysis.
The threshold is understood to be interpreted as the amount of
ionizing energy deposited in the active volume of the far detector
and may span APAs.

\begin{table}[htbf]
  \centering
  \begin{tabular}{|l||l|l|l|}
    \hline
    Topic & min thresh & max thresh & trigger \\
    \hline
    \hline
    beam & 50 MeV & 500 MeV & spill \\
    \hline
    pdk & $\sim$few MeV & 100 MeV (see K) & self+see K+nuc. dexcite \\
    \hline
    atm$\nu$ & $\sim$few MeV & 50 MeV &
    self+see Michele $e$ \\
    atm$\nu$ & $\sim$100 MeV & 500 MeV &
    self+no Michele $e$ \\
    \hline
    snb & $\sim$MeV & 5 MeV & self+ZS stream+dump \\
    \hline
    noise & full stream?&$\sim$MeV& min bias?\\
    \hline
    rad &$\sim 0$ & $\sim$MEV & self+minbias\\
    \hline
    calib & ??? & ??? & ??? \\
    \hline
  \end{tabular}
  \caption{Summary thresholds required for different study topics
    considering \textbf{signal only}.
  Below the minimum threshold no further impact on the physics is
  expected.
  Above the maximum threshold the physics is negatively impacted.
  The ``???'' indicate input is still needed.}
  \label{tab:physrates}
\end{table}

\section{Other estimates}

\fixme{rough list for now, these numbers need to be moved into their
  proper subsection}
\begin{itemize}
\item 5 minutes reco CPU / cosmic muon (scale by energy dep to get
  beam event reco CPU?)
  (tj)
\item Want 10-100$\times$ MC events as data for beam events.
  (tj)
  How frequent?
  How long to retain?
\item cosmic muons are about 1M events/year in 30kt (tj)
Can scale in-beam rates by energy deposition to this assuming constant
byte-per-MeV/cm deposited?
\item Square all numbers with those in the science book  
\end{itemize}

