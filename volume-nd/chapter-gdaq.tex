\chapter{Near Detector Global DAQ and Global Computing (WBS 130.03.06)}
\label{ch:global-daq}

\fixme{This is from rev 4041 from the old repository at lbne.bnl.gov/svn/documents/cdr/volume-3-ND/chapter-06-gdaq.tex. I cannot find a date associated with it, but it looks right, at least at first glance. -- Anne}

\section{Introduction}
\label{sec:gdac-intro}

The purpose of the Global Data Acquisition System (GDAQ) is to collect 
raw data from each detector system in the Near Detector Complex (NDC), 
issue global triggers, add timing data from a global position system 
(GPS), and build events. Each NDC detector has its own data acquisition 
system that connects to the GDAQ.  

The GDAQ is made up of three parts: 
Master Global DAQ (M-GDAQ), Neutrino Global DAQ (N-GDAQ), which connects to the neutrino-measurement detectors (ND), and Beamline 
Global DAQ (B-GDAQ), which connects to the BLM.

%\fixme{Where the neutrino GDAQ goes with the detectors for neutrino measurements and the beamline GDAQ goes with the detectors that make beamline measurements, right? AH, HGB: Yes, see last sentence added above.}

The computing system encompasses two major activities: online computing (the required
slow-control systems) and offline computing for further development of the measurement strategy and for simulation work on technical systems.

\section{Global DAQ}

\subsection{Master Global DAQ}
\label{subsec:M-GDAQ}

%\subsubsection{Requirements and Specifications}
\subsubsection{Introduction} %header added by AH

The Master Global DAQ (M-GDAQ) is generally controlled by the DUNE 
Central Run Control to start and stop runs.
%\fixme{Where is this described? Add ref. AH; HGB: Actually, I don't know. Logically this should be described somewhere in the Neutrino Beamline volume, but I couldn't find any section dedicated to run control so far.}
It also has its own run 
control and event builder so that it can run independently of the DUNE 
Central Run Control, e.g. while the beam is off or for calibration or 
test runs. It communicates with the Neutrino GDAQ and 
Beamline GDAQ, and provides two-way trigger processing to and from 
the Near Detectors and beamline. Based on the T2K Near Detector DAQ~\cite{ref:T2KNDDAQ},
%\fixme{add ref. AH; HGB: Ref:T2KNDDAQ added, reference info at http://ieeexplore.ieee.org/xpls/abs_all.jsp?arnumber=5750358 }
the proposed system will use Gigabit Ethernet as the primary data-transfer 
medium.

%\fixme{I got rid of "Hall" on "Neutrino Hall GDAQ" above and in section title below; I don't want to introduce another term unnecessarily. Also check 'near detectors' above; it had said 'ND' but I don't think it's just for the neutrino detector.} 3/1/12


The M-GDAQ incorporates a GPS system %for the NDC
 for precision 
time-stamping of beam spills and/or detector events and to provide clock 
and synchronization signals to the individual detector and beamline 
DAQs.

\subsubsection{Design Considerations} %header added by AH

The detailed M-GDAQ
%\fixme{the M-GDAQ? AH  HGB: Yes, fixed.}
requirements will have to be developed and specified 
over time once details of each Near Site subdetector have been 
established.  
The M-GDAQ will need to provide typical DAQ functions including triggering, event-building, data monitoring, transfer, storage, logging, and so on.   The general system requirements of the M-GDAQ can be found in the NDC requirements documentation~\cite{nd_requirements_doc}.

In addition, the M-GDAQ will need to allow standalone data-taking of individual detector DAQs for 
debugging or calibration, and allow secure remote access, storage and monitoring 
capability via firewall to DUNE collaboration member institutions.

%provide triggers and clock-synchronization signals to the individual system GDAQs 
%\item  Provide sustainable data-transfer and logging rates via Gigabit Ethernet consistent with the scientific demands of the DUNE experiment
%\item  Collect and sort data from the N-GDAQ and B-GDAQ, build global events, format the final event data and save to local storage
%\item Allow inclusion or exclusion of individual detectors from data-taking on a run-by-run basis.
%\item  Allow standalone data-taking of individual detector DAQs for debugging or calibration
%\item  Provide a local run-control user interface to allow data-taking or debugging independently of DUNE central run control
%\item  Include slow-control system with environmental monitoring, and sensor and front-end electronics monitoring
%\item  Provide an interface for accessing and monitoring online processes and data on a (near) real-time basis for quality control
%\item  Allow secure remote access, storage and monitoring capability via firewall to DUNE collaboration member institutions

\subsubsection{Reference Design}

The central component of the M-GDAQ is a computer array serving as the 
back-end for the Near Detector Complex GDAQ. It will be a scalable and 
flexible system that can be expanded or upgraded as needed and as 
computer technology improves over time. The actual %amount 
quantity of computers needed for this system and their
specifications will depend on the %amount 
number of channels and expected data rates of the individual %ND 
neutrino
and beamline detectors, %which have not been finalized in detail yet. 
for which the details have not yet been finalized. 
Using the T2K Near-Detector DAQ experience as a reference, 
it is safe to assume that a Gigabit Ethernet structure is more than sufficient for the 
data transmissions between %the ND and beamline detectors 
the DUNE Near Detectors and the GDAQ.
% if T2K Near Detector DAQ experience can be used as a reference example.
%\fixme{What is the distinction between the ND and the beamline detectors? I see the beamline detectors as being part of the ND, the other part being the nu detector.  I've changed it to the `beamline and neutrino detector systems'  above. AH; HGB: Thanks. Sounds better, yes.}

\begin{cdrfigure}[Near Detector Master Global DAQ (M-GDAQ) block diagram]{GDAQ_M_Block}{Near 
Detector Master Global DAQ (M-GDAQ) block diagram. 
The green and yellow blocks belong to the whole GDAQ.  Green blocks are
part of the M-GDAQ only, while the yellow blocks summarize the N-GDAQ and B-GDAQ.  Other colored blocks are not
part of the GDAQ, but shown here for presenting the connections between
the GDAQ and other parts of the DUNE project.}
%\includegraphics[width=7in,angle=0]{MGDAQ_BlockDiagram}
\end{cdrfigure}

Figure~\ref{fig:GDAQ_M_Block} shows the conceptual design of the M-GDAQ 
based on a computer array connected with the N-GDAQ and B-GDAQ via 
Gigabit Ethernet and other components for triggering and clock 
synchronization. For starting and stopping runs the DUNE Central Run Control 
connects to the M-GDAQ via optical fiber. A GPS subsystem is an 
integral part of the M-GDAQ; it provides precision time stamps of each 
global-trigger event.  As an option, the GPS timing data can be 
transferred to the DUNE Far Detector via a private network (fiber) link 
to allow real-time processing of DUNE beam data with Far Detector data -- 
as it is done at T2K with T2K's beamline (and near detector) and \superk
as T2K's Far Detector. 
% -- if desired.

\begin{cdrfigure}[DAQ back-end module designed for T2K ND280]{T2KBEB}{DAQ back-end module designed for T2K ND280, allowing four selectable functions via FPGA firmware}
%\includegraphics[width=6in,angle=0]{T2K_BEB_Photo}
\end{cdrfigure}

A specially designed Global Trigger electronics module processes 
beam-spill trigger(s) coming 
both from the DUNE central run control and from the beamline and neutrino GDAQs,
then provides a global trigger for 
the GPS system. 
Then, 
depending on settings either from the M-GDAQ local run control or DUNE 
Central Run Control, it sends back trigger signals to the N-GDAQ and B-GDAQ.  
The Master Clock module, another specially 
designed electronics module, provides clock- and time-synchronization 
signals generated by the GPS system and forwards them to the N-GDAQ and 
B-GDAQ systems.  These two special electronics modules can be made of 
commercial components based on Field Programmable Gate Arrays (FPGA) 
that can be programmed with high functional complexity and flexibility. 
For example, the T2K UK group designed and built a universal back-end board for 
the ND280 DAQ, shown in Figure~\ref{fig:T2KBEB}, that used a Xilinx Virtex-II 
FPGA as its central smart-electronics component. The FPGA firmware was 
written such that the back-end board could be configured for four 
selectable functions: master clock, slave clock, global or cosmic 
trigger, and readout merger.

The M-GDAQ back-end computer array collects and sorts the detector data 
coming from the N-GDAQ and B-GDAQ, then builds the event, includes 
time stamps from the GPS, formats the event, then %finally 
saves the 
result to local storage, e.g., disk arrays.

A local run-control option is added to the M-GDAQ %\fixme{the M-GDAQ? AH; HGB: Yes, fixed.}
to allow independent 
detector runs when the DUNE Central Run Control is not being used, e.g.,  
when the beam is off for maintenance or repair or when detector 
calibration runs are scheduled. The local run-control system will consist of three or four 
desktop workstations with monitors and keyboards that can also be used 
for (near) real-time event displays of several detector sections and 
online data-processing and monitoring.

Slow-control electronics is included in the M-GDAQ  %\fixme{the M-GDAQ? AH; HGB: Yes, fixed.}
with environmental 
sensors (temperature, humidity, etc.) to monitor the surroundings and 
status of the DAQ electronics.

Finally, a firewall between the M-GDAQ  %\fixme{the M-GDAQ? AH; HGB: Yes, fixed.}
network and outside world allows 
remote access, monitoring and offline data storage of the GDAQ online 
system by DUNE collaboration institutes.

%%%%%%%%%%%%%%%%%%%%%%%%%%%%%%%%%%
\subsection{Neutrino Global DAQ}
\label{subsec:N-GDAQ}

%\subsubsection{Requirements and Specifications}
\subsubsection{Introduction} %heading added by AH

The Neutrino GDAQ (N-GDAQ) will communicate with the data-acquisition 
systems of each neutrino-detector system in the Near Detector Hall %. The N-GDAQ will 
and manage the local trigger information. % local to the Near Detector Hall. 
It will 
collect triggers from each of these detector systems and make some local-trigger 
decisions. It will also pass on global-trigger information from the %ND (ND means neutrino not near) 
M-GDAQ  %\fixme{the M-GDAQ? AH; HGB: Yes, fixed.}
or higher levels to each detector system. The N-GDAQ will have its 
own run control to allow for calibration or other special runs without 
requiring interaction with the %ND
M-GDAQ  %\fixme{the M-GDAQ? AH; HGB: Yes, fixed.}
or DUNE Central Run Control. It 
will also have the capacity to build events using detectors %physically 
located in the Near Detector Hall exclusively (i.e., the neutrino detectors).

\subsubsection{Reference Design}

Similar to the M-GDAQ subsystem, the N-GDAQ will 
mainly consist of a scalable back-end computer array, inter-connected to 
the individual neutrino-detector DAQs via Gigabit Ethernet, plus electronics 
modules for trigger-processing and clock synchronization, as shown in Figure~\ref{fig:GDAQ_N_Block}. The %amount
quantity of 
computers required for the back-end system is highly dependent on the 
number of channels and expected data rates of the individual %ND 
neutrino detectors. Based on the T2K Near Detector DAQ experience, one back-end computer 
should be able to handle approximately 3,000 channels for sustainable and 
continuous runs. Assuming a total of 180,000 channels for all %ND
neutrino detectors combined, for example, 60 back-end computers  would be needed.

\begin{cdrfigure}[Neutrino Global DAQ (N-GDAQ) block 
diagram]{GDAQ_N_Block}{Neutrino Global DAQ (N-GDAQ) block 
diagram}
%\includegraphics[width=7in,angle=0]{NGDAQ_BlockDiagram}
\end{cdrfigure}

Trigger signals from each neutrino detector will be collected and 
pre-processed by a trigger-electronics module, similar in design to the 
global trigger or master-clock modules %shown in 
of the M-GDAQ setup. Depending on the run mode, this module could feed 
local-trigger decisions back to the detector DAQs for data collection, or 
it could
forward global triggers to the neutrino-detector DAQs from the M-GDAQ or 
higher levels.

A slave-clock electronics module, similar to the master-clock module in 
the M-GDAQ setup, distributes clock- and time-synchronization signals 
from the M-GDAQ to all neutrino detectors.

The N-GDAQ is usually controlled by the %ND
M-GDAQ  %\fixme{M-GDAQ?; HGB: Yes, fixed.}
or DUNE Central Run 
Control.  But it will also have its own local run-control setup, 
consisting of several desktop workstations with displays to allow 
independent local runs 
%without %NDM-GDAQ   %\fixme{M-GDAQ?; HGB: Yes, fixed.}or DUNE Central Control, 
that include %ND
neutrino  detectors exclusively. This is useful during detector commissioning, calibration 
runs, standalone cosmic runs and other runs where the beam is stopped or 
not needed. %This 
The N-GDAQ will include the capability for event building, 
formatting, local data storage, and real-time event- and process-monitoring displays.


\subsection{Beamline Global DAQ}
\label{subsec:B-GDAQ}

%\subsubsection{Requirements and Specifications}
\subsubsection{Introduction} %heading addded by AH
Similar to the Neutrino GDAQ and neutrino detectors, the Beamline GDAQ (B-GDAQ) collects 
information from all %beamline detector systems in purview of the NDC in the beamline region.
the beamline-measurement detectors.
 It manages trigger information and has the capacity 
to determine when to issue triggers local to the beamline-measurement system. It 
also has two-way communication with the %NDC
M-GDAQ %\fixme{M-GDAQ?; HGB: Yes, fixed.}
regarding trigger 
information.

The B-GDAQ has its own run control and event builder to allow for runs that 
only include beamline detectors, and is illustrated in Figure~\ref{fig:GDAQ_B_Block}.

\begin{cdrfigure}[Beamline Global DAQ (B-GDAQ) block 
diagram]{GDAQ_B_Block}{Beamline Global DAQ (B-GDAQ) block 
diagram}
%\includegraphics[width=7in,angle=0]{BGDAQ_BlockDiagram}
\end{cdrfigure}

\subsubsection{Reference Design}

The reference design of the B-GDAQ is %proposed to be 
identical to the 
N-GDAQ design, except that it is connected to the beamline-detector DAQs. 
As does the N-GDAQ, it 
consists of a back-end computer array with Gigabit Ethernet layer, 
and specialized electronics modules to process triggers from the 
beamline detectors, send back global triggers from the %NDE
M-GDAQ  %\fixme{M-GDAQ?; HGB: yes, fixed.}
to the 
beamline detectors, and distribute clock and synchronization signals. 
Again, a local run-control setup with its own event-building capability, 
data formatter, local storage and local event display and process-monitoring 
allows disconnection of the B-GDAQ from the %NDC
M-GDAQ  %\fixme{M-GDAQ?; HGB: Yes, fixed.}
or DUNE 
Central Run Control for standalone runs including beamline 
detectors exclusively.


\section{Global Computing}
The computing system encompasses two major activities: %The first is 
online computing (% which consists of 
the required slow-control systems) and %.  The second is 
offline 
computing which is costed off-Project, but is important to consider.
Offline computing is needed to complete 
the work outlined in the Measurement Strategy described in Chapter~\ref{ch:meas-strat} and the simulation work %important 
for the technical systems.

The required slow-control (online) computing systems will be defined when the Project moves 
from the conceptual-design to the preliminary-design phase.

For offline computing, resources are currently being provided by Fermilab, 
LANL and various universities.  Project-wide resources are currently 
being developed at Fermilab and Brookhaven.
