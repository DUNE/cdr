
\chapter{Measurements at External Facilities}
\label{ch:ext-meas}

\fixme{Starting from the fall 2012 CDR content}

The technical components that would be needed to implement the strategies 
described in this chapter are outside the scope of the LBNE NDC conceptual 
design. This information is included in the CDR because it complements the conceptual 
design and expands the NDC capabilities to more closely meet the mission need without increasing the project cost.


\section{External Neutrino-Beam Measurements}
As discussed in Section~\ref{subsec:nu-meas-strat}, LBNE's strategy for neutrino-beam measurements includes making measurements of the Far Detector response to a known flux of neutrinos, and NuMI is the only appropriate beamline to use for the neutrino source.

To implement this strategy, appropriate detectors will need to be built.  A plausible scenario would be a liquid argon TPC detector of 20-30 tons in the current location of the Minerva experiment, in front of the MINOS near detector. In that way, the MINOS near detector could be used to measure the charge of muons exiting the TPC. The TPC would be designed using the same readout techonology that is used in the LBNE Far Detector.  Once an optimal detector arrangement is determined, LBNE would use the same beam simulation and same muon measurements to apply that knowledge to the LBNE beam and Far Detector.

\section{External Hadron-Production Measurements}
\label{sec:hadron}

Uncertainties on hadron production translate into uncertainties in the neutrino fluxes in the LBNE Far Detector, since the hadrons decay into neutrinos in the decay pipe.  
Therefore, the accuracy to which the neutrino flux must be known at the Far Detector directly determines the required accuracy 
of the hadron-production measurements, since the LBNE conceptual design does not include a near neutrino detector. This chapter outlines the LBNE strategy for augmenting the capabilities of the BLM with external measurements of secondary-beam particles.  

\section{Background}
A complete knowledge of the momenta and decay points of the kaons, pions and muons would be sufficient to completely predict the un-oscillated flux of neutrinos at the Near and Far Detector
locations. This would require knowledge of:

\begin{itemize}
\item the phase-space distribution of the initial proton beam
\item details of all materials present in the target, horn and decay pipe areas
\item  the electromagnetic focusing characteristics of the magnetic horn
\item the detailed development of the hadron cascade, spawned by the
initial proton, that passes through the target/horn/decay pipe
\item the meson-to-neutrino decay rates
\end{itemize}

With careful engineering design and careful control of the materials in the
target area, all of these items can be simulated accurately except
hadronic cascades in the target, horn and decay
pipe. The simulation of the hadronic cascade requires accurate knowledge of the 
hadron scattering cross sections, for which there are no first-principle
calculations. These cross sections must therefore rely on models, which in
turn require hadron-production measurements that span particle type,
particle energy and the various materials found in the target, horn
and decay pipe. 

 At the present time, a sufficient body of hadron-production
measurements does not exist to achieve LBNE's desired accuracy of 4-5\%, as determined by the irreducible 
error on the statistical uncertainty for the appearance-measurement background, although this is expected to improve over time. 
As the BLM system described in Chapter~\ref{ch:blm} cannot meet this requirement alone, a near-far comparison will be more complicated than in certain other neutrino-oscillation experiments, e.g.,  MINOS experiment~\cite{gnumi-validation}. 

\section{Strategy}

The current approach is to rely on measurements made externally (outside the scope of LBNE) to calibrate detector response and flux simulations, and to relate 
these measurements to LBNE. This would be done through the use of a common simulation code and through measurements of tertiary muons in both LBNE and the external facility, using nearly identical tertiary muon-measurement systems. 

In order to keep the uncertainty in the near/far event-rate ratio from being
limited by systematic uncertainties in the flux, the LBNE flux simulation 
 must be accurate at the 4-5\% level.
Efforts at this stage are intended to understand the effect of the uncertainties in hadron-production in the beamline on overall LBNE
sensitivities, to determine what further measurements may be needed
by LBNE and to estimate their potential cost to the Project.

The measurements that LBNE would require from an external facility begin with the primary hadron-production cross
sections in the proton-target material, followed by similar studies in
thick targets, and finally hadron yields after passage through the complete target and
focusing-horn system. In addition, hadron-interaction cross sections
on materials in the decay pipe and absorber can be important in
flux calculations.

External hadron-production measurements are expected to play
a critical role once the Far Detector has accumulated sufficient
statistics toward the end of the running period to make systematic errors
on the flux a dominant source of error in the oscillation measurement. 

\section{Use of External Facilities for Measurements}

Historically, a number of hadron-production experiments have
contributed directly to the outcome of neutrino experiments
by measuring meson production from the proton targets used
by those experiments, and hence providing a constraint on their neutrino fluxes. 
For example, the HARP data\cite{ref:HARP} contributed directly to
MiniBooNE and the SPY\cite{ref:SPY} experiment contributed directly to
NOMAD. Since their contributions were crucial to those neutrino experiments, 
it is also expected that LBNE will require some dedicated hadron-production measurements.
In the future, the MIPP experiment at Fermilab is planning to contribute its
measurements to the NOvA experiment, and the NA61
experiment\cite{Abgrall:2011ae} is contributing to the T2K
experiment.


The most suitable apparatus for LBNE's hadron-production measurements
is the collection of equipment and detectors used by the MIPP experiment at Fermilab~
\cite{Isenhower:2006zp}.   
A full suite of LBNE-related
hadron-production measurements would require the installation of the LBNE horn-focusing
elements and associated power supplies in front of a future
incarnation of MIPP in the meson area at Fermilab.
This kind of effort could be within the scope of the LBNE Project and could be postponed
until after LBNE construction or even after LBNE operations have
stopped. 

A proposal for using the NA61 experiment, see Figure \ref{fig:NA61Scheme}, 
is also being developed since it is currently operating
in the H2 beamline at CERN. NA61 does not have the complete suite of particle identification that MIPP does, but it could provide very useful hadron-production data for predicting neutrino fluxes at LBNE.
A pilot run of 120~GeV/c protons interacting on a thin 4\% graphite target 
was taken by NA61 during July, 2012. That run is in the process of being analyzed and will explore the capabilities of the NA61 experiment for LBNE purposes. 

\begin{figure}[htbp]
\centering
%\includegraphics[width=6in]{NA61Drawing.pdf}
\caption[NA61]
{A schematic drawing of the CERN NA61 detector, a hadron production and heavy ion experiment 
designed to measure hadrons over a large part of the relevant phase for 
neutrino experiments. The TPCs, shown in blue, can separate pions from protons 
and kaons.}
\label{fig:NA61Scheme}
\end{figure}
\begin{cdrfigure}[short]{label}{long}
  %\includegraphics[width=0.8\textwidth]{file}
\end{cdrfigure}
