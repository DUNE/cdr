\chapter{Introduction}  
\label{ch:intro}

\fixme{Starting from the fall 2012 CDR content}

%\input{common/lbne-intro}
\clearpage
\section{Introduction to the DUNE Near Detectors (WBS 130.03)}
\label{vol:intro}

%\fixme{Anne reworked this section 8/27}

LBNE will measure neutrino oscillations through high-statistics 
and high-precision measurements in the Far Detector. 
 In order to achieve the ultimate neutrino-oscillation sensitivity, DUNE must be able to accurately and 
precisely predict signal and background events in the Far Detector.

The role of the DUNE Near Detector Complex (NDC) L2 Project is to 
minimize the systematic uncertainties on 
the long-baseline oscillation program and to thus maximize the 
oscillation-physics potential of the Far Detector. It is important that the
 Near Detectors not limit the sensitivity of the 
DUNE long-baseline neutrino-oscillation measurements. 
They should aid the analysis of electron-neutrino 
appearance, the primary oscillation channel, and muon-neutrino disappearance. 

The NDC will be installed downstream of the LBNF beamline at Fermilab. The site is shown in Figure~\ref{fig:nscf_layout-v3}.

\begin{cdrfigure}[LBNF Overall Project Layout at Fermilab]{nscf_layout-v3}{LBNF Overall Project Layout at Fermilab; NDC will be located near LBNF 30}
  %\includegraphics[width=0.8\textwidth]{file}
\end{cdrfigure}

Ideally, the NDC design would comprise both a beamline measurement system to determine neutrino fluxes and spectra, as well as near neutrino detectors to characterize the LBNF neutrino beam and
the response of the Far Detector to those neutrinos.  The characterization would be sufficient to reduce the systematic errors
on the predicted Far-Detector event rate (a function of neutrino oscillation 
parameters, e.g., $\theta_{13}$ and $\delta$) to a level below expected statistical uncertainties.
This would impose certain requirements on
the precision with which the neutrino flux must be known at the Far Detector, since
oscillations manifiest themselves as changes in the relative fluxes of the three neutrino
flavors.

Due to financial constraints, this L2 Project has produced a conceptual design for only beamline measurement and data acquisition systems, discussed in Chapters~\ref{ch:blm} and~\ref{ch:global-daq}, respectively.  
That said, the traditional strategy in neutrino-oscillation experiments has been to measure the un-oscillated events in a near neutrino detector and to calibrate the simulation of the neutrino flux
and detector response on the basis of those measurements. 
The current approach relies on measurements made outside the scope of DUNE to calibrate the detector response and flux simulations, and to relate those measurements to DUNE through the use of common simulation code and measurements of the tertiary muon flux in both locations with nearly identical
 measurement systems. 
These additional elements of the measurement strategy are discussed in Chapter~\ref{ch:ext-meas}. 

\subsection{Neutrino-Beam Measurement Strategy}
\label{subsec:nu-meas-strat}

The purpose of the DUNE Near Detector effort is to make a suite of measurements that
characterize the LBNF neutrino beam and the response of the Far Detector to the neutrinos.

The neutrino-beam measurement strategy will be two-pronged. It will entail measurements of the Far Detector response to a known flux of neutrinos, presumably a neutrino beam similar in nature to the LBNF beam. It will also entail the development of a detailed beamline simulation supported by direct measurements of critical aspects of the beam, 
for example, a well-developed suite of hadron-production measurements. Validation of the simulation via measurements of the secondary 
beam, such as measurements of tertiary muons, will be essential.

It is also essential to maintain a direct connection between the neutrino measurements not performed in the LBNF beam and the LBNF beam simulation at all levels. 
In other words, the simulation code, target material, detector configuration and supporting tertiary-beam measurements must be common and must be validated for both beams. 

NuMI is the only beamline similar enough to the LBNF beamline design to be a candidate for use in the measurements. The NuMI beam has been calibrated by measuring the event rate in the MINOS near detector under a number of different beam configuations, i.e., several horn currents and target positions. The result has been a reasonably well understood neutrino flux. Those neutrino measurements were also supported by the NuMI muon-system measurements, as least for the high-energy part of the neutrino beam where the NuMI muon monitors were sensitive. 

%moved pgraph to ch 4 per EM 8/30/12

LBNF is planning to install a new set of muon-measurement devices into the 
NuMI beam as part of its prototyping effort. 
%\fixme{where do these come from? (Anne)} 
These devices are designed to measure the spectrum of muons at lower energies, starting at 2~GeV, as they exit the absorber. This will enable sensitivity to pions, which produce neutrinos at energies above 2~GeV. The NuMI muon measurements will then be repeated at LBNF for a variety of beam configurations as was done in NuMI. The combined set of measurements, in addition to the NuMI neutrino measurements, should provide a sound basis for predicting event rates at the DUNE Far Detector.
The prototype devices need to be tested in the NuMI muon alcoves prior to CD-2, as they represent the only measurement linking the NuMI beam, where the near-detector neutrino measurements are made, to the LBNF beam. Waiting until after CD-2 would introduce a risk associated with achieving the DUNE physics goals.  More details of the program are in Volume 1.

\subsection{Reference Design Overview}  % No change to this section 8/28. Anne.

The NDC reference design consists of a beamline-measurements system (BLM) and a Global Data Acquisition system (GDAQ).

The BLM will be located in the region of the Absorber Complex at the downstream end 
of the decay region to measure the muon fluxes from hadron decay, as shown in Figure~\ref{fig:schematic-nu-beamline-ndc}. The absorber itself is part of the Beamline L2 Project and is discussed in Volume 2 of this CDR. 

\begin{cdrfigure}[Cartoon of the LBNF neutrino beamline components]{schematic-nu-beamline-ndc}{A cartoon of the LBNF neutrino beamline showing the major components of the neutrino beam. From left to right (the direction of the beam): the beam window, horn-protection baffle, target, the two toroidal focusing horns, decay pipe and absorber. The BLM system will be to the right of the absorber.}
%\includegraphics[width=.9\textwidth]{../volume-2-beam/figures/Fig_C3_Intro_1}\end{cdrfigure}
\end{cdrfigure}

This BLM system is intended to determine the neutrino fluxes and spectra
and to monitor the beam profile on a spill-by-spill basis, and will operate for the life of the
experiment. 
Figure~\ref{fig:MuonSystemsOverview} shows the downstream side of the absorber and a conceptual 
layout of the BLM muon systems.
The first set 
of muon measurement devices, from  left to right, are three
variable-pressure, gas Cherenkov counters. Following these counters 
is the 5 $\times$ 5 ion-chamber array that is
 mounted directly to the rear wall of the absorber. 
Finally, there is a set of stopped-muon counters 
 interspersed between walls of steel ``blue blocks''. The blue blocks 
provide several
 depths at which to monitor the muons as they range out and stop 
in the material.

\begin{cdrfigure}[Layout of muon monitors]{MuonSystemsOverview}{A model of the three muon monitor systems placed downstream of the absorber, in the muon alcove.}
%\includegraphics[width=5in,angle=0]{Alcove_w_detectors.pdf}
\end{cdrfigure}

The GDAQ will collect raw data from each
detector system in the NDC, issue global triggers, add timing data
from a global position system (GPS) and build events.


\section{Participants}

The design for the DUNE Near Detector Complex (NDC) is being carried out by an DUNE L2 Project team, headed at Los Alamos National Laboratory (LANL) 
with participants from universities, in conjunction with outside contractors.  
The BLM are planned for construction at 
Fermilab.

The DUNE NDC is managed by the 
Level 2 Manager for the Near Detector Complex L2 Project. The supporting team 
includes a Level 3 
Manager for each of the NDC's principal systems: Beamline Measurement System, Global Data Acquisition and Computing. The organization is illustrated to WBS Level 4 in Figure~\ref{fig:nd-org}.

\begin{cdrfigure}[Organization chart for the NDC L2 Project to L4]{nd-org}{Organization chart for the NDC L2 Project (to WBS Level 4)}
%\includegraphics[width=.9\textwidth]{ndc-org-no-names}
\end{cdrfigure}

The Conventional Facilities Level 3 Near Site Manager is the LBNF Project liaison with the NDC L2 Project 
to ensure the L2 project civil requirements are met; 
this person is responsible for all LBNF conventional facilities scope at the Near Site. 

Interaction between Fermilab facility engineers, DUNE NDC design teams, and
design consultants consists of 
weekly telephone conferences, periodic design interface workshops, and
electronic mail. An integration document resulting from these discussions will allow the NDC L2 Project team to coordinate activities and ensure that the design efforts remain on track to satisfy all requirements.


%%%%%%%%%%%%%%%%%%%%%%%%%%% Data for the Acro/Abbrev/Units list  %%%%%%%%%%%%%%%%%

\nomenclature{ADC}{analog-to-digital converter}
\nomenclature{ASIC}{application-specific integrated circuit}
\nomenclature{BEB}{back-end board}
\nomenclature{BLM}{beamline-measurement system}
\nomenclature{B-GDAQ}{beamline global data acquisition}
\nomenclature{CC}{charged current (interaction)}
\nomenclature{CCQE}{charged current quasi-elastic (interaction)}
\nomenclature{CERN}{European Organization for Nuclear Research}
\nomenclature{CNGS}{Neutrino Beam to Gran Sasso (at CERN)}
\nomenclature{DAQ}{data acquisition}
\nomenclature{DUNE}{Deep Underground Neutrino Experiment}
\nomenclature{ECAL}{electromagnetic calorimeter}
\nomenclature{ESH}{Environment, Safety and Health}
\nomenclature{eV}{electron-Volt, unit of energy (also keV, MeV, GeV, etc.)}
\nomenclature{FEB}{front-end board}
\nomenclature{FGT}{Fine-Grained Tracker}
\nomenclature{FPGA}{field-programmable gate array}
\nomenclature{FRA}{Fermi Research Alliance}
\nomenclature{g-2}{the New Muon g-2 Experiment at Fermilab}
\nomenclature{GDAQ}{global data acquisition}
\nomenclature{GPS}{Global Position System}
\nomenclature{HEP}{high energy physics}
\nomenclature{ICARUS}{Imaging Cosmic And Rare Underground Signals (experiment at LNGS)}
\nomenclature{K2K}{``From KEK to \superk,'' long-baseline neutrino-oscillation experiment}
\nomenclature{LANL}{Los Alamos National Laboratory}
\nomenclature{LAr}{liquid argon}
\nomenclature{LArTPC}{Liquid Argon Time Projection Chamber}
\nomenclature{LArTPCT}{Liquid Argon Time Projection Chamber Tracker system}
\nomenclature{LAr-FD}{(LBNE) Liquid Argon Far Detector}
\nomenclature{LBNF}{Long-Baseline Neutrino Facility}
\nomenclature{LHC}{Large Hadron Collider (at CERN)}
\nomenclature{LNGS}{Gran Sasso National Laboratory}
\nomenclature{M-GDAQ}{master global data acquisition}
\nomenclature{m}{meter (also nm, micron, mm, cm, km) }
\nomenclature{MicroBooNE}{A 100-ton LArTPC located along Fermilab's Booster neutrino beamline}
\nomenclature{MINERvA}{A neutrino-scattering experiment that uses the NuMI beamline at Fermilab}
\nomenclature{MiniBooNE}{Booster Neutrino Experiment (at Fermilab)}
\nomenclature{MINOS}{Main Injector Neutrino Oscillation Search experiment at Fermilab}
\nomenclature{MIPP}{Main Injector Particle Production Experiment (at Fermilab)}
\nomenclature{MPPC}{multi-pixel photon counter}
\nomenclature{MuID}{muon-identification detector}
\nomenclature{NC}{neutral current (interaction)}
\nomenclature{ND}{(Near Site) neutrino detector }
\nomenclature{NDC}{Near Detector Complex; refers to the L2 Project under LBNE }
\nomenclature{NOMAD}{Neutrino Oscillation Magnetic Detector, experiment at CERN}
\nomenclature{NA61}{NA61/SHINE, experiment at CERN that studies hadron production in hadron-nucleus and nucleus-nucleus collisions}
\nomenclature{N-GDAQ}{neutrino global data acquisition}
\nomenclature{NOvA}{NuMI Off-Axis Neutrino Appearance experiment at Fermilab}
\nomenclature{NuMI}{Neutrino beam to MINOS}
\nomenclature{PMT}{photomultiplier tube}
\nomenclature{QE}{quasi-elastic (interaction)}
\nomenclature{STT}{straw-tube tracker}
\nomenclature{Super-K}{Super-Kamiokande, a neutrino detector in Japan}
\nomenclature{T2K}{Tokai-to-Kamioka, a long-baseline neutrino oscillation experiment in Japan}
\nomenclature{T2K ND280}{Near Detector of the Tokai-to-Kamioka (T2K) experiment, located 280m from the beamline target}
\nomenclature{TDC}{time-to-digital converter}
\nomenclature{TFB}{T2K front-end board}
\nomenclature{TPC}{time projection chamber}
\nomenclature{TR}{transition radiation}
\nomenclature{TRIP-t}{Trigger and Pipeleine with timing (full custom ASIC designed at Fermilab)}
\nomenclature{WBS}{Work Breakdown Structure}
\nomenclature{T}{Tesla; unit of magnetic field strength}
\nomenclature{UA1}{Experiment at CERN that ran from 1986 to 1993}
\nomenclature{UV}{ultra-violet}
\nomenclature{V}{Volt}
\nomenclature{W}{watt (also mW, kW, MW) }
\nomenclature{WLS}{wavelength shifter}

